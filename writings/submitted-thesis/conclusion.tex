

\chapter{Conclusion}

\label{ch:conclusion}

This thesis has laid the foundations for reasoning about asynchronous
multi-agent domains in the situation calculus. As highlighted by our
initial investigations and implementation of the multi-agent Golog
variant MIndiGolog, the standard reasoning and planning machinery
of the situation calculus often depends on an assumption of synchronicity.
In many cases, this synchronicity is enforced by requiring all actions
to be publicly observable.

We identified three main limitations of the situation calculus when
trying to extend its reach into asynchronous domains. First, it generates
fully-ordered sequences of actions as the output of the Golog execution
planning process, which cannot be feasibly executed in the face of
hidden actions. Second, its standard account of agent-level knowledge
cannot effectively handle arbitrarily-long sequences of hidden actions.
Finally, it lacks a formal account of reasoning about group-level
epistemic modalities such as common knowledge, which are vital for
managing coordination in multi-agent domains.

At the core of our approach to overcoming these limitations is a new,
explicit representation of the local perspective of each agent. By
formalising what each agent \emph{observes} when a particular set
of actions is performed, and its corresponding local \emph{view} in
each situation, we are able to approach reasoning and planning in
a principled way without making any assumptions about the dynamics
of the domain. In particular, we can explicitly define and represent
asynchronous domains as those in which some action occurrences generate
no observations; in other words, domains in which agents cannot determine
how many actions have been performed.

Building on this foundation, we have developed four key extensions
to the reasoning and planning machinery of the situation calculus
that work to overcome its current limitations in asynchronous multi-agent
domains.


\section{Contributions}

Our first key contribution defines a partially-ordered branching action
structure to replace raw situation terms as the output of the Golog
execution planning process. Called \emph{joint executions}, they represent
a set of many possible histories that could constitute a legal execution
of the program. They allow independent actions to be performed independently,
while ensuring that inter-agent synchronisation is always possible
when required. By formulating these requirements explicitly in terms
of the local view available to each agent, we identify joint executions
that are feasible to perform in the world despite potential asynchronicity
in the domain.

The utility of these structures was demonstrated by implementing an
offline execution planner that produces joint executions as its output.
By imposing some simple restrictions on the theory of action, the
planner is able to reason about joint executions without having to
explicitly consider the exponentially-many possible histories of such
a partially-ordered structure. It can thus make use of the standard
reasoning machinery of the situation calculus developed in existing
Golog implementations.\\


Second, we have characterised a kind of inductive query that we call
a \emph{property persistence} query. These queries are restricted
enough to be amenable to a special-purpose reasoning algorithm based
on a meta-level fixpoint calculation. A simple iterative approximation
algorithm was presented and shown to be complete for several interesting
cases. More importantly, we have shown that such queries can always
be replaced with a uniform formula called the \emph{persistence condition,}
in a way that integrates well with the standard regression operator.
This allows certain second-order aspects of our formulation to be
{}``factored out'' and handled using special-purpose tools, while
maintaining the use of regression as the primary reasoning technique.\\


The third major contribution is a powerful new account of individual-level
epistemic reasoning, in which an agent's knowledge is expressed directly
in terms of its local observations. In asynchronous domains agents
are required to account for arbitrarily-long sequences of hidden actions
and must therefore perform some inductive reasoning. By precisely
characterising the inductive component of their reasoning in terms
of a property persistence query, we factor it out of the reasoning
process and provide a regression-based technique for answering knowledge
queries.

Basing our formalism explicitly on an agent's observations provides
two important benefits. It makes our formalism robust to theory elaboration,
so that our theorems and our regression rule apply unmodified as more
complex knowledge-producing actions are added to the domain. Second,
it means that a situated agent can directly use our regression rules
to reason about its own knowledge using only its local information.

We have also demonstrated that if the theory of action is known to
be synchronous, then our regression rule does not require inductive
reasoning and it reduces to a simple syntactic transformation. Our
account of individual-level knowledge thus provides a flexible new
formalism that is comparable in reasoning complexity to the standard
account of knowledge for synchronous domains, while at the same time
extending gracefully to asynchronous domains where inductive reasoning
is required.\\


Finally, we have introduced a powerful new language of complex epistemic
modalities to the situation calculus. Based on an epistemic interpretation
of dynamic logic, these modalities are expressive enough to formulate
a regression rule for common knowledge while still permitting arbitrarily-long
sequences of hidden actions to be handled using the persistence condition
operator.

Our work provides the first formal account of effective reasoning
about common knowledge in the situation calculus. If the domain is
restricted to be synchronous, our regression rule does not require
inductive reasoning and common knowledge can be reasoned using a purely
syntactic manipulation, a powerful new ability in its own right. By
basing our formalism on an explicit representation of each agent's
local view, it can also extend gracefully to handle asynchronous domains
in which some inductive reasoning is required.\\


These contributions provide a powerful fundamental framework for the
situation calculus to represent and reason about asynchronous multi-agent
domains.


\section{Future Work}

Throughout the thesis, we have also identified areas where further
work is required to bring our new techniques together with practical
systems built on the situation calculus. Work continues on developing
a MIndiGolog execution planner capable of coordinating \emph{online}
execution of a shared program in asynchronous domains, building on
the techniques we have developed here. The most promising avenues
of future research are summarised below.

Chapter \ref{ch:cknowledge} developed a precise characterisation
of the processes required to reason about common knowledge in the
situation calculus, and demonstrated that it is decidable in principle
in finite domains by conversion into PDL. Unfortunately existing PDL
provers cannot be used for even very simple domains, due to an exponential
blowup in problem size during this conversion. We have some initial
intuitions on constructing a \emph{free-variable} prover for our modalities
to avoid performing this conversion in its entirety, but there is
still significant implementation work to be done before the technique
can be applied in practical domains.

There is also the possibility of identifying \emph{fragments} of our
epistemic language that have more tractable reasoning procedures,
but are still powerful enough to capture the regression of common
knowledge under certain domain restrictions. A promising starting
point would be to extend the relativised common knowledge operator
of \citep{vanBenthem06lcc} to the first-order case, and determine
whether it suffices for regressing common knowledge when all actions
are public, as it does in propositional formalisms.

While a possible-worlds formulation of knowledge as developed in this
thesis provides an excellent theoretical foundation for epistemic
reasoning, possible-worlds reasoning can be highly intractable in
practice. In Section \ref{sec:Knowledge:Advances} we suggested a
way to combine the formalism for tractable literal-level knowledge
of \citep{demolombe00tractable_sc_belief} with our new observation-based
semantics, which would be interesting to explore in more detail. It
would also be interesting to extend the approach of \citep{demolombe00tractable_sc_belief}
with a literal-level account of common knowledge, to enable approximate
reasoning about group-level epistemic modalities. Such a formalism
would need to decompose disjunctions inside epistemic paths, and it
is far from clear whether this could be done in a principled manner.

Finally, agents in a cooperative team need not consider \emph{all}
possible actions of their teammates, since they know that the other
agents will behave according to some protocol. We have identified
a potential approach based on the work of \citep{fritz08congolog_sin_trans}
that would allow protocols to be expressed as Golog programs, but
significant work will be required to produce a detailed theory based
on these ideas.

