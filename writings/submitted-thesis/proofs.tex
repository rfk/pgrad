

\chapter{Detailed Proofs}

\label{ch:proofs}

This appendix contains complete proofs for various lemmas and theorems
throughout the paper, along with some additional lemmas.\medskip{}


\begin{thmext}
[\ref{thm:PstN-works}] For any $n\in\mathbb{N}$, $\Pst_{\Dt}^{n}(\phi,\alpha)[\sigma]$
iff $\phi$ holds in $\sigma$ and in all successors of $\sigma$
reached by performing at most $n$ actions satisfying $\alpha$:\begin{multline*}
\Dt\,\models\,\Pst_{\Dt}^{n}(\phi,\alpha)[\sigma]\,\equiv\,\\
\bigwedge_{i\leq n}\forall a_{1},\dots,a_{i}:\,\left(\bigwedge_{j\leq i}\alpha[a_{j},do([a_{1},\dots,a_{j-1}],\sigma)]\,\,\rightarrow\,\,\phi[do([a_{1},\dots,a_{i}],\sigma)]\right)\end{multline*}

\end{thmext}
\begin{proof}
By induction on the natural numbers. For $n=0$ we have $\phi[\sigma]\equiv\phi[\sigma]$
by definition. For the inductive case, we expand the definition of
$\Pst_{\Dt}^{n}(\phi,\alpha)[\sigma]$ to get the following for the
LHS:\[
\Pst_{\Dt}^{n-1}(\phi,\alpha)[\sigma]\wedge\forall a:\,\Reg_{\Dt}(\alpha[a,\sigma])\rightarrow\Reg_{\Dt}(\Pst_{\Dt}^{n-1}(\phi,\alpha)[do(a,\sigma)])\]


By the inductive hypothesis we can equate $\Pst_{\Dt}^{n-1}(\phi,\alpha)[\sigma]$
in this LHS with all but the $i=n$ clause from the RHS conjunction,
and we suppress them on both sides. If we also drop the regression
operators we are left with the following to establish:\begin{multline*}
\Dt\,\models\,\forall a:\,\alpha[a,\sigma]\rightarrow\Pst_{\Dt}^{n-1}(\phi,\alpha)[do(a,\sigma)]\,\equiv\,\\
\forall a_{1},\dots,a_{n}:\,\left(\bigwedge_{j\leq n}\alpha[a_{j},do([a_{1},\dots,a_{j-1}],\sigma)]\,\,\rightarrow\,\,\phi[do([a_{1},\dots,a_{n}],\sigma)]\right)\end{multline*}


We can again use the inductive hypothesis on $\Pst_{\Dt}^{n-1}$ in
the LHS of this equivalence. If we then distribute the $\alpha[a,\sigma]$
implication over the outermost conjunction and collect quantifiers,
we obtain the following for the LHS:\begin{multline*}
\bigwedge_{i\leq n-1}\forall a,a_{1},\dots,a_{i}:\\
\left(\alpha[a,\sigma]\wedge\bigwedge_{j\leq i}\alpha[a_{j},do([a,a_{1},\dots,a_{j-1}],\sigma)]\rightarrow\phi[do([a,a_{1},\dots,a_{i}],\sigma)]\right)\end{multline*}


Renaming $a\Rightarrow a_{1}$, $a_{1}\Rightarrow a_{2}$, \ldots{},$a_{i}\Rightarrow a_{i+1}$,
we see that each of the $i<n-1$ clauses on the LHS is equivalent
to one of the $i<n$ clauses that have been suppressed on the RHS.
The remaining $i=n-1$ clause is equivalent to the required RHS: \[
\forall a_{1},\dots,a_{n}:\,\left(\bigwedge_{j\leq n}\alpha[a_{j},do([a_{1},\dots,a_{j-1}],\sigma)]\,\,\rightarrow\,\,\phi[do([a_{1},\dots,a_{n}],\sigma)]\right)\]


We therefore have the desired equivalence. 
\end{proof}
\medskip{}


\begin{lemma}
\label{lem:pbu-implies-legal}For situation terms $s$ and $s'$,
and agent $agt$:\[
\Dt\cup\Dt_{K}^{obs}\,\models\, Legal(s)\wedge s\leq_{\LbU(agt)}s'\,\rightarrow\, Legal(s')\]

\end{lemma}
\begin{proof}
Since $\LbU$ implies $Legal$, $s\leq_{LbU(agt)}s'$ implies $s\leq_{Legal}s'$.
Given the definition of $Legal(s)$ as $root(s)\leq_{Legal}s$, we
have the implication as desired. 
\end{proof}
\medskip{}


\begin{lemma}
\label{lem:pbu-implies-view}For situation terms $s$ and $s'$, and
agent $agt$:\[
\Dt\cup\Dt_{K}^{obs}\,\models\, s\leq_{\LbU(agt)}s'\,\rightarrow\, View(agt,s')=View(agt,s)\]

\end{lemma}
\begin{proof}
By induction on situations, using the definition of $View$ from equation
(\ref{eq:view_defn}). In the base case of $s'=s$ the lemma is trivial.
For the inductive case let $s'=do(c,s'')$, with $s\leq_{LbU(agt)}s''$
and $View(agt,s'')=View(agt,s)$ by the inductive hypothesis. By the
definition of $\leq_{\LbU(agt)}$ we have that $Obs(agt,c,s'')=\{\}$.
By the definition of $View$ we then have that $View(agt,s')=View(agt,s'')$,
which equals $View(agt,s)$ giving the lemma as required. 
\end{proof}
\medskip{}


\begin{lemma}
\label{lem:K-implies-root}For situation terms $s$ and $s''$, and
agent $agt$:\[
\Dt\cup\Dt_{K}^{obs}\,\models\, K(agt,s'',s)\,\rightarrow\, K_{0}(agt,root(s''),root(s))\]

\end{lemma}
\begin{proof}
By induction on situations. The lemma is trivial in the base case
of $Init(s)$. For the $do(c,s)$ case, suppose that we have $K(agt,s'',do(c,s))$.
Then by equation (\ref{eqn:new_k_ssa}) there is some $s'$ such that
$s'\sqsubseteq s''$ and $K(agt,s',s)$. Then $root(s'')$ = $root(s')$,
and $K_{0}(root(s'),root(s))$ by the inductive hypothesis, giving
the required result. 
\end{proof}
\medskip{}


\begin{lemma}
\label{lem:K-implies-legal}For situation terms $s$ and $s''$, and
agent $agt$:\[
\Dt\cup\Dt_{K}^{obs}\,\models\, K(agt,s'',s)\,\rightarrow\, Legal(s'')\]

\end{lemma}
\begin{proof}
By induction on situations. All $s'$ such that $K_{0}(agt,s',s)$
are initial and therefore legal. So using equation (\ref{eqn:new_k_s0})
in the base case, if $K(agt,s'',s)$ then there must be an $s'$ such
that $Init(s')$ and $s'\leq_{\LbU(agt)}s''$, making $s''$ legal
by Lemma \ref{lem:pbu-implies-legal}. For the $do(c,s)$ case, equation
(\ref{eqn:new_k_ssa}) ensures that $do(c',s')\leq_{\LbU(agt)}s''$
for some $s'$,$c'$ satisfying $K(agt,s',s)$ and $Legal(c',s')$.
So $s'$ is legal by the inductive hypothesis, making $s''$ legal
by Lemma \ref{lem:pbu-implies-legal} as required. 
\end{proof}
\medskip{}


\begin{lemma}
\label{lem:K-implies-view}For situation terms $s$ and $s''$, and
agent $agt$:\[
\Dt\cup\Dt_{K}^{obs}\,\models\, K(agt,s'',s)\,\rightarrow View(agt,s'')=View(agt,s)\]

\end{lemma}
\begin{proof}
By induction on situation terms. For the base case of $Init(s)$,
using equation \eqref{eqn:new_k_s0}, $K(agt,s'',s)$ implies that
there must be an $s'$ such that $Init(s')$ and $s'\leq_{\LbU(agt)}s''$.
Therefore $View(s'')$ = $View(s')$ = $\epsilon$ = $View(s)$ as
required.

For the $do(c,s)$ case, suppose $Obs(agt,c,s)=\{\}$. We have $View(agt,do(c,s))$
= $View(agt,s)$, while equation \eqref{eqn:new_k_ssa} gives us $K(agt,s'',s)$,
which yields $View(agt,s'')$ = $View(agt,s)$ by the inductive hypothesis.

Alternately, suppose $Obs(agt,c,s)\neq\{\}$, then equation \eqref{eqn:new_k_ssa}
gives us $s'$,$c'$ such that $do(c',s')\leq_{\LbU(agt)}s''$, $Obs(agt,c,s)$
= $Obs(agt,c',s')$, and $K(agt,s',s)$. By the inductive hypothesis
$View(agt,s')=View(agt,s)$, and we have the following: $View(agt,s'')$
= $View(agt,do(c',s'))$ = $Obs(agt,c,s)\cdot View(agt,s')$. This
is in turn equal to $View(agt,do(c,s))$ as required. 
\end{proof}
\medskip{}


\begin{thmext}
[\ref{thm:k_obs_equiv}] For any agent $agt$ and situations
$s$ and $s''$:\begin{multline*}
\Dt\cup\Dt_{K}^{obs}\,\models K(agt,s'',s)\equiv\\
K_{0}(root(s''),root(s))\wedge Legal(s'')\wedge View(agt,s'')=View(agt,s)\end{multline*}

\end{thmext}
\begin{proof}
For the \emph{if} direction, we simply combine Lemmas \ref{lem:K-implies-root},
\ref{lem:K-implies-legal} and \ref{lem:K-implies-view}. For the
\emph{only-if} direction we proceed by induction on situations. In
the base case of $Init(s)$, the $\exists s'$ part of equation (\ref{eqn:new_k_s0})
is trivially satisfied by $root(s'')$ and the equivalence holds as
required.

For the inductive case with $do(c,s)$, we have two sub-cases to consider.
Suppose $Obs(agt,c,s)=\{\}$: then $View(agt,s'')$ = $View(agt,do(c,s))$
= $View(agt,s)$ and $K(agt,s'',s)$ holds by the inductive hypothesis,
satisfying the equivalence in equation (\ref{eqn:new_k_ssa}). Alternately,
suppose $Obs(agt,c,s)\neq\{\}$: then we have:\[
View(agt,do(c,s))=Obs(agt,c,s)\cdot View(agt,s)=View(agt,s'')\]
 Since $s''$ is legal, this implies there there is some $s'$,$c'$
satisfying $Obs(agt,c',s')$ = $Obs(agt,c,s)$, $View(agt,s')$ =
$View(agt,s)$ and $do(c',s')\leq_{\LbU(agt)}s''$ . This is enough
to satisfy the $\exists s',c'$ part of equation (\ref{eqn:new_k_ssa})
and so the equivalence holds as required. 
\end{proof}
\medskip{}


\begin{thmext}
[\ref{thm:Reg_Knows}] Given a basic action theory $\Dt\cup\Dt_{K}^{obs}$
and a uniform formula $\phi$:\[
\Dt\cup\Dt_{K}^{obs}\,\models\,\Knows(agt,\phi,s)\equiv\Reg(\Knows(agt,\phi,s))\]

\end{thmext}
\begin{proof}
To obtain this result, we must establish that our new regression rules
in equations \eqref{eqn:R_do_c_s} and \eqref{eqn:R_s0} are equivalences
under the theory of action $\Dt\cup\Dt_{K}^{obs}$. The mechanics
of the proof mirror the analogous proof in \citep{scherl03sc_knowledge},
but with the addition of a persistence condition application.

For notational clarity we define the abbreviation $\mathbf{PEO}(agt,\phi,o,s)$
(for {}``persists under equivalent observations'') which states
that $\phi$ holds in all legal futures of $s$ compatible with observations
$o$:\begin{multline*}
\mathbf{PEO}(agt,\phi,o,s)\,\isdef\,\\
\forall c':\, Obs(agt,c',s)=o\wedge Legal(c',s)\rightarrow\left[\forall s':\, do(c',s)\leq_{\LbU(agt)}s'\rightarrow\,\phi[s']\right]\end{multline*}
 Expanding the definition of the $\Knows$ macro at $do(c,s)$, and
applying the successor state axiom from equation \eqref{eqn:new_k_ssa}
to the $K(agt,s'',do(c,s))$ term, we can produce the following:

\begin{align*}
\Knows(agt,\phi,do(c,s))\equiv\, & \forall s'':\, K(agt,s'',do(c,s))\,\rightarrow\,\phi[s'']\\
\equiv\, & \exists o:\, Obs(agt,c,s)=o\\
 & \wedge\,\left[o=\{\}\rightarrow\forall s':\, K(agt,s',s)\rightarrow\phi[s']\right]\\
 & \wedge\,\left[o\neq\{\}\rightarrow\forall s':\, K(agt,s',s)\rightarrow\mathbf{PEO}(agt,\phi,o,s')\right]\end{align*}


Noting that both conjuncts contain sub-formulae matching the form
of the $\Knows$ macro, it can be substituted back in to give:\begin{align*}
\mathbf{Knows}(agt,\phi,do(c,s))\equiv\, & \exists o:\, Obs(agt,c,s)=o\\
 & \wedge\,\left[o=\{\}\rightarrow\mathbf{Knows}(agt,\phi,s)\right]\\
 & \wedge\,\left[o\neq\{\}\rightarrow\mathbf{Knows}(agt,\mathbf{PEO}(agt,\phi,o),s)\right]\end{align*}


For $\mathbf{PEO}(agt,\phi,o,s')$ to legitimately appear inside the
$\mathbf{Knows}$ macro it must be uniform in the situation variable
$s'$. Applying the persistence condition and regressing to make the
expression uniform, we develop the following equivalence:\[
\mathbf{PEO}(agt,\phi,o,s)\equiv\forall c':\, Obs(agt,c',s)=o\wedge Legal(c',s)\rightarrow\Reg(\Pst(\phi,\LbU(agt)),c')\]


Suppressing the situation term in this uniform formula gives the regression
rule from equation \eqref{eqn:R_do_c_s} as required.

For $S_{0}$, a straightforward transformation of equations \eqref{eqn:Knowledge:knows_def}
and \eqref{eqn:new_k_s0} gives:\[
\mathbf{Knows}(agt,\phi,S_{0})\equiv\forall s:\, K_{0}(agt,s,S_{0})\rightarrow\left[\forall s':\, s\leq_{\LbU(agt)}s'\rightarrow\phi[s']\right]\]
 Applying the persistence condition operator, this can easily be re-written
as:\[
\mathbf{Knows}(agt,\phi,S_{0})\equiv\forall s\,.\, K_{0}(agt,s,S_{0})\rightarrow\Pst(\phi,\LbU(agt))[s]\]


This matches the form of the definition for $\KnowsZ$, which we can
substitute in to give:\[
\Knows(agt,\phi,S_{0})\equiv\KnowsZ(agt,\Pst(\phi,\LbU(agt)),S_{0})\]


Since all situations reachable by $K_{0}$ are initial, and since
regression preserves equivalence, it is valid to use $\Reg(\psi[S_{0}])^{-1}$
on the enclosed formula to give:\[
\Knows(agt,\phi,S_{0})\equiv\KnowsZ(agt,\Reg(\Pst(\phi,\LbU(agt))[S_{0}])^{-1},S_{0})\]


This is the regression rule from equation \eqref{eqn:R_s0} as required.
Our regression rules are thus equivalences under the theory $\Dt\cup\Dt_{K}^{obs}$,
and the theorem holds. 
\end{proof}
\medskip{}


\begin{thmext}
[\ref{thm:Reg_KnowsO}] Given a basic action theory $\Dt$
and a uniform formula $\phi$:\[
\Dt\cup\Dt_{K}^{obs}\,\models\,\Knows(agt,\phi,v)\equiv\Reg(\Knows(agt,\phi,v))\]

\end{thmext}
\begin{proof}
Recall the definition of $\Knows(agt,\phi,v)$ as follows:\begin{multline*}
\mathbf{Knows}(agt,\phi,v)\,\isdef\,\\
\forall s:\, View(agt,s)=v\wedge root(s)=S_{0}\wedge Legal(s)\rightarrow\mathbf{Knows}(agt,\phi,s)\end{multline*}


We also have the following simple corollary of Theorem \ref{thm:k_obs_equiv}:\begin{multline*}
\Dt\cup\Dt_{K}^{obs}\models\forall s,s',s'':\, root(s)=root(s')\wedge View(s)=View(s')\\
\wedge K(agt,s'',s)\rightarrow K(agt,s'',s')\end{multline*}


The definition of $\Knows(agt,\phi,v)$ is thus equivalent to:\begin{multline*}
\mathbf{Knows}(agt,\phi,v)\,\equiv\,\,\neg\exists s:\, View(agt,s)=v\wedge root(s)=S_{0}\wedge Legal(s)\\
\vee\,\exists s:\, View(agt,s)=v\wedge root(s)=S_{0}\wedge Legal(s)\wedge\mathbf{Knows}(agt,\phi,s)\end{multline*}


We thus need to find a single witness situation rather than examining
all situations with that view. We proceed by induction over views.
For the $\epsilon$ case, $S_{0}$ serves as an appropriate witness
since it is always legal, $View(agt,S_{0})=\epsilon$ and $root(S_{0})=S_{0}$.
Applying the regression rule for $\Knows(agt,\phi,S_{0})$ gives us
the same expression as applying the regression rule for $\Knows(agt,\phi,\epsilon)$.
So if $\Reg(\Knows(agt,\phi,\epsilon))$ holds then so does $\Knows(agt,\phi,S_{0})$.
Using $S_{0}$ as a witness we conclude that $\Knows(agt,\phi,\epsilon)$
iff $\Reg(\Knows(agt,\phi,\epsilon))$ as desired.

For the inductive $o\cdot v$ case we split on whether there is any
situation having that view. Suppose there is no such situation, then
the definition of $\Knows(agt,\phi,o\cdot v)$ is trivially satisfied
and the agent must know all statements. We need to show that the regression
of $\Knows(agt,\phi,o\cdot v)$ is always entailed by the domain in
this case. The regressed expression is:\[
\mathbf{Knows}(agt,\forall c:\, Obs(agt,c)=o\wedge Legal(c)\rightarrow\Reg(\Pst(\phi,\LbU(agt)),c),v)\]


If there is no situation having view $v$, then there is also no situation
having view $o\cdot v$, and the above is entailed by the inductive
hypothesis in this case.

Alternately, suppose there is a situation $s$ having view $v$ but
no legal situation having view $o\cdot v$. Then all situations $s'$
that have view equal to $v$ must satisfy $\neg\exists c:\, Obs(agt,c,s')=o\wedge Legal(c,s')$,
otherwise we could construct a situation with view $o\cdot v$. Since
these situations $s'$ are the only ones that can be $K$-related
to $s$, the antecedent in the above implication is falsified at all
such situations, and the regressed expression is equivalent to $\Knows(agt,\top,v)$
which is trivially entailed.

Finally, suppose there is a legal situation $do(c,s)$ having view
$o\cdot v$. We can assume without loss of generality that $Obs(agt,c,s)=o$
and $View(agt,s)=v$. Regressing $\Knows(agt,\phi,do(c,s))$ in this
case will produce:\begin{alignat*}{1}
\Reg(\Knows(agt,\phi,do(c,s))=\,\, & \exists o':\, Obs(agt,c,s)=o'\\
 & \wedge\,\left[o'=\{\}\,\rightarrow\,\Knows(agt,\phi,s)\right]\\
 & \wedge\,\left[o'\neq\{\}\,\rightarrow\,\Knows(agt,\forall c':\, Obs(agt,c')=o'\right.\\
 & \,\,\,\,\,\,\,\,\,\,\,\wedge Legal(c')\rightarrow\left.\Reg(\Pst(\phi,\LbU(agt)),c'),s)\right]\end{alignat*}


We have that $Obs(agt,c,s)=o$ and $o\neq\{\}$, so this is equivalent
to:\[
\Knows(agt,\forall c':\, Obs(agt,c')=o\wedge Legal(c')\rightarrow\Reg(\Pst(\phi,\LbU(agt)),c'),s)\]


Since this matches the form of $\Reg(\Knows(agt,\phi,o\cdot v))$,
and we have that the view of $s$ is $v$, this will be entailed by
the domain precisely when the regression of $\Knows(agt,\phi,o\cdot v)$
is entailed by the domain.

Thus if there is no legal situation with view $v$ then $\Reg(\Knows(agt,\phi,v))$,
is always entailed, while if there is such a situation $s$ then $\Reg(\Knows(agt,\phi,v))$
is equivalent to $\Reg(\Knows(agt,\phi,s))$. The regression rules
over observations are thus equivalences as desired. 
\end{proof}
\medskip{}


\begin{lemma}
\label{lem:TrnA_works}For any epistemic path $\pi$:\begin{multline*}
\Dt\cup\Dt_{K}^{obs}\models\,\KDoZ(\pi,do(c,s),s'')\,\equiv\,\exists\mu,\mu',c',s':\\
\mu(x)=c\wedge\mu'(x)=c'\wedge\left(c'\neq\{\}\wedge s''=do(c',s')\,\vee\, c'=\{\}\wedge s''=s'\right)\\
\wedge\KDoZ(\Trn_{a}(\pi,x),\mu,s,\mu',s')\end{multline*}

\end{lemma}
\begin{proof}
Proceed by cases, covering each path operator in turn. For the base
case of an individual agent, we have:\begin{align*}
\KDoZ(agt,do(c,s),s'')\,\equiv\, & K_{0}(agt,s'',do(c,s))\end{align*}
 \[
\TrnA(agt,x)=\,\, z\Leftarrow Obs(agt,x)\,;\, agt\,;\,\exists x\,;\,?Legal(x)\vee x=\{\}\,;\,?Obs(agt,x)=z\]
 Expanding $\KDoZ(\TrnA(agt,x),\mu,s,\mu',s')$ thus produces:\begin{multline*}
\KDoZ(\TrnA(agt,x),\mu,s,\mu',s')\equiv z\Leftarrow Obs(agt,\mu(x),s)\,\wedge\,\exists s'':\, K_{0}(agt,s'',s)\,\wedge\\
\left(Legal(\mu'(x),s'')\vee\mu'(x)=\{\}\right)\,\wedge\, Obs(agt,\mu'(x),s'')=z\,\wedge\, s''=s'\end{multline*}
 Note that $\mu$ and $\mu'$ are never applied to a variable other
than $x$. When we substitute this into the RHS of the hypothesis,
$\mu(x)$ and $\mu'(x)$ are asserted to be $c$ and $c'$ respectively,
so they can be simplified away to give:\begin{multline*}
\Dt\cup\Dt_{K}^{obs}\models K(agt,s'',do(c,s))\equiv\\
\exists c',s':\left(c'\neq\{\}\wedge s''=do(c',s')\,\vee\, s''=s'\wedge c'=\{\}\right)\\
\wedge K_{0}(agt,s,s')\,\wedge\,\left(Legal(c',s')\vee c'=\{\}\right)\wedge\, Obs(agt,c,s)=Obs(agt,c',s')\end{multline*}
 This is the successor state axiom for $K_{0}$, which is trivially
entailed by the domain.\\


For the $?\phi$ case, we have:\[
\KDoZ(?\phi,do(c,s),s'')\equiv\phi[do(c,s)]\wedge s''=do(c,s)\]
 \[
\TrnA(?\phi,x)=\,\,?\Reg(\phi,x)\]


Giving:\[
\KDoZ(\TrnA(?\phi,x),\mu,s,\mu',s')\equiv\Reg(\phi,x)[s]\wedge s=s'\wedge\mu=\mu'\]
 Substituting into the RHS of the hypothesis, this asserts that $c=c'$
and hence $s''=do(c,s)$, so the hypothesis is clearly entailed.\\


The case for $\exists y$ is trivial as $\KDoZ(\exists y,s,s')\equiv s=s'$.\\


The inductive cases are straightforward as $\Trn_{a}$ is simply pushed
inside each operator. We will take the $\pi^{*}$ case as an example.
The inductive hypothesis gives:\begin{multline*}
\KDoZ(\pi,do(c,s),s'')\,\equiv\exists\mu,\mu',c',s':\,\mu(x)=c\wedge\mu'(x)=c'\\
\wedge\left(c'\neq\{\}\wedge s''=do(c',s')\,\vee\, s''=s'\wedge c'=\{\}\right)\wedge\KDoZ(\Trn_{a}(\pi,x),\mu,s,\mu,'s')\end{multline*}
 We can apply reflexive transitive closure to both sides of this equivalence,
along with two rearrangements: the LHS is expanded to the four-argument
form with $\exists\mu,\mu''$ at its front, and the rigid tests on
the RHS are taken outside the $RTC$ operation. This produces the
following equivalence:\begin{multline}
\exists\mu,\mu'':\, RTC[\KDoZ(\pi,\mu,do(c,s),\mu'',s'')]\,\equiv\\
\exists\mu,\mu',c',s':\,\mu(x)=c\wedge\mu'(x)=c'\wedge\\
(c'\neq\{\}\wedge s''=do(c',s')\,\vee\, s'=\{\}\wedge s''=s'c\wedge RTC[\KDoZ(\Trn_{a}(\pi,x),\mu,s,\mu,'s')]\label{eq:rtc-inductive-hyp}\end{multline}
 Using the definitions of $\KDoZ$ and $\TrnA$ we have:\[
\KDoZ(\pi^{*},do(c,s),s'')\equiv\exists\mu,\mu'':\, RTC[\KDoZ(\pi,\mu,do(c,s),\mu'',s'')]\]
 \[
\KDoZ(\TrnA(\pi^{*},x),\mu,s,\mu',s')\equiv RTC[\KDoZ(\TrnA(\pi,x),\mu,s,\mu',s')]\]
 Substituting these into the $RTC$ of the inductive hypothesis from
equation \eqref{eq:rtc-inductive-hyp} gives us the $\KDoZ(\pi^{*})$
cases we need to satisfy the theorem. 
\end{proof}
\medskip{}


\begin{thmext}
[\ref{thm:Trn-respects-epi-paths}] For any epistemic path
$\pi$:\begin{multline*}
\Dt\cup\Dt_{K}^{obs}\models\,\KDoZ(\pi,do(c,s),s'')\,\equiv\\
\exists c',s':\,\left(c'\neq\{\}\wedge s''=do(c',s')\,\vee\, c'=\{\}\wedge s''=s'\right)\wedge\KDoZ(\Trn(\pi,c,c'),s,s')\end{multline*}

\end{thmext}
\begin{proof}
Recall the rule for $\Trn(\pi,c,c')$:\[
\Trn(\pi,c,c')\isdef\,\,\, x\Leftarrow c\,;\,\TrnA(\pi,x)\,;\,?x=c'\]
 Expanding $\KDoZ$ for this rule:\[
\KDoZ(\Trn(\pi,c,c'),s,s')\equiv\exists\mu,\mu':\,\mu(x)=c\,\wedge\mu'(x)=c'\wedge\KDoZ(\TrnA(\pi,x),\mu,s,\mu',s')\]
 We can thus substitute $\KDoZ(\Trn(\pi,c,c'),s,s')$ into the RHS
of Lemma \ref{lem:TrnA_works} to get the required result. 
\end{proof}
\medskip{}


\begin{thmext}
[\ref{thm:Reg_PKnowsZ}] For any epistemic path $\pi$, uniform
formula $\phi$ and action $c$:\begin{gather*}
\Dt\cup\Dt_{K}^{obs}\,\models\,\PKnowsZ(\pi,\phi,do(c,s))\,\equiv\,\forall c':\,\PKnowsZ(\Trn(\pi,c,c'),\Reg(\phi,c'),s)\end{gather*}

\end{thmext}
\begin{proof}
The mechanics of this proof mirror that of Theorem \ref{thm:Reg_Knows}:
we expand the $\PKnowsZ$ macro, apply Theorem \ref{thm:Trn-respects-epi-paths}
as a successor state axiom for $\KDoZ$, re-arrange to eliminate existential
quantifiers, then collect terms back into forms that match $\PKnowsZ$.
We begin with the following:\begin{multline*}
\PKnowsZ(\pi,\phi,do(c,s))\,\equiv\,\,\forall s'':\,\KDoZ(\pi,do(c,s),s'')\,\rightarrow\,\phi[s'']\\
\shoveleft{\equiv\forall s'':\,\left[\exists c',s':\,\left(c'\neq\{\}\wedge s''=do(c',s')\,\vee\, c'=\{\}\wedge s''=s'\right)\right.}\\
\shoveright{\left.\wedge\,\KDoZ(\Trn(\pi,c,c'),s,s')\right]\rightarrow\,\phi[s'']}\\
\shoveleft{\equiv\forall s'',c',s':\,\left[\left(c'\neq\{\}\wedge s''=do(c',s')\,\vee\, c'=\{\}\wedge s''=s'\right)\right.}\\
\left.\wedge\,\KDoZ(\Trn(\pi,c,c'),s,s')\right]\rightarrow\,\phi[s'']\end{multline*}


Case-splitting on the disjunction, we see that:\begin{gather*}
s''=do(c',s')\wedge c'\neq\{\}\,\rightarrow\,\left(\phi[s'']\equiv\Reg(\phi,c')[s']\right)\end{gather*}
 \[
s''=s'\wedge c'=\{\}\,\rightarrow\,\left(\phi[s'']\equiv\Reg(\phi,c')[s']\right)\]


This allows us to remove the variable $s''$ from the consequent of
the implication, making it redundant in the antecedent and allowing
us to eliminate it entirely. Folding the quantification over $s'$
back into the $\PKnowsZ$ macro completes the proof:\begin{multline*}
\PKnowsZ(\pi,\phi,do(c,s))\\
\shoveleft{\equiv\,\forall s'',c',s':\,\left[\left(c'\neq\{\}\wedge s''=do(c',s')\,\vee\, c'=\{\}\wedge s''=s'\right)\right.}\\
\shoveright{\left.\wedge\,\KDoZ(\Trn(\pi,c,c'),s,s')\right]\rightarrow\,\Reg(\phi,c')[s']}\\
\shoveleft{\equiv\,\forall c',s':\,\KDoZ(\Trn(\pi,c,c'),s,s')\,\rightarrow\,\Reg(\phi,c')[s']}\\
\shoveleft{\equiv\,\forall c':\,\PKnowsZ(\Trn(\pi,c,c'),\Reg(\phi,c'),s)}\,\,\,\,\,\,\,\,\,\,\,\,\,\,\,\,\,\,\,\,\,\,\,\,\,\,\,\,\,\,\,\,\,\,\,\,\,\,\,\,\,\,\,\,\,\,\,\,\,\,\,\,\,\,\,\,\,\,\,\,\,\,\,\,\,\,\,\,\,\,\,\,\,\,\,\,\,\,\,\,\,\,\,\,\,\,\,\,\,\,\,\,\,\,\,\,\,\,\,\,\,\,\,\,\,\,\,\,\,\,\,\,\,\,\,\,\,\,\,\,\,\,\,\,\,\,\,\end{multline*}


This is the theorem as required. 
\end{proof}
\medskip{}


\begin{lemma}
\label{lem:KDoZ_E1_impl_KDoZ}For any epistemic path $\pi$:\[
\Dt\cup\Dt_{K}^{obs}\,\models\,\KDoZ(\Trn(\pi,\{\},\{\}),s,s')\,\rightarrow\,\KDoZ(\pi,s,s')\]

\end{lemma}
\begin{proof}
By a case analysis on the epistemic path operators. For the base case
of an individual agent, we have:\begin{align*}
\Trn(agt,\{\},\{\})\,=\, & x\Leftarrow\{\}\,;\,\exists z\,;\,?Obs(agt,x)=z\,;\, agt\,;\\
 & \,\,\,\,\,\,\exists x\,;\,?Legal(x)\vee x=\{\}\,;\,\,?Obs(agt,x_{a})=z\,;\,?x=\{\}\\
=\, & \exists z\,;\,?z=\{\}\,;\, agt\,;\,?Legal(\{\})\vee\{\}=\{\}\,;\,?z=\{\}\\
=\, & agt\end{align*}


So the hypothesis is clearly entailed. For the $?\phi$ case:\begin{align*}
\Trn(?\phi,\{\},\{\})=\, & x\Leftarrow\{\}\,;\,?\Reg(\phi,x)\,;\,?x=\{\}\\
=\, & ?\Reg(\phi,\{\})\\
=\, & ?\phi\end{align*}


So the hypothesis is clearly entailed. For the $\exists z$ case:\begin{align*}
\Trn(\exists z,\{\},\{\})=\, & x\Leftarrow\{\}\,;\,\exists z\,;\,?x=\{\}\\
=\, & \exists z\end{align*}


So the hypothesis is clearly entailed. The inductive cases are then
straightforward, by choosing $x=\{\}$ uniformly whenever $\exists x$
is encountered in the translated path. 
\end{proof}
\medskip{}


\begin{thmext}
[\ref{thm:En_impl_En-1}] For any epistemic path $\pi$: \[
\Dt\cup\Dt_{K}^{obs}\,\models\,\PKnowsZ(\pi,\phi,\mathcal{E}^{1}(s))\rightarrow\PKnowsZ(\pi,\phi,s)\]

\end{thmext}
\begin{proof}
Expanding the macros, we have:\[
\left(\forall s'':\,\KDoZ(\pi,do(\{\},s),s'')\rightarrow\phi[s'']\right)\,\,\rightarrow\,\,\left(\forall s':\,\KDoZ(\pi,s,s')\rightarrow\phi[s']\right)\]


Using equation \ref{thm:Trn-respects-epi-paths} on the LHS gives:\begin{multline*}
\left(\forall s'':\,\exists c',s':\,\left(c'\neq\{\}\wedge s''=do(c',s')\,\vee\, c'=\{\}\wedge s''=s'\right)\right.\\
\left.\wedge\KDoZ(\Trn(\pi,\{\},c'),s,s')\rightarrow\phi[s'']\right)\,\,\rightarrow\,\,\left(\forall s':\,\KDoZ(\pi,s,s')\rightarrow\phi[s']\right)\end{multline*}


We can strengthen the antecedent by dropping the $do(c',s')$ case;
if the implication holds with this stronger antecedent then it must
hold in its weaker form above. We obtain:\begin{multline*}
\left(\forall s'':\,\exists c',s':\, s''=s'\wedge c'=\{\}\wedge\KDoZ(\Trn(\pi,\{\},c'),s,s')\rightarrow\phi[s'']\right)\,\,\rightarrow\\
\left(\forall s':\,\KDoZ(\pi,s,s')\rightarrow\phi[s']\right)\end{multline*}


Simplifying away the variables $s''$ and $c'$ gives:\[
\left(\forall s':\,\KDoZ(\Trn(\pi,\{\},\{\}),s,s')\rightarrow\phi[s']\right)\,\,\rightarrow\,\,\left(\forall s':\,\KDoZ(\pi,s,s')\rightarrow\phi[s']\right)\]


This implication is a trivial consequence of lemma \ref{lem:KDoZ_E1_impl_KDoZ},
so the theorem holds. 
\end{proof}
\medskip{}


\begin{thmext}
[\ref{thm:Reg_PKnows}] Given a basic action theory $\Dt$
and a uniform formula $\phi$:\[
\Dt\cup\Dt_{K}^{obs}\,\models\,\PKnows(\pi,\phi,s)\equiv\Reg(\PKnows(\pi,\phi,s))\]

\end{thmext}
\begin{proof}
By induction on situation terms and the natural numbers. In the base
case of $S_{0}$ we have:\begin{gather*}
\Reg(\PKnows(\pi,\phi,S_{0}))\,\isdef\,\Pst(\PKnowsZ(\pi,\phi),Empty)[S_{0}]\end{gather*}


The key part of this proof is to demonstrate that for any $n$:\[
\Pst^{n}(\PKnowsZ(\pi,\phi),Empty)[S_{0}]\,\equiv\,\PKnowsZ(\pi,\phi,\mathcal{E}^{n}(S_{0}))\]


Begin with $n=1$, and we have:\begin{multline*}
\Pst^{1}(\PKnowsZ(\pi,\phi),Empty)[S_{0}]\,\equiv\\
\PKnowsZ(\pi,\phi,S_{0})\wedge\forall c:\, Empty(c)\rightarrow\PKnows(\pi,\phi,do(c,S_{0}))\end{multline*}


By the definition of $Empty$, we need only consider $c=\{\}$ and
this is equivalent to:\begin{multline*}
\PKnowsZ(\pi,\phi,S_{0})\wedge\PKnows(\pi,\phi,do(\{\},S_{0}))\\
\equiv\PKnowsZ(\pi,\phi,S_{0})\wedge\PKnowsZ(\pi,\phi,\mathcal{E}^{1}(S_{0}))\end{multline*}


By Theorem \ref{thm:En_impl_En-1} this is equivalent to $\PKnowsZ(\pi,\phi,\mathcal{E}^{1}(S_{0}))$.
A similar construction will produce the general case for $\Pst^{n}$
and $\mathcal{E}^{n}$. Since $\Pst(\PKnowsZ(\pi,\phi),Empty)$ is
equivalent to $\Pst^{n}(\PKnowsZ(\pi,\phi),Empty)$ for all $n$,
it is also equivalent to the definition of $\PKnows$ and the regression
rule is an equivalence as required.

In the $do(c,s)$ case, we can repeat the above reasoning to demonstrate
that the persistence condition accounts for all empty actions inserted
\emph{after} $c$. Pushing the application of $\mathcal{E}^{n}$ past
$c$, we obtain:\[
\PKnows(\pi,\phi,do(c,s))\,\equiv\,\bigwedge_{n\in\mathbb{N}}\Pst(\PKnowsZ(\pi,\phi),Empty)[do(c,\mathcal{E}^{n}(s))]\]


Applying the regression rule for $\PKnowsZ$ to handle $c$, we obtain:\[
\PKnows(\pi,\phi,do(c,s))\,\equiv\,\bigwedge_{n\in\mathbb{N}}\Reg(\Pst(\PKnowsZ(\pi,\phi),Empty),c)[\mathcal{E}^{n}(s)]\]


Finally, we need the following property of $\UnZ(\PKnowsZ(\pi,\phi,s))$,
which is a direct consequence of the definition of $\PKnows:$\[
\bigwedge_{n\in\mathbb{N}}\PKnowsZ(\pi,\phi,\mathcal{E}^{n}(s))\equiv\PKnows(\pi,\phi,s)\equiv\UnZ(\PKnowsZ(\pi,\phi,s))\]


From Theorem \ref{thm:En_impl_En-1} and the definition of $\Reg(\PKnowsZ(\pi,\phi,s))$,
the expression generated by $\Reg(\Pst(\PKnowsZ(\pi,\phi),Empty),c)$
will be in the form of a finite conjunction of $\PKnowsZ$ statements,
so we can apply $\UnZ$ to remove the infinite conjunction by capturing
it within $\PKnows$ from the inductive hypothesis:

\[
\PKnows(\pi,\phi,do(c,s)))\,\equiv\,\UnZ(\Reg(\Pst(\PKnowsZ(\pi,\phi),Empty),c)[s])\]


This is the theorem as required. 
\end{proof}
\medskip{}


\begin{thmext}
[{{[}{\ref{thm:Sync-PKnows-PKnowsZ}}]}] Let $\Dt_{sync}$ be a
synchronous basic action theory, then:\[
\Dt_{sync}\cup\Dt_{K}^{obs}\models\forall s:\,\PKnows(\pi,\phi,s)\,\equiv\,\PKnowsZ(\pi,\phi,s)\]

\end{thmext}
\begin{proof}
It suffices to show that:\[
\Dt_{sync}\cup\Dt_{K}^{obs}\models\PKnowsZ(\pi,\phi,s)\,\rightarrow\PKnowsZ(\pi,\phi,\mathcal{E}^{1}(s))\]


Then by Theorem \ref{thm:En_impl_En-1} we have that $\PKnowsZ(\pi,\phi,s)$
is enough to establish $\PKnowsZ(\pi,\phi,\mathcal{E}^{n}(s))$ for
any $n$, which establishes the infinite conjunction in the definition
of $\PKnows(\pi,\phi,s)$ as required. Regressing $\PKnowsZ(\pi,\phi,\mathcal{E}^{1}(s))$:\[
\Reg(\PKnowsZ(\pi,\phi,do(\{\},s)))\,\Rightarrow\,\forall c:\,\PKnowsZ(\Trn(\pi,\{\},c),\Reg(\phi,c),s)\]


$\Trn(\pi,\{\},c)$ will have the following form:\[
\Trn(\pi,\{\},c)\,\Rightarrow\,\,\,\,\,\, x\Leftarrow\{\}\,;\,\TrnA(\pi,x)\,;\,?x=c\]


For $x$ to take on a new value while traversing this path, it will
have to cross one of the regressed $agt$ steps in $\TrnA(\pi,x)$,
which have the following form:\[
z\,\Leftarrow Obs(agt,x)\,;\, agt\,;\,\exists x\,;\,?Legal(x)\vee x=\{\}\,;\,?Obs(agt,x)=z\]


Entering this path with $x$ set to $\{\}$ will bind $z$ to $\{\}$.
$x$ can then take on any new value that is legal and has $Obs(agt,x)=z=\{\}$.
But the domain is synchronous, so by definition there are no such
legal actions, and $x$ therefore remains set to $\{\}$ along the
entire path.

For any value of $c$ other than $\{\}$, there will be no situations
reachable by this regressed path and $\PKnowsZ$ will be vacuously
true. We can thus simplify away the quantification over $c$ to get:\[
\PKnowsZ(\Trn(\pi,\{\},\{\}),\Reg(\phi,\{\}),s)\]


Since $x$ is aways bound to $\{\}$, the tests in the regressed $agt$
steps in $\TrnA(\pi,x)$ are always satisfied. Likewise, the regressed
$?\phi$ steps in $\TrnA(\pi,x)$ always have the form $?\Reg(\phi,x)$.
Since $\Reg(\phi,\{\})$ is always equivalent to $\phi$, we conclude
that in synchronous domains the path $\Trn(\pi,\{\},\{\})$ is precisely
equivalent the path $\pi$. This gives us the equivalence between
$\PKnows$ and $\PKnowsZ$ as required. 
\end{proof}
\medskip{}


\begin{lemmaext}
[\ref{lem:Pknows_LbU_S0}] For any $agt$ and $\phi$:\[
\Dt\cup\Dt_{K}^{obs}\,\models\,\PKnows(agt,\phi,S_{0})\,\equiv\,\PKnowsZ(agt,\Pst(\phi,LbU(agt)),S_{0})\]

\end{lemmaext}
\begin{proof}
Recall the regression rule for $\PKnows$ at $S_{0}$:\[
\PKnows(agt,\phi,S_{0})\,\equiv\,\Pst(\PKnowsZ(agt,\phi),Empty)[S_{0}]\]
 Begin by considering the sequence of calculations required to calculate
the regression of $\Pst^{1}(\PKnowsZ(agt,\phi))$. First, we perform
some simplification on $\Trn(agt,\{\},c')$:\begin{align*}
\Trn(agt,\{\},c')=\,\,\, & x\Leftarrow\{\}\,;\, z\Leftarrow Obs(agt,x)\,;\, agt\,;\\
 & \,\,\,\,\,\,\exists x\,;\,?Legal(x)\vee x=\{\}\,;\,\,?Obs(agt,x)=z\,;\,?x=c'\\
=\,\,\, & z\Leftarrow Obs(agt,\{\})\,;\, agt\,;\,?Legal(c')\vee c'=\{\}\,;\,?Obs(agt,c')=z\\
=\,\,\, & agt\,;\,?Obs(agt,c')=\{\}\wedge\left(Legal(c')\vee c'=\{\}\right)\\
=\,\,\, & agt\,;\,?\left(\LbU(agt,c')\vee c'=\{\}\right)\end{align*}
 Now we can use this in the calculation of $\Pst^{1}(\PKnowsZ(agt,\phi))$,
recalling the simplifications used in Theorem \ref{thm:Reg_PKnows}:

\begin{multline*}
\Pst^{1}(\PKnowsZ(agt,\phi))[S_{0}]\equiv\,\,\Reg(\PKnows_{0}(agt,\phi,do(\{\},S_{0})))\\
\shoveleft{\equiv\,\,\forall c':\,\PKnowsZ(\Trn(agt,\{\},c'),\Reg(\phi,c'),S_{0})}\\
\shoveleft{\equiv\,\,\forall c':\,\PKnowsZ(agt\,;\,?\left(\LbU(agt,c')\vee c'=\{\}\right),\Reg(\phi,c'),S_{0})}\\
\shoveleft{\equiv\,\,\PKnowsZ(agt,\forall c':\,\left(\LbU(agt,c')\vee c'=\{\}\right)\rightarrow\Reg(\phi,c'),S_{0})}\\
\shoveleft{\equiv\,\,\PKnowsZ(agt,\phi\,\wedge\,\forall c':\, LbU(agt,c')\rightarrow\Reg(\phi,c'),S_{0})}\\
\shoveleft{\equiv\,\,\PKnowsZ(agt,\Pst^{1}(\phi,\LbU(agt)),S_{0})}\,\,\,\,\,\,\,\,\,\,\,\,\,\,\,\,\,\,\,\,\,\,\,\,\,\,\,\,\,\,\,\,\,\,\,\,\,\,\,\,\,\,\,\,\,\,\,\,\,\,\,\,\,\,\,\,\,\,\,\,\,\,\,\,\,\,\,\,\,\,\,\,\,\,\,\,\,\,\,\,\,\,\,\,\,\,\,\,\,\,\,\,\,\,\,\,\,\,\,\,\,\,\,\,\,\,\,\,\,\,\,\,\,\end{multline*}


Using the same construction, we can show that in general:\begin{align*}
\Pst^{n}(\PKnowsZ(agt,\phi))[S_{0}]\equiv\,\, & \PKnowsZ(agt,\Pst^{1}(\Pst^{n-1}(\phi,\LbU(agt)),\LbU(agt)))\\
\equiv\,\, & \PKnowsZ(agt,\Pst^{n}(\phi,\LbU(agt)))[S_{0}]\end{align*}


Clearly the fixpoint calculation in the regression of $\PKnows$ at
$S_{0}$ is the same as the fixpoint calculation used to find $\Pst(\phi,\LbU(agt))$.
Therefore, we have the required:\[
\PKnows(agt,\phi,S_{0})\,\equiv\,\PKnowsZ(agt,\Pst(\phi,\LbU(agt)),S_{0})\]

\end{proof}
\medskip{}


\begin{lemmaext}
[\ref{lem:Pknows_LbU_do}] For any $agt$, $\phi$, $c$ and
$s$:\begin{multline*}
\PKnows(agt,\phi,do(c,s))\,\equiv\,\exists z:\, Obs(agt,c,s)=z\\
\wedge\,\left[z=\{\}\,\rightarrow\,\PKnows(agt,\Pst(\phi,\LbU(agt)),s)\right]\\
\left[z\neq\{\}\,\rightarrow\,\PKnows(agt,\forall c':\,\left(Legal(c')\wedge Obs(agt,c')=z\right)\right.\\
\left.\rightarrow\Reg(\Pst(\phi,\LbU(agt)),c),s)\right]\end{multline*}

\end{lemmaext}
\begin{proof}
Repeating the calculations from Lemma \ref{lem:Pknows_LbU_S0} on
$\Pst(\PKnowsZ(agt,\phi))[do(c,s)]$, and pushing the application
of $\mathcal{E}^{n}$ past the actions $c$, we obtain the following:\[
\PKnows(agt,\phi,do(c,s))\,\equiv\,\bigwedge_{n\in\mathbb{N}}\PKnowsZ(agt,\Pst(\phi,\LbU(agt)),do(c,\mathcal{E}^{n}(s)))\]


Regressing the RHS over the actions $c$, we obtain:\begin{multline*}
\PKnows(agt,\phi,do(c,s))\,\\
\shoveleft{\equiv\,\forall c':\,\bigwedge_{n\in\mathbb{N}}\PKnowsZ(\Trn(agt,c,c'),\Reg(\Pst(\phi,\LbU(agt)),c'),\mathcal{E}^{n}(s))}\\
\shoveleft{\equiv\,\forall c':\,\PKnows(\Trn(agt,c,c'),\Reg(\Pst(\phi,\LbU(agt)),c'),s)}\,\,\,\,\,\,\,\,\,\,\,\,\,\,\,\,\,\,\,\,\,\,\,\,\,\,\,\,\,\,\,\,\,\,\,\,\,\,\,\,\,\,\,\,\,\,\,\,\,\,\,\,\,\,\,\,\end{multline*}


Now, let us expand and re-arrange $\Trn(agt,c,c')$:\begin{align*}
\Trn(agt,c,c')=\,\,\, & x\Leftarrow c\,;\, z\Leftarrow Obs(agt,x)\,;\, agt\,;\\
 & \,\,\,\,\,\,\exists x\,;\,?Legal(x)\vee x=\{\}\,;\,\,?Obs(agt,x)=z\,;\,?x=c'\\
=\,\,\, & z\Leftarrow Obs(agt,c)\,;\, agt\,;\,?Legal(c')\vee c'=\{\}\,;\,?Obs(agt,c')=z\\
=\,\,\, & z\Leftarrow Obs(agt,c)\,;\\
 & \,\,\,\,\,\,\left(z=\{\}\,;\, agt\,;\,?Legal(c')\vee c'=\{\}\,;\,?Obs(agt,c')=\{\}\right)\\
 & \,\,\,\cup\left(z\neq\{\}\,;\, agt\,;\,?Legal(c')\vee c'=\{\}\,;\,?Obs(agt,c')=z\right)\\
=\,\,\, & z\Leftarrow Obs(agt,c)\,;\,\left(z=\{\}\,;\, agt\,;\,?c'=\{\}\right)\\
 & \,\,\,\cup\left(z=\{\}\,;\, agt\,;\,?LbU(agt,c')\right)\\
 & \,\,\,\cup\left(z\neq\{\}\,;\, agt\,;\,?Legal(c')\wedge Obs(agt,c')=z\right)\end{align*}


Substituting this back into the RHS, we can bring the leading tests
outside the macro and split the $\cup$ into a conjunction to give:\begin{multline*}
\PKnows(agt,\phi,do(c,s))\,\equiv\,\forall c':\,\exists z:\, Obs(agt,c,s)=z\\
\wedge\,\PKnows(\left(?z=\{\};\, agt\right),\Reg(\Pst(\phi,\LbU(agt)),\{\}),s)\\
\wedge\,\PKnows(\left(?z=\{\};agt;?LbU(agt,c')\right),\Reg(\Pst(\phi,\LbU(agt)),c'),s)\\
\wedge\,\PKnows(\left(?z\neq\{\};agt;?Legal(c')\wedge Obs(agt,c')=z\right),\Reg(\Pst(\phi,\LbU(agt)),c'),s)\end{multline*}


Extracting the remaining tests from these paths, removing regression
over the empty action, and pushing the quantification over $c'$ into
its narrowest scope:\begin{multline*}
\PKnows(agt,\phi,do(c,s))\,\equiv\,\exists z:\, Obs(agt,c,s)=z\\
\wedge\,\left[z=\{\}\rightarrow\PKnows(agt,\Pst(\phi,\LbU(agt)),s)\right]\\
\wedge\,\left[z=\{\}\rightarrow\PKnows(agt,\forall c':\, LbU(agt,c')\,\rightarrow\,\Reg(\Pst(\phi,\LbU(agt)),c'),s)\right]\\
\wedge\,\left[z\neq\{\}\rightarrow\PKnows(agt,\forall c':\left(Legal(c')\wedge Obs(agt,c')=z\right)\right.\\
\left.\rightarrow\Reg(\Pst(\phi,\LbU(agt)),c'),s)\right]\end{multline*}


To complete the proof, we need the following property of the persistence
condition, which follows directly from its definition:\[
\Dt\,\models\,\left(\forall c:\,\alpha[c,s]\rightarrow\Reg(\Pst(\phi,\alpha),c)[s]\right)\,\equiv\,\Pst(\phi,\alpha)[s]\]


Using this we see that the two $z=\{\}$ clauses are equivalent, and
we can drop the more complicated one to get the theorem as required. 
\end{proof}


\begin{lemmaext}
[\ref{lem:Knows_impl_KnowsLbU}] For any $agt$, $\phi$ and
$s$:\[
\Dt\cup\Dt_{K}^{obs}\,\models\,\Knows(agt,\phi,s)\,\equiv\,\Knows(agt,\Pst(\phi,LbU(agt)),s)\]

\end{lemmaext}
\begin{proof}
By induction on situations, using the regression rules for knowledge,
and the following straightforward properties of the persistence condition:\[
\forall s':\,\Pst(\phi,\alpha)[s]\,\wedge\, s\leq_{\alpha}s'\,\rightarrow\,\Pst(\phi,\alpha)[s']\]
 \[
\Pst(\Pst(\phi,\alpha),\alpha)[s]\,\equiv\,\Pst(\phi,\alpha)[s]\]


For $s=S_{0}$ the regression rule in equation \eqref{eqn:R_s0} gives
us the following:\begin{align*}
\Knows(agt,\phi,S_{0})\equiv & \KnowsZ(agt,\Pst(\phi,LbU(agt)),S_{0})\\
\equiv & \forall s':\, K_{0}(agt,s',S_{0})\,\rightarrow\,\Pst(\phi,LbU(agt))[s']\end{align*}


Which, by the above properties of $\Pst$, yields:\[
\Knows(agt,\phi,S_{0})\equiv\forall s',s'':\, K_{0}(agt,s',S_{0})\wedge s'\leq_{LbU(agt)}s''\,\rightarrow\,\Pst(\phi,LbU(agt))[s'']\]


This matches the form of the $\Knows$ macro, and can be restructured
to give the required:\[
\Knows(agt,\phi,S_{0})\,\equiv\,\Knows(agt,\Pst(\phi,LbU(agt)),S_{0})\]


For the $do(c,s)$ the inductive hypothesis gives us:\[
\Knows(agt,\phi,s)\equiv\Knows(agt,\Pst(\phi,LbU(agt)),s)\]


We have two sub-cases to consider. If $Obs(agt,c,s)=\{\}$ then the
regression rule in equation \eqref{eqn:R_do_c_s} gives us:\[
\Knows(agt,\phi,do(c,s))\equiv\Knows(agt,\phi,s)\]
 \[
\Knows(agt,\Pst(\phi,LbU(agt)),do(c,s))\equiv\Knows(agt,\Pst(\phi,LbU(agt)),s)\]


These can be directly equated using the inductive hypothesis, so the
theorem holds in this case. Alternately, if $Obs(agt,c,s)\neq\{\}$
then the regression rule gives:\begin{multline*}
\Knows(agt,\phi,do(c,s))\equiv\exists o:\, Obs(agt,c,s)=o\,\wedge\\
\Knows(agt,\forall c':\, Legal(c')\wedge Obs(agt,c')=o\,\rightarrow\,\Reg(\Pst(\phi,LbU(agt)),c'),s)\end{multline*}
 \begin{multline*}
\Knows(agt,\Pst(\phi,LbU(agt)),do(c,s))\equiv\exists o:\, Obs(agt,c,s)=o\,\wedge\\
\Knows(agt,\forall c':\, Legal(c')\wedge Obs(agt,c')=o\,\\
\rightarrow\,\Reg(\Pst(\Pst(\phi,LbU(agt)),LbU(agt)),c'),s)\end{multline*}


Simplifying the second equation using the properties of $\Pst$ gives:\begin{multline*}
\Knows(agt,\Pst(\phi,LbU(agt)),do(c,s))\equiv\exists o:\, Obs(agt,c,s)=o\,\wedge\\
\Knows(agt,\forall c':\, Legal(c')\wedge Obs(agt,c')=o\,\rightarrow\,\Reg(\Pst(\phi,LbU(agt)),c'),s)\end{multline*}


Which matches the equivalence for $\Knows(agt,\phi,do(c,s))$, as required.
\end{proof}

