\documentclass[times, 10pt, twocolumn]{article}
\pagestyle{empty}
\setcounter{secnumdepth}{3}
\setcounter{tocdepth}{3}


\makeatletter

\providecommand{\tabularnewline}{\\}

\usepackage{latex8}
\usepackage{times}
\usepackage{verbatim}
\usepackage{calc}
\usepackage{subfigure}
\usepackage{amsmath}
\usepackage{amsthm}
\newtheorem{lemma}{Lemma}
\newtheorem{thm}{Theorem}

\makeatother

\begin{document}

\title{Towards High-Level Programming for Distributed Problem Solving\thanks{This research was supported by a grant from the Australian Research Councils (ARCs) Centre for Perceptive and Intelligent Machines in Complex Environments (PIMCE).}}

\author{Ryan F. Kelly and Adrian R. Pearce\\
 NICTA Victoria Research Laboratory\\
 Department of Computer Science and Software Engineering\\
 The University of Melbourne\\
 Victoria, 3010, Australia\\
 \{rfk,adrian\}@csse.unimelb.edu.au}

\maketitle
\thispagestyle{empty}

\begin{abstract}
We present a new approach to distributed problem solving based on
high-level program execution. While this technique has proven itself
for single-agent systems based on the Golog language, several challenges
are encountered when moving to a multi-agent setting. Key to our approach
is a better representation of the dynamics of multi-agent teams by
means of the following features: a robust combination of true concurrency
of actions with the interleaved concurrency of ConGolog; an explicit
notion of time to assist coordination; and semantic support for predictable
exogenous actions (also called {}``natural actions''). The result
is MIndiGolog, a new Golog variant suitable for distributed, cooperative
execution by a multi-agent team. 
\end{abstract}

\Section{Introduction}

In distributed problem solving applications, a team of agents must
cooperate closely to achieve a shared goal. Such teams can often be
conceptualized as a single agent with distributed sensing, reasoning
and acting capabilities, which leads to a two-stage technique for
programming such teams: adapt a formalism from single-agent programming
to specify the tasks to be performed, then provide a coordination
strategy allowing the team to execute the specification cooperatively.
This approach has been successfully employed by platforms such as
STEAM \cite{tambe97flexible_teamwork}, SharedPlans \cite{grosz99planning_together},
and TAEMS \cite{decker95designing_coordination}, which specify tasks
with variants of the Hierarchical Task Networks (HTN) formalism.

While HTN is a popular and powerful approach to task specification,
an increasingly popular alternative is \emph{high-level program execution}
as embodied by the Golog programming language \cite{levesque97golog}.
By {}``high-level program'' is meant a program whose primitive components
are domain-specific actions, connected by standard programming constructs,
and that may contain nondeterministic operators. The primary advantage
of this approach is \emph{controlled nondeterminism}, allowing some
program parts to be fully specified while others may involve arbitrary
amounts of nondeterminism, or even goal-based planning. Compared to
standard HTN, Golog provides a more natural
representation of many tasks
and is based on a more sophisticated logic of action \cite{son00htn_golog}.
Extensions to Golog have introduced further advantageous
features such as concurrent execution by interleaving of actions (ConGolog \cite{giacomo00congolog})
and a combination of
online execution with offline planning (IndiGolog \cite{giacomo99indigolog}).

Motivated by these advantages, our research program aims to build
a distributed problem solving system based on cooperative execution
of shared high-level programs. This paper represents a first step
towards that goal. We integrate several existing extensions to the
situation calculus into the Golog language to better represent the
dynamics of a multi-agent team. Key among these is true concurrency
of actions, which we combine with the interleaved concurrency of ConGolog
to give a flexible account of concurrent execution. An explicit notion
of time is incorporated to enrich the world model and to assist in
coordination between agents. The concept of natural actions is also
tightly integrated into the language, to allow agents to predict the
behavior of their teammates and environment. We name the resulting
language {}``MIndiGolog'' for {}``Multi-Agent IndiGolog''.

The paper proceeds as follows: section \ref{sec:Background} gives
some background on the situation calculus and Golog;
section \ref{sec:MIndiGolog} develops the semantics
of MIndiGolog
and shows how distributed logic programming techniques
can facilitate the shared execution of MIndiGolog programs; section
\ref{sec:Related-Work} discusses related work and section \ref{sec:Conclusions}
concludes with a summary of our results and ongoing research. The
work is motivated and illustrated throughout by examples from a simple
multi-agent domain, in which a team of robotic chefs must be programmed
to cooperatively prepare a meal.

\Section{Background\label{sec:Background}}

\SubSection{The Situation Calculus}

The situation calculus is a first-order logic formalism for modeling
dynamic worlds, with the following key features: \emph{Actions} are
functions denoting individual instantaneous events that can cause
the state of the world to change; \emph{Situations} are histories
of the actions that have occurred, with $S_{0}$ being the initial
situation and successive situations built up using the function $do(a,s)$;
\emph{Fluents} are predicates or functions representing properties
of the world that may change from one situation to another. The special
fluent $Poss(a,s)$ indicates when it is possible to perform an action
in a situation. For a detailed description consult \cite{pirri99contributions_sitcalc}.

A collection of situation calculus statements $\mathcal{D}$ describing
a dynamic world is referred to as a \emph{theory of action}, and queries
about the behavior or evolution of the world are posed as logical
entailment queries relative to this theory.

\textbf{Concurrency.} In the basic situation calculus only
a single action can occur at any instant. While suitable for most
single-agent domains, this limitation is emphatically not suitable
for multi-agent systems - several actions can easily occur simultaneously
if performed by different agents. Modeling this \emph{true concurrency}
is necessary to avoid problems with conflicting or incompatible actions.
There is also the potential to utilize concurrency to execute tasks
more efficiently. Clearly a solid account of concurrency is required
for programming multi-agent teams.

The work of \cite{lin92sc_conc,reiter96sc_nat_conc} adds true concurrency
to the situation calculus by replacing action terms with sets of actions
that are performed simultaneously. All functions and predicates that
take an action are modified to take sets of actions instead. For example,
$do(a,s)$ becomes $do(\{ a_{1},a_{2},...\},s)$.

This seemingly simple modification introduces a complication - a combination
of actions is not guaranteed to be possible even if each of the individual
actions are. For example, two agents may not be able to acquire the
same resource at the same time. This is known as the precondition
interaction problem \cite{pinto94temporal} and is an area of ongoing
research. For our purposes it is addressed by introducing a predicate
$Conflicts(c,s)$ which is true when the actions in $c$ are in conflict
and cannot be performed together.

\textbf{Time.} An explicit notion of time can make coordination
between agents easier, as joint actions may be performed at a particular
time. It also allows a richer description of the world, for example
when stating the baking time of a cake.

The standard approach to time in the situation calculus is that of
\cite{reiter96sc_nat_conc,pinto94temporal}. Each action gains an
extra argument indicating the time at which is was performed. Time
itself can be represented by any appropriately-behaved sequence, such
as integers or reals, whose axiomatisation must be included in the
theory of action. The functions $time$ and $start$ are introduced
to give the performance time of an action and the start time of a
situation respectively. The start time of the initial situation may
be defined arbitrarily.

An additional predicate $Coherent$ is defined to ensure that the
performance time is the same for all members in a set of concurrent
actions. The $Poss$ fluent for concurrent actions can then be defined
to ensure that the temporal relation between past and future situations
is respected, as well as accounting for conflicting actions%
\footnote{As usual, lower-case terms are variables and free variables are implicitly
universally quantified%
}:\begin{multline*}
Poss(c,s)\equiv\forall a.\left[a\in c\rightarrow Poss(a,s)\right]\\
\wedge\,\neg Conflicts(c,s)\,\wedge\, time(c)>start(s)\,\wedge\, Coherent(c)\end{multline*}
 This representation is accompanied by a standard approach to actions
with a finite duration: they are decomposed into instantaneous $start$
and $end$ actions and a fluent indicating that the action is in progress.
For example, a long-running task may be represented by the actions
$beginTask$ and $endTask$ along with a fluent $doingTask$.

\textbf{Natural actions.} These are a special class of exogenous
actions, those actions which occur outside of an agent's control \cite{reiter96sc_nat_conc}.
They are classified according to the following requirement: natural
actions must occur at their predicted times, provided no earlier actions
prevent them from occurring. For example, a timer will ring at the
time it was set for, unless it is switched off. The action $endTask$
from above is another example - it must occur whenever it is possible,
which is at the time when the agent finishes the task. In domains
where many agents may be simultaneously engaged in many long-running
tasks, strong semantic support for natural actions will therefore
be of significant benefit.

Natural actions are indicated by the truth of the predicate $Natural(a)$.
The times at which natural actions may occur are specified by the
$Poss$ predicate as usual. For example, suppose that the fluent $TimerSet(m,s)$
represents the fact that a timer is set to ring in $m$ minutes in
situation $s$. The possibility predicate for the $ringTimer(t)$
action would be:\begin{multline*}
Poss(ringTimer(t),s)\equiv\\
\exists m.\left[TimerSet(m,s)\wedge t=start(s)+m\right]\end{multline*}
 The timer may thus ring only at its predicted time. To enforce the
requirement that natural actions \emph{must} occur whenever possible,
a predicate $Legal(s)$ is introduced which is true only for situations
that respect this requirement. Legal situations are the only situations
that could be brought about in the real world:\begin{multline*}
Legal(S_{0})\equiv True\\
\shoveleft{Legal(do(c,s))\equiv Legal(s)\wedge Poss(c,s)}\\
\wedge\,\forall a.\left[Natural(a)\wedge Poss(a,s)\rightarrow\left[a\in c\vee t<time(a)\right]\right]\end{multline*}

An important concept when dealing with natural actions is the least
natural time point (LNTP) of a situation, defined as the earliest
time at which a natural action may occur. We assume that the theory
of action avoids certain pathological cases, so that absence of an
LNTP implies that no natural actions are possible.\begin{multline*}
Lntp(s,t)\equiv\\
\exists a.\left[Natural(a)\wedge Poss(a,s)\wedge time(a)=t\right]\wedge\\
\forall a.\left[Natural(a)\wedge Poss(a,s)\rightarrow t\leq time(a)\right]\end{multline*}

\SubSection{Golog}

Golog \cite{levesque97golog} is a declarative agent programming language
based on the situation calculus. Testimony to its success are its
wide range of applications and many extensions to provide additional
functionality (\cite{giacomo00congolog,giacomo99indigolog,Ferrein2005readylog}).
We use {}``Golog'' to refer to the family of languages based on
this technique, including ConGolog \cite{giacomo00congolog} and IndiGolog
\cite{giacomo99indigolog}.

To program an agent using Golog one specifies a situation calculus
theory of action, and a program consisting of actions from the theory
connected by programming constructs such as if-then-else, while loops,
and nondeterministic choice. Table \ref{tbl:Golog-Operators} lists
some of the operators available in various incarnations of the language.%
\begin{table}
\caption{Some Golog operators\label{tbl:Golog-Operators}}
\begin{centering}\begin{tabular}{|c|c|}
\hline 
Operator&
Meaning\tabularnewline
\hline
\hline 
$a$&
Execute action $a$ in the world\tabularnewline
\hline 
$\phi?$&
Proceed if condition $\phi$ is true\tabularnewline
\hline 
$\delta_{1};\delta_{2}$&
Execute $\delta_{1}$followed by $\delta_{2}$\tabularnewline
\hline 
$\delta_{1}|\delta_{2}$&
Execute either $\delta_{1}$ or $\delta_{2}$\tabularnewline
\hline 
$\pi(x)\delta(x)$&
Nondet. select arguments for $\delta$\tabularnewline
\hline 
$\delta*$&
Execute $\delta$ zero or more times\tabularnewline
\hline 
$\mathbf{if}\,\phi\,\mathbf{then}\,\delta_{1}\,\mathbf{else}\,\delta_{2}$&
Exec. $\delta_{1}$ if $\phi$ holds, $\delta_{2}$ otherwise\tabularnewline
\hline 
$\mathbf{while\,}\phi\mathbf{\, do}\,\delta$&
Execute $\delta$ while $\phi$ holds\tabularnewline
\hline 
$\mathbf{proc}P(\overrightarrow{x})\delta(\overrightarrow{x})\mathbf{end}$&
Procedure definition\tabularnewline
\hline 
$\delta_{1}||\delta_{2}$&
Concurrent execution (ConGolog)\tabularnewline
\hline 
$\Sigma\delta$&
Plan execution offline (IndiGolog)\tabularnewline
\hline
\end{tabular}\par\end{centering}
\end{table}

In line with the idea of high-level program execution, the agent's
control program may be nondeterministic. It is the task of the agent
to plan a deterministic instantiation of the program, a sequence of
actions that can legally be performed in the world. Such a sequence
is called a \emph{legal execution} of the Golog program.

Two predicates $Trans$ and $Final$ define the semantics for each
operator. $Trans(\delta,s,\delta',s')$ holds when executing a step
of program $\delta$ can cause the world to move from situation $s$
to situation $s'$, after which $\delta'$ remains to be executed.
It thus characterizes single steps of computation. $Final(\delta,s)$
holds when program $\delta$ may legally terminate its execution in
situation $s$. We base our work on the semantics of IndiGolog, which
builds on ConGolog and is the most feature-full of the standard Golog
variants. The full semantics are available in the references, but
as an example consider equation (\ref{eqn:trans_conc_orig}), which
specifies the concurrent-execution operator as an \emph{interleaving}
of computation steps. It states that it is possible to single-step
the concurrent execution of $\delta_{1}$ and $\delta_{2}$ by performing
either a step from $\delta_{1}$ or a step from $\delta_{2}$, with
the remainder $\gamma$ left to execute concurrently with the other
program:
\begin{multline}
Trans(\delta_{1}||\delta_{2},s,\delta',s')\equiv\\
\shoveright{\exists\gamma.Trans(\delta_{1},s,\gamma,s')\wedge\delta'=(\gamma||\delta_{2})}\\
\vee\,\exists\gamma.Trans(\delta_{2},s,\gamma,s')\wedge\delta'=(\delta_{1}||\gamma)\label{eqn:trans_conc_orig}
\end{multline}

Clearly there are two notions of concurrency to be considered: the
possibility of performing several actions at the same instant (\emph{true
concurrency}), and the possibility of interleaving the execution of
several programs (\emph{interleaved concurrency}). These were combined
in \cite{pinto99tcongolog} by modifying Golog to incorporate sets
of concurrent actions. However, they give a semantics which may call
for actions to be performed that are not possible and which can result
in unintuitive program behavior. A key aspect of our work is a more
robust integration of these two notions of concurrency.

If the theory of action $\mathcal{D}$ is enriched with $Trans$ and
$Final$, planning an execution of a Golog program $\delta$ is basically
a theorem proving task as shown in equation (\ref{eqn:golog_execution}).
Here $Trans*$ indicates reflexive transitive closure. The situation
$s$ gives a sequence of actions forming a legal execution of the
program.
\begin{equation}
\mathcal{D}\models\exists s.\left[Trans*(\delta,S_{0},\delta',s)\wedge Final(\delta',s)\right]\label{eqn:golog_execution}
\end{equation}

In IndiGolog agents can also proceed without planning a full terminating
execution of their program, by searching for a legal {}``next step''
action $a$ such that $\mathcal{D}\models\exists a\,.\, Trans*(\delta,s,\delta',do(a,s))$.
The search operator ($\Sigma$) controls which parts of the program
are subject to full execution planning, providing fine-grained control
over nondeterminism and the amount of planning work required.

As an example of a multi-agent task specification in Golog, consider
a program $MakeSalad$ that instructs a team of agents to prepare
a simple salad:
\begin{multline}
\mathbf{proc}\, MakeSalad(dest)\label{eqn:MakeSalad}\\
\left[\pi(agt)(ChopTypeInto(agt,Lettuce,dest))\,||\right.\\
\pi(agt)(ChopTypeInto(agt,Carrot,dest))\,||\\
\left.\pi(agt)(ChopTypeInto(agt,Tomato,dest))\right]\,;\\
\pi(agt)\left[acquire(agt,dest)\,;\,\right.\\
beginTask(agt,mix(dest,1))\,;\\
\left.\, release(agt,dest)\right]\,\,\mathbf{end}\end{multline}

The sub-procedure $ChopTypeInto$ (not shown) picks an object of
the given type and an available chopping board, chops the object using
the board, then transfers it into the destination container. $MakeSalad$
tells the agents to do this for a lettuce, a carrot and a tomato,
then mix the ingredients together for 1 minute. Note the nondeterminism
in this program - the agent assigned to handling each ingredient is
not specified ($\pi$ construct), nor is the order in which they should
be added ($||$ construct). There is thus considerable scope for cooperation
between agents to effectively carry out this task.

While this is a valid program in standard IndiGolog, executing it
using the existing semantics would be far from ideal. The explicit
temporal component described above must to added to IndiGolog to accommodate
the $mix(dest,1)$ task. The lack of true concurrency would mean only
one agent could act at a time, while others would remain idle. And
since there is no support for natural actions, IndiGolog would fail
to find a legal execution of this program: it would find that the
final action $release$ cannot be performed after doing $beginTask$,
as our theory of action ensures agents can only be doing one thing
at a time. But it would not determine that the natural action $endTask$
will occur after one minute and enable to program to finish.

Our new Golog variant, MIndiGolog, is designed to produce executions
of such programs in a manner that overcome these limitations, and
is thus suitable for specifying tasks to be performed by multi-agent
teams in distributed problem solving applications.

\Section{MIndiGolog\label{sec:MIndiGolog}}

We have integrated three extensions to the situation calculus with
the semantics of IndiGolog to better model the dynamics of a multi-agent
setting. These extensions allow agents to represent time, concurrently-occurring
actions, and natural actions in a robust way.

\SubSection{Time}

It is clear from the background section that the approach of \cite{reiter96sc_nat_conc}
to modeling time is complicated by the presence of concurrent actions.
To avoid the need for the $Coherent$ predicate, we attach the temporal
argument to each situation rather than to each action. The successor
situation function $do(a,s)$ becomes $do(a,t,s)$, to indicate {}``action
$a$ was performed at time $t$ in situation $s$''. The possibility
predicate $Poss(a,s)$ likewise becomes $Poss(a,t,s)$. The semantics
of IndiGolog trivially accommodate this change, and
$Coherent$ and $time$ are no longer needed.

\SubSection{Concurrency}

While it is straightforward to modify the IndiGolog $Trans$ rule
for primitive actions to accept sets of concurrent actions, there
are deeper implications for the concurrency operator. This is implemented
by accepting a transition from either of the two programs as a transition
for the pair \cite{giacomo00congolog}. In the presence of true concurrency,
this is insufficient. Suppose program $\delta_{1}$ may be transitioned
by performing actions $c_{1}$, and $\delta_{2}$ may be transitioned
by performing actions $c_{2}$. As noted in \cite{pinto99tcongolog},
it should be possible to exploit true concurrency by performing both
simultaneously, i.e. $c_{1}\cup c_{2}$. However, this introduces
several complications that \cite{pinto99tcongolog} does not address.

First, precondition interaction means that $c_{1}\cup c_{2}$ may
not be possible even if the individual actions are. The transition
clause must ensure that the combination of the two sets of actions
is possible. Another issue arises when two programs can legitimately
be transitioned by executing the same action. Consider the following
programs which add ingredients to a bowl:
\begin{multline*}
\delta_{1}=placeIn(Thomas,Flour,Bowl)\,;\\
\shoveright{placeIn(Thomas,Sugar,Bowl)}\\
\shoveleft{\delta_{2}=placeIn(Thomas,Flour,Bowl)\,;}\\
placeIn(Thomas,Egg,Bowl)
\end{multline*}

Executing $\delta_{1}||\delta_{2}$ should result in the bowl containing
two units of flour, one unit of sugar and an egg. However, an individual
transition for both programs is $c_{1}=c_{2}=placeIn(Thomas,Flour,Bowl)$.
Naively executing $c_{1}\cup c_{2}$ to transition both programs would
add only one unit of flour. Alternately, consider two programs waiting
for a timer to ring:
\begin{gather*}
\delta_{1}=ringTimer\,;\, acquire(Thomas,Bowl)\\
\delta_{2}=ringTimer\,;\, acquire(Richard,Board)
\end{gather*}

Both programs should be allowed to proceed with a single occurrence
of the $ringTimer$ action, because it is an aspect of the environment.
To avoid unintuitive (and potentially dangerous) behavior, concurrent
execution must not be allowed to transition both programs using the
same \emph{agent-initiated} action. If an agent-initiated action may
be safely skipped, it can be enclosed in an appropriate if-then-else
or choice construct.

Taking these factors into account, we develop the improved transition
rule for concurrency in equation (\ref{eqn:trans_conc_new}). The
first two lines are the original interleaved concurrency clause from
ConGolog, while the remainder characterizes the above considerations
for true concurrency. This robust combination allows the language
to more accurately reflect the concurrency present in multi-agent
teams. As with ConGolog and IndiGolog, our semantics make no attempt
to maximize concurrency or otherwise differentiate between potential
executions, they only state which transitions can be legally performed.
\begin{multline}
Trans(\delta_{1}||\delta_{2},s,\delta',s')\equiv\\
\shoveright{\exists\gamma\,.\, Trans(\delta_{1},s,\gamma,s')\wedge\delta'=(\gamma||\delta_{2})}\\
\shoveright{\vee\exists\gamma\,.\, Trans(\delta_{2},s,\gamma,s')\wedge\delta'=(\delta_{1}||\gamma)}\\
\shoveright{\vee\exists c_{1},c_{2},\gamma_{1},\gamma_{2},t\,.\, Trans(\delta_{1},s,\gamma_{1},do(c_{1},t,s))}\\
\wedge Trans(\delta_{2},s,\gamma_{2},do(c_{2},t,s))\wedge Poss(c_{1}\cup c_{2},t,s)\\
\shoveright{\wedge\forall a.\left[a\in c_{1}\wedge a\in c_{2}\rightarrow Natural(a)\right]}\\
\wedge\delta'=(\gamma_{1}||\gamma_{2})\wedge s'=do(c_{1}\cup c_{2},t,s)\label{eqn:trans_conc_new}
\end{multline}
\SubSection{Natural actions}

The formalism of \cite{reiter96sc_nat_conc} is adopted, with simple
modifications for our handling of time. While planning with natural
actions has previously been done in Golog \cite{pirri00planning_nat_acts},
the programmer was required to explicitly check for any possible natural
actions and ensure that they appear in the execution. We significantly
lower the burden on the programmer by guaranteeing that all legal
program executions result in legal situations. MIndiGolog agents will
plan for the occurrence of natural actions without having them explicitly
mentioned in the program. They may optionally be included in the program,
instructing the agents to wait for the action to occur before proceeding.

This is achieved using a new $Trans$ clause for the case of a single
action $c$, as shown in equation (\ref{eqn:trans_prim_new}). If
$s$ has an LNTP $t_{n}$ and corresponding set of natural actions
$c_{n}$, a transition can be made in three ways: perform $c$ at
a time before $t_{n}$ (fourth line), perform it along with the natural
actions at $t_{n}$ (fifth line), or wait for the natural actions
to occur (sixth line). If there is no LNTP, then $c$ may be performed
at any time greater than $start(s)$.
\begin{multline}
Trans(c,s,\delta',s')\equiv\\
\exists t,t_{n},c_{n}\,.\, Lntp(s,t_{n})\wedge t\geq start(s)\wedge\\
\forall a.\left[Natural(a)\wedge Poss(a,t_{n},s)\equiv a\in c_{n}\right]\wedge\\
\left[t<t_{n}\wedge Poss(c,t,s)\wedge s'=do(c,t,s)\wedge\delta'=Nil\right.\\
\vee Poss(c\cup c_{n},t_{n},s)\wedge s'=do(c\cup c_{n},t,s)\wedge\delta'=Nil\\
\vee\left.s'=do(c_{n},t_{n},s)\wedge\delta'=c\right]\\
\vee\,\neg\exists t_{n}\,.\, Lntp(s,t_{n})\wedge\exists t\,.\, Poss(c,t,s)\wedge\\
t\geq start(s)\wedge s'=do(c,t,s)\wedge\delta'=Nil\label{eqn:trans_prim_new}\end{multline}

The occurrence of natural actions may also cause test conditions within
the program to become satisfied, so a new $Trans$ clause for $\phi?$
is also required as shown in equation (\ref{eqn:trans_cond_new}).
This permits a program consisting of a single test condition to make
a transition if the condition is satisfied, or if a natural action
occurs: \footnote{$\phi[s]$ should be read as {}``$\phi$ holds in situation $s$''}
\begin{multline}
Trans(\phi?,s,\delta',s')\,\,\equiv\,\,\phi[s]\wedge\delta'=Nil\wedge s'=s\\
\shoveright{\vee\,\exists t_{n},c_{n}\,.\, Lntp(s,t_{n})\wedge\delta'=\phi?\wedge s'=do(c_{n},t_{n}s)}\\
\wedge\,\forall a.\left[Natural(a)\wedge Poss(a,t_{n},s)\rightarrow a\in c_{n}\right]\label{eqn:trans_cond_new}
\end{multline}

A MIndiGolog execution will thus contain all natural actions that
will occur, regardless of whether they were considered explicitly
by the programmer.

Contrast this with the standard handling of exogenous events in IndiGolog,
which is achieved by executing the main program concurrently with
a program that generates exogenous actions:\[
\delta_{main}\,||\,\left(\pi(a)(Exog(a)?\,;\, a)\right)^{*}\]
 This allows the program to make a legal transition regardless of
what exogenous actions occur - an approach suitable for dealing with
arbitrary exogenous actions which may occur at any time, but suboptimal
for handling \emph{predictable} exogenous actions. Our approach allows
the agents to directly predict the natural actions that will occur
and automatically include them in a planned execution.

\SubSection{Legality of the semantics}

Let a \emph{MIndiGolog Theory of Action} be a theory of action in
the situation calculus enhanced with time, true concurrency and natural
actions, augmented with the predicates $Trans$ and $Final$ from
IndiGolog, modified according to equations (\ref{eqn:trans_conc_new}),
(\ref{eqn:trans_prim_new}) and (\ref{eqn:trans_cond_new}). All legal
executions of a MIndiGolog program derived from such a theory of action
produce legal situations.
\begin{lemma} Let $\mathcal{D}$ be a MIndiGolog theory of action.
Then:\label{thm:trans_legal}
\begin{multline*}
\mathcal{D}\models\forall s,s',\delta,\delta'.Legal(s)\wedge Trans(\delta,s,\delta',s') \\ \rightarrow Legal(s')
\end{multline*}
\end{lemma} \begin{proof} By induction on the structure of $\delta$.
That the theorem holds for the modified $Trans$ clauses of equations
(\ref{eqn:trans_conc_new}), (\ref{eqn:trans_prim_new}) and (\ref{eqn:trans_cond_new})
is straightforward, and no other clause constructs new situation terms.
\end{proof} 

\begin{thm} Let $\mathcal{D}$ be a MIndiGolog theory of action.
Then: \[
\mathcal{D}\models\forall s',\delta,\delta'\,.\, Trans*(\delta,S_{0},\delta',s')\rightarrow Legal(s')\]
Thus, all legal executions of a MIndiGolog program produce
legal situations. \end{thm} \begin{proof} From lemma \ref{thm:trans_legal},
the legality of $S_{0}$, and the properties of transitive closure.
\end{proof} 

Like standard IndiGolog, MIndiGolog will only call for actions to
be performed if it is actually possible to perform them. MIndiGolog
has the added advantage of effortlessly incorporating any natural
actions that may occur.


\SubSection{Example execution}

The effect of our new semantics can be seen in figure \ref{cap:example_trace},
which shows one possible legal execution of the $MakeSalad$ program
using the new semantics of MIndiGolog%
\footnote{A Prolog implementation of the MIndiGolog semantics, from which this
trace was obtained, is available at http://www.csse.unimelb.edu.au/\textasciitilde{}rfk/golog/%
} in a domain with thee agents. Note the occurrence of several actions
within each \textbf{do} statement, demonstrating the integration of
true concurrency into the language. Note also the incorporation of
natural $endTask$ actions into the program, and the explicit occurrence
time for each action. (in this trace the occurrence times have been
instantiated to their earliest possible value).

%
\begin{figure*}[t]
\centering \subfigure[One possible execution of the $MakeSalad$ program with three agents,
using MIndiGolog. Multiple actions occur
at each step.]{ \label{cap:example_trace} \framebox{\makebox[0.97\textwidth]{%
\begin{minipage}[t][1\totalheight]{0.5\textwidth}%
{\small \verbatiminput{listings/output_makeSalad_1.txt}}%
\end{minipage}%
%
\begin{minipage}[t][1\totalheight]{0.45\textwidth}%
{\small \verbatiminput{listings/output_makeSalad_2.txt}}%
\end{minipage}%
}} } \subfigure[One possible execution of the $MakeSalad$ program with three
agents, using IndiGolog. Only one agent acts at
each step.]{ \label{cap:indigolog_trace} \framebox{\makebox[0.97\textwidth]{%
\begin{minipage}[t][1\totalheight]{0.5\textwidth}%
{\small \verbatiminput{listings/output_makeSalad_ig_1.txt}}%
\end{minipage}%
%
\begin{minipage}[t][1\totalheight]{0.45\textwidth}%
{\small \verbatiminput{listings/output_makeSalad_ig_2.txt}}%
\end{minipage}%
}} } 


\caption{Example executions of the $MakeSalad$ program}
\end{figure*}


For comparison purposes, we added a temporal element and inserted
explicit $endTask$ actions into $MakeSalad$ to allow a legal execution
to be found using the standard IndiGolog semantics. Such an execution
is shown in figure \ref{cap:indigolog_trace}. Note that the lack
of true concurrency means only a single agent can act at each step,
leaving the other two agents completely idle. The ability to reduce
idle time by performing actions concurrency is clearly an advantage
for distributed problem solving applications.


\SubSection{Distributed execution planning}

As the existing languages based on Golog have demonstrated, an interpreter
can be conveniently constructed using a logic programming language
such a Prolog. We have followed the style of \cite{giacomo00congolog,giacomo99indigolog}
to build interpreters for our language in both Prolog and Oz \cite{vanroy03mozart_logic},
a multi-paradigm programming language with strong support for distributed
computing.

One powerful feature of Oz is the ability to use several networked
computers to search for solutions to a logic program in parallel.
Since the task of planning a MIndiGolog execution is encoded as a
logic program, this immediately allows a team of agents to distribute
the execution planning workload. Below we briefly summarize our implementation;
the full version is available from the author's website.

MIndiGolog programs are represented in Oz as record terms (which are
similar to Prolog data terms) with the name of the record representing
an operator and its features the arguments. Actions are also encoded
as records. As in Prolog, uppercase terms in Oz represent variables.
For example, the program:\[
\pi(agt)\left[acquire(agt,Bowl);acquire(agt,Knife)\right]\]


is represented as follows:

{\small \verbatiminput{listings/goloz_ex_prog.oz} }{\small \par}

The predicates $Trans$ and $Final$ have a straightforward encoding
as Oz procedures, using the \textbf{case} statement to encode each
individual clause using pattern matching, and the \textbf{choice}
statement to explicitly introduce choice points:

{\small \verbatiminput{listings/goloz_trans.oz} }{\small \par}

{\small \verbatiminput{listings/goloz_final.oz} }{\small \par}

A procedure $Do(\delta,s,s')\equiv Trans*(\delta,s,\delta',s')\wedge Final(\delta',s')$
is defined that determines a legal execution \textbf{\emph{Sp}} for
a given program \textbf{\emph{D}}:

{\small \verbatiminput{listings/goloz_do.oz} }{\small \par}

This \textbf{\emph{Do}} procedure can then be passed to the parallel
search functionality to plan a program execution. Here {}``agent1''
and {}``agent2'' are the DNS names of agents in the team, and {}``Goloz''
is an Oz functor (basically, a portable piece of code) that exports
the \textbf{\emph{Do}} procedure defined above:

{\small \verbatiminput{listings/goloz_do_parallel.oz} }{\small \par}

When this code is run, it will utilize the computational resources
of both agents to plan a legal execution of a given MIndiGolog program.
This requires that the same information is available to each agent,
which restricts the technique to fully-observable domains. We are
currently developing an algorithm for cooperative execution of MIndiGolog
programs that utilizes such distribution of the planning workload.


\Section{Related work\label{sec:Related-Work}}

That Golog shows promise for multi-agent teams is evidenced by the
success of \cite{Ferrein2005readylog} with a RoboCup soccer team
executing a shared Golog program. However, the semantics of their
Golog variant {}``ReadyLog'' remain largely single-agent and do
not address concerns such as: the possibility of performing actions
concurrently and the coordination of concurrent actions; differing
knowledge or beliefs between team members; sharing the computational
workload of planning; and predicting the behavior of teams members
and the environment in the face of many concurrently-executing tasks.
MIndiGolog overcomes some of these limitations, while our ongoing
work on cooperative execution will address the others.

As stated earlier, there has been much promising work on distributed
problem solving systems using the Hierarchical Task Networks formalism
(\cite{tambe97flexible_teamwork,decker95designing_coordination,grosz99planning_together},
among others). We believe high-level program execution to have several
clear advantages over HTN, in particular the ready availability of
controlled nondeterminism. Combined with familiar programming constructs
such as loops and if-then-else, it provides a very powerful formalism
for expressing complex behaviors and tasks \cite{Gabaldon02htn_in_golog,son00htn_golog}.
Golog also benefits from a logic of action rich enough to capture
many challenging aspects of multi-agent domains (such as time and
concurrency) while remaining computationally feasible.

Note that this paper focuses on task specification using Golog and
does not deal with coordination between team members. We are currently
developing techniques for cooperative execution of MIndiGolog programs
based on these successful approaches to executing HTN specifications.


\Section{Conclusions and future work\label{sec:Conclusions}}

Our work integrates several important extensions to the situation
calculus and Golog to better model the dynamics of multi-agent teams.
Specifically, MIndiGolog combines true and interleaved concurrency,
an explicit account of time, and seamless integration of natural actions.
It defines legal executions of high-level programs that are suitable
for cooperative execution by a multi-agent team.

Since the semantics of MIndiGolog are based on first-order logic,
existing techniques for distributed logic programming can be used
to share the execution-planning workload between agents. In fully-observable
domains, the parallel search capabilities of Oz can be used directly.
We are currently developing a more sophisticated coordination strategy
to augment these techniques and allow cooperative execution of MIndiGolog
programs by a team of autonomous agents in partially observable domains.

Such coordination strategies are typically based on explicit mental
attitudes such as knowledge and intention. A key aspect of our recent
work has been the development of a computationally-feasible account
of knowledge in partially observable domains \cite{kelly07sc_know_obs}.
This requires efficient reasoning about what cannot be changed by
certain types of action, and we have developed a technique for answering
such {}``persistence queries'' under some simple assumptions \cite{kelly07sc_persistence}.
An implementation of these techniques will form the base of our distributed
problem solving system.

This paper thus represents a first step towards providing the advantages
of Golog (such as controlled nondeterminism, powerful programming
constructs, and a rich logic of action) for task specification for
multi-agent teams. While significant work remains to be done to produce
a full distributed problem solving system, our current implementations
of MIndiGolog in Prolog and Oz, particularly combined with distributed
logic programming techniques, have already proven useful for programming
the behavior of multi-agent teams in fully observable domains.\bibliographystyle{latex8}
\bibliography{../../../library/references}


\end{document}
