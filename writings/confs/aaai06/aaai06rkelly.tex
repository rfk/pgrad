
\documentclass[letterpaper]{article}

\usepackage{aaai}
\usepackage{times}
\usepackage{helvet}
\usepackage{courier}

\usepackage{verbatim}
\usepackage{amsmath}
\usepackage{amsthm}

\title{High-Level Program Execution for Multi-Agent Teams}

%\author{Ryan Kelly \and Adrian Pearce \\
%Department of Computer Science and Software Engineering \\
%NICTA Victoria Laboratory \\
%The University of Melbourne \\
%Victoria, 3010, Australia \\
%\{rfk,adrian\}@csse.unimelb.edu.au}

\nocopyright

\newtheorem{theorem}{Theorem}

\begin{document}

\maketitle

\begin{abstract}
We present MIndiGolog, a promising approach to programming multi-agent teams
based on high-level program execution. While the concept of high-level program
execution, as embodied by the programming language Golog, has recently arisen
as a practical
alternative to traditional planning for single-agent systems, several
challenges are encountered when moving to a multi-agent setting.  We introduce
the following enhancements to make Golog more suitable for multi-agent teams:
integration of true concurrency of actions with the interleaved concurrency
of ConGolog; incorporation of an explicit notion of time to enrich the world
model and assist coordination; and tight integration of predictable
exogenous actions (often called ``natural actions'') into the semantics of
the language.
\end{abstract}

\section{Introduction}

As described in \cite{giacomo99indigolog}, \emph{high-level program execution}
is emerging as a practical alternative to traditional plan
synthesis. By ``high-level program'' is meant a program whose
primitive operations are domain-specific actions, and which may be
incompletely specified due to nondeterministic constructs. Rather
than a planner finding a sequence of actions leading from an initial
state to some goal state, the task is to
find a sequence of actions which constitute an execution of this program.

This task can range from executing a fully-specified program to synthesizing
an action sequence from a completely nondeterministic program, thus
subsuming both deterministic agent programming and traditional planning.
The primary advantage of this approach is \emph{controlled nondeterminism},
allowing some aspects of the program to remain unspecified while avoiding
an exponential increase in planning difficulty for large programs.
One very successful application of this technique is the Golog \cite{levesque97golog}
family of programming languages, which have provided clear benefits
for single-agent programming that include: the ability to control
the amount of nondeterminism; sophisticated symbolic reasoning about
the world; and formal verification of program properties, in some
cases using automated tools.

When a team of agents must cooperate closely in order to achieve some
shared goal, they can often be conceptualized as a single agent with
distributed sensing, reasoning and acting capabilities. Behaviors
and goals are naturally ascribed to the team rather than to individual
members.
This suggests high-level program execution as an attractive
approach, potentially providing the benefits described above for applications
involving multi-agent teams.

Pursuant to this goal, we combine several existing extensions to the
Golog language to make it suitable for representing multi-agent teams.
Key among these is the notion of true concurrency of actions, which
we combine with the interleaved concurrency found in the language
ConGolog to give a flexible account of concurrent
execution. An explicit notion of time is incorporated, both to enrich
the world model and to assist in coordination between agents. The
concept of natural actions is also tightly
integrated into the language, to allow agents to predict the behavior
of each other and their environment. We name the resulting language
``MIndiGolog'' for ``Multi-Agent IndiGolog''.

\section{The Situation Calculus and Golog}

The representation of a dynamic world in the situation calculus consists
of first-order logic statements modeling the world in terms of actions,
situations, and fluents.
For space reasons only a brief overview is given here.
A detailed treatment
is available in \cite{pirri99contributions_sitcalc}.

\emph{Actions} are function terms in the logic denoting the ways
in which the state of the world can be changed.
The evolution of the world is handled using \emph{situations}, which
are histories of actions that have been performed. The initial situation
is represented by the term $S_{0}$, and terms representing other situations
are constructed using the function $do(a,s)$ which represents the
situation resulting from performing action $a$ in situation $s$.
Properties of the world are described by \emph{fluents}, which are
functions or predicates taking a situation as their final argument.
The conditions under which an action can be performed are given in
terms of the \emph{possibility fluent} $Poss(a,s)$. The truth values
of fluents are specified by defining what is true in the initial situation,
and collecting the effects of various actions into \emph{successor
state axioms} characterizing how they change from one situation to
the next.

Formally, a dynamic world is described using a \emph{theory of action}
$\mathcal{D}$, a collection of (mostly) first-order axioms describing
how the world behaves. It contains axioms describing: the foundational
axioms of the calculus; action precondition axioms; successor state
axioms; unique names axioms for actions; and a description of the
initial situation:
\begin{equation}
\label{eqn:sc_action_theory}
\mathcal{D}=\Sigma\cup\mathcal{D}_{ap}\cup\mathcal{D}_{ss}\cup\mathcal{D}_{una}\cup\mathcal{D}_{S_{0}}
\end{equation}

Statements about the world are then evaluated relative to this theory
of action.

Golog is a declarative agent programming language based on the situation
calculus, developed by \cite{levesque97golog}. Testimony
to its success are its wide range of applications and many extensions
to provide additional functionality, some of which will be discussed below.
In this paper we use the term ``Golog'' to refer to the family
of languages based on this technique.

To program an agent using Golog one specifies a situation calculus
theory of action, and a program consisting of actions from the world
connected using programming constructs such as if-then-else, while
loops, and nondeterministic choice. Table \ref{tbl:Golog-Operators}
lists some of the operators available in various incarnations of
the language.

\begin{table}[t]
\begin{center}\begin{tabular}{|c|c|}
\hline 
Operator&
Meaning\tabularnewline
\hline
\hline 
$\phi?$&
Proceed only if condition $\phi$ is true\tabularnewline
\hline 
$\delta_{1};\delta_{2}$&
Execute program $\delta_{1}$followed by $\delta_{2}$\tabularnewline
\hline 
$\delta_{1}|\delta_{2}$&
Execute either $\delta_{1}$ or $\delta_{2}$\tabularnewline
\hline 
$\pi(x)\delta(x)$&
Nondet. select arguments for $\delta$\tabularnewline
\hline 
$\delta*$&
Execute $\delta$ zero or more times\tabularnewline
\hline 
$\mathbf{if}\,\phi\,\mathbf{then}\,\delta_{1}\,\mathbf{else}\,\delta_{2}$&
Exec. $\delta_{1}$ if $\phi$ holds, $\delta_{2}$ otherwise\tabularnewline
\hline 
$\mathbf{while\,}\phi\mathbf{\, do}\,\delta$&
Execute $\delta$ while $\phi$ holds\tabularnewline
\hline 
$\delta_{1}||\delta_{2}$&
Concurrently execute $\delta_{1}$and $\delta_{2}$\tabularnewline
\hline 
$\Sigma\delta$&
Find and perform legal exec. of $\delta$\tabularnewline
\hline 
$\mathbf{proc}P(\overrightarrow{x})\delta(\overrightarrow{x})\mathbf{end}$&
Procedure definition\tabularnewline
\hline
\end{tabular}\end{center}


\caption{Some Golog Operators\label{tbl:Golog-Operators}}
\end{table}


In line with the idea of high-level program execution, the agent's
control program may be nondeterministic. It is the task of the agent to
find a deterministic instantiation of the program, a sequence of actions
that can legally be performed in the world. Such a sequence is called a ``legal
execution'' of the Golog program.

The semantics of the Golog operators are typically defined recursively
using logical formulae. The most flexible semantics have proven to
be the transition semantics introduced with the extension ConGolog
\cite{giacomo00congolog} to allow concurrent execution of several
programs. Two predicates $Trans$ and $Final$ are defined for each
operator. Intuitively, $Trans(\delta,s,\delta',s')$ is true precisely
when executing a single step of program $\delta$ can cause the world
to move from situation $s$ to situation $s'$, after which $\delta'$
remains to be executed. It thus characterizes single steps of computation.
$Final(\delta,s)$ is true when program
$\delta$ may legally terminate its execution in situation $s$.  For example,
equation (\ref{eqn:trans_conc_orig}) characterizes transition for the 
concurrent execution construct of ConGolog:
\footnote{We follow the convention that lower-case free variables are
implicitly universally quantified}
\begin{multline}
\label{eqn:trans_conc_orig}
Trans(\delta_{1}||\delta_{2},s,\delta',s')\equiv\\
\shoveright{\exists\gamma. Trans(\delta_{1},s,\gamma,s')\wedge \delta'=(\gamma||\delta_2)}\\
\vee\exists\gamma. Trans(\delta_{2},s,\gamma,s')\wedge \delta'=(\delta_1 || \gamma)
\end{multline}

In words, this states that it is possible to single-step the concurrent
execution of $\delta_1$ and $\delta_2$ by performing either a step from
$\delta_1$ or a step from $\delta_2$, with the remainder $\gamma$ left
to execute
concurrently with the other program.
It thus specifies concurrent execution as the interleaving of steps from
the two programs.

The original Golog was conceived as on offline planner - the agent
determines a complete legal execution of the program before any actions
are performed in the world.  If the theory of action $\mathcal{D}$ is enriched
with $Trans$ and $Final$, this is a theorem proving task to find a suitable
situation $s$:
\begin{equation}
\label{eqn:golog_execution}
\mathcal{D}\models\exists s.\left[Trans*(\delta,S_{0},\delta',s)\wedge Final(\delta',s)\right]
\end{equation}

Here $Trans*$ is the reflexive transitive closure of $Trans$, or at least a
first-order approximation thereof.
While this guarantees that a legal execution
will be found if possible, it can be impractical for large programs
and cannot account for unexpected occurrences or information obtained
from run-time sensing.  IndiGolog \cite{giacomo99indigolog} is an extension
that allows planning to be interleaved with performing the actions in the world.

As an example, consider a program $MakeSalad$ that instructs a team of
agents to prepare a simple salad.
\begin{multline}
\label{eqn:MakeSalad}
\mathbf{proc}\,MakeSalad(dest)\\
\left[\pi(agt)(ChopTypeInto(agt,Lettuce,dest))\,||\right.\\
\pi(agt)(ChopTypeInto(agt,Carrot,dest))\,||\\
\left.\pi(agt)(ChopTypeInto(agt,Tomato,dest))\right]\,;\\
\pi(agt)\left[acquire(agt,dest)\,;\right.\\
 \shoveright{begin\_task(agt,mix(dest,1))\,;\quad}\\
 \shoveright{end\_task(agt,mix(dest,1))\,;\quad}\\
 \left.release(agt,dest)\right]\,\,\mathbf{end}
\end{multline}

The procedure $ChopTypeInto$ (not shown) picks an object of the given
type and an available chopping board, chops the object using the board,
then transfers it into the destination container. $MakeSalad$ tells
the agents to do this for a lettuce, a carrot and a tomato, then mix
the ingredients together for 1 minute. The agent assigned to handling
each ingredient is not specified ($\pi$ construct), nor is the order in which
they should be added ($||$ construct). 

While this program could be executed using the existing semantics of ConGolog
or IndiGolog, the result would be suboptimal.  As these languages lack true
concurrency, only a single agent could act at each step.
Predictable exogenous actions such
as $end\_task$ also require special handling in these languages.  MIndiGolog
is designed to produce executions of such programs that are more suitable for
multi-agent teams.

\section{MIndiGolog Semantics}

We have integrated three extensions to the situation calculus with the
semantics of ConGolog and IndiGolog to overcome
challenges encountered when moving to a multi-agent setting.  These
extensions allow the agents to represent time, concurrently-occurring actions,
and natural actions.

\subsubsection{Time:}
An explicit notion of time can make coordination between agents
easier, as joint actions may be performed at a particular time.  It
also allows a richer description of the world - for example, it becomes possible
to specify that cake must be baked for 30 minutes.

Several authors have successfully incorporated a temporal component
into the situation calculus, notably \cite{pinto94temporal} and
\cite{reiter96sc_nat_conc}.  We follow the ideas of Reiter but, to
avoid complications with concurrent actions, we attach an explicit 
occurrence time to each situation rather than each action.

The successor situation function $do(a,s)$
becomes $do(a,t,s)$, to indicate ``action $a$ was performed at
time $t$ in situation $s$''. Likewise, the possibility predicate
$Poss(a,s)$ becomes $Poss(a,t,s)$, meaning ``it is possible to
perform action $a$ at time $t$ in situation $s$''.  The existing semantics
of IndiGolog are trivially modified to accommodate this.
Time may be represented by any appropriately-behaved sequence (e.g. reals,
integers), and no commitment is made about the starting time of $S_0$.
The function $start(s)$ is added to give the starting time of other situations:
\begin{equation}
\label{eqn:sit_start}
start(do(a,t,s))=t
\end{equation}

\subsubsection{Concurrency:} 
For multi-agent teams, concurrency refers to two distinct ideas -
the possibility of performing several actions at the same instant,
and the possibility of interleaving the execution of several high-level
programs. These are referred to as \emph{true concurrency} and \emph{interleaved
concurrency} respectively.

The work of \cite{lin92sc_conc,reiter96sc_nat_conc} provides an extension
to the situation calculus that allows several primitive actions to
occur at the same time. Action terms are replaced with sets of actions
to be performed simultaneously. All functions and predicates that
take an action are modified to accept sets of actions instead. For
example, $do$ and $Poss$ become $do(\{ a_{1},a_{2},...\},t,s)$
and $Poss(\{ a_{1},a_{2},...\},t,s)$.

As mentioned earlier, concurrent execution of multiple programs
was developed in the language ConGolog with the concurrency operator
$\delta_1||\delta_2$.
These two notions of concurrency were combined by
\cite{pinto99tcongolog}, who modified ConGolog to incorporate sets
of concurrent actions. We follow a similar approach, but place
additional restrictions on the semantics to ensure that programs are
well-behaved and to incorporate an account of time.

While it is straightforward to modify the clause of $Trans$
for primitive actions to accept sets of concurrent actions, there are
deeper implications for the concurrency operator. In ConGolog/IndiGolog it
is implemented by accepting a transition from either of the two programs
as a transition for the pair (equation \ref{eqn:trans_conc_orig}).
In the presence of true concurrency, this is insufficient. Suppose program
$\delta_1$ may be transitioned by performing actions $c_1$, and
$\delta_2$ may be transitioned by performing actions $c_2$.
The potential for true concurrency should be exploited by performing both $c_1$
and $c_2$ simultaneously, i.e. $c_1 \cup c_2$.

This approach introduces several complications. First, the combination
of actions $c_{1}\cup c_{2}$ is not guaranteed to be possible even if the
individual actions are.  For example, an agent may not perform two actions
at once, or two agents may not be able to acquire the same resource at the
same time.
This is known as the precondition interaction problem \cite{pinto94temporal}
and is an area of ongoing research.

For our purposes this is addressed by introducing a predicate $Conflicts(c,t,s)$
which is true when the actions in $c$ are in conflict and cannot
be performed together. The possibility axiom for concurrent actions
is then:
\begin{multline}
\label{eqn:poss_conc_acts}
Poss(c,t,s)\equiv\\
\forall a\left[a\in c\rightarrow Poss(a,t,s)\right]\wedge\neg Conflicts(c,t,s)
\end{multline}

Another issue arises when two programs can legitimately be transitioned
by executing the same action. Consider the following programs which
add ingredients to a bowl:
\begin{multline}
\delta_{1}=place\_ in(Thomas,Flour,Bowl)\,;\\
   \shoveright{place\_ in(Thomas,Sugar,Bowl)}\\
\shoveleft{\delta_{2}=place\_ in(Thomas,Flour,Bowl)\,;}\\
   place\_ in(Thomas,Egg,Bowl)
\end{multline}

Executing $\delta_{1}||\delta_{2}$ should result in the bowl containing
two units of flour, one unit of sugar and an egg. However, individual
transitions for both programs are $c_{1}=c_{2}=place\_ in(Thomas,Flour,Bowl)$.
Naively executing $c_1 \cup c_2$ for the combination would add only one unit
of flour. Alternately, consider two programs
waiting for a timer to ring (a natural action, not initiated by any agent):
\begin{multline}
\delta_{1}=ring\_ timer(OvenTimer)\,;\\
   \shoveright{acquire(Thomas,Bowl)}\\
\shoveleft{\delta_{2}=ring\_ timer(OvenTimer)\,;}\\
    acquire(Richard,Board)
\end{multline}

Both programs should be allowed to proceed with a single occurrence
of the $ring\_ timer$ action. Thus to ensure consistency, 
concurrent execution must not be allowed
to transition both programs through the occurrence of the same
\emph{agent-initiated} action.  If-then-else or
choice operators should be used to allow agent-initiated actions to be
skipped if they are not necessary.

Taking these factors into account, we derive an improved transition
rule for concurrency are follows:
\begin{multline}
\label{eqn:trans_conc_new}
Trans(\delta_{1}||\delta_{2},s,\delta',s')\equiv\\
\shoveright{\exists\gamma. Trans(\delta_{1},s,\gamma,s')\wedge \delta'=(\gamma||\delta_2)}\\
\shoveright{\vee\exists\gamma. Trans(\delta_{2},s,\gamma,s')\wedge \delta'=(\delta_1 || \gamma)}\\
\shoveright{\vee\exists c_{1},c_{2},\gamma_{1},\gamma_{2},t\,.Trans(\delta_{1},s,\gamma_{1},do(c_{1},t,s))}\\
\wedge Trans(\delta_{2},s,\gamma_{2},do(c_{2},t,s))\\
\wedge Poss(c_{1}\cup c_{2},t,s)\\
\wedge\forall a\left[a\in c_{1}\wedge a\in c_{2}\rightarrow Natural(a)\right]\\
\wedge\delta'=(\gamma_{1}||\gamma_{2})\wedge s'=do(c_{1}\cup c_{2},t,s)
\end{multline}

The first two lines are the original interleaved concurrency clause from
ConGolog, while the remainder
characterizes the above considerations for true concurrency.
This robust combination allows
the language to more accurately reflect the concurrency present in
multi-agent teams.

\subsubsection{Natural Actions:}
These are a special class of exogenous actions, those
actions which occur outside of an agent's control.
Introduced to the situation
calculus by \cite{pinto94temporal} and expanded by \cite{reiter96sc_nat_conc},
they are classified according to the following requirement: natural
actions must occur at their predicted times, provided no earlier actions
prevent them from occurring. For example, a timer will ring at the
time it was set for, unless it is switched off.
They are especially important in a multi-agent context for predicting the
behavior of other agents, such as when they will finish a particular
long-running task.

The methodology of Reiter is adopted, suitably
modified for our handling of time. Natural actions are indicated by the
truth of the predicate $Natural(a)$. As usual, the times at which
natural actions may occur are specified by the $Poss$ predicate.
For example, suppose that the fluent $Timer\_Set(ID,m,s)$ represents
the fact that a particular timer is set to ring in $m$ minutes in
situation $s$. The possibility predicate for the $ring\_timer(ID)$
action would be:
\begin{multline}
Poss(ring\_timer(ID),t,s)\equiv\\
\exists m.\left[Timer\_Set(ID,m,s)\wedge t=start(s)+m\right]
\end{multline}

An important concept when dealing with natural actions it the least
natural time point (LNTP) of a situation. This is defined as the earliest
time at which a natural action may occur:
\begin{multline}
\label{eqn:lntp_def}
Lntp(s,t)\equiv \exists a\left[Natural(a)\wedge Poss(a,t,s)\right]\wedge\\
\forall a,t_{a}\left[Natural(a)\wedge Poss(a,t_{a},s)\rightarrow t\leq t_{a}\right]
\end{multline}

Note that the LNTP need not exist for a given situation.
We will assume that the domain axiomatisation
is designed to avoid certain pathological cases, so that absence of an
LNTP implies that no natural actions are possible.

To enforce the requirement that natural actions must occur when possible,
a predicate $Legal(s)$ is introduced which is true only for situations
which respect this requirement (assuming no precondition interaction
among natural actions).  Legal situations are the only situations that could
actually be brought about in the real world:
\begin{multline}
\label{eqn:legal_def}
Legal(S_0) \equiv True\\
\shoveleft{Legal(do(c,t,s)) \equiv}\\
Legal(s)\wedge Poss(c,t,s)\wedge start(s)\leq t \, \wedge\\
\forall a,t_{a}.\,Natural(a)\wedge Poss(a,t_{a},s)\rightarrow\left[a\in c\vee t<t_{a}\right]
\end{multline}

While planning with natural actions has previously been done in Golog
\cite{pirri00planning_nat_acts}, the programmer
was required to explicitly check for the possible occurrence of natural
actions and ensure that executions of the program resulted in legal
situations. Our integration significantly lowers the burden on the
programmer by guaranteeing that all program executions produce legal
situations.
MIndiGolog agents will plan for the occurrence of natural actions
without having them explicitly mentioned in the program.  They may
optionally be included in the program, instructing the agents to wait for
the action to occur before proceeding.

This is achieved using a new $Trans$ clause for programs consisting of a single
action $c$, shown in equation (\ref{eqn:trans_prim_new}).
If $s$ has an LNTP $t_n$ and corresponding
set of natural actions $c_n$, a transition can be made in three ways:
perform $c$ at a time before $t_n$ (third line), perform it along with
the natural actions at $t_n$ (fourth line), or wait for the natural
actions to occur (fifth line).  If there is no LNTP, then $c$ may be
performed at any time greater than $start(s)$.
\begin{multline}
\label{eqn:trans_prim_new}
Trans(c,s,\delta',s')\equiv \\
  \exists t,t_n,c_n.\,Lntp(s,t_n) \wedge t\geq start(s) \wedge\\
      \forall a.\left[\,Natural(a)\wedge Poss(a,t_n,s)\equiv c\in c_n\right]\wedge\\
      \left[ t<t_n\wedge Poss(c,t,s)\wedge s'=do(c,t,s)\wedge \delta'=Nil\right.\\
            \vee Poss(c\cup c_n,t_n,s)\wedge s'=do(c\cup c_n,t,s)\wedge \delta'=Nil\\
            \vee \left.s'=do(c_n,t_n,s)\wedge \delta'=c\right]\\
  \vee\,\neg\exists t_{n}.\,Lntp(s,t_{n})\wedge\exists t.\,Poss(c,t,s)\wedge\\
  t\geq start(s)\wedge s'=do(c,t,s)\wedge\delta'=Nil
\end{multline}

\subsubsection{MIndiGolog:}
Let a \emph{MIndiGolog Theory of Action} be a theory
of action in the situation calculus enhanced with time, true concurrency and
natural actions, augmented with the predicates $Trans$ and $Final$ from
IndiGolog, modified according to equations (\ref{eqn:trans_conc_new}) and
(\ref{eqn:trans_prim_new}).
IndiGolog programs can be executed relative to this
theory of action as in equation (\ref{eqn:golog_execution}),
and the resulting legal
executions will be suitable for execution by multi-agent teams.
In particular, all legal executions of a MIndiGolog
program produce legal situations:

\begin{theorem}
\label{thm:trans_legal}
Let $\mathcal{D}$ be a MIndiGolog theory of action. Then:
\begin{multline*}
\mathcal{D} \models \forall s,s',\delta,\delta'.Legal(s)\wedge Trans(\delta,s,\delta',s')\\
\rightarrow Legal(s')
\end{multline*}
\end{theorem}

\begin{proof}
By induction on the structure of $\delta$.
Consider first the base case of a single concurrent action $\delta=c$.  This
$Trans$ clause (\ref{eqn:trans_prim_new}) has four disjuncts. We treat
here only the
case where actions are performed before the LNTP, other
cases employ similar reasoning.
Assuming this case, (\ref{eqn:trans_prim_new}) reduces to:
\begin{multline*}
Trans(c,s,Nil,do(c,t,s)) \equiv \exists t_n . Lntp(s,t_n) \wedge \\
  t < t_n \wedge ti \geq start(s) \wedge Poss(c,t,s)
\end{multline*}
By the definition of $Lntp$ (\ref{eqn:lntp_def}) and given $t < t_n$,
no natural actions
can be possible before or at time $t$, allowing us to add:
\begin{multline*}
Trans(c,s,Nil,do(c,t,s)) \equiv \exists t_n . Lntp(s,t_n) \wedge \\
  t < t_n \wedge t \geq start(s) \wedge Poss(c,t,s) \wedge\\
  \forall a,t_a \left[ Natural(a) \wedge Poss(a,t_a,s) \rightarrow t < t_a \right]
\end{multline*}
Given $Legal(s)$ and equation (\ref{eqn:legal_def}), it is then clear that for
this case:
\begin{multline*}
\mathcal{D} \models Legal(s) \wedge Trans(c,s,Nil,do(c,t,s))\\
\rightarrow Legal(do(c,t,s))
\end{multline*}
Next consider the case of concurrent execution of two programs.  The $Trans$
clause  for concurrency (\ref{eqn:trans_conc_new}) only introduces a
new situation term when both
sub-programs transition simultaneously, with other cases trivially handled by
the inductive hypothesis.  Assuming this case, (\ref{eqn:trans_conc_new})
reduces to:
\begin{multline*}
Trans(\delta_{1}||\delta_{2},s,\delta',s')\equiv \exists c_{1},c_{2},\gamma_{1},\gamma_{2},t\,.\\
Trans(\delta_{1},s,\gamma_{1},do(c_{1},t,s))\,\wedge\\
Trans(\delta_{2},s,\gamma_{2},do(c_{2},t,s))\wedge\\
Poss(c_{1}\cup c_{2},t,s)\,\wedge\\
\forall a\left[a\in c_{1}\wedge a\in c_{2}\rightarrow Natural(a)\right]\,\wedge\\
\delta'=(\gamma_{1}||\gamma_{2})\wedge s'=do(c_{1}\cup c_{2},t,s)
\end{multline*}
Both $do(c_1,t,s)$ and $do(c_2,t,s)$ will be legal by the inductive hypothesis,
hence $start(s) \leq t$ and both $c_1$ and $c_2$ contain any required natural
actions.
Since $Trans$ requires that $Poss(c_1 \cup c_2,t,s)$ all conjuncts in
equation (\ref{eqn:legal_def}) are satisfied, and:
\begin{equation*}
\mathcal{D} \models Legal(s) \wedge Trans(\delta_1 || \delta_2,s,\delta,s'))
\rightarrow Legal(s')
\end{equation*}
No other $Trans$ clauses introduce new situation terms, making
the remaining induction trivial.
\end{proof}

\begin{theorem}
Let $\mathcal{D}$ be a MIndiGolog theory of action. Then:
\begin{equation*}
\mathcal{D} \models \forall s',\delta,\delta'.\,Trans*(\delta,S_0,\delta',s')
\rightarrow Legal(s')
\end{equation*}
Consequently, all legal executions of a MIndiGolog program respect
the occurrence of natural actions.
\end{theorem}

\begin{proof}
Immediate from Theorem \ref{thm:trans_legal}, the legality of $S_0$, and
the properties of transitive closure.
\end{proof}

\begin{figure*}[t]
\centering
\framebox{
\begin{minipage}[t]{0.5\linewidth}
\small
\verbatiminput{listings/output_makeSalad_1.txt}
\end{minipage}
\begin{minipage}[t]{0.45\linewidth}
\small
\verbatiminput{listings/output_makeSalad_2.txt}
\end{minipage}
}
\caption{One possible execution of the $MakeSalad$
program with three agents. Variables $\_ U$, $\_ T$, etc give
the occurrence times of each action, constrained to ensure all situations
are legal.}\label{cap:example_trace}
\end{figure*}

The effect of our enhancements can be seen in Figure
\ref{cap:example_trace}, which shows one possible legal execution of the
$MakeSalad$ program (\ref{eqn:MakeSalad}). Notice the occurrence of several
actions within each \textbf{do} statement, demonstrating the integration
of true concurrency into the language.
Note also the explicit occurrence time for
each step, allowing the agents to properly synchronize their actions.
This time is not fixed, but constrained to synchronize with the natural
actions included in the program.

\section{Related Work}

The utility of the high-level program execution paradigm is highlighted
by the many extensions which have been developed for the base Golog
language. Prominent among these are the already mentioned \emph{ConGolog}
\cite{giacomo00congolog} and 
\emph{IndiGolog} \cite{giacomo99indigolog}
Another useful extension is \emph{DTGolog} \cite{boutilier00dtgolog}, which
combines the situation calculus with Markov Decision Processes to
allow agents to determine the optimal execution of a program in the face
of probabilistic action outcomes.

There has also been work on extending the situation calculus to allow
richer world models, notably that of Pinto \cite{pinto94temporal}
and Reiter \cite{reiter96sc_nat_conc}. 
However, the incorporation of such enrichments with extensions to Golog
remains a topic of ongoing work.
It is at this crossroads
that MIndiGolog is positioned. It combines the
collective capabilities of several of these extensions to represent and program
the behavior of multi-agent teams.

While work with Golog has predominantly focused on single-agent systems,
there have been several notable applications of the technique to multi-agent
systems. The Cognitive Agent Specification Language \cite{shapiro02casl}
uses ConGolog to describe the behavior of agents in a multi-agent
setting, and an automated theorem prover is in development that can
verify properties of the programs. It differs from our work in that
it is designed for open multi-agent systems, with each agent having
its own Golog program defining its individual behavior. By contrast, we seek
to develop a single Golog program that is cooperatively executed by
all agents. It also appears to focus on modeling and simulation
of agent systems more than actually executing the programs.

Closer to the spirit of this paper is the work of
\cite{Ferrein2005readylog} on ReadyLog, a Golog derivative designed
for highly dynamic domains. It has been successfully used to control
multi-agent teams in the RoboCup soccer tournament by having the agents
execute a shared Golog program. ReadyLog agents
each individually determine a legal execution of the program, allowing
them to coordinate without communication.  We aim to allow the agents to
communicate and pool their computational resources in planning an
execution.
ReadyLog integrates
the facilities of concurrency, online execution and decision theory
and can be seen as a unification of ConGolog, IndiGolog and DTGolog.
We believe our enhancements to be compatible with and complimentary 
to this work.

It should be noted that our work does not treat many higher-level concepts
relevant
to teamwork, such as team formation and the development of joint goals and
intentions.  MIndiGolog programs allow a given team of agents to
plan the performance of a specific task.  We believe the approach
to be complimentary to
the wide variety of formalisms for modeling and programming higher-level
aspects of teamwork.

\section{Conclusions}

In this paper, we have integrated of several important
extensions to the situation calculus and Golog to provide a new language
suitable for expressing high-level programs for multi-agent teams.
Specifically, MIndiGolog combines both true and interleaved concurrency,
an explicit account of time, and seamless integration of natural actions.

Since planning a legal execution in MIndiGolog is a theorem-proving
task, existing tools for distributed logic programming can be used to
allow a team of agents to share the planning workload.  Preliminary
tests using the distributed programming capabilities of the Oz language
\cite{schulte00oz_parallel} have yielded promising results.
We are presently exploring the potential of this approach for a range
of distributed programming applications, and the integration of additional
techniques from the Golog community. Key among
these is the significant work on representing knowledge and belief
in Golog. As many applications involve partially observable worlds,
different agents must be capable of having different beliefs about
the world.  This will in turn require more complex strategies for
sharing the planning workload.

As we have shown, the combination of state-of-the-art
techniques from Golog is highly capable of representing tasks and
behavior
for multi-agent teams. Combined with facilities for distributed logic
programming, MIndiGolog promises to provide the advantages of the high-level
program execution paradigm for multi-agent teams.

\bibliography{/storage/uni/pgrad/library/references}
\bibliographystyle{aaai}

\end{document}
