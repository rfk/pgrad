\chapter{Property Persistence}\label{ch:persistence}
%\minitoc

This chapter will be an expansion of my conference paper on this topic \cite{kelly07sc_persistence}.

Remaining work: further characterization of completeness of the algorithm, ways to deal with interacting effects.

We should also talk more generally about reasoning, maybe formalise
"reasoning in the fluent domain" and prove it's applicability.

Also some chat about decidability, e.g. using typing of functions to
guarantee a finite herbrand universe (\cite{levesque04krr_book}, pp69)

Perhaps the stuff about shannon graphs can also be merged into this chapter?
Maybe it's two chapters?

\section{The Persistence Problem}

\begin{itemize}
\item Formal definition
\item Examples of why it's important
\end{itemize}

\section{The Persistence Condition}

\begin{itemize}
\item Definition of $\mathcal{P}$, $\mathcal{P}^{1}$ operators
\item Proof that $\mathcal{P}$ is a least-fixed-point
\item Justification that it's an "effective" technique
\item Techniques for ensuring completness
\end{itemize}

\section{Calculating $\mathcal{P}$}

\begin{itemize}
\item Naive algorithm: definition, shortcomings
\item Algorithm based on explicit effect axioms
\item Dealing with interacting effects
\end{itemize}


