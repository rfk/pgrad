\chapter{Reasoning}\label{ch:reasoning}
%\minitoc

\section{Effective Reasoning}

\begin{itemize}
\item Careful qualification of what we mean by "Effective"
\item Review basics of reasoning in more detail, esp. regression.
\item SitCalc as a re-writing system
\item Introduce "uniformize" operator to compliment regression
\item reasoning in the "fluent domain"
\item using typing of functions to guarantee a finite herbrand universe (\cite{levesque04krr_book}, pp69) and therefore decidability
\end{itemize}

\section{Property Persistence}

\begin{itemize}
\item Formal definition
\item Examples of why it's important
\item Why it cant be done using standard regression
\end{itemize}

\section{The Persistence Condition}

\begin{itemize}
\item Definition of $\mathcal{P}$, $\mathcal{P}^{1}$ operators
\item Proof that $\mathcal{P}$ is a least-fixed-point
\item Justification that it's an "effective" technique
\item Techniques for ensuring completness
\end{itemize}

\section{Calculating $\mathcal{P}$}

\begin{itemize}
\item Naive algorithm: definition, shortcomings
\item Algorithm based on explicit effect axioms
\item Dealing with interacting effects
\end{itemize}

\section{TODO}
\begin{itemize}
\item Using constraint solvers to reason about time
\item Handling interacting effects, action preconditions
\end{itemize}

