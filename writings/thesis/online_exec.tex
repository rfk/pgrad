\chapter{Online Execution}\label{ch:online-exec}
%\minitoc

Results for this chapter exist in a quite preliminary form as a draft conference paper, but I don't forsee any major stumbling blocks in completing the work.

Remaining work: devise algorithm for communicating to establish action safety, formalize interaction between online and offline execution.

\section{Knowing what to do}
\begin{itemize}
\item Agents may be unsure exactly what remains to be executed
\item Encoding the program as a fluent, so it can be dealt with epistemically
\item S.S.A. based on transition semantics
\item Using natural actions to transition test conditions
\end{itemize}

\section{Coordination using Social Laws}
\begin{itemize}
\item Use an ordering to specify which actions are prefered
\item Define the notion of a "safe" action
\item Perform a joint action when it's safety is common knowledge among actors
\item Communicating to determine safety of actions
\item ?? Ways to increase efficiency of this approach ??
\end{itemize}

\section{Managing the Search Operator}
\begin{itemize}
\item beginplan() and endplan() actions observable by all
\item planning in the face of change - "restart" actions incorporated into planning procedure
\end{itemize}

