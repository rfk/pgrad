\chapter{Reasoning: In Theory}\label{ch:reasoning-theory}
%\minitoc

This chapter will be an overview of reasoning techniques in the situation
calculus.  A large part will be an expansion of my conference paper on this topic \cite{kelly07sc_persistence}.

Remaining work: further characterization of completeness of the persistence algorithm, ways to deal with interacting effects.

Perhaps the stuff about shannon graphs can also be merged into this chapter?
Maybe it's two chapters?

\section{Effective Reasoning}

\begin{itemize}
\item Review basics of reasoning in more detail, esp. regression.
\item SitCalc as a re-writing system
\item Decidability
  \begin{itemize}
  \item use typing of functions to guarantee a finite herbrand universe (\cite{levesque04krr_book}, pp69)
  \end{itemize}
\item reasoning in the "fluent domain"
\end{itemize}

\section{Property Persistence}

\begin{itemize}
\item Formal definition
\item Examples of why it's important
\end{itemize}

\section{The Persistence Condition}

\begin{itemize}
\item Definition of $\mathcal{P}$, $\mathcal{P}^{1}$ operators
\item Proof that $\mathcal{P}$ is a least-fixed-point
\item Justification that it's an "effective" technique
\item Techniques for ensuring completness
\end{itemize}

\section{Calculating $\mathcal{P}$}

\begin{itemize}
\item Naive algorithm: definition, shortcomings
\item Algorithm based on explicit effect axioms
\item Dealing with interacting effects
\end{itemize}


