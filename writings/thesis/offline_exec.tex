\chapter{Offline Execution}\label{ch:offline-exec}
%\minitoc

\section{What is Planning?}
\begin{itemize}
\item In this setting, it is the process of \emph{resolving nondeterminism}
\item Planning should produce a program for the agent to follow, that does
not itself require deliberation to perform. \cite{levesque96what_is_planning,giacomo04sem_delib_indigolog}
\item Must account for outcomes of sensing actions - some sort of branch/case statement
\item Must be epistemically feasible (show formalisation from \cite{giacomo04sem_delib_indigolog})
\end{itemize}

\section{What is Team Planning?}
\begin{itemize}
\item Produce a program for each agent
\item Programs must include the necessary coordination actions/conditions
\item Any interleaving of the programs must be a valid legal execution
\item similar idea in strips: \cite{boutilier01partialorder_conc}
\end{itemize}

\section{Resolving Non-Determinism}
\begin{itemize}
\item Hard Problem: removing only as much non-determinism as necessary
\item Easier Problem: re-inserting non-determinism where possible
\item Idea:  plan by first \emph{completely} resolving nondeterminism, but
maintain auxiliary information that allows actions to be nondeterministically 
interleaved where a legal execution is maintained.
\end{itemize}

