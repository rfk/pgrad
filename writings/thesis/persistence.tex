
\chapter{Property Persistence}

\label{ch:persistence} %\minitoc



\section{Effective Reasoning}

\begin{itemize}
\item Careful qualification of what we mean by \char`\"{}Effective\char`\"{} 
\item Review basics of reasoning in more detail, esp. regression. 
\item SitCalc as a re-writing system 
\item reasoning in the \char`\"{}fluent domain\char`\"{}: \[
\mathcal{D}_{una}\models\forall s.\phi(\bar{x},s)\,\,\,\,\,\, iff\,\,\,\,\,\,\mathcal{D}_{una}\models\phi(\bar{x})\]

\item using typing of functions to guarantee a finite herbrand universe
(\cite{levesque04krr_book}, pp69) and therefore decidability 
\end{itemize}

\section{Property Persistence}

\begin{itemize}
\item Formal definition 
\item Examples of why it's important 
\item Why it cant be done using standard regression 
\end{itemize}

\section{The Persistence Condition}

\begin{itemize}
\item Definition of $\mathcal{P}$, $\mathcal{P}^{1}$ operators 
\item Proof that $\mathcal{P}$ is a least-fixed-point 
\item Justification that it's an \char`\"{}effective\char`\"{} technique 
\item Techniques for ensuring completness 
\end{itemize}

\section{Calculating $\mathcal{P}$}

\begin{itemize}
\item Naive algorithm: definition, shortcomings 
\item Algorithm based on explicit effect axioms 
\item Handling advanced features: interacting effects, natural actions 
\item TODO: interacting preconditions and effects 
\end{itemize}

