


\chapter{Property Persistence}

\label{ch:persistence}

This chapter extends the standard effective reasoning techniques of
the situation calculus to deal with certain types of universally-quantified
query. Recall that the foundational axioms of the situation calculus
include a second-order induction axiom to define the set of all situations,
meaning that regression cannot be applied to queries that universally
quantify over situation terms. Unfortunately there are many reasoning
tasks that require such quantification, including the formalism we
will developed in chapter TODO.

In this chapter we study a subset of such queries which we refer to
as \emph{property persistence queries}. We introduce a new approach
that is similar in spirit to the regression operator of \citep{reiter91frameprob}:
define an operator that transforms a formula $\phi$ into a formula
$\mathcal{P}_{\mathcal{D}}[\phi]$, such that $\phi$ persists in
$s$ if and only if $\mathcal{P}_{\mathcal{D}}[\phi]$ holds in $s$.
We term this the \emph{persistence condition} of $\phi$, and show
how to calculate it in a form suitable for effective automated reasoning.

The chapter proceeds as follows: TODO


\section{Property Persistence Queries}

Given some property $\phi$ and situation $s$, a persistence query
asks whether $\phi$ will hold in all situations in the future of
$s$:\[
\mathcal{D}\models\forall s'\,.\, s\sqsubseteq s'\,\rightarrow\,\phi[s']\]


More generally, one may wish to limit the futures considered to those
brought about by a certain class of actions $\alpha$:\[
\mathcal{D}\models\forall s'\,.\, s\le_{\alpha}s'\rightarrow\phi[s']\]
 In words this states {}``given that all subsequent actions satisfy
$\alpha$, $\phi$ will remain true'' or, more succinctly, {}``$\phi$
persists under $\alpha$''. We term queries of this form \emph{property
persistence} \emph{queries}, and they are involved in many useful
reasoning tasks. The following are a small selection:

\textbf{Goal Impossibility:} Given a goal $G$, establish that there
is no possible situation in which that goal is satisfied:\[
\mathcal{D}\models\forall s\,.\, S_{0}\leq s\rightarrow\neg G(s)\]


\textbf{Goal Futility:} Given a goal $G$ and situation $s$, establish
that the goal cannot be satisfied in any possible situation in the
future of $s$:\[
\mathcal{D}\models\forall s'\,.\, s\leq s'\rightarrow\neg G(s')\]
 Note how this is different from goal impossibility: while the agent
may have initially been able to achieve its goal, the actions that
have subsequently been performed have rendered the goal unachievable.
Agents would be well advised to avoid such situations.

\textbf{Checking State Constraints:} This is a variant of goal impossibility
- show that the constraint can never be violated.

\textbf{Need for Cooperation:} To establish that it is absolutely
necessary to cooperate with another agent, an agent must determine
persistence with respect to actions performed by itself:\[
\mathcal{D}\models\forall s'\,.\, s\leq_{OwnAction}s'\rightarrow\neg G[s']\]
 Assuming that the action description predicate $OwnAction$ has been
defined to identify actions performed by the agent in question, this
states that no situation in the future of $s$ in which all actions
were performed by that agent can satisfy $G$. If this is the case,
it will need to cooperate with another agent in order to achieve its
goal.

\textbf{Knowledge with Hidden Actions:} In a subsequent chapter, we
develop an account of knowledge in the situation calculus where some
actions may be hidden from some agents. To facilitate reasoning in
this formalism, agents must be able to reason about what cannot be
changed by actions that they cannot observe, a form of persistence
query.


\section{The Persistence Condition}

For implementing persistence queries in practical systems, we clearly
need to transform the query into a form suitable for effective reasoning.
Specifically, we will transform a persistence query based at $s$
into the evaluation of a uniform formula at $s$, which can be done
effectively using the regression operator. We need some transformation
of a property $\phi$ and action description predicate $\alpha$ into
a uniform formula $\mathcal{P}_{\mathcal{D}}[\phi,\alpha]$ that is
true at precisely the situations in which $\phi$ persists under $\alpha$,
given a particular action theory $\mathcal{D}$. We call $\mathcal{P}_{\mathcal{D}}$
the \emph{persistence condition} of $\phi$ under $\alpha$.

\begin{defn}
The persistence condition $\mathcal{P}_{\mathcal{D}}[\phi,\alpha]$
is a uniform formula that is the weakest precondition for the persistence
of $\phi$ under $\alpha$, given a basic action theory $\mathcal{D}$
without the initial situation axioms. That is:\label{def:persistence-condition}\[
\mathcal{D}-\mathcal{D}_{S_{0}}\models\forall s\,.\,\left(\mathcal{P_{D}}[\phi,\alpha][s]\,\equiv\,\forall s'\,.\, s\leq_{\alpha}s'\rightarrow\phi[s']\right)\]

\end{defn}
Defining $\mathcal{P}_{\mathcal{D}}$ to be independent of the initial
world state allows an agent to calculate it regardless of what (if
anything) is known about the actual state of the world. To see how
$\mathcal{P}_{\mathcal{D}}$ may be calculated, consider the weaker
notion of a formula \emph{persisting to depth $n$} in a situation:

\begin{defn}
A uniform formula $\phi$ persists to depth 1 under $\alpha$ in situation
$s$ when the formula $\mathcal{P}_{\mathcal{D}}^{1}[\phi,\alpha]$
holds in $s$, as defined by:\label{def:persists-depth-n}\[
\mathcal{P}_{\mathcal{D}}^{1}[\phi,\alpha][s]\,\isdef\,\phi[s]\,\wedge\,\forall a\,.\,\mathcal{R_{D}}[\alpha(a,s)\rightarrow\phi[do(a,s)]]\]
 More generally, for any $n\geq0$, a uniform formula $\phi$ persists
to depth $n$ under $\alpha$ in situation $s$ when the formula $\mathcal{P}_{\mathcal{D}}^{n}[\phi,\alpha]$
holds in $s$, as defined by:\[
\mathcal{P}_{\mathcal{D}}^{0}[\phi,\alpha]\,\isdef\,\phi\]
 \[
\mathcal{P}_{\mathcal{D}}^{n}[\phi,\alpha]\,\isdef\,\mathcal{P}_{\mathcal{D}}^{1}[\mathcal{P}_{D}^{n-1}[\phi,\alpha],\alpha]\]

\end{defn}
Note that $\mathcal{P}_{\mathcal{D}}^{1}$ is a literal encoding of
the requirement {}``$\phi$ holds in $s$ and in all its direct successors''.
Since $\alpha$ is an action description predicate and $\phi$ is
a uniform formula, the expression $\alpha(a,s)\rightarrow\phi[do(a,s)]$
can always be regressed and the result will always be uniform in $s$.
Successive applications of $\mathcal{P}_{\mathcal{D}}^{1}$ assert
persistence to greater depths.

Intuitively, one would expect $\mathcal{P}_{\mathcal{D}}[\phi,\alpha]$
to be a fixed-point of $\mathcal{P}_{\mathcal{D}}^{1}[\phi,\alpha]$,
since $\mathcal{P}_{\mathcal{D}}[\phi,\alpha]$ implies persistence
up to any depth. Such a fixed-point can then be calculated using standard
iterative approximation techniques. To show that this is in fact the
case we require a number of theoretical results, presented below.


\subsection{Formal Development}

We begin by adapting two existing results involving induction to operate
with our generalized $\leq_{\alpha}$ notation, and be based at situations
other than $S_{0}$:

\begin{prop}
For any action description predicate $\alpha$, the foundational axioms
of the situation calculus entail the following induction principle:\label{prop:a-order-induction}\begin{multline*}
\forall W,s\,.\,\, W(s)\wedge\left[\forall a,s'\,.\,\alpha(a,s')\wedge s\leq_{\alpha}s'\right.\\
\left.\wedge W(s')\rightarrow W(do(a,s'))\right]\rightarrow\forall s'\,.\, s\leq_{\alpha}s'\rightarrow W(s')\end{multline*}

\end{prop}
\begin{proof}
A trivial adaptation of theorem 1 in \citep{Reiter93proving}. 
\end{proof}
\begin{prop}
For any basic action theory $\mathcal{D}$, uniform formula $\phi$
and action description predicate $\alpha$:\label{prop:a-order-reduction}\begin{gather*}
\mathcal{D}-\mathcal{D}_{S_{0}}\models\forall s\,.\,\phi[s]\rightarrow\left(\forall s'\,.\, s\leq_{\alpha}s'\rightarrow\phi[s']\right)\\
\mathit{iff}\\
\mathcal{D}_{una}\models\forall s,a\,.\,\phi[s]\wedge\mathcal{R}_{\mathcal{D}}[\alpha(a,s)]\rightarrow\mathcal{R}_{\mathcal{D}}[\phi[do(a,s)]]\end{gather*}

\end{prop}
\begin{proof}
A straightforward generalization of Lemma 5 in \citep{Lin94-StateConstraints},
utilizing Proposition \ref{prop:a-order-induction}. 
\end{proof}
Proposition \ref{prop:a-order-reduction} will be key in our algorithm
for calculating the persistence condition. It allows one to establish
the result {}``if $\phi$ holds in $s$, then $\phi$ persists in
$s$'' by checking entailment of a uniform formula by the unique
names axioms, a straightforward first-order reasoning task.

Next we must formalize some basic relationships between $\mathcal{P}_{\mathcal{D}}$
and $\mathcal{P}_{\mathcal{D}}^{n}$, as follows:

\begin{lemma}
Given a basic action theory $\mathcal{D}$, uniform formula $\phi$
and action description predicate $\alpha$, then for any $n$:\label{lem:p-equiv-p(pn)}\begin{multline*}
\mathcal{D}-\mathcal{D}_{S_{0}}\models\forall s\,.\,\left[\,\,\forall s'\,.\,\left(s\leq_{\alpha}s'\rightarrow\phi[s']\right)\right.\\
\left.\equiv\forall s'\,.\,\left(s\leq_{\alpha}s'\rightarrow\mathcal{P}_{D}^{n}[\phi,\alpha][s']\right)\,\,\right]\end{multline*}
 That is, $\phi$ persists under $\alpha$ iff $\,\mathcal{P}_{\mathcal{D}}^{n}[\phi,\alpha]$
persists under $\alpha$. 
\end{lemma}
\begin{proof}
By the definition of $\mathcal{P}_{\mathcal{D}}^{n}$, any $s'$ that
falsifies the right-hand side of this equivalence will also falsify
the left-hand side, and vice-versa. 
\end{proof}
\begin{lemma}
Given a basic action theory $\mathcal{D}$, uniform formula $\phi$
and action description predicate $\alpha$, then for any $n$:\label{lem:p-implies-pn}\[
\mathcal{D}-\mathcal{D}_{S_{0}}\models\forall s\,.\,\left(\mathcal{P_{D}}[\phi,\alpha][s]\rightarrow\mathcal{P}_{\mathcal{D}}^{n}[\phi,\alpha][s]\right)\]

\end{lemma}
\begin{proof}
$\mathcal{P_{D}}[\phi,\alpha]$ implies the persistence of $\phi$
by definition, which implies $\mathcal{P}_{\mathcal{D}}^{n}[\phi,\alpha]$
by Lemma \ref{lem:p-equiv-p(pn)}. 
\end{proof}
We are now equipped to prove the major theorem of this paper: that
if $\mathcal{P}_{\mathcal{D}}^{n}[\phi,\alpha]$ implies $\mathcal{P}_{\mathcal{D}}^{n+1}[\phi,\alpha]$,
then $\mathcal{P}_{\mathcal{D}}^{n}[\phi,\alpha]$ is equivalent to
the persistence condition for $\phi$ under $\alpha$.

\begin{thm}
Given a basic action theory $\mathcal{D}$, uniform formula $\phi$
and action description predicate $\alpha$, then for any $n$:\label{thm:p(pn)-equiv-p}\begin{gather}
\mathcal{D}_{una}\models\forall s\,.\,\mathcal{P}_{\mathcal{D}}^{n}[\phi,\alpha][s]\rightarrow\mathcal{P}_{\mathcal{D}}^{n+1}[\phi,\alpha][s]\label{eqn:pn_persists}\\
\mathit{iff}\nonumber \\
\mathcal{D}-\mathcal{D}_{s_{0}}\models\forall s\,.\,\mathcal{P}_{\mathcal{D}}^{n}[\phi,\alpha][s]\equiv\mathcal{P_{D}}[\phi,\alpha][s]\label{eqn:pn_equiv_persists}\end{gather}

\end{thm}
\begin{proof}
For the \emph{if} direction, we first see by the definition of $\mathcal{P}_{\mathcal{D}}^{1}$
that equation (\ref{eqn:pn_persists}) is equivalent to:\begin{multline*}
\mathcal{D}_{una}\models\forall s,a\,.\,\mathcal{P}_{\mathcal{D}}^{n}[\phi,\alpha][s]\wedge\mathcal{R}_{\mathcal{D}}[\alpha(a,s)]\\
\rightarrow\mathcal{R}_{\mathcal{D}}[\mathcal{P}_{\mathcal{D}}^{n}[\phi,\alpha][do(a,s)]]\end{multline*}
 which by Proposition \ref{prop:a-order-reduction} implies the persistence
of $\mathcal{P}_{\mathcal{D}}^{n}[\phi,\alpha]$ under $\alpha$.
By Lemma \ref{lem:p-equiv-p(pn)} this implies the persistence of
$\phi$ under $\alpha$, which in turn implies $\mathcal{P_{D}}[\phi,\alpha]$,
giving:\[
\mathcal{D}-\mathcal{D}_{s_{0}}\models\forall s\,.\,\mathcal{P}_{\mathcal{D}}^{n}[\phi,\alpha][s]\rightarrow\mathcal{P_{D}}[\phi,\alpha][s]\]
 By Lemma \ref{lem:p-implies-pn} this implication is in fact an equivalence,
yielding equation (\ref{eqn:pn_equiv_persists}) as required.

The \emph{only if} direction is a straightforward reversal of this
reasoning: $\mathcal{P_{D}}[\phi,\alpha]$ implies the persistence
of $\phi$, which implies the persistence of $\mathcal{P}_{\mathcal{D}}^{n}[\phi,\alpha]$,
which implies (\ref{eqn:pn_persists}) by Proposition \ref{prop:a-order-reduction}. 
\end{proof}
Since $\mathcal{P}_{\mathcal{D}}^{n+1}[\phi,\alpha]\rightarrow\mathcal{P}_{\mathcal{D}}^{n}[\phi,\alpha]$
by definition, equation (\ref{eqn:pn_persists}) identifies $\mathcal{P}_{\mathcal{D}}^{n}[\phi,\alpha]$
as a fixed-point of the $\mathcal{P}_{\mathcal{D}}^{1}$ operator,
as our initial intuition suggested.


\section{Calculating $\mathcal{P}$}

Since we can easily calculate $\mathcal{P}_{\mathcal{D}}^{n}[\phi,\alpha]$
for any $n$, we have a straightforward algorithm for determining
$\mathcal{P_{D}}[\phi,\alpha]$: search for an $n$ such that\[
\mathcal{D}_{una}\models\forall s\,.\,\left(\mathcal{P}_{\mathcal{D}}^{n}[\phi,\alpha][s]\rightarrow\mathcal{P}_{\mathcal{D}}^{n+1}[\phi,\alpha][s]\right)\]
 Since we expect $\mathcal{P}_{\mathcal{D}}^{n}[\phi,\alpha]$ to
be simpler than $\mathcal{P}_{\mathcal{D}}^{n+1}[\phi,\alpha]$, we
should look for the smallest such $n$. Algorithm \ref{alg:calc_p}
presents an iterative procedure for doing just that.

%
\begin{algorithm}
\caption{Calculate $\mathcal{P}_{\mathcal{D}}[\phi,\alpha]$}


\label{alg:calc_p} \begin{algorithmic} \STATE $\mathtt{pn}\Leftarrow\phi$
\STATE $\mathtt{pn1}\Leftarrow\mathcal{P}_{\mathcal{D}}^{1}[\mathtt{pn},\alpha]$
\WHILE{$\mathcal{D}_{una}\not\models\forall s\,.\,\mathtt{pn}[s]\rightarrow\mathtt{pn1}[s]$}
\STATE $\mathtt{pn}\Leftarrow\mathtt{pn1}$ \STATE $\mathtt{pn1}\Leftarrow\mathcal{P}_{\mathcal{D}}^{1}[\mathtt{pn},\alpha]$
\ENDWHILE \STATE return $\mathtt{pn}$ \end{algorithmic} 
\end{algorithm}


Note that the calculation of $\mathcal{P}_{\mathcal{D}}^{1}[\phi,\alpha]$
is a straightforward syntactic transformation and so requires no further
treatment.


\subsection{Correctness}

If algorithm \ref{alg:calc_p} terminates, it terminates returning
a value of $pn$ for which equation (\ref{eqn:pn_persists}) holds.
By Theorem \ref{thm:p(pn)-equiv-p} this value of $pn$ is equivalent
to the persistence condition for $\phi$ under $\alpha$. The algorithm
therefore correctly calculates the persistence condition.

Note that equation (\ref{eqn:pn_persists}) holds when $\mathcal{P}_{\mathcal{D}}^{n}[\phi,\alpha]$
is unsatisfiable for any situation, as it appears in the antecedent
of the implication. The algorithm thus correctly returns an unsatisfiable
condition (equivalent to $false$) when $\phi$ can never persist
under $\alpha$.


\subsection{Completeness}

As Theorem \ref{thm:p(pn)-equiv-p} is an equivalence, the only source
of incompleteness will be failure to terminate. Algorithm \ref{alg:calc_p}
may fail to terminate for two reasons: the loop condition may never
be satisfied, or the first-order logical inference in the loop condition
may be undecidable and fail to terminate.

The later indicates that the basic action theory $\mathcal{D}$ is
undecidable. While this is a concern, it affects more than just our
algorithm - any system implemented around such an action theory will
be incomplete. Thus, with respect to this source of incompleteness,
our algorithm is no more incomplete than any larger system it would
form a part of.

The former is of more direct consequence to our work, and raises two
questions: is the persistence condition guaranteed to exist, and is
it guaranteed to be reachable in a finite number of iterations? Since
$\mathcal{P}_{\mathcal{D}}^{1}$ is clearly monotone, the constructive
proof of Tarski's fixed-point theorem \citep{cousot79constructive_tarski}
guarantees the existence of a fixed-point that can be calculated via
transfinite iteration, as performed by our algorithm. Moreover, this
will be the least fixed-point greater than $\phi$, a satisfying confirmation
of correctness since the persistence condition must be the weakest
precondition for the persistence of $\phi$.

Unfortunately, there is no guarantee that this fixed-point can be
reached via \emph{finite} iteration, which is required for termination.
Indeed, it is straightforward to construct a fluent for which the
algorithm never terminates: consider a fluent $Fp(x,s)$ taking integers
$x$, that is affected by a single action that makes it false whenever
$Fp(x+1,s)$ is false. Letting $\alpha$ be vacuously true, the sequence
of iterations produced by our algorithm would be:\begin{gather*}
\mathcal{P}_{\mathcal{D}}^{1}[Fp(x,s)]\equiv Fp(x,s)\wedge Fp(x+1,s)\\
\mathcal{P}_{\mathcal{D}}^{2}[Fp(x,s)]\equiv Fp(x,s)\wedge Fp(x+1,s)\wedge Fp(x+2,s)\\
\vdots\\
\mathcal{P}_{\mathcal{D}}^{n}[Fp(x,s)]\equiv\bigwedge_{i=0}^{i=n}Fp(x+i,s)\end{gather*}
 The persistence condition in this case is $\mathcal{P}_{\mathcal{D}}[Fp(x,s)]\equiv\forall y\,.\, x\leq y\rightarrow Fp(y,s)$,
and while this is the limit of the iteration it is clearly unachievable
in any finite number of steps.

$\mathcal{P}_{\mathcal{D}}^{1}$ operates over the set of equivalence
classes of formulae uniform in $s$, and the theory of fixed-points
requires that this set be a \emph{well-founded partial order} to guarantee
termination of an iterative approximation algorithm. There are certain
classes of basic action theory for which this well-foundedness can
be guaranteed. The most obvious is theories with finite action and
object domains, in which the set of equivalence classes is finite.
Another is the case where successor-state axioms refer only to the
direct arguments of their fluents, in which case repeated applications
of $\mathcal{P}_{\mathcal{D}}^{1}$ will pick out a subset of the
equivalence classes that refers to a finite number of objects, giving
the same guarantee.

In general, termination requires that repeated applications of the
successor-state axioms (via the regression operator) do not construct
infinite chains. We are currently investigating syntactic restrictions
on successor state axioms that can enforce this requirement, and have
found it to be easily met in practice. We are also investigating more
advanced fixed-point algorithms that may give better efficiency and
termination guarantees.


\section{Discussion}


\subsection{Effectiveness}

Our algorithm replaces a single reasoning task based on the full action
theory $\mathcal{D}$ with a series of reasoning tasks based on the
unique names axioms $\mathcal{D}_{una}$. Is this a worthwhile trade-off
in practice? The following points weigh strongly in favor of our approach:

First and foremost, we avoid the need for the second-order induction
axiom. All the reasoning tasks can be performed using standard first-order
reasoning, for which there are many high-quality automated provers.
Second, the calculation of $\mathcal{P_{D}}$ only reasons based on
the unique names axioms, which as discussed is a comparatively straightforward
task. Third, $\mathcal{P}_{\mathcal{D}}[\phi,\alpha][s]$ is in a
form amenable to regression, a standard tool for effective reasoning
in the situation calculus. Fourth, the persistence condition for a
given $\phi$ and $\alpha$ can be cached and re-used for a series
of related queries about different situations, a significant gain
in amortized efficiency. Finally, in realistic domains we expect many
properties to fail to persist beyond a few situations into the future,
meaning that our algorithm will require few iterations in a large
number of cases.

Of course, we also inherit the potential disadvantage of the regression
operator: the length of $\mathcal{P_{D}}[\phi,\alpha]$ may be exponential
in the length of $\phi$. As with regression, our experience thus
far has been that this is rarely a problem, and is more than compensated
for by the reduced number of axioms required for reasoning.


\subsection{Applications}

The persistence condition is readily applicable to the persistence
query problems given in Section \ref{sub:Property-Persistence}. Since
$\mathcal{P}_{\mathcal{D}}$ yields a uniform formula, the techniques
outlined in Section \ref{sub:Effective-Reasoning} can be directly
applied to the following transformed queries.

\textbf{Goal Impossibility:} Given a goal $G$, establish that there
is no possible situation in which that goal is satisfied:\[
\mathcal{D}\models\mathcal{P_{D}}[\neg G,Poss][S_{0}]\]
 The persistence condition of $\neg G$ with respect to action possibility
allows goal impossibility to be checked easily.

\textbf{Goal Futility:} Given a goal $G$ and situation $s$, establish
that the goal cannot be satisfied in any possible situation in the
future of $s$:\[
\mathcal{D}\models\mathcal{P_{D}}[\neg G,Poss][s]\]
 Precisely the same formula is required for checking goal impossibility
and goal futility. This highlights the advantage of re-using the persistence
condition at multiple situations. Our approach makes it feasible for
an agent to check for goal futility each time it considers performing
an action, and avoid situations that would make its goals unachievable.

\textbf{Checking State Constraints:} This can be handled as per goal
impossibility above. A state constraint $\phi$ should always satisfy:\[
\mathcal{D}_{una}\models\phi\equiv\mathcal{P}_{\mathcal{D}}[\phi,Poss]\]


If this is not the case then $\mathcal{P}_{\mathcal{D}}[\phi,Poss]$
indicates the additional conditions that are necessary to ensure that
$\phi$ persists, which may be useful in adjusting the action theory
to accommodate the constraint.

\textbf{Need for Cooperation:} To establish that it is absolutely
necessary to cooperate with another agent, an agent must determine
persistence with respect to actions performed by itself:\[
\mathcal{D}\models\mathcal{P_{D}}[\neg G,OwnAction][s]\]


\textbf{Knowledge under Partial Observability:} In recent unpublished
work, we have used the persistence condition to formulate a regression
procedure for knowledge formulae when actions may be unobservable.
To know $\phi$, the agent must know that $\phi$ will persist under
any actions that it would be unable to observe, resulting in a regression
rule like the following:\begin{multline*}
\mathcal{R}_{\mathcal{D}}[\mathbf{Knows}(\phi,do(a,s))]=\\
\mathbf{Knows}(\mathcal{R}_{\mathcal{D}}[\mathcal{P}_{\mathcal{D}}[\phi,CantObs][do(a,s)]],s)\end{multline*}



\subsection{Related Work}

While there is a rich and diverse literature base for the situation
calculus, there appears to have been little work dealing with universally
quantified queries. The work of \citet{Reiter93proving} shows how
to handle such queries manually using an appropriate instantiation
of the induction axiom, but makes no mention of automating this reasoning.

Other work considering persistence focuses exclusively on verifying
state constraints. These are uniform formulae that must hold in every
possible situation, a highly specialized form of persistence query.
The work of \citet{Lin94-StateConstraints} {}``compiles away''
the induction axiom when verifying a state constraint, by means of
the following equivalence:\begin{gather*}
\Sigma\cup\mathcal{D}_{una}\cup\mathcal{D}_{ss}\models\phi[S_{0}]\rightarrow\left(\forall s\,.\, S_{0}\leq s\rightarrow\phi[s]\right)\\
\mathit{iff}\\
\mathcal{D}_{una}\models\forall s,a\,.\,\phi[s]\wedge\mathcal{R}_{\mathcal{D}}[Poss(a,s)]\,\rightarrow\,\mathcal{R}_{\mathcal{D}}[\phi[do(a,s)]]\end{gather*}
 Verification of a state constraint can thus be reduced to reasoning
about a universally quantified uniform formula using only the unique
names axioms, a comparatively straightforward reasoning task. This
problem is also approached by \citet{bertossi96automating}, who develop
a system for automatically verifying state constraints based on an
induction theorem prover.

However, there are many issues related to persistence that are not
addressed by such work, including: persistence at situations other
than $S_{0}$; action description predicates other than $Poss$; how
to combine notions of persistence and regression; and determining
what additional conditions may be necessary to guarantee the persistence
of $\phi$. As our treatment of persistence can provide a concrete
basis for these considerations, it is significantly more general than
existing work.


\subsection{Conclusions}

We have developed an algorithm that transforms persistence queries,
a very general and useful class of situation calculus query, to a
form that is amenable to standard techniques for effective reasoning
in the situation calculus. The algorithm is based on iterative application
of the standard regression operator, and uses only first-order reasoning
with a small number of axioms. Our algorithm is shown to be correct,
and complete given some basic restrictions on the theory of action.

Our approach generalizes previous work on persistence in several important
ways. It can consider sequences of actions satisfying a range of conditions,
not just the standard ordering over possibility, enabling us to treat
problems such as need for cooperation and knowledge under partial
observability. It can establish that properties persist in the future
of an arbitrary situation, not necessarily the initial situation,
enabling us to answer the question of goal futility. The results of
calculating the persistence condition can be cached, allowing for
example the goal futility question to be efficiently posed on a large
number of situations once the persistence condition has been calculated.
Finally, the fact that $\mathcal{P_{D}}$ operates similarly to the
standard regression operator $\mathcal{R}_{\mathcal{D}}$ allows them
to interact in non-trivial ways, such as in the regression of knowledge
formulae under partial observability of actions.

As a result, we have significantly increased the scope of queries
that can be effectively posed in practical systems built on the situation
calculus.

