

\chapter{Property Persistence}

\label{ch:persistence}

This chapter develops a new inductive reasoning technique for the
situation calculus that can handle certain types of universally-quantified
query. As discussed in Chapter \ref{ch:observations}, for an agent
in an asynchronous domain to reason about the world based on its local
information, it needs to pose queries that universally quantify over
situation terms. Unfortunately such queries cannot be handled using
the regression operator, and have thus far been beyond the reach of
automated reasoning systems for the situation calculus.

We study a restricted subset of universally-quantified queries that
we refer to as \emph{property persistence queries}, introducing an
approach to reasoning about them that is similar in spirit to the
standard regression operator: transform the query into a form more
amenable to automated reasoning. A new meta-operator $\Pst_{\Dt}$
is defined such that $\phi$ persists in $s$ if and only if $\Pst_{\Dt}(\phi)$
holds in $s$. We term the formula generated by this operator the
\emph{persistence condition} of $\phi$.

The persistence condition is shown to be a fixpoint of applications
of the regression operator, which can be calculated using an iterative
approximation algorithm. The resulting formula can then be used in
combination with standard regression-based reasoning techniques, allowing
the inductive component of the reasoning to be {}``factored out''
and approached using a special-purpose reasoning algorithm. The technique
is always sound, and is complete in several interesting cases.

In Chapter \ref{ch:observations} we identified a universally-quantified
query that must be answered by an agent to reason about its own world
based on its local view. This query is \emph{not }in a form that that
can be handled directly using the persistence condition. However,
Chapter \ref{ch:knowledge} will demonstrate how to combine the techniques
developed in this chapter with a new formalism for explicit epistemic
reasoning, and will allowing an agent to reason effectively about
its own knowledge using a combination of regression and property persistence.

The chapter proceeds as follows: after some more detailed background
material on inductive reasoning in the situation calculus in Section
\ref{sec:Persistence:Background}, we formally define the class of
property persistence queries in Section \ref{sec:Persistence:Definitions},
along with several examples of practical queries that are of this
form. Section \ref{sec:Persistence:Condition} defines the persistence
condition operator and demonstrates that it is equivalent to the result
of a meta-level fixpoint calculation. Section \ref{sec:Persistence:Calculating}
presents a simple iterative algorithm for calculating the persistence
condition, and discusses its correctness, completeness, and effectiveness.
We conclude with some general discussion in Section \ref{sec:Persistence:Discussion}.


\section{Background\label{sec:Persistence:Background}}

While there is a rich and diverse literature base for the situation
calculus, there appears to have been little work on reasoning about
universally quantified queries. The work of \citet{Reiter93proving}
shows how to handle such queries manually using an appropriate instantiation
of the second-order induction axiom, but makes no mention of automating
this reasoning.

Other work considering queries that universally quantify over situations
focuses exclusively on verifying state constraints. These are uniform
formulae that must hold in every possible situation, a highly specialised
form of the more general persistence queries we define in this chapter.
The work of \citet{Lin94-StateConstraints} shows that the induction
axiom can be {}``compiled away'' when verifying a state constraint,
by means of the following equivalence:\begin{gather*}
\Dt\,\models\,\phi[S_{0}]\rightarrow\left(\forall s:\, S_{0}\leq s\rightarrow\phi[s]\right)\\
\mathit{iff}\\
\Dt_{una}\models\forall s,a:\,\phi[s]\wedge\Reg_{\Dt}(Poss(a,s))\,\rightarrow\,\Reg_{\Dt}(\phi[do(a,s)])\end{gather*}
 The set $\Dt_{una}$ here performs the same role as our background
axioms $\Dt_{bg}$ but contains only the unique names axioms for actions.
Verification of a state constraint can thus be reduced to reasoning
about a universally quantified uniform formula using only the static
background theory, a comparatively straightforward reasoning task
which we call \emph{static domain reasoning}. Verification of state
constraints was also approached by \citet{bertossi96automating},
who develop an automatic constraint verification system using an induction
theorem prover.

However, there are many issues that are not addressed by work specific
to state constraints. What if we are interested in the future of some
arbitrary situation $\sigma$, rather than only $S_{0}$? What if
want to restrict future actions according to an arbitrary action description
predicate? Can we integrate a method for handling universally-quantified
queries with existing regression techniques? Our treatment of property
persistence can provide a concrete basis for these considerations,
and is hence significantly more general than this existing work.

Another field that deals with induction over situations is the verification
of ConGolog programs. \citet{ternovska97congolog_fixpoint} show how
to formulate various safety, liveness, and starvation properties of
a ConGolog program as fixpoint queries in second-order logic. A preliminary
model-checker capable of verifying these properties is described in
\citep{ternovska02congolog_model_checker}. \citet{classen08golog_properties}
develop a logic of ConGolog programs in $\mathcal{ES}$, a variant
of the situation calculus based on modal logic. They demonstrate that
properties of a program can be verified using an iterative fixpoint
computation similar to the one we propose in this chapter.

As we shall see, property persistence queries are equivalent to a
particular kind of safety property of a ConGolog program, so our work
is in some ways less general than that described above. This means,
however, that we can be more specific in our algorithm and approach.
These ConGolog verifiers are designed to operate in isolation, while
we seek a method of handling universally-quantified queries that can
integrate directly with the existing meta-theoretical reasoning machinery
of the situation calculus, in particular with the regression operator.

Finally, let us introduce an important property of situations first
formally identified by \citet{savelli06sc_quantum_levels}: that universal
quantification over situation terms is equivalent to an infinite conjunction
over the \emph{levels} of the tree of situations:\begin{gather*}
\Dt\,\models\,\forall s:\,\psi(s)\\
\mathrm{iff}\\
\Dt\,\models\,\bigwedge_{n\in\mathbb{N}}\forall a_{1},\dots,a_{n}:\,\psi(do([a_{1},\dots,a_{n}],S_{0}))\end{gather*}


This is a direct consequence of the induction axiom for situations,
which restricts situations to be constructed by performing some countable
number of actions in the initial situation. This property also guarantees
that for any existential statement about situations, we can pick a
\emph{named witness} to that statement which is of the form $do([a_{1},\dots a_{n}],S_{0})$
for some $n\in\mathbb{N}$.

While we do not use this result directly in this chapter, it captures
an important intuition about situation terms that is fundamental to
the operation of our approach.


\section{Property Persistence Queries\label{sec:Persistence:Definitions}}

Let us now formally define the kinds of query that will be approached
in this chapter. Given some property $\phi$ and situation $\sigma$,
a \emph{property persistence query} asks whether $\phi$ will hold
in all situations in the future of $\sigma$: \[
\Dt\models\forall s:\,\sigma\sqsubseteq s\,\rightarrow\,\phi[s]\]


More generally, one may wish to limit the futures under consideration
to those brought about by actions satisfying a certain action description
predicate $\alpha$, which is easily accomplished using the $\leq_{\alpha}$
macro. We thus have the following definition of a persistence query:

\begin{defnL}
[{Property~Persistence~Query}] Let $\phi$ be a uniform
formula, $\alpha$ an action description predicate, and $\sigma$
a situation term. Then a property persistence query is a query of
the form:\[
\Dt\models\forall s:\,\sigma\le_{\alpha}s\rightarrow\phi[s]\]

\end{defnL}
In words, a persistence query states that {}``$\phi$ holds in $\sigma$,
and assuming all subsequent actions satisfy $\alpha$, $\phi$ will
continue to hold''. For succinctness we will henceforth describe
this as {}``$\phi$ persists under $\alpha$''. Queries of this
form are involved in many useful reasoning tasks, of which the following
are a small selection:\\


\textbf{Goal Impossibility:} Given a goal $G$, establish that there
is no legal situation in which that goal is achieved:\[
\Dt\models\forall s:\, S_{0}\leq_{Legal}s\rightarrow\neg G(s)\]


\textbf{Goal Futility:} Given a goal $G$ and situation $\sigma$,
establish that the goal cannot be achieved in any legal future of
$\sigma$:\[
\Dt\models\forall s:\,\sigma\leq_{Legal}s\rightarrow\neg G(s)\]


Note how this differs from goal impossibility: while the agent may
have initially been able to achieve its goal, the actions that have
subsequently been performed have rendered the goal unachievable. Agents
would be well advised to avoid such situations.\\


\textbf{Checking State Constraints:} Given a state constraint $SC$,
show that the constraint holds in every legal situation:\[
\Dt\models\forall s:\, S_{0}\leq_{Legal}s\rightarrow SC(s)\]
 This can be seen as a variant of goal impossibility, by showing that
the constraint can never be violated.\\


\textbf{Need for Cooperation:} Given an agent $agt$, goal $G$ and
situation $\sigma$, establish that no sequence of actions performed
by that agent can achieve the goal. Suppose we define $MyAction$
to identify the agent's own actions:\[
MyAction(a,s)\,\isdef\, actor(a)=agt\]


Then the appropriate query is:\[
\Dt\models\forall s:\,\sigma\leq_{MyAction}s\rightarrow\neg G(s)\]
 If this is the case, the agent will need to seek cooperation from
another agent in order to achieve its goal.\\


\textbf{Knowledge with Hidden Actions:} An agent reasoning about its
own knowledge in asynchronous domains must account for arbitrarily-long
sequences of hidden actions. To establish that it knows $\phi$, it
must establish that $\phi$ cannot become false through a sequence
of hidden actions:\[
\Dt\models\forall s:\,\sigma\leq_{Hidden}s\rightarrow\phi[s]\]


This last case is our main motivation for the developments in this
chapter, and we will explore the use of property persistence in this
context in detail in Chapter \ref{ch:knowledge}. The other examples
are designed to show that persistence queries are quite a general
form of query, and the techniques developed in this chapter thus have
application beyond our specific use of them in the remainder of this
thesis.

Unfortunately, persistence queries do not meet the criteria for regressable
formulae found in Definition \ref{def:Background:Regressable-Formulae},
since they quantify over situation terms. Such queries therefore cannot
be handled using the standard regression operator. Indeed, since universal
quantification over situation terms requires the use of a second order
induction axiom, current systems needing to answer such queries must
resort to second-order theorem proving. This is hardly an attractive
prospect for effective automated reasoning.


\section{The Persistence Condition\label{sec:Persistence:Condition}}

To implement practical systems that can perform persistence queries,
we clearly need to transform the query into a form suitable for effective
automated reasoning. Our approach is to transform a property persistence
query at $\sigma$ into the evaluation of a uniform formula at $\sigma$.
This transformed query can then be handled effectively using the standard
regression operator.

To achieve this we need some transformation of a property $\phi$
and action description predicate $\alpha$ into a uniform formula
$\Pst_{\Dt}(\phi,\alpha)$ that is true at precisely the situations
in which $\phi$ persists under $\alpha$. We call such a formula
the \emph{persistence condition} of $\phi$ under $\alpha$.

\begin{defnL}
[{Persistence~Condition}] The persistence condition of $\phi$
under $\alpha$, denoted $\Pst_{\Dt}(\phi,\alpha)$, is a uniform
formula that is equivalent to the persistence of $\phi$ under $\alpha$
with respect to a basic action theory $\Dt$ without the initial situation
axioms. Formally:\label{def:persistence-condition}\[
\Dt-\Dt_{S_{0}}\models\forall s:\,\left(\Pst_{\Dt}(\phi,\alpha)[s]\,\equiv\,\forall s':\, s\leq_{\alpha}s'\rightarrow\phi[s']\right)\]

\end{defnL}
Defining $\Pst_{\Dt}$ to be independent of the initial world state
allows an agent to calculate it regardless of what (if anything) is
known about the actual state of the world -- after all, an agent may
not know all the details of $\Dt_{S_{0}}$, and we still want it to
be able to use this technique.

This definition alone clearly does not make the task of answering
a persistence query any easier, since it gives no indication of how
the persistence condition might be calculated in practice. Indeed,
we have not yet even shown whether such a formula actually exists.
In order to establish these results, we first need to define the weaker
notion of a formula \emph{persisting to depth $n$} in a situation.

Since we wish to establish our technique as a general reasoning mechanism
for the situation calculus, we drop the assumption that concurrent
actions are in use for the duration of this chapter. Note that nothing
in our definitions precludes the use of various situation calculus
extensions as described in Section \ref{sec:Background:Extensions}.

\begin{defnL}
[{Persistence~to~Depth~1}] A uniform formula $\phi$ persists
to depth 1 under $\alpha$ in situation $s$ when the formula $\Pst_{\Dt}^{1}(\phi,\alpha)[s]$
holds, as defined by:\label{def:persists-depth-n}\[
\Pst_{\Dt}^{1}(\phi,\alpha)\,\isdef\,\phi^{-1}\,\wedge\,\forall a:\,\Reg_{\Dt}(\alpha[a,s])^{-1}\rightarrow\Reg_{\Dt}(\phi[do(a,s)])^{-1}\]

\end{defnL}
Note that $\Pst_{\Dt}^{1}$ is a literal encoding of the requirement
{}``$\phi$ holds in $s$ and in all its direct successors'', using
the standard regression operator $\Reg_{\Dt}$ and the situation-suppression
operator $\phi^{-1}$ to produce a situation-suppressed uniform formula.
Without the use of regression, the definition would appear as follows:\[
\PstDI(\phi,\alpha)[s]\,\equiv\,\phi[s]\,\wedge\,\forall a:\,\alpha[a,s]\,\rightarrow\,\phi[do(a,s)]\]
 Since $\alpha$ is an action description predicate and $\phi$ is
a uniform formula, the expressions $\Reg_{\Dt}(\alpha[a,s])^{-1}$
and $\Reg(\phi[do(a,s)])^{-1}$ are always defined and always produce
uniform formulae. Successive applications of $\Pst_{\Dt}^{1}$ can
then assert persistence to greater depths:

\begin{defnL}
[{Persistence~to~Depth~N}] For any $n\geq0$, a uniform
formula $\phi$ persists to depth $n$ under $\alpha$ in situation
$s$ when the formula $\Pst_{\Dt}^{n}(\phi,\alpha)[s]$ holds, as
defined by:\begin{gather*}
\Pst_{\Dt}^{0}(\phi,\alpha)\,\isdef\,\phi\\
\Pst_{\Dt}^{n}(\phi,\alpha)\,\isdef\,\Pst_{\Dt}^{1}(\Pst_{D}^{n-1}(\phi,\alpha),\alpha)\end{gather*}

\end{defnL}
The following theorem confirms that $\Pst_{\Dt}^{n}$ operates according
to this intuition -- that for any sequence of actions of length $i=0$
to $i=n$, if each action satisfies $\alpha$ in the situation it
is executed in, then $\phi$ will hold after executing those actions.

\begin{thm}
For any $n\in\mathbb{N}$, $\Pst_{\Dt}^{n}(\phi,\alpha)$ holds in
$\sigma$ iff $\phi$ holds in $\sigma$ and in all successors of
$\sigma$ reached by performing at most $n$ actions satisfying $\alpha$:\label{thm:PstN-works}\begin{multline*}
\Dt\,\models\,\Pst_{\Dt}^{n}(\phi,\alpha)[\sigma]\,\equiv\,\\
\bigwedge_{i\leq n}\forall a_{1},\dots,a_{i}:\,\left(\bigwedge_{j\leq i}\alpha[a_{j},do([a_{1},\dots,a_{j-1}],\sigma)]\,\,\rightarrow\,\,\phi[do([a_{1},\dots,a_{i}],\sigma)]\right)\end{multline*}

\end{thm}
\begin{proofsketch}
By induction on the natural numbers. For $n=0$ we have $\phi[\sigma]\equiv\phi[\sigma]$
by definition. For the inductive case, we expand the definition of
$\Pst_{\Dt}^{n}(\phi,\alpha)[\sigma]$ to get the following for the
LHS:\[
\Pst_{\Dt}^{n-1}(\phi,\alpha)[\sigma]\wedge\forall a:\,\Reg_{\Dt}(\alpha[a,\sigma])\rightarrow\Reg_{\Dt}(\Pst_{\Dt}^{n-1}(\phi,\alpha)[do(a,\sigma)])\]


Substituting for $\Pst_{\Dt}^{n-1}$ using the inductive hypothesis
gives us a conjunction ranging over $i\leq n-1$, with universally
quantified variables $a_{1},\dots,a_{i}$, and we must establish the
$i=n$ case. Pushing this conjunction inside the scope of the $\forall a$
quantifier, we find we can rename $a\Rightarrow a_{1}$, $a_{1}\Rightarrow a_{2}$
etc to get the required expression. 
\end{proofsketch}
The $\Pst_{\Dt}^{n}$ operator thus allows us to express the persistence
of a formula to any given depth using a simple syntactic translation
based on regression. Intuitively, one would expect $\Pst_{\Dt}(\phi,\alpha)$
to be some sort of fixpoint of $\Pst_{\Dt}^{1}(\phi,\alpha)$, since
$\Pst_{\Dt}(\phi,\alpha)$ must imply persistence up to any depth.
Such a fixpoint could then be calculated using standard iterative
approximation techniques. The remainder of this section is devoted
to verifying this intuition.

We begin by adapting two existing results involving induction from
the situation calculus literature, so that they operate with our generalised
$\leq_{\alpha}$ notation and can be based at situations other than
$S_{0}$.

\begin{prop}
For any action description predicate $\alpha$, the foundational axioms
of the situation calculus entail the following induction principle:\label{prop:a-order-induction}\begin{multline*}
\forall W,s:\,\, W(s)\wedge\left[\forall a,s':\,\alpha[a,s']\wedge s\leq_{\alpha}s'\wedge W(s')\rightarrow W(do(a,s'))\right]\\
\rightarrow\forall s':\, s\leq_{\alpha}s'\rightarrow W(s')\end{multline*}

\end{prop}
\begin{proof}
A trivial adaptation of Theorem 1 in \citep{Reiter93proving}. 
\end{proof}
\begin{prop}
For any basic action theory $\Dt$, uniform formula $\phi$ and action
description predicate $\alpha$:\label{prop:a-order-reduction}\begin{gather*}
\Dt-\Dt_{S_{0}}\models\forall s:\,\phi[s]\rightarrow\left(\forall s':\, s\leq_{\alpha}s'\rightarrow\phi[s']\right)\\
\mathrm{iff}\\
\Dt_{bg}\models\forall s,a:\,\phi[s]\wedge\Reg_{\Dt}(\alpha[a,s])\rightarrow\Reg_{\Dt}(\phi[do(a,s)])\end{gather*}

\end{prop}
\begin{proof}
A straightforward generalisation of the model-construction proof of
Lemma 5 in \citep{Lin94-StateConstraints}, utilising Proposition
\ref{prop:a-order-induction}. 
\end{proof}
Proposition \ref{prop:a-order-reduction} will be key in our algorithm
for calculating the persistence condition. It allows one to establish
the result {}``if $\phi$ holds in $s$, then $\phi$ persists in
$s$'' by using static domain reasoning, a comparatively straightforward
reasoning task.

We next formalise some basic relationships between $\Pst_{\Dt}$ and
$\Pst_{\Dt}^{n}$.

\begin{lemma}
Given a basic action theory $\Dt$, uniform formula $\phi$ and action
description predicate $\alpha$, then for any $n$:\label{lem:p-equiv-p(pn)}\[
\Dt-\Dt_{S_{0}}\models\forall s:\,\left(\forall s':\, s\leq_{\alpha}s'\rightarrow\phi[s']\right)\,\equiv\,\left(\forall s':\, s\leq_{\alpha}s'\rightarrow\Pst_{D}^{n}(\phi,\alpha)[s']\right)\]
 That is, $\phi$ persists under $\alpha$ iff $\,\Pst_{\Dt}^{n}[\phi,\alpha]$
persists under $\alpha$. 
\end{lemma}
\begin{proof}
Since $\Pst_{\Dt}^{n}[\phi,\alpha]$ implies $\phi$ by definition,
the \emph{if} direction is trivial. For the \emph{only-if} direction
we proceed by induction on $n$.

For the base case, assume that $\phi$ persists but $\Pst_{\Dt}^{1}(\phi,\alpha)$
does not, then we must have some $s'$ with $s\leq_{\alpha}s'$ and
$\neg\Pst_{\Dt}^{1}(\phi,\alpha)[s']$. By the definition of $\Pst_{\Dt}^{1}$,
this means that:\[
\neg\left(\phi[s']\,\wedge\,\forall a:\,\alpha[a,s']\rightarrow\phi[do(a,s')]\right)\]


Since $\phi$ persists it must hold at $s'$, so there must be some
$a$ such that $\alpha[a,s']$ and $\neg\phi[do(a,s')]$. But $s\leq_{\alpha}do(a,s')$
and so $\phi[do(a,s')]$ must hold by our assumption that $\phi$
persists, and we have a contradiction.

For the inductive case, assume that $\Pst_{\Dt}^{n-1}(\phi,\alpha)$
persists but $\Pst_{\Dt}^{n}(\phi,\alpha)$ does not. By definition
we have $\Pst_{\Dt}^{n}(\phi,\alpha)=\Pst_{\Dt}^{1}(\Pst_{\Dt}^{n-1}(\phi,\alpha),\phi)$,
and we repeat the base case proof using $\phi'=\Pst_{\Dt}^{n-1}(\phi,\alpha)$
in place of $\phi$ to obtain a contradiction. 
\end{proof}
\begin{lemma}
Given a basic action theory $\Dt$, uniform formula $\phi$ and action
description predicate $\alpha$, then for any $n$:\label{lem:p-implies-pn}\[
\Dt-\Dt_{S_{0}}\models\forall s:\,\left(\Pst_{\Dt}(\phi,\alpha)[s]\rightarrow\Pst_{\Dt}^{n}(\phi,\alpha)[s]\right)\]

\end{lemma}
\begin{proof}
$\Pst_{\Dt}(\phi,\alpha)$ implies the persistence of $\phi$ by definition.
If $\phi$ persists at $s$, then by Lemma \ref{lem:p-equiv-p(pn)}
we have that $\Pst_{\Dt}^{n}(\phi,\alpha)$ persists at $s$ . Since
the persistence of $\Pst_{\Dt}^{n}(\phi,\alpha)$ at $s$ implies
that $\Pst_{\Dt}^{n}(\phi,\alpha)$ holds at $s$ by definition, we
have the lemma as desired. 
\end{proof}
We are now equipped to prove the major theorem of this chapter: that
if $\Pst_{\Dt}^{n}(\phi,\alpha)$ implies $\Pst_{\Dt}^{n+1}(\phi,\alpha)$,
then $\Pst_{\Dt}^{n}(\phi,\alpha)$ is the persistence condition for
$\phi$ under $\alpha$.

\begin{thm}
Given a basic action theory $\Dt$, uniform formula $\phi$ and action
description predicate $\alpha$, then for any $n$:\label{thm:p(pn)-equiv-p}\begin{gather}
\Dt_{bg}\models\forall s:\,\Pst_{\Dt}^{n}(\phi,\alpha)[s]\rightarrow\Pst_{\Dt}^{n+1}(\phi,\alpha)[s]\label{eqn:pn_persists}\\
\mathit{iff}\nonumber \\
\Dt-\Dt_{s_{0}}\models\forall s:\,\Pst_{\Dt}^{n}(\phi,\alpha)[s]\equiv\Pst_{\Dt}(\phi,\alpha)[s]\label{eqn:pn_equiv_persists}\end{gather}

\end{thm}
\begin{proof}
For the \emph{if} direction, we begin by expanding equation \eqref{eqn:pn_persists}
using the definition of $\Pst_{\Dt}^{1}$ to get the equivalent form:\begin{gather*}
\Dt_{bg}\models\forall s:\,\Pst_{\Dt}^{n}(\phi,\alpha)[s]\rightarrow\Pst_{\Dt}^{1}(\Pst_{\Dt}^{n}(\phi,\alpha),\alpha)[s]\\
\Dt_{bg}\models\forall s:\,\Pst_{\Dt}^{n}(\phi,\alpha)[s]\rightarrow\left(\Pst_{\Dt}^{n}(\phi,\alpha)[s]\wedge\forall a:\,\Reg_{\Dt}(\alpha[a,s])\rightarrow\Reg_{\Dt}(\phi[do(a,s)])\right)\\
\Dt_{bg}\models\forall s,a:\,\Pst_{\Dt}^{n}(\phi,\alpha)[s]\wedge\forall a:\,\Reg_{\Dt}(\alpha[a,s])\rightarrow\Reg_{\Dt}(\phi[do(a,s)])\end{gather*}
 By Proposition \ref{prop:a-order-reduction}, equation \eqref{eqn:pn_persists}
thus lets us conclude that $\Pst_{\Dt}^{n}(\phi,\alpha)$ persists
under $\alpha$. By Lemma \ref{lem:p-equiv-p(pn)} this is equivalent
to the persistence of $\phi$ under $\alpha$, which is equivalent
to $\Pst_{\Dt}(\phi,\alpha)$ by definition, giving:\[
\Dt-\Dt_{s_{0}}\models\forall s:\,\Pst_{\Dt}^{n}(\phi,\alpha)[s]\rightarrow\Pst_{\Dt}(\phi,\alpha)[s]\]
 By Lemma \ref{lem:p-implies-pn} this implication is an equivalence,
yielding equation (\ref{eqn:pn_equiv_persists}) as required.

The \emph{only if} direction is a straightforward reversal of this
reasoning: $\Pst_{\Dt}(\phi,\alpha)$ implies the persistence of $\phi$,
which implies the persistence of $\Pst_{\Dt}^{n}(\phi,\alpha)$, which
yields equation (\ref{eqn:pn_persists}) by Proposition \ref{prop:a-order-reduction}. 
\end{proof}
Since $\Dt_{bg}\models\Pst_{\Dt}^{n+1}(\phi,\alpha)\rightarrow\Pst_{\Dt}^{n}(\phi,\alpha)$
by definition, equation (\ref{eqn:pn_persists}) identifies $\Pst_{\Dt}^{n}(\phi,\alpha)$
as a fixpoint of the $\Pst_{\Dt}^{1}$ operator, as our initial intuition
suggested. In fact, we can use the constructive proof of Tarski's
fixpoint theorem \citep{cousot79constructive_tarski} to establish
that the persistence condition always exists for a given $\phi$ and
$\alpha$.

\begin{thm}
Given a uniform formula $\phi$ and action description predicate $\alpha$,
the persistence condition $\Pst_{\Dt}(\phi,\alpha)$ always exists,
and is unique up to equivalence under the static background theory
$\Dt_{bg}$. \label{thm:p-always-exists} 
\end{thm}
\begin{proof}
Let $L$ be the subset of the Lindenbaum algebra of the static background
theory $\Dt_{bg}$ containing only sentences uniform in $s$. $L$
is thus a boolean lattice in which each element is a set of sentences
uniform in $s$ that are equivalent under $\Dt_{bg}$. $L$ is a complete
lattice with minimal element the equivalence class of $\bot$ and
maximal element the equivalence class of $\top$. Fixing $\alpha$,
$\PstDI$ is a function whose domain and range are the elements of
$L$.

By definition, we have that $\PstDI(\phi,\alpha)\,\rightarrow\,\phi$,
and $\PstDI$ is thus a \emph{monotone decreasing} function over $L$.
By the constructive proof of Tarski's fixpoint theorem, $\PstDI$
must have a unique greatest fixpoint less than the equivalence class
of $\phi$, which can be determined by transfinite iteration of the
application of $\PstDI$. By Theorem \ref{thm:p(pn)-equiv-p}, this
fixpoint is the equivalence class of $\Pst_{\Dt}(\phi,\alpha)$ under
$\Dt_{bg}$. 
\end{proof}
This theorem legitimates the use of the persistence condition for
reasoning about property persistence queries -- for any persistence
query at situation $\sigma$, there is a unique (up to equivalence)
corresponding query that is uniform in $\sigma$ and is thus amenable
to standard effective reasoning techniques of the situation calculus.

Of course, it remains to actually calculate the persistence condition
for a given $\phi$ and $\alpha$. The definition of $\PstD(\phi,\alpha)$
as a fixpoint suggests that it can be calculated by iterative approximation,
which we discuss in the next section.


\section{Calculating $\PstD$\label{sec:Persistence:Calculating}}

Since we can easily calculate $\Pst_{\Dt}^{n}(\phi,\alpha)$ for any
$n$, we have a straightforward algorithm for determining $\Pst_{\Dt}(\phi,\alpha)$:
search for an $n$ such that\[
\Dt_{bg}\models\forall s:\,\left(\Pst_{\Dt}^{n}(\phi,\alpha)[s]\rightarrow\Pst_{\Dt}^{n+1}(\phi,\alpha)[s]\right)\]
 Since we expect $\Pst_{\Dt}^{n}(\phi,\alpha)$ to be simpler than
$\Pst_{\Dt}^{n+1}(\phi,\alpha)$, we should look for the smallest
such $n$. Algorithm \ref{alg:calc_p} presents an iterative procedure
for doing just that.

%
\begin{algorithm}
\caption{Calculate $\Pst_{\Dt}(\phi,\alpha)$}


\label{alg:calc_p} \begin{algorithmic} \STATE $\mathtt{pn}\Leftarrow\phi$
\STATE $\mathtt{pn1}\Leftarrow\Pst_{\Dt}^{1}(\mathtt{pn},\alpha)$
\WHILE{$\Dt_{bg}\not\models\forall s:\,\mathtt{pn}[s]\rightarrow\mathtt{pn1}[s]$}
\STATE $\mathtt{pn}\Leftarrow\mathtt{pn1}$ \STATE $\mathtt{pn1}\Leftarrow\Pst_{\Dt}^{1}[\mathtt{pn},\alpha]$
\ENDWHILE \RETURN $\mathtt{pn}$ \end{algorithmic} 
\end{algorithm}


Note that the calculation of $\Pst_{\Dt}^{1}(\phi,\alpha)$ is a straightforward
syntactic transformation, so we do not present an algorithm for it.


\subsection{Correctness}

If Algorithm \ref{alg:calc_p} terminates, it terminates returning
a value of $pn$ for which equation (\ref{eqn:pn_persists}) holds.
By Theorem \ref{thm:p(pn)-equiv-p} this value of $pn$ is equivalent
to the persistence condition for $\phi$ under $\alpha$. The algorithm
therefore correctly calculates the persistence condition.

In particular, note that equation (\ref{eqn:pn_persists}) holds when
$\Pst_{\Dt}^{n}(\phi,\alpha)$ is unsatisfiable for any situation,
as it appears in the antecedent of an implication. The algorithm thus
correctly returns an unsatisfiable condition (equivalent to $\bot$)
when $\phi$ can never persist under $\alpha$.


\subsection{Completeness}

Since Theorem \ref{thm:p(pn)-equiv-p} is an equivalence, the persistence
condition is always the fixpoint of $\PstDI$. From Theorem \ref{thm:p-always-exists}
this fixpoint always exists and can be calculated by transfinite iteration.
Therefore, the only source of incompleteness in our algorithm will
be failure to terminate. Algorithm \ref{alg:calc_p} may fail to terminate
for two reasons: the loop condition may never be satisfied, or the
first-order logical inference in the loop condition may be undecidable
and fail to terminate.

The later case indicates that the basic action theory $\Dt$ is undecidable.
While this is a concern, it affects more than just our algorithm --
any system implemented around such an action theory will be incomplete.
With respect to this source of incompleteness, our algorithm is no
more incomplete than any larger reasoning system it would form a part
of.

The former case is of more direct consequence to our work. Unfortunately,
there is no guarantee in general that the fixpoint can be reached
via \emph{finite} iteration, which is required for termination of
Algorithm \ref{alg:calc_p}.

Indeed, it is straightforward to construct a fluent for which the
algorithm never terminates: consider a fluent $F(x,s)$ that is affected
by a single action that makes it false whenever $F(suc(x),s)$ is
false. Letting $\alpha$ be vacuously true, the sequence of iterations
produced by our algorithm would be:\begin{gather*}
\Pst_{\Dt}^{1}(F(x,s))\equiv F(x,s)\wedge F(suc(x),s)\\
\Pst_{\Dt}^{2}(F(x,s))\equiv F(x,s)\wedge F(suc(x),s)\wedge F(suc(suc(x)),s)\\
\vdots\\
\Pst_{\Dt}^{n}(F(x,s))\equiv\bigwedge_{i=0}^{i=n}F(suc^{i}(x),s)\end{gather*}
 The persistence condition in this case is clearly: \[
\Pst_{\Dt}(F(x,s))\equiv\forall y:\, x\leq y\rightarrow F(y,s)\]
 While this is equivalent to the infinite conjunction produced as
the limit of iteration in our algorithm, it will not be found after
any finite number of steps.

As discussed in the proof of Theorem \ref{thm:p-always-exists}, $\Pst_{\Dt}^{1}$
operates over the boolean lattice of equivalence classes of formulae
uniform in $s$, and the theory of fixpoints requires that this lattice
be \emph{well-founded} to guarantee termination of an iterative approximation
algorithm such as Algorithm \ref{alg:calc_p}. We must therefore identify
restricted kinds of basic action theory for which this well-foundedness
can be guaranteed.

The most obvious case is theories in which the action and object sorts
are finite. In such theories the lattice of equivalence classes of
formulae uniform in $s$ is finite, and any finite lattice is well-founded.
These theories also have the advantage that the static domain reasoning
performed by Algorithm \ref{alg:calc_p} can be done using propositional
logic, meaning it is decidable and so providing a strong termination
guarantee.

Alternately, suppose all successor state axioms and action description
predicates have the following restricted form, where the terms in
$\vars{y}$ are a subset of the terms in $\vars{x}$ and $\Phi_{F}$,
$\Pi_{ADP}$ mention no terms other than $\vars{x}$, $a$ and $s$:

\begin{gather*}
F(\vars{x},do(a,s))\,\equiv\,\bigwedge_{i=1}^{n}a=a_{i}(\vars{y}_{i})\wedge\Phi_{F}(\vars{x},a,s)\\
ADP(\vars{x},a,s)\,\equiv\,\bigwedge_{i=1}^{n}a=a_{i}(\vars{y})\wedge\Pi_{ADP}(\vars{x},a,s)\end{gather*}


Under such theories, applications of $\PstDI$ will introduce no new
terms into the query, apart from finitely many action terms $a_{i}$.
The range of $\PstDI$ applied to $\phi$ is then a finite subset
of the lattice of equivalence classes of formulae uniform in $s$,
again guaranteed well-foundedness and terminating calculation of $\Pst_{\Dt}$.

Of course this is a very strong restriction on the structure of the
theory, as the successor state axioms are not able to contain any
quantifiers. It does demonstrate, however, that certain syntactical
restrictions on $\Dt$ are able to guarantee terminating calculation
of $\Pst_{\Dt}$. It seems there should be a more general {}``syntactic
well-foundedness'' restriction that can be applied to these axioms,
but we have not successfully formulated one at this stage.

In a similar vein, suppose that the theory of action is \emph{context
free} \citep{reiter97progression}. In such theories successor state
axioms have the following form:\[
F(\vars{x},do(a,s))\,\equiv\,\Phi_{F}^{+}(\vars{x},a)\,\vee\,\left(F(\vars{x},s)\,\wedge\,\neg\Phi_{F}^{-}(\vars{x},a)\right)\]


The effects of an action are thus independent of the situation it
is performed in. \citet{levesque98what_robots_can_do} demonstrate
that such theories have a finite state space, again ensuring our algorithm
operates over a finite lattice and hence guaranteeing termination.
Context free domains are surprisingly expressive; for example, domains
described in the style of STRIPS operators are context free.

From a slightly different perspective, suppose that $\phi$ can never
persist under $\alpha$, so that $\PstD(\phi,\alpha)\equiv\bot$.
Further suppose that $\Dt$ has the \emph{compactness} property as
in standard first-order logic. Then the {}``quantum levels'' of
\citet{savelli06sc_quantum_levels} guarantee that there is a fixed,
finite number of actions within which $\neg\phi$ can always be achieved.
In this case Algorithm \ref{alg:calc_p} will determine $\PstD(\phi,\alpha)\equiv\bot$
within finitely many iterations.

It would also be interesting to determine whether known decidable
variants of the situation calculus (such as \citep{yu07twovar_sitcalc})
are able to guarantee termination of the fixpoint construction, or
whether more sophisticated fixpoint algorithms can be applied instead
of simple iterative approximation. Investigating such algorithms would
be a promising avenue for future research.

The important point here is not that we can guarantee completeness
in general, but that we have precisely characterised the inductive
reasoning necessary to answer property persistence queries, and shown
that it can always be replaced by the evaluation of a uniform formula
at the situation in question.


\subsection{Effectiveness\label{sec:Persistence:Effectiveness}}

Our algorithm replaces a single reasoning task based on the full action
theory $\Dt$ with a series of reasoning tasks based on the static
background theory $\Dt_{bg}$. Is this a worthwhile trade-off in practice?
The following points weigh strongly in favour of our approach.

First and foremost, we avoid the need for the second-order induction
axiom. All the reasoning tasks can be performed using standard first-order
reasoning, for which there are high-quality automated provers. Second,
the calculation of $\Pst_{\Dt}$ performs only \emph{static doing
reasoning}, which as discussed in Chapter \ref{ch:background} is
a comparatively straightforward task which can be made decidable under
certain conditions. Third, $\Pst_{\Dt}(\phi,\alpha)[s]$ is in a form
amenable to regression, a standard tool for effective reasoning in
the situation calculus. Fourth, the persistence condition for a given
$\phi$ and $\alpha$ can be cached and re-used for a series of related
queries about different situations, a significant gain in amortised
efficiency. Finally, in realistic domains we expect many properties
to fail to persist beyond a few situations into the future, meaning
that our algorithm will require few iterations in a large number of
cases.

Of course, we also inherit the potential disadvantage of the regression
operator: the length of $\Pst_{\Dt}(\phi,\alpha)$ may be exponential
in the length of $\phi$. As with regression, our experience has been
that this is rarely a problem in practice, and is more than compensated
for by the reduced complexity of the resulting reasoning task.


\subsection{Applications}

The persistence condition is readily applicable to the example persistence
query problems given in Section \ref{sec:Persistence:Definitions}.
All of the transformed queries can then be answered using standard
regression.\\


\textbf{Goal Impossibility:} Given a goal $G$, establish that there
is no legal situation in which that goal is satisfied:\[
\Dt\,\models\,\Pst_{\Dt}(\neg G,Legal)[S_{0}]\]
 The persistence condition of $\neg G$ with respect to action legality
allows goal impossibility to be checked easily.\\


\textbf{Goal Futility:} Given a goal $G$ and situation $\sigma$,
establish that the goal cannot be satisfied in any legal future situation
from $\sigma$:\[
\Dt\,\models\,\Pst_{\Dt}(\neg G,Legal)[\sigma]\]
 Precisely the same formula is required for checking goal impossibility
and goal futility. This highlights the advantage of re-using the persistence
condition at multiple situations. Our approach makes it feasible for
an agent to check for goal futility each time it considers performing
an action, and avoid situations that would make its goals unachievable.\\


\textbf{Checking State Constraints:} Given a state constraint $SC$,
show that the constraint holds in every legal situation:\[
\Dt\,\models\,\Pst_{\Dt}(SC,Legal)[S_{0}]\]


However, since we want a state constraint to \emph{always} persist,
it must satisfy the following equivalence:

\[
\Dt_{bg}\models\phi\equiv\Pst_{\Dt}(\phi,Legal)\]


If this equivalence does not hold then $\Pst_{\Dt}(\phi,Legal)$ indicates
the additional conditions that are necessary to ensure that $\phi$
persists, which may be used to adjust the action theory to enforce
the constraint. This particular application has strong parallels to
the approach to state constraints developed by \citet{Lin94-StateConstraints}.\\


\textbf{Need for Cooperation:} Given an agent $agt$, goal $G$ and
situation $\sigma$, establish that no sequence of actions performed
by that agent can achieve the goal:\[
\Dt\models\Pst_{\Dt}(\neg G,MyAction)[\sigma]\]


\textbf{Knowledge with Hidden Actions:} In Chapter \ref{ch:knowledge}
we will develop a regression rule for knowledge that uses the persistence
condition to account for arbitrarily-long sequences of hidden actions.
While we defer the details to that chapter, the general form of the
rule is:\[
\Reg_{\Dt}(\Knows(\phi,do(a,s)))\isdef\Knows(\Reg_{\Dt}(\Pst_{\Dt}(\phi,Hidden),a),s)\]



\section{Discussion\label{sec:Persistence:Discussion}}

In this chapter we have developed an algorithm that transforms property
persistence queries, a very general and useful class of situation
calculus query, to a form that is amenable to standard techniques
for effective reasoning in the situation calculus. The algorithm replaces
a second-order induction axiom with a meta-level fixpoint calculation
based on iterative application of the standard regression operator.
It is shown to be correct, and also complete in some interesting cases.

Our approach generalises previous work on universally-quantified queries
in several important ways. It can consider sequences of actions satisfying
a range of conditions, not just the standard ordering over action
possibility, enabling us to treat problems such as need for cooperation
and knowledge with hidden actions. It can establish that properties
persist in the future of an arbitrary situation, not necessarily the
initial situation, enabling us to answer the question of goal futility.
The results of calculating the persistence condition can be cached,
allowing for example the goal futility question to be efficiently
posed on a large number of situations once the persistence condition
has been calculated.

Most importantly for the remainder of this thesis, we have \emph{factored
out} the inductive reasoning required to answer these queries. Work
on increasing the effectiveness of this inductive reasoning, and on
guaranteeing a terminating calculation in stronger classes of action
theory, can now proceed independently from the development of formalisms
that utilise persistence queries. We will henceforth use $\PstD$
as a kind of {}``black box'' operator to formulate regression rules
within our framework, dropping the explicit $\Dt$ subscript as we
do for the regression operator.

As noted in Section \ref{sec:Persistence:Background}, our use of
fixpoints in this chapter has much in common with the study of properties
of ConGolog programs by \citep{ternovska97congolog_fixpoint,classen08golog_properties}.
Indeed, a property persistence query is equivalent to a safety query
stating that the property $\phi$ never becomes false during execution
of the following program:\[
\delta_{P\alpha}\isdef\left(\pi(a,\,\alpha[a]?\,;\, a)\right)^{*}\]


Formally:\begin{gather*}
\Dt\,\models\,\forall s:\,\sigma\leq_{\alpha}s\,\rightarrow\,\phi[s]\\
\mathrm{iff}\\
\Dt\cup\Dt_{golog}\,\models\,\forall s,\delta:\, Trans^{*}(\delta_{P\alpha},\sigma,\delta,s)\,\rightarrow\,\phi[s]\end{gather*}


Since we intend to use persistence queries as part of a larger reasoning
apparatus, rather than as a stand-alone query, we cannot directly
leverage the existing work on verifying ConGolog programs. However,
given the similarity between the approaches, we are confident that
advances in reasoning effectively about ConGolog programs will also
advance our ability to effectively answer persistence queries.

This chapter has thus significantly increased the scope of queries
that can be posed when building systems upon the situation calculus.
In the coming chapters, the persistence condition operator will allow
us to factor out certain inductive aspects of reasoning, treating
them as separate, well-defined components of the overall reasoning
machinery.

