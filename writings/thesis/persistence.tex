\chapter{Property Persistence}\label{ch:persistence}
%\minitoc

This chapter will be an expansion of my conference paper on this topic \cite{kelly07sc_persistence}.

Remaining work: further characterization of completeness of the algorithm, ways to deal with interacting effects.


\section{The Persistence Problem}

\begin{itemize}
\item Formal definition
\item Examples of why it's important
\end{itemize}

\section{The Persistence Condition}

\begin{itemize}
\item Definition of $\mathcal{P}$, $\mathcal{P}^{1}$ operators
\item Proof that $\mathcal{P}$ is a least-fixed-point
\item Justification that it's an "effective" technique
\item Techniques for ensuring completness
\end{itemize}

\section{Calculating $\mathcal{P}$}

\begin{itemize}
\item Naive algorithm: definition, shortcomings
\item Algorithm based on explicit effect axioms
\item Dealing with interacting effects
\end{itemize}


