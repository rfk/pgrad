

\chapter{Conclusion}

\label{ch:conclusion}

This thesis has laid the foundations for reasoning about asynchronous
multi-agent domains in the situation calculus. As highlighted by our
initial investigations and implementation of the multi-agent Golog
variant MIndiGolog, the standard reasoning and planning machinery
of the situation calculus often depends on an assumption of synchronicity.
In many cases, synchronicity is enforced by requiring all actions
to be publicly observable.

We identified three main limitations of the situation calculus when
trying to extend its reach into asynchronous domains. It generates
fully-ordered sequences of actions as the output of the Golog execution
planning process, which cannot be feasibly executed in the face of
hidden actions. Its standard account of agent-level knowledge cannot
effectively handle arbitrarily-long sequences of hidden actions. Finally,
it lacks a formal account of reasoning about group-level epistemic
modalities such as common knowledge, which are vital for coordination
in multi-agent domains.

At the core of our approach to overcoming these limitations is a new,
explicit representation of the local perspective of each agent. By
formalising what each agent \emph{observes} when a particular set
of actions is performed, and its corresponding local \emph{view} in
each situation, we are able to approach reasoning and planning in
a principled way without making any assumptions about the dynamics
of the domain. In particular, we can explicitly define and represent
asynchronous domains as those in which some action occurrences generate
no observations; in other words, domains in which agents cannot determine
how many actions have been performed.

Building on this foundation, we have developed four key extensions
to the reasoning and planning machinery of the situation calculus
that work to overcome its current limitations in asynchronous multi-agent
domains.


\section{Contributions}

Our first key contribution defines a partially-ordered branching action
structure to replace raw situation terms as the output of the Golog
execution planning process. Called \emph{joint executions}, these
structures allow independent actions to be performed independently,
while ensuring that inter-agent synchronisation is always possible
when required. By formulating these requirements explicitly in terms
of the local view available to each agent, we identify joint executions
that are feasible to perform in the world despite potential asynchronicity
in the domain.

The utility of these structures was demonstrated by implementing an
offline execution planner that produces joint executions as its output.
By imposing some simple restrictions on the theory of action, the
planner is able to reason about joint executions without having to
explicitly consider the exponentially-many possible histories of such
a partially-ordered structure. It thus uses standard regression techniques
for effective reasoning.\\


Second, we have characterised a kind of inductive query that we call
a \emph{property persistence} query. These queries are restricted
enough to be amenable to a special-purpose reasoning algorithm based
on a meta-level fixpoint calculation. A simple iterative approximation
algorithm was presented and shown to be complete for several interesting
cases. More importantly, we have shown that such queries can always
be replaced with a uniform formula called the \emph{persistence condition,}
in a way that integrates well with the standard regression operator.
This allows certain second-order aspects of our formulation to be
{}``factored out'' and handled using special-purpose tools, while
maintaining the use of regression as the primary reasoning technique.\\


The third major contribution is a powerful new account of individual-level
epistemic reasoning, in which an agent's knowledge is expressed directly
in terms of its local observations. In asynchronous domains agents
are required to account for arbitrarily-long sequences of hidden actions
and must therefore perform some inductive reasoning. By precisely
characterising inductive component of their reasoning in terms of
a property persistence query, we factor it out of the reasoning process
and provide a regression rule for answering knowledge queries.

Basing our formalism explicitly on an agent's observations provides
two important benefits. It makes our formalism robust to theory elaboration,
as our theorems and our regression rule apply unmodified as more complex
knowledge-producing actions are added into the axioms defining observations
and views. Second, it means that a situated agent can directly use
our regression rules to reason about its own knowledge using only
its local information.

We have also demonstrated that if the theory of action is known to
be synchronous, then our regression rule does not require inductive
reasoning and it reduces to a simple syntactic transformation. Our
account of individual-level knowledge thus provides a flexible new
formalism that is comparable in reasoning complexity to the standard
account of knowledge for synchronous domains, while at the same time
extending gracefully to asynchronous domains where inductive reasoning
is required.\\


Finally, we have introduced a powerful new language of complex epistemic
modalities to the situation calculus. Based on an epistemic interpretation
of dynamic logic, these modalities are expressive enough to formulate
a regression rule for common knowledge while still permitting arbitrarily-long
sequences of hidden actions to be handled separately by the persistence
condition operator.

Our work provides the first formal account of effective reasoning
about common knowledge in the situation calculus, even in purely synchronous
domains. As in the case of individual-level knowledge, restricting
the domain to be synchronous means our regression rule does not require
inductive reasoning, so the complexity of reasoning about common knowledge
is comparable to that of reasoning about individual-level knowledge
in such domains. Our formalism again has the advantage of extending
gracefully to asynchronous domains in which inductive reasoning is
required.\\


These contributions provide a powerful fundamental framework for the
situation calculus to represent and reason about asynchronous multi-agent
domains.


\section{Future Work}

Throughout the thesis, we have also identified areas where further
work is required to bring our new techniques together with practical
systems built on the situation calculus. Work continues on developing
a MIndiGolog execution planner capable of coordinating \emph{online}
execution of a shared program by a team of agents, building on the
techniques we have developed here. The most pressing avenues of future
research are summarised below.

While a possible-worlds formulation of knowledge as developed in this
thesis provides an excellent theoretical foundation for epistemic
reasoning, possible-worlds reasoning can be highly intractable in
practice. In Section \ref{sec:Knowledge:Advances} we identified an
existing approach to more tractable epistemic reasoning that could
be combined with our formalism, and provides a promising avenue for
future work.

Chapter \ref{ch:cknowledge} developed a precise characterisation
of the processes required to reason about common knowledge in the
situation calculus, and demonstrated that it is decidable in finite
domains, but existing implementation techniques are not sufficient
to take advantage of this ability in practical systems. A naive translation
of our complex epistemic modalities into PDL quickly produces an exponential
blowup in problem size. While we have some initial intuitions on constructing
a free-variable based prover to overcome this limitation, there is
still much work to be done before our technique can be applied in
practical domains.

Two open questions remain about our complex epistemic modalities.
First, is there a \emph{less} powerful epistemic language that can
still capture a regression rule for common knowledge? We have shown
that our formalism generates epistemic paths with a quite a regular
structure, but it is unclear whether the required structure can be
formulated in a less expressive kind of path language. TODO: express
in terms of {}``bounds''.

Second, is there the possibility of performing \emph{approximate}
reasoning about common knowledge? This would require a set {}``disjunctive
knowledge'' axioms similar to those used by TODO in the case of individual
knowledge, but it is not clear how such axioms would be formulated
in the presence of the transitive closure necessary to express common
knowledge. Both of these questions have significant potential for
ongoing work.

Finally, agents in a cooperative team need not consider \emph{all}
possible actions of their teammates, since they know that the other
agents will behave according to some protocol. While we have identified
a potential approach to such reasoning that would allow protocols
to be expressed as Golog programs, significant work will be required
to produce a workable theory based on these ideas.

