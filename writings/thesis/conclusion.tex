

\chapter{Conclusion}

\label{ch:conclusion}

This thesis has laid the foundations for effective reasoning with
the situation calculus in asynchronous multi-agent domains.

We began by constructing a multi-agent variant of Golog in Chapter
\ref{ch:mindigolog}, both to motivate the use of the situation calculus
in a multi-agent setting and to highlight some of the limitations
that it has traditionally faced. The remainder of the thesis was devoted
to systematically overcoming those limitations.

At the core of our approach is the explicit representation of the
local perspective of each agent developed in Chapter \ref{ch:observations},
in which agents make a set of \emph{observations} in response to each
action occurrence. The agent's \emph{view} in a particular situation
is its history of obserations, but excluding the case where the set
of observations was empty. These structures are direct analogues to
the actions and situations that represent the omniscient perspective
of the system designer. By reifying the local perspective as terms
in the logic, and by explicitly allowing for the case when some actions
are hidden from some agents, we have been able to construct a variety
of reasoning and planning tools on top of a common, principled foundation.

The first of three key contributions was a partially-ordered branching
action structure designed to replace the use of raw situation terms
in the Golog execution planning process. This structure, a \emph{joint
execution} as defined in Chapter \ref{ch:jointexec}, leverages the
notion of a view to let the planning process explicitly reason about
inter-agents coordination of actions. The resulting plans allow the
agents to perform independent actions independently while ensuring
they can always syncrhonise their actions when necessary, and that
each agent always knows what actions to perform next. They thus provide
a much more powerful formalism with which the situation calculus can
represent and plan the cooperative execution of shared tasks in asynchronous
domains.

Our second contribution was to identify a restricted form of universally
quantified query that is amenable to more effective automated reasoning
than open-ended inductive theorem proving. Dubbed \emph{property persistence
queries}, we have shown that they are always equivalence to the evaluation
of a uniform formula at a given situation. This uniform formula, which
we call the \emph{persistence condition} of the query, can be calculated
using a meta-level fixpoint approximation which provably terminates
in some interesting cases. The inductive reasoning required to answer
such queries can thus be {}``factored out'' as a separate calculation,
and approached using specialised tools that are separate from the
rest of the reasoning process.

Finally, we have constructed a powerful new account of multi-agent
\emph{knowledge} in the situation calculus in Chapters \ref{ch:knowledge}
and \ref{ch:cknowledge}. To correctly characterise knowledge in asynchronous
domains, the agents are required to inductively reason about arbitarily-long
sequences of hidden actions. We have shown that this inductive reasoning
can be precisely characterised using the persistence condition, allowing
us to factor it out of the query and construct a regression rule for
effective reasoning about knowledge. Also unique in our formalism
is the provision of a regression rule for \emph{common knowledge}
based on a more expressive epistemic language. These features have
enabled us to reason effectively about group-level knowledge in asynchronous
domains such as the {}``party invitation'' example, which have previously
been beyond the reach of automated reasoning in the situation calculus.

Despite the signicant increase in expressive power of our epistemic
language, and the use of a meta-level fixpoint calculation for inductive
reasoning, we have demonstrated that our formalism maintains an important
property of existing work in the situation calculus: that if non-\noun{Situation}
sorts are restricted to be finitely enumerable, queries can be effectively
{}``propositionalised'' and reasoning becomes robustly decidable.
We have also highlighted some key areas in which future work could
identify less onerous restrictions on the theory of action that will
permit more efficient reasoning procedures.

Using our explicit account of the local perspective of an agent, we
have given a formal definition of \emph{sychronicity} in the situation
calculus that parallels the definitions used in epistemic reasoning
literature. We have shown that inductive reasoning is not necessary
for reasoning about knowledge in synchronous domains, and our new
regression rules all reduce back to simple syntactic transformations.
This means that our formalism can be directly adopted for synchronous
domains, providing powerful new abilities such as reasoning about
common knowledge without a significant increase in reasoning complexity.

The clearest avenue for future work from this thesis is to merge explcit
knowledge-based reasonining with the MIndiGolog execution planner.
Although we have identified the basics of the necessary control algorithm,
we do not have a prover efficient enough to make use of this technique.
TODO: free-variable FODL prover, implemented on top of Shannon Graphs
formalism.


\section{Further Work}

\begin{itemize}
\item Computational Efficiency - decidable fragments, belief instead of
knowledge. 
\item Implementation -- need free-variable PDL prover 
\end{itemize}
