\chapter{Background}\label{ch:background}
%\minitoc

\section{Distributed Problem Solving}

\begin{itemize}
\item Quickly highlight difference between open multi-agent systems and
distributed problem solving - explain what we're \emph{not} interested in.
\item Overview of DPS formalisms, task representation (HTN, TAEMS, etc...)
\item Briefly introduce HLP approach, discuss advantages.
\item Highlight success of Readylog team, but discuss limitations.
\item Overview of coordination techniques, esp. social laws for coordination without communication
\item Overview of distributed planning techniques
\end{itemize}

\section{First-Order Logic}

\begin{itemize}
\item Basic definitions
\item A good reference for more information (maybe \cite{fitting96fol_book}?)
\end{itemize}

\section{The Situation Calculus}

\begin{itemize}
\item Basic Sitcalc: origins with McCarthy \cite{McCHay69sitcalc}, formalizations by Reiter et al \cite{reiter01kia,pirri99contributions_sitcalc,levesque98sc_foundations}
\item Our own customizations, e.g. "Action Description Predicates" \cite{kelly07sc_persistence}
\item Concurrent Actions, Continuous Time, Natural Actions \cite{pinto94temporal,reiter96sc_nat_conc}
\item Reasoning (Regression, Decidable Fragments)
\item Related approaches (Fluent Calculus, Event Calculus)
  \begin{itemize}
  \item justify focusing on sitcalc by highlighting deep links between the formalisms
  \end{itemize}
\item Briefly highlight things we \emph{don't} consider - nondeterministic/probabilistic actions, decision theoretic aspects, etc - and explain why.
\end{itemize}

\section{High-Level Program Execution}

\begin{itemize}
\item Basic Golog, operators, semantics \cite{levesque97golog}
\item Highlight the vast body of similar approaches, e.g. Dynamic Logic
\item So why golog? - full power of sitcalc, answer extraction for planning.
\item Discuss idea of 'Legal Execution' in depth
\item ConGolog \cite{giacomo00congolog}.  Also some discussion of related
semantics, such as CCS 
\item IndiGolog - highlight online/offline distinction and connections to coordination and planning \cite{giacomo99indigolog}
\item TConGolog - highlight deficiencies \cite{pinto99tcongolog}
\item Other Gologs (DTGolog, HTNGolog, GTGolog)
  \begin{itemize}
  \item for completeness only, we have already established that we aren't operating in domains appropriate for them
  \item But, take time to highlight rich cross-pollination between works, and that our results similarly have potential for integration with these approaches
  \end{itemize}
\end{itemize}

\section{Epistemic Reasoning}

\begin{itemize}
\item Knowledge, Distributed Knowledge, Common Knowledge \cite{halpern90knowledge_distrib,fagin95}
\item Importance of Common Knowledge for Coordination
\item Knowledge in the Situation Calculus
  \begin{itemize}
  \item Reasoning about Knowledge \cite{moore80know_act,scherl03sc_knowledge}
  \item Use in a multi-agent setting (??what's the best reference)
  \item Concurrency and time \cite{scherl03conc_knowledge}
  \item Shortcomings of multi-agent extensions
  \item Decidable fragments, weakening for computational efficiency
  \end{itemize}
\end{itemize}

\section{The Oz Programming Language}

\begin{itemize}
\item Basic introduction \cite{vanroy99mozart}
\item Oz for Logic Programming \cite{vanroy03mozart_logic}
\item Distributed Logi Programming: Parallel Search \cite{schulte00oz_parallel}
\item include some simple example programs
\end{itemize}

