

\chapter{Background}

\label{ch:background}

This chapter covers general background material for the thesis and
provides a brief overview of the related literature. We defer more
specific technical details and discussion of related work to the individual
chapters that follow, where it can be presented in the appropriate
context.

Readers familiar with the situation calculus are encouraged to briefly
review this chapter. While it does not present any new results, it
does introduce some novel notation and definitions which will be needed
later in the thesis. They are introduced here to maintain consistency
of the presentation. The introductory material on the Mozart programming
platform may also be helpful.

We begin by introducing the base language of the situation calculus
in section \ref{sec:Background:The-Situation-Calculus}, illustrated
using examples from the {}``cooking agents'' domain. Section \ref{sec:Background:Golog}
introduces the Golog family of programming languages, which are the
standard formalism for representing complex tasks in the situation
calculus. Related formalisms for reasoning about action and change
are briefly discussed in section \ref{sec:Background:Related-Formalisms}.
Finally, section \ref{sec:Background:Mozart/Oz} introduces the Mozart
programming system, which will be used to implement the various techniques
developed throughout the thesis. Basic familiarity with formal logic
is assumed throughout; readers requiring background on such material
may find a gentle introduction in \citep{kelly96logic} and a more
detailed treatment in \citep{fitting96fol_book}.


\section{The Situation Calculus\label{sec:Background:The-Situation-Calculus}}

The situation calculus is a powerful formalism for describing and
reasoning about dynamic worlds. It was first introduced by \citet{McCHay69sitcalc}
and has since been significantly expanded and formalised \citep{reiter91frameprob,pirri99contributions_sitcalc}.
We use the particular variant due to Reiter et. al. at the University
of Toronto, sometimes called the {}``Toronto school'' or {}``situations-as-histories''
version. The formalisation below is based on the standard definitions
from \citep{levesque98sc_foundations,pirri99contributions_sitcalc,reiter01kia},
but has been slightly generalised to accommodate both existing extensions
to the situation calculus and our own forthcoming extensions.

Readers familiar with the situation calculus should note some slightly
modified notation: the unique names axioms $\Dt_{una}$ are incorporated
into a general background theory $\Dt_{bg}$; the $Poss$ fluent is
subsumed by a general class of \emph{action description predicates}
defined in $\Dt_{ad}$; we parameterise the {}``future situations''
predicate $s\sqsubset s'$ to assert that all intervening actions
satisfy a given predicate using the notation $s<_{\alpha}s'$; and
we use the single-step variant of the regression operator, with corresponding
definitions of regressable formulae.


\subsection{Notation\label{sec:Background:SC:Notation}}

The language $\mathcal{L}_{sitcalc}$ of the situation calculus is
a many-sorted language of first-order logic with equality, augmented
with a second-order induction axiom, containing the following disjoint
sorts:

\begin{itemize}
\item \emph{\noun{Action}} terms are functions denoting individual instantaneous
events that can cause the state of the world to change; 
\item \noun{Situation} terms are histories of the actions that have occurred
in the world, with the initial situation represented by $S_{0}$ and
successive situations built using the function $do\,:\, Action\times Situation\rightarrow Situation$; 
\item \noun{Object} terms represent any other object in the domain. 
\end{itemize}
\emph{Fluents} are predicates or functions that represent properties
of the world that may change between situations, and so take a situation
term as their final argument. Predicates and functions that do not
take a situation term are called \emph{rigid}. We use the term \emph{primitive
fluent} to describe fluents that are directly affected by actions,
rather than being defined in terms of other fluents. No functions
other than $S_{0}$ and $do$ produce values of sort \noun{Situation.}

For concreteness, let us present some formulae from an example domain
that will be used throughout the thesis. In the {}``cooking agents''
domain a group of robotic chefs inhabit a kitchen containing various
ingredients and utensils, and they must cooperate to prepare a meal.
Some example statements from this domain include {}``Joe is not holding
the knife initially'', {}``Jim is holding the knife after he picks
it up'' and {}``It is only possible to pick up an object if nobody
is holding it''. Formally:\begin{gather*}
\neg Holding(Joe,Knife1,S_{0})\\
Holding(Jim,Knife1,do(pickup(Jim,Knife1),S_{0}))\\
Poss(pickup(agt,obj),s)\equiv\neg\exists agt_{2}:\, Holding(agt_{2},obj,s)\end{gather*}


Here $Holding$ is a primitive fluent while $Poss$ is defined in
terms of $Holding$.\medskip{}


$\mathcal{L}_{sitcalc}$ contains the standard alphabet of logical
connectives, countably infinitely many variables of each sort, countably
infinitely many predicates of each arity, etc; for a complete definition,
consult the foundational paper by \citet{pirri99contributions_sitcalc}.
We follow standard naming conventions for the situation calculus:
upper-case roman names indicate constants; lower-case roman names
indicate variables; greek characters indicate meta-variables or formula
templates. All axioms universally close over their free variables
at outermost scope. The notation $\vars{t}$ indicates a vector of
terms of context-appropriate arity and type. The connectives $\wedge$,
$\neg$, $\exists$ are taken as primitive, with $\vee$, $\rightarrow$,
$\equiv$, $\forall$ defined in the usual manner.\\


In multi-agent domains it is customary to introduce the sort \noun{Agent
}as a sub-sort of \noun{Object} to explicitly represent the agents
operating in the world, and we will do so here. As seen in the example
formulae above, the first argument of each action term gives the performing
agent.

\medskip{}
 Complex properties of the state of the world are represented using
\emph{uniform formulae}. These are basically logical combinations
of fluents referring to a common situation term. For the moment we
restrict ourselves to \emph{objective} uniform formulae; this definition
will be revised to include statements about knowledge when we introduce
that material.

\begin{defnL}
[{Uniform~Terms}] Let $\sigma$ be a fixed situation term,
$r$ an arbitrary rigid function symbol, $f$ an arbitrary fluent
function symbol, and $x$ a variable that is not of sort \noun{Situation}.
Then the terms uniform in $\sigma$ are the smallest set of syntactically-valid
terms satisfying:\[
\tau\,::=x\,|\, r(\vars{\tau})\,|\, f(\vars{\tau},\sigma)\]

\begin{defnL}
[{Uniform~Formulae}] Let $\sigma$ be a fixed situation
term, $R$ an arbitrary rigid predicate, $F$ an arbitrary primitive
fluent predicate, $\tau$ an arbitrary term uniform in $\sigma$,
and $x$ an arbitrary variable that is not of sort \noun{Situation}.
Then the formulae uniform in $\sigma$ are the smallest set of syntactically-valid
formulae satisfying:\[
\phi::=F(\vars{\tau},\sigma)\,|\, R(\vars{\tau})\,|\,\tau_{1}=\tau_{2}\,|\,\phi_{1}\wedge\phi_{2}\,|\,\neg\phi\,|\,\exists x:\phi\]

\end{defnL}
\end{defnL}
The important aspect of this definition is that the formula refers
to no situation other than $\sigma$, which appears as the final argument
of all fluents in the formula. In particular, uniform formulae cannot
quantify over situations or compare situation terms, and cannot contain
non-primitive fluents.

The meta-variable $\phi$ is used throughout to refer to an arbitrary
uniform formula. Since they represent some aspect of the state of
the world, it is frequently useful to evaluate uniform formulae at
several different situation terms. The notation $\phi[s']$ represents
a uniform formula with the particular situation $s'$ inserted into
all its fluents. We may also completely suppress the situation term
to simplify the presentation, using $\phi^{-1}$ to represent a uniform
formula with the situation argument removed from all its fluents.
For example, if $\phi=Holding(Jim,Knife,s)\wedge Holding(Joe,Bowl,s)$,
then we have:\begin{gather*}
\phi[s']\,=\, Holding(Jim,Knife,s')\wedge Holding(Joe,Bowl,s')\\
\phi^{-1}\,=\, Holding(Jim,Knife)\wedge Holding(Joe,Bowl)\end{gather*}


Note that these are strictly meta-level operations, corresponding
to possibly quite complex sentences from the underlying logic. They
are \emph{not} terms or operators from the logic itself.


\subsection{Axioms\label{sec:Background:SC:Axioms}}

The dynamics of a particular domain are captured by a set of sentences
from $\mathcal{L}_{sitcalc}$ called a \emph{basic action theory}.
Queries about the behaviour of the world are posed as logical entailment
queries relative to this theory.

\begin{defnL}
[{Basic~Action~Theory}] A basic action theory, denoted
$\Dt$, is a set of situation calculus sentences (of the specific
syntactic form outlined below) describing a particular dynamic world.
It consists of the following disjoint sets: the foundational axioms
of the situation calculus ($\Sigma$); action description axioms defining
preconditions etc for each action ($\Dt_{ad}$); successor state axioms
describing how primitive fluents change between situations ($\Dt_{ssa}$);
axioms describing the value of primitive fluents in the initial situation
($\Dt_{S_{0}}$); and axioms describing the static background facts
of the domain ($\Dt_{bg}$):\[
\Dt=\Sigma\cup\Dt_{ad}\cup\Dt_{ssa}\cup\Dt_{S_{0}}\cup\Dt_{bg}\]

\end{defnL}
These axioms must satisfy some simple consistency criteria to constitute
a valid domain description; see \citep{pirri99contributions_sitcalc}
for the details. This definition is slightly broader than the standard
definitions found in the literature \citep{levesque98sc_foundations,pirri99contributions_sitcalc,reiter01kia},
and is designed to accommodate a variety of extensions to the situation
calculus in a uniform manner.

We assume an arbitrary, but fixed, basic action theory.


\subsubsection{Background Axioms}

The set $\Dt_{bg}$ characterises the static aspects of the domain,
and contains all axioms defining rigid predicates or functions. In
particular, it must contain a set of unique names axioms asserting
that action terms with different types or arguments are in fact different,
e.g.:\begin{gather*}
pickup(agt,obj)\neq drop(agt,obj)\\
pickup(agt_{1},obj_{1})=pickup(agt_{2},obj_{2})\,\rightarrow\, agt_{1}=agt_{2}\,\wedge\, obj_{1}=obj_{2}\end{gather*}


It also contains domain closure axioms for the sorts \noun{Action,
Agent} and \noun{Object}, and defines the function $actor(a)$ to
give the agent performing an action. The background axioms are a generalisation
of the set $\Dt_{una}$ commonly found in the literature, which contains
only the unique names axioms.


\subsubsection{Successor State Axioms}

The set $\Dt_{ssa}$ contains one \emph{successor state axiom} for
each primitive fluent in the domain. These axioms provide an elegant
monotonic solution to the frame problem for that fluent \citep{reiter91frameprob}
which has been instrumental to the popularity and utility of the situation
calculus. They have the following general form: \[
F(\vars{x},do(a,s))\equiv\Phi_{F}^{+}(\vars{x},a,s)\,\,\vee\,\, F(\vars{x},s)\wedge\neg\Phi_{F}^{-}(\vars{x},a,s)\]


Here $\Phi_{F}^{+}$ and $\Phi_{F}^{-}$ are formulae uniform in $s$.
This may be read as {}``$F$ is true after performing $a$ if $a$
made it true, or it was previously true and $a$ did not make it false''.
For example, the dynamics of the $Holding$ fluent may be specified
using:\begin{multline*}
Holding(agt,obj,do(a,s))\,\equiv\, a=pickup(agt,obj)\\
\vee\,\, Holding(agt,obj,s)\wedge a\neq drop(agt,obj)\end{multline*}


When we do not wish to call attention to the special form of successor
state axioms, we will denote the axiom for fluent $F$ by the simpler
form:\[
F(\vars{x},do(a,s))\,\equiv\,\Phi_{F}(\vars{x},a,s)\]


For functional fluents, $\Dt_{ssa}$ contains a similar axiom to specify
the value $v$ of the fluent after an action has occurred:\[
f(\vars{x},do(a,s))=v\,\equiv\,\Phi_{f}(v,\vars{x},a,s)\]



\subsubsection{Action Description Predicates}

The set $\Dt_{ad}$ generalises the standard \emph{action precondition
axioms} \citep{pirri99contributions_sitcalc} to define fluents that
describe various aspects of the performance of an action, which we
call \emph{action description predicates}. These are the only non-primitive
fluents permitted in a basic action theory. The predicate $Poss(a,s)$
is the canonical example, indicating whether it is possible to perform
an action in a given situation. The set $\Dt_{ad}$ contains a single
axiom of the following form, defining the complete set of preconditions
for the action variable $a$, where $\Pi_{Poss}$ is a formula uniform
in $s$:\[
Poss(a,s)\,\equiv\,\Pi_{Poss}(a,s)\]


Note that this is a slight departure from the standard approach of
\citep{pirri99contributions_sitcalc}, in which the preconditions
for each action type are enumerated individually. The more restrictive
approach presented here embodies a domain-closure assumption on the
\noun{Action} sort. For example, if there are finitely many action
types then $\Pi_{Poss}$ is simply the disjunction of the precondition
axioms for each action type. The single-axiom form is necessary when
quantifying over {}``all possible actions'' and has been widely
used in the literature \citep{vassos08progression_future_queries,savelli06sc_quantum_levels}.

In principle, any number of predicates and functions can be defined
in this way - a common example is the sensing-result function $SR(a,s)$
introduced in section TODO - and the general notation of action description
predicates allows us to treat all of them uniformly. We will use the
meta-variable $\alpha$ to represent an arbitrary action description
predicate.

In preparation for the coming material on natural actions in section
\ref{sub:Background:Natural-Actions}, let us introduce an action
description predicate $Legal$ that identifies actions that can be
legally executed in the real world. In the basic situation calculus,
it is simply equivalent to $Poss$:\begin{gather*}
Legal(a,s)\,\equiv\, Poss(a,s)\end{gather*}


As shown by the above, it is often useful define new action description
predicates in terms of simpler existing ones, rather than directly
in terms of the primitive fluents of the domain. As long as the definitions
are well-founded they can be expanded down to primitive fluents when
constructing the basic action theory.


\subsubsection{Foundational Axioms}

The foundational axioms $\Sigma$ ensure that situations form a branching-time
account of the world state. There is a distinguished situation $S_{0}$
called the \emph{initial situation}. Situations in general form a
tree structure with the initial situation at the root and $do(a,s)$
constructing the successor situation resulting when the action $a$
is performed in $s$. All situations thus produced are distinct:\[
do(a_{1},s_{1})=do(a_{2},s_{2})\,\rightarrow\, a_{1}=a_{2}\,\wedge\, s_{1}=s_{2}\]


We abbreviate the performance of several successive actions by writing:\[
do([a_{1},\dots,a_{n}],s)\,\isdef\, do(a_{n},do(\dots,do(a_{1},s)))\]


The relation $s\sqsubset s'$ indicates that $s'$ is in the future
of $s$ and is defined as follows:\begin{gather*}
\neg(s\sqsubset S_{0})\\
s\sqsubset do(a,s')\equiv s\sqsubseteq s'\end{gather*}


Here $s\sqsubseteq s'$ is the standard abbreviation for $s\sqsubset s'\vee s=s'$.
There is also a second-order induction axiom asserting that all situations
must be constructed in this way, which is needed to prove statements
that universally quantify over situations \citep{Reiter93proving}:\[
\forall P:\,\left[P(S_{0})\wedge\forall s,a:\,\left(P(s)\rightarrow P(do(a,s))\right)\right]\,\rightarrow\,\forall s:\, P(s)\]


The notation for {}``in the future of'' can be extended to consider
only those futures in which all actions satisfy a particular action
description predicate. We include a relation $<_{\alpha}$ for each
action description predicate $\alpha$, with the following definitions:\begin{gather*}
\neg\left(s<_{\alpha}S_{0}\right)\\
s<_{\alpha}do(a,s')\equiv s\leq_{\alpha}s'\wedge\alpha(a,s')\end{gather*}


The \emph{legal situations} are those in which every action was legal
to perform in the preceding situation. These are of fundamental importance,
as they are the only situations that could be reached in the real
world:\begin{eqnarray*}
Legal(s) & \equiv & S_{0}\leq_{Legal}s\end{eqnarray*}



\subsubsection{Initial State Axioms}

The set $\Dt_{S_{0}}$ describes the actual state of the world before
any actions are performed. It is a collection of sentences uniform
in $S_{0}$ stating what holds in the initial situation. In many domains
the initial state can be completely specified, so $\Dt_{S_{0}}$ is
often in a closed form suitable for efficient automated reasoning.


\subsection{Reasoning}

One of the main attractions of the situation calculus is the existence
of effective reasoning procedures for certain types of query. These
are generally based on syntactic manipulation of a query into a form
that is more amenable to reasoning -- for example, because it can
be proven without using some of the axioms from $\Dt$.


\subsubsection{Types of Reasoning}

In the general case, answering a query about a basic action theory
$\Dt$ is a theorem-proving task in second-order logic (denoted SOL)
due to the induction axiom included in the foundational axioms:\[
\Dt\models_{SOL}\psi\]
 This is clearly problematic for effective automated reasoning, but
fortunately there exist particular syntactic forms for which some
of the axioms in $\mathcal{D}$ are not required.

If a query existentially quantifies over situation terms, it can be
answered without the induction axiom (denoted $I$) and thus using
only first-order logic (FOL) \citep{pirri99contributions_sitcalc}:\[
\Dt\models_{SOL}\exists s:\,\psi(s)\,\,\,\,\mathrm{iff}\,\,\,\,\Dt-\{I\}\models_{FOL}\exists s:\,\psi(s)\]


While this is a substantial improvement over requiring a second-order
theorem prover, it is still far from an effective technique. Effective
reasoning requires that the set of axioms be reduced as much as possible.

In their work on state constraints, \citet{Lin94-StateConstraints}
show how to reduce the task of verifying a state constraint to a reasoning
task we call \emph{static domain reasoning}, where only the background
axioms need to be considered:\[
\Dt_{bg}\models_{FOL}\forall s:\,\phi[s]\]


This is a major improvement because universal quantification over
situation terms usually requires the second-order induction axiom.
Their work has shown that this requirement can be circumvented in
some cases.

Simpler still are queries uniform in the initial situation, which
can be answered using only first-order logic and a limited set of
axioms:\[
\mathcal{D}\models_{SOL}\phi[S_{0}]\,\,\,\,\,\mathit{\mathrm{iff}}\,\,\,\,\,\Dt_{S_{0}}\cup\Dt_{bg}\models_{FOL}\phi[S_{0}]\]


We call such reasoning \emph{initial} \emph{situation reasoning}.
Since the axioms $\Dt_{S_{0}}\cup\Dt_{bg}$ often satisfy the closed-world
assumption, provers such as Prolog can be employed to handle this
type of query quite effectively.


\subsubsection{Regression}

The principle tool for effective reasoning in the situation calculus
is the regression meta-operator $\Reg_{\Dt}$ , a syntactic manipulation
that encodes the preconditions and effects of actions into the query
itself, meaning fewer axioms are needed for the final reasoning task
\citep{pirri99contributions_sitcalc}. The idea is to reduce a query
about some future situation to a query about the initial situation
only.

There are two styles of regression operator commonly defined in the
literature: the single-pass operator as defined in \citep{pirri99contributions_sitcalc}
which reduces to $S_{0}$ in a single application, the the single-step
operator as defined in \citep{scherl03sc_knowledge} which operates
one action at a time. We use the single-step variant because it is
the more expressive of the two -- while it is straightforward to define
the single-pass operator in terms of the single-step operator, the
reverse is not the case.

Regression is only defined for a certain class of formulae, the \emph{regressable
formulae}.

\begin{defnL}
[{Regressable~Terms}] Let $\sigma$ be an arbitrary situation
term, $x$ an arbitrary variable not of sort situation, $r$ an arbitrary
rigid function and $f$ an arbitrary fluent function. Then the regressable
terms are the smallest set of syntactically-valid terms satisfying:
\[
\nu::=\sigma\,|\, x\,|\, f(\vars{\nu},\sigma)\,|\, r(\vars{\nu})\]

\begin{defnL}
[{Regressable~Formulae}] Let $\sigma$ be an arbitrary situation
term, $x$ an arbitrary variable not of sort situation, $\nu$ an
arbitrary regressable term, $R$ an arbitrary rigid predicate, $F$
an arbitrary primitive fluent predicate, and $\alpha$ an arbitrary
action description predicate. Then the regressable formulae are the
smallest set of syntactically-valid formulae satisfying: \[
\varphi::=F(\vars{\nu},\sigma)\,|\,\alpha(\vars{\nu},a,\sigma)\,|\, R(\vars{\nu})\,|\,\nu_{1}=\nu_{2}\,|\,\neg\varphi\,|\,\varphi_{1}\wedge\varphi_{2}\,|\,\exists x:\,\varphi\]

\end{defnL}
\end{defnL}
Regressable formulae are more general than uniform formulae -- in
particular, they can contain action description predicates and may
mention different situation terms. They cannot, however, quantify
over situation terms or compare situations using the $\sqsubset$
predicate.

With these definitions in place, the operation of $\Reg_{\Dt}$ is
defined by a set of \emph{regression rules} as shown below.

\begin{defnL}
[{Regression~Operator}] Let $R$ be a rigid predicate, $\alpha$
be an action description predicate with axiom $\alpha(\vars{\nu},a,s)\equiv\Pi_{\alpha}(a,s)$
in $\Dt_{ad}$, and $F$ be a primitive fluent with axiom $F(\vars{x},s)\equiv\Phi_{F}(\vars{x},s)$
in $\Dt_{ssa}$. Then the regression of $\phi$, denoted $\Reg_{\Dt}(\phi)$,
is defined according to the following rules:\begin{gather*}
\Reg_{\Dt}(\varphi_{1}\wedge\varphi_{2})\,\isdef\,\Reg_{\Dt}(\varphi_{1})\wedge\Reg_{\Dt}(\varphi_{2})\\
\Reg_{\Dt}(\exists x:\,\varphi)\,\isdef\,\exists x:\,\Reg_{\Dt}(\varphi)\\
\Reg_{\Dt}(\neg\varphi)\,\isdef\,\neg\Reg_{\Dt}(\varphi)\\
\Reg_{\Dt}(\alpha(\vars{\nu},a,\sigma))\,\isdef\,\Reg_{\Dt}(\Pi_{\alpha}(\vars{\nu},a,\sigma))\\
\Reg_{\Dt}(F(\vars{\nu},do(a,\sigma)))\,\isdef\,\Phi_{F}(\vars{\nu},a,\sigma)\\
\Reg_{\Dt}(F(\vars{\nu},s))\,\isdef\,\Phi_{F}(\vars{\nu},a,s)\\
\Reg_{\Dt}(F(\vars{\nu},S_{0}))\,\isdef\,\Phi_{F}(\vars{\nu},a,S_{0})\end{gather*}

\end{defnL}
Since $\Dt$ is fixed, we will henceforth drop the subscript and simply
write $\Reg$ for the regression operator.

We have omitted some technical details here, such as the handling
of functional fluents; consult \citep{pirri99contributions_sitcalc}
for the details. The key point is that each application of the regression
operator replaces action description predicates with their definitions
from $\Dt_{ad}$ and primitive fluents with their successor state
axioms from $\Dt_{ssa}$, {}``unwinding'' a single action from each
$do(a,\sigma)$ situation term in the query. If the situation term
is not constructed using $do$, it is left unchanged.

When dealing with situation-suppressed uniform formulae, we will use
a two-argument operator $\Reg(\phi,a)$ to indicate the regression
of $\phi$ over the action $a$. It should be read as a shorthand
for $\Reg(\phi[do(a,s)])^{-1}$ using the situation-suppression operator
from section \ref{sec:Background:SC:Notation}.

Let us briefly state some important properties of the regression operator.
First, and most importantly, it preserves equivalence of formulae:

\begin{prop}
Let $\varphi$ be a regressable formula, then $\Dt\,\models\,\varphi\,\equiv\,\Reg(\varphi)$ 
\end{prop}
Any formula uniform in $do(a,s)$ is regressable, and the result is
uniform in $s$:

\begin{prop}
Let $\phi$ be uniform in $do(a,s)$, then $\Reg(\phi)$ is uniform
in $s$ 
\end{prop}
Let $\Reg^{*}$ denote repeated applications of $\Reg$ until the
formula remains unchanged. Such applications can transform a query
about some future situation into a query about the initial situation
only:

\begin{prop}
Let $\phi$ be uniform in \emph{$do([a_{1}\dots a_{n}],S_{0})$,}
then $\Reg^{*}(\phi)$ is uniform in $S_{0}$ 
\end{prop}
This last property is key to effective reasoning in the situation
calculus. As discussed above, queries uniform in $S_{0}$ are much
easier to answer. The axioms $\Dt_{ad}$ and $\Dt_{ssa}$ are essentially
{}``compiled into'' the query by the $\Reg^{*}$ operator. While
an efficiency gain is not guaranteed, regression has proven a very
effective technique in practice \citep{levesque97golog,pirri99contributions_sitcalc}.

One shortcoming of the regression operator is the potential for an
exponential growth in the length of the query \citep{reiter91frameprob}.
While this cannot be completely overcome in general, there are certain
restrictions that can be placed on the domain to ensure it is not
a problem. One example is domains with \emph{context free} successor-state
axioms \citep{reiter01kia}, where the components $\Phi_{F}^{+}(\vars{x},a,s)$
and $\Phi_{F}^{-}(\vars{x},a,s)$ in the successor state axiom for
$F$ are independent of the situation $s$.

\begin{defnL}
[{Context~Free~Successor-State~Axioms}] A successor state
axiom is context-free if it has the following form, where $\Phi_{F}^{+}$
and $\Phi_{F}^{-}$ are independent of $s$:\[
F(\vars{x},do(a,s))\,\equiv\,\Phi_{F}^{+}(\vars{x},a)\,\vee\, F(\vars{x},s)\wedge\neg\Phi_{F}^{-}(\vars{x},a)\]

\end{defnL}
In other words, the effects of each action depend only on the action,
not on the situation it is executed in. Surprisingly many domains
satisfy this restriction; for example, the axiom for $Holding(agt,obj,do(a,s))$
present earlier is context-free. Context-free successor state axioms
have been proven to increase the size of the regressed query at most
linearly with the number of regression steps performed.

As we shall see, requiring context-free successor state axioms can
also improve the effectiveness of other reasoning procedures.


\subsubsection{Decidability}

Even given the use of regression to reduce the number of axioms required,
reasoning still requires first-order logic and is thus only semi-decidable
in general. Practical systems implemented on top of the situation
calculus typically enforce additional restrictions on the domain in
order to gain decidability. The most common, since it is straightforward
to implement in Prolog, assumes that axioms $\Dt_{bg}$ and $\Dt_{S_{0}}$
satisfy the closed-world assumption. This allows initial situation
reasoning to be performed using Prolog instead of a full first-order
prover.

TODO: quality this further: prolog $\neq$ decidability

Another option is to assume that the \noun{Action} and \noun{Object}
domains are finite. This allows quantification over these variables
to be replaced with finite conjunctions or disjunctions, essentially
{}``propositionalising'' the domain \citep{giacomo99impl_robots,levesque04krr_book}.
Both static domain and initial situation reasoning can then be performed
using propositional logic, which is decidable.

Recent work by \citet{yu07twovar_sitcalc} has shown how to model
domains using to the two-variable fragment of first-order logic. Since
this fragment is decidable in general, both static domain and initial
situation reasoning are decidable in such domains.


\subsubsection{Inductive Reasoning}

One class of query that cannot be answered effectively using regression
is formulae that quantify over situations. Examples of such queries
include verifying state constraints ({}``for all situations, the
constraint is satisfied'') and determining the impossibility of a
goal ({}``for all situations, the goal is not satisfied''). The
difficulty comes from the induction axiom.

\citet{Reiter93proving} has shown how the induction axiom is necessary
to prove statements that universally quantify over situation terms.
This work demonstrates the use of the axiom in manual proofs, but
offers no procedure for answering such queries automatically.

Other work considering inductive reasoning has focused exclusively
on verifying state constraints \citep{Lin94-StateConstraints,bertossi96automating}.
While it is possible to automate this verification in some cases,
there are currently no general-purpose automated tools for handling
queries that universally quantify over situation terms.

It is this limitation, more than any other, that has restricted the
situation calculus to synchronous domains. In asynchronous domains
agents must account for potentially arbitrarily-long sequences of
hidden actions, which requires universal quantification over situation
terms. In chapter \ref{ch:persistence} we develop a new reasoning
technique to overcome this limitation.


\subsubsection{Progression}

While regression has proven quite an effective technique in practice,
it has an obvious shortcoming modelling domains with long histories
- the computation required to reason about the current state of the
world increases with each action performed.

An alternative approach is \emph{progression}, in which the initial
state of the world $\Dt_{S_{0}}$ is updated with each action performed,
to give a new set of axioms describing the state of the current situation.
Although this increases the upfront complexity when an action is performed,
this work is amortised over many queries about the updated state.
\citet{thielscher04case_for_progression} makes a compelling case
that progression gives better runtime performance in domains with
many actions. Why, then, do we focus only on regression in this thesis?

The theoretical foundations of progression in the situation calculus
were laid out by \citet{reiter97progression} and come with an important
caveat: the progression of a first-order database is not always first-order
definable. While it is possible to define first-order progressions
of a database that are valid for restricted classes of query, a first-order
progressed database cannot be complete in general \citep{vassos08progression_future_queries}.
As such, work on progression in the situation calculus has focused
on restricted queries or restricted databases for which progression
can be defined \citep{liu05sc_progression_knowledge,vassos07progression}.
By contrast, the regression operator is both sound and complete for
answering a broad range of queries.

In this thesis, we develop formalisms and reasoning techniques for
problems which have not been approach before in the situation calculus.
Our first priority must be a sound and complete reasoning tool, for
which regression is a good match. Advanced techniques such a progression
are placed squarely under {}``future work'' at this stage.


\subsection{Extensions}

The base language of the situation calculus may seem rather simplistic,
lacking many features that would be desirable for modelling rich multi-agent
domains. However, it is possible to significantly enrich the domain
features that can be modelled while maintaining the elegance and simplicity
of the base situation calculus. We now discuss several such extensions
that are important when modelling multi-agent domains.


\subsubsection{Concurrent Actions}

In the basic situation calculus only a single action can occur at
any instant. While suitable for most single-agent domains, this limitation
is emphatically not suitable for multi-agent systems - several actions
can easily occur simultaneously if performed by different agents.
Modelling this \emph{true concurrency} is necessary to avoid problems
with conflicting or incompatible actions. There is also the potential
to utilise concurrency to execute tasks more efficiently. Clearly
a solid account of concurrency is required for programming multi-agent
teams.

The work of \citep{lin92sc_conc,reiter96sc_nat_conc,pinro95reasoning_time}
adds true concurrency to the situation calculus by replacing action
terms with \emph{sets} of actions that are performed simultaneously.
The additional sort \noun{Concurrent} is added to $\mathcal{L}_{sitcalc}$,
and the appropriate axioms for set theory are added to $\Dt_{bg}$.
All functions and predicates that take an \noun{Action} term are are
modified to take a \noun{Concurrent} term instead. For example, $do(a,s)$
becomes $do(\{a_{1},a_{2},...\},s)$. Successor state axioms are modified
to test for set membership rather than equality of action terms. For
example, the successor state axiom for $Holding$ would become:\begin{multline*}
Holding(agt,obj,do(c,s))\equiv pickup(agt,obj)\in c\\
\vee\,\, Holding(agt,obj,s)\,\wedge drop(agt,obj)\not\in c\end{multline*}


Since it operates solely by replacing formulae with their equivalents,
the regression operator is unchanged by this extension.

There is, however, a subtle complication in axiomatising action description
predicates such as $Poss$: interaction between primitive actions.
A combination of actions is not guaranteed to be possible even if
each of the individual actions are. For example, two agents may not
be able to acquire the same resource at the same time. This is known
as the precondition interaction problem and has undergone extensive
research \citep{pinto94temporal,pinto98interacting_effects,pinto00action_interaction}.
We make no explicit commitment towards a solution for this problem.
Rather, we assume that the axioms in $\Dt_{ad}$ contain the necessary
logic to account for precondition interaction for all action description
predicates.

Since true concurrency is such an important aspect of multi-agent
systems, we will assume concurrent actions are in use throughout the
rest of the thesis.


\subsubsection{Time}

An explicit notion of time can make coordination between agents easier,
as joint actions may be performed at a particular time. It also allows
a richer description of the world, particularly in domains such as
the cooking agents where time can play an important part in tasks
to be performed.

The standard approach to time in the situation calculus is that of
\citep{pinto94temporal,reiter96sc_nat_conc}. An additional sort \noun{Timepoint}
is introduced, which can be any appropriately-behaved sequence such
as integers or reals. The axiomatisation of timepoints is added to
$\Dt_{bg}$, and each action gains an extra argument indicating the
time at which is was performed. The functions $time$ and $start$
are introduced to give the performance time of an action and the start
time of a situation respectively. The start time of the initial situation
is typically defined to be zero.

However, this approach does not integrate cleanly with concurrent
actions: it requires an additional predicate $Coherent$ to ensure
that the performance time is the same for all members in a set of
concurrent actions \citep{reiter96sc_nat_conc}. The legal situations
must be restricted to those in which all actions were coherent.

To avoid this extra complexity, we follow the approach taken in the
related formalism of the Fluent Calculus \citep{martin03conc_flux}
and attach the temporal component to the set of concurrent actions
itself, rather than to each individual action.

Predicates and functions taking terms of sort \noun{Action} are modified
to take \noun{Concurrent\#Timepoint} pairs, e.g. $do(c,s)$ becomes
$do(c\#t,s)$. The new function $start$ is added to the background
theory with the following definition:\[
start(do(c\#t,s))=t\]


We must ensure that successor situations have later start times than
their preceeding situations, by modifying the definition of $Legal$:\[
Legal(c\#t,s)\,\equiv\, Poss(c\#t,s)\wedge start(s)<t\]


Introducing timepoints does not affect the regression operator, but
does increase the complexity of reasoning as $\Dt_{bg}$ now contains
the axioms of number theory. In practise, we limit predicates about
time to express only \emph{linear} relationships, and employ a linear
constraint solver to handle the temporal reasoning component.


\subsubsection{Natural Actions\label{sub:Background:Natural-Actions}}

Natural actions are a special class of exogenous actions, those actions
which occur outside of an agent's control \citep{reiter96sc_nat_conc}.
They are classified according to the following requirement: natural
actions must occur if it is possible for them to occur, unless an
earlier action prevents them. For example, a timer will ring at the
time it was set for, unless it is switched off. Such actions are used
to model the predictable behaviour of the environment.

Natural actions are identified by the truth of the predicate $Natural(a)$.
The times at which natural actions may occur are specified by the
$Poss$ predicate just like ordinary actions. For example, suppose
that the fluent $TimerSet(m,s)$ represents the fact that a timer
is set to ring in $m$ minutes in situation $s$. The possibility
predicate would entail:\[
Poss(ringTimer(t),s)\equiv\exists m:\,\left[TimerSet(m,s)\wedge t=start(s)+m\right]\]
 The timer may thus ring only at its predicted time. To enforce the
requirement that natural actions \emph{must} occur whenever possible,
the action description predicate $Legal(c\#t,s)$ is modified to ensure
that $c\#t$ is not legal if natural actions could occur at some earlier
time:\begin{multline*}
Legal(c\#t,s)\equiv Poss(c\#t,s)\\
\wedge\,\,\forall a,t':\,\left[Natural(a)\wedge Poss(\{a\}\#t',s)\,\rightarrow\,\left(a\in c\vee t<t'\right)\right]\end{multline*}
 Thus it is only legal to perform actions $c$ at time $t$ if no
natural actions can occur in $s$ at a time less than $t$.

The \emph{least natural time point} (or {}``LNTP'') of a situation
is defined as the earliest time at which a natural action may occur.
Rather than adding another axiom, this can be defined using a simple
macro: \begin{multline*}
\LNTP(s,t)\isdef\exists a:\left[Natural(a)\wedge Poss(\{a\}\#t,s)\right]\wedge\\
\forall a',t':\left[Natural(a')\wedge Poss(\{a'\}\#t,s)\rightarrow t\leq t'\right]\end{multline*}


We assume that the theory of action avoids certain pathological cases
\citep{reiter01kia}, so that absence of an LNTP implies that no natural
actions are possible. The LNTP is important when planning in the presence
of natural actions -- one cannot plan to perform some actions at time
$t$ if $t$ is greater than the least natural timepoint of the current
situation.

We also define a related concept, the set of \emph{pending natural
actions}, as the set of all natural actions that are possible at the
least natural time point:\begin{multline*}
\PNA(s,c)\isdef\exists t:\,\LNTP(s,t)\wedge\forall a:\left[Natural(a)\wedge Poss(\{a\}\#t,s)\equiv a\in c\right]\\
\vee\neg\exists t:\,\LNTP(s,t)\wedge c=\{\}\end{multline*}


TODO: LNTP condition: $\exists a:Natural(a)\wedge Poss(a,s)\rightarrow\exists t:\LNTP(s,t)$


\subsubsection{Long-Running Tasks}

Although all actions in the situation calculus are instantaneous,
it is still possible to model long-running tasks that have a finite
duration. They are decomposing them into instantaneous $beginTask$
and $endTask$ actions, and a fluent $DoingTask$ indicating that
a task is in progress \citep{pinto94temporal}.

In the presence of long-running tasks, a robust account of natural
actions is very important - the $endTask$ must be a natural action
to ensure that any task that is initiated eventually terminates at
the appropriate time.


\subsubsection{Summary}

As can be seen from the discussion above, it is possible to enrich
the situation calculus with some very powerful domain features while
still maintaining the basic structure of the language, as well as
retaining regression as an effective reasoning technique.

TODO: revisit this section to summarise how and when we use these
extensions - after we've actually used them in later chapters :-)


\subsection{Applications}

The most prominent application of the situation calculus has been
the programming language Golog \citep{levesque97golog} and its descendants,
which will be discussed in detail in section \ref{sec:Background:Golog}.
That the Situation Calculus and Golog are viable tools for specifying
and implementing agent behaviour is highlighted by the many successful
applications of the techniques in different planning tasks, including:
open-world planning \citep{Finzi00open_world_sitcalc}; planning with
natural actions \citep{pirri00planning_nat_acts}; planning under
uncertainty \citep{baier03golog_planning}; control and coordination
in robotic soccer \citep{Ferrein2005readylog}; and specifying the
behaviour of computer-controlled characters in video games \citep{jacobs05unrealgolog}.

In more theoretical settings, the situation calculus has enabled reasoning
about multi-agent games \citep{delgrande01sitcalc_cleudo} and automated
verification of multi-agent system specifications \citep{shapiro02casl,lesperance05ecasl}.
It has formed the basis for a standard process specification language
in manufacturing \citep{gruninger04psl} and has been applied to the
automated composition of semantic web services \citep{mcilraith02golog_web_services}.

TODO: more on applications - e.g. why are we discussing them here?
Do we even need to?


\section{Golog\label{sec:Background:Golog}}

Golog is a declarative agent programming language based on the situation
calculus \citep{levesque97golog}. Testimony to its success are its
wide range of applications and many extensions to provide additional
functionality \citep{giacomo00congolog,giacomo99indigolog,Ferrein2005readylog}.
We use the name {}``Golog'' to refer to the family of languages
based on this technique, including ConGolog \citep{giacomo00congolog}
and IndiGolog \citep{giacomo99indigolog}.


\subsection{Notation}

To program an agent using Golog one specifies a situation calculus
theory of action, and a program consisting of actions from the theory
connected by programming constructs such as if-then-else, while loops,
and nondeterministic choice. Table \ref{tbl:Background:Golog-Operators}
lists the some of the operators available in various incarnations
of the language.%
\begin{table}[h]
 

\begin{centering}
\begin{tabular}{|c|c|}
\hline 
Operator  & Meaning\tabularnewline
\hline
\hline 
$Nil$  & The empty program\tabularnewline
\hline 
$a$  & Execute action $a$ in the world\tabularnewline
\hline 
$\phi?$  & Proceed if condition $\phi$ is true\tabularnewline
\hline 
$\delta_{1};\delta_{2}$  & Execute $\delta_{1}$followed by $\delta_{2}$\tabularnewline
\hline 
$\delta_{1}|\delta_{2}$  & Execute either $\delta_{1}$ or $\delta_{2}$\tabularnewline
\hline 
$\pi(x,\delta(x))$  & Nondet. select arguments for $\delta$\tabularnewline
\hline 
$\delta*$  & Execute $\delta$ zero or more times\tabularnewline
\hline 
$\mathbf{if}\,\phi\,\mathbf{then}\,\delta_{1}\,\mathbf{else}\,\delta_{2}$  & Exec. $\delta_{1}$ if $\phi$ holds, $\delta_{2}$ otherwise\tabularnewline
\hline 
$\mathbf{while\,}\phi\mathbf{\, do}\,\delta$  & Execute $\delta$ while $\phi$ holds\tabularnewline
\hline 
$\mathbf{proc}P(\overrightarrow{x})\delta(\overrightarrow{x})\mathbf{end}$  & Procedure definition\tabularnewline
\hline 
$\delta_{1}||\delta_{2}$  & Concurrent execution\tabularnewline
\hline 
$\Sigma\delta$  & Plan execution offline\tabularnewline
\hline
\end{tabular}
\par\end{centering}

\caption{Golog Operators used in this thesis\label{tbl:Background:Golog-Operators} }

\end{table}


Readers familiar with dynamic logic will recognise some of these operators,
but others are unique to first-order formalisms such as Golog. Many
Golog operators are nondeterministic and may be executed in several
different ways. It is the task of the agent to plan a deterministic
instantiation of the program, a sequence of actions that can legally
be performed in the world. Such a sequence is called a \emph{legal
execution} of the program.

To get a feel for how these operators can be used, consider some example
programs. Figure \ref{fig:Background:Golog:Washing-Dishes} shows
a simple program for Bob to wash the dishes. It makes use of the nondeterministic
{}``pick'' operator to select and clean a dish that needs washing,
and does so in a loop until no dirty dishes remain. The legal executions
of this program are sequences of $clean(Bob,d)$ actions, one for
each dirty dish in the domain, performed in any order.

%
\begin{figure}
\begin{centering}
\framebox{%
\begin{minipage}[t][1\totalheight]{0.85\columnwidth}%
\begin{flalign*} \mathbf{while}\,\exists d:\, Dirty(d)\,\mathbf{do}\\
 \pi(d,\, clean(Bob,d))\\
 \mathbf{end}\end{flalign*} %
\end{minipage}} 
\par\end{centering}

\caption{A Golog program for washing the dishes\label{fig:Background:Golog:Washing-Dishes}}

\end{figure}


Figure \ref{fig:Background:Golog:MakeSalad} shows a program that
we will return to in subsequent chapters, giving instructions for
how to prepare a simple salad. The procedure $ChopTypeInto$ (not
shown) directs the specified agent to acquire an ingredient of the
specified type, chop it, and place it into the indicated bowl. The
procedure $MakeSalad$ nondeterministically selects an agent to do
this for a lettuce, a carrot, and a tomato. Note the nondeterminism
in this program: the agent assigned to handling each ingredient is
not specified ($\pi$ construct), nor is the order in which they should
be processed ($||$ construct). There is thus considerable scope for
cooperation between agents to effectively carry out this task.

%
\begin{figure}
\begin{centering}
\framebox{%
\begin{minipage}[t][1\totalheight]{0.85\columnwidth}%
\begin{multline*}
\mathbf{proc}\, MakeSalad(dest)\\
\left[\pi(agt,\, ChopTypeInto(agt,Lettuce,dest))\,||\right.\\
\pi(agt,\, ChopTypeInto(agt,Carrot,dest))\,||\\
\left.\pi(agt,\, ChopTypeInto(agt,Tomato,dest))\right]\,;\\
\pi(agt,\,\left[acquire(agt,dest)\,;\,\right.\\
beginTask(agt,mix(dest,1))\,;\\
\left.\, release(agt,dest)\right])\,\,\mathbf{end}\end{multline*}
 %
\end{minipage}} 
\par\end{centering}

\caption{A Golog program for making a salad\label{fig:Background:Golog:MakeSalad}}

\end{figure}



\subsection{Semantics}

The original semantics of Golog were defined using macro-expansion
\citep{levesque97golog}. The macro $\Do(\delta,s,s')$ was defined
to be true if program $\delta$ could be successfully executed in
situation $s$, leaving the world in situation $s'$. However, these
semantics could not support the concurrent execution of two programs
and were modified with the introduction of ConGolog \citep{giacomo00congolog}
to use two predicates $Trans(\delta,s,\delta',s')$ and $Final(\delta,s)$
which are capable of representing single-steps of execution of the
program.

The predicate $Trans(\delta,s,\delta',s')$ holds when executing a
step of program $\delta$ can cause the world to move from situation
$s$ to situation $s'$, after which $\delta'$ remains to be executed.
It thus characterises single steps of computation. The predicate $Final(\delta,s)$
holds when program $\delta$ may legally terminate its execution in
situation $s$. We base our work on the semantics of IndiGolog, which
builds on ConGolog and is the most feature-full of the standard Golog
variants. The full semantics are available in the references \citep{giacomo00congolog,giacomo99indigolog},
but we present some illustrative examples below.

The transition rule for a programing consisting of a single action
is straightforward -- it transitions by performing the action, provided
it is possible in the current situation. Such a program may not terminate
its execution since the action remains to be performed:\begin{alignat*}{1}
Trans(a,s,\delta',s')\,\equiv\, & \, Poss(a,s)\wedge\delta'=Nil\wedge s'=do(a,s)\\
Final(a,s)\,\equiv\, & \, false\end{alignat*}


The transition rule for a test operator proceeds only if the test
is true, leaving the situation unchanged, and likewise cannot terminate
execution until the test has been satisfied:\begin{alignat*}{1}
Trans(?\phi,s,\delta',s')\,\equiv\, & \,\phi[s]\wedge\delta'=Nil\wedge s'=s\\
Final(?\phi,s)\,\equiv\, & \, false\end{alignat*}


Now consider a simple nondeterministic operator, the {}``choice''
construct that performs one of two alternate programs:\begin{alignat*}{1}
Trans(\delta_{1}|\delta_{2},s,\delta',s')\,\equiv\, & \, Trans(\delta_{1},s,\delta',s')\,\vee\, Trans(\delta_{2},s,\delta',s')\\
Final(\delta_{1}|\delta_{2},s)\,\equiv\, & \, Final(\delta_{1},s)\,\vee\, Final(\delta_{2},s)\end{alignat*}


It is possible for this operator to transition in two different ways
- by executing a step of execution from the first program, or a step
of execution from the second program. Slightly more complicated, but
of fundamental important in the next chapter, is the semantics of
the concurrency operator:\begin{alignat*}{1}
Trans(\delta_{1}||\delta_{2},s,\delta',s')\,\equiv\, & \,\exists\gamma:\, Trans(\delta_{1},s,\gamma,s')\,\wedge\,\delta'=(\gamma||\delta_{2})\\
 & \,\vee\,\exists\gamma:\, Trans(\delta_{2},s,\gamma,s')\,\wedge\,\delta'=(\delta_{1}||\gamma)\\
Final(\delta_{1}||\delta_{2},s)\,\equiv\, & \, Final(\delta_{1},s)\,\wedge\, Final(\delta_{2},s)\end{alignat*}


This rule specifies the concurrent-execution operator as an \emph{interleaving}
of computation steps. It states that it is possible to single-step
the concurrent execution of $\delta_{1}$ and $\delta_{2}$ by performing
either a step from $\delta_{1}$ or a step from $\delta_{2}$, with
the remainder $\gamma$ left to execute concurrently with the other
program

Clearly there are two notions of concurrency to be considered in the
situation calculus: the possibility of performing several actions
at the same instant (\emph{true concurrency}), and the possibility
of interleaving the execution of several programs (\emph{interleaved
concurrency}). \citet{pinto99tcongolog} have modified ConGolog to
incorporate sets of concurrent actions in an attempt to integrate
these two forms of concurrency. However, their semantics may call
for actions to be performed that are not possible and can also produce
unintuitive program behaviour in some cases. A key aspect of our work
in chapter \ref{ch:mindigolog} is a more robust integration of these
two notions of concurrency.

TODO: semantics of search operator? see what's needed for next chapter.\\


We have omitted many details here that are not relevant to this thesis,
such as the second-order axioms necessary to handle recursive procedure
definitions. We will denote by $\Dt_{golog}$ the $Trans$ and $Final$
axioms defining the Golog operators.


\subsection{Executing Planning}

Planning an execution of a Golog program $\delta$ can be reduced
to a theorem proving task as shown in equation (\ref{eqn:Background:golog_execution}).
Here $Trans^{*}$ indicates the standard second-order definition for
the reflexive transitive closure of $Trans$.\begin{equation}
\Dt\cup\Dt_{golog}\models\exists s,\delta':\,\left[Trans^{*}(\delta,S_{0},\delta',s)\wedge Final(\delta',s)\right]\label{eqn:Background:golog_execution}\end{equation}


A constructive proof of this query would produce an instantiation
of $s$, a situation term giving a sequence of actions constituting
a legal execution of the program. In the original Golog and in ConGolog
this forms the entirety of the execution planning process, as these
variants require a full legal execution to be planned before any actions
are performed in the world.

%
\begin{algorithm}[t]
 

\caption{The Golog/ConGolog Execution Algorithm}


\label{alg:golog_exec} \begin{algorithmic} \STATE Find a situation
$s$ such that: $\Dt\cup\Dt_{golog}\models\exists\delta':\ \left[Trans^{*}(\delta,S_{0},\delta',s)\wedge Final(\delta',s)\right]$
\FOR{each action in the resulting situation term} \STATE execute
that action \ENDFOR \end{algorithmic} 
\end{algorithm}


By contrast, IndiGolog allows agents to proceed without planning a
full terminating execution of their program, instead searching for
a legal next step $a$ in the current situation $\sigma$ such that:\[
\Dt\cup\Dt_{golog}\,\models\,\exists a,\delta':\, Trans^{*}(\delta,\sigma,\delta',do(a,\sigma))\]


This next step is then performed immediately, and the process repeats
until a terminating configuration is reached. This allows the agents
to include sensing actions in their programs without having to plan
for every possible outcome of the action.

%
\begin{algorithm}[t]
 

\caption{The IndiGolog Execution Algorithm}


\label{alg:indigolog_exec} \begin{algorithmic} \STATE $s\ \Leftarrow\ S_{0}$
\WHILE{$\Dt\cup\Dt_{golog}\not\models Final(\delta,s)$} \STATE
Find an action $a$ and program $\delta'$ such that: $\Dt\cup\Dt_{golog}\models Trans^{*}(\delta,s,\delta',do(a,s))$
\STATE Execute the action $a$ \STATE $s\ \Leftarrow\ do(a,s)$
\STATE $\delta\ \Leftarrow\ \delta'$ \ENDWHILE \end{algorithmic} 
\end{algorithm}


TODO: semantics of plan operator

TODO: histories, planning with sensing results, as needed for JE chapter.


\subsection{Extensions}

There has been a wide range of Golog extensions developed which we
will not consider in this thesis. Among them have been extensions
to include decision-theoretic \citep{boutilier00dtgolog} and game-theoretic
aspects \citep{finzi03gtgolog,finzi05pogtgolog}, additional control
operators such as partially-ordered sequences of actions \citep{son00htn_golog}
and hierarchical task networks \citep{Gabaldon02htn_in_golog,Son04golog+htn+time},
synchronisation between the individual programs of a team of agents
\citep{farinelli07team_golog}, and accounting for continuous change
and event triggering \citep{grosskreutz00ccgolog}.

While we will not consider these Gologs in any detail, we do note
that each has been a relatively straightforward matter of extending
the underlying situation calculus theory and/or the semantics of the
Golog operators, and as a result there has been rich cross-pollination
between different works. We therefore expect that our work may in
turn be combined with some of these extensions to provide an even
richer formalism.


\section{Related Formalisms}

\label{sec:Background:Related-Formalisms}

There are a range of related formalisms for reasoning about knowledge,
action and change, which we do not directly consider in this thesis.
Most closely related to the situation calculus are the fluent calculus
of \citet{thielscher98fluent_calculus} and the event calculus of
\citet{kowalski86event_calculus}.

The fluent calculus is based explicitly on the use of progression
for solving the projection problem, and so maintains an explicit representation
of the state of the world which is updated as actions are performed.
It can derived from the situation calculus by transforming successor
state axioms into state update axioms the explicitly add and remove
fluents from the state \citep{thielscher99fluentcalc_from_sitcalc}.
Notably, it is relatively straightforward to interpret Golog programs
on top of the fluent calculus \citep{thielscher05golog_in_flux}.
As discussed in section TODO, it would be interesting to translate
our regression-based ideas into a progression-based formalism such
as this.

The event calculus is slightly further removed, in that it contains
a single linear timeline rather than the branching time structure
of the situation calculus. This makes it more suitable for representing
some domains and posing some queries, but less suitable for other;
a detailed comparison can be found in \citep{kowalski97reconcile_sitcalc_evtcalc,belleghem97sitcalc_evtcalc}.
Like the situation calculus, it has found significant applications
in a logical approach to planning and agent control \citep{shanahan00ec_planner}.
One particular strength of the event calculus is in planning with
partially-ordered sequences of events, highlighted by a reimplementation
of Golog using the event calculus that supports partially-ordered
plans \citep{pereira04ec_golog}. We add a similar ability to the
situation calculus in this thesis.

There is also the family of approaches known as {}``dynamic epistemic
logic'', which are based on modal logic \citep{baltag98pa_ck,vanBenthem06lcc,vanBentham06tree_of_knowledge}
These formalisms are typically propositional rather than first-order,
and focus more on reasoning about knowledge and communication than
on modelling a changing dynamic world. However, there are still strong
similarities with the situation calculus \citep{vanbentham07ml_sitcalc}.
In chapter \ref{ch:cknowledge} we will adapt some techniques from
these formalisms to model common knowledge in the situation calculus.\\


There have been many attempts to unite the various action formalisms
into a unifying theory of action, including \citep{belleghem95combine_sitcalc_evtcalc,kowalski97reconcile_sitcalc_evtcalc,thielscher06reconcile_sc_fc,thielscher07unifying_action_calculus},
but there is yet to emerge a clear standard in this regard. In the
meantime, we find the notation and meta-theory of the situation calculus
particularly suitable for expressing our main ideas, and hope that
the strong underlying similarities between the major action formalisms
will allow the these ideas to transcend the specifics of the situation
calculus.


\section{Mozart/Oz\label{sec:Background:Mozart/Oz}}

One of the main advantages of the situation calculus and Golog are
their straightforward implementation as a logic program. As the dominant
implementation of logic programming, Prolog is typically used for
such implementations. In this thesis we use Mozart, a multi-paradigm
programming system with some unique features that are particularly
suited to our work.

The Mozart system \citep{vanroy99mozart} is an implementation of
the Oz programming language \citep{vanRoyHaridi04ctm} with strong
support for logic programming and distributed computing. While a full
explanation of its features is well outside the scope of this thesis,
we provide a short introduction to the subset of its features we will
be using -- in particular, doing prolog-style logic programming in
Oz. Familiarity with logic programming in the style of prolog is assumed.

Terms, variables and unification in Mozart work similarly to prolog,
although arguments in compound terms are separated by whitespace rather
than a comma. Predicates are implemented as ordinary procedures, so
all clauses for a predicate must be contained in a single procedure.
Figure \ref{fig:Background:Naive-List-Reverse} shows a Mozart implementation
of a classic Prolog example predicate, naive list reverse. Some things
to note about this example include:

\begin{itemize}
\item The syntax for procedure definition is $\mathbf{proc}\,\{Name\, Arg\,\dots\,\}$ 
\item The syntax for procedure calls is $\{Name\, Arg\,\dots\,\}$ 
\item The $\mathbf{case}$ statement is used to pattern match the contents
of a variable 
\item Local variables must be explicitly introduced using the keyword $\mathbf{in}$ 
\item Mozart separates functionality into modules, such as $List$ 
\end{itemize}
%
\begin{figure}[t]
\programinput{listings/background/Reverse.oz}

\caption{Naive List Reverse implemented in Mozart/Oz\label{fig:Background:Naive-List-Reverse}}

\end{figure}


Procedures in Mozart are deterministic by default, and there is no
default search strategy for exploring different alternatives. Instead,
Mozart provides independent facilities for creating choicepoints and
for exploring procedures that contain choicepoints. The result is
a much more flexible, although syntactically more cumbersome, approach
to logic programming \citep{lpinoz99}.

The creation of choice points is explicit in Mozart, and performed
using the $\mathbf{choice}$ keyword. To demonstrate, consider another
classic Prolog example: the nondeterministic list member predicate
demonstrated in figure \ref{fig:Background:Nondet-Member}. In the
case of the empty list, $Member$ simply fails. For a non-empty list,
$Member$ explicitly creates a \emph{choice point} with two options
- either bind $E$ to the head of the list, or bind $E$ to a member
of the tail of the list.

%
\begin{figure}[t]
\programinput{listings/background/Member.oz}

\caption{Nondeterministic List Member implemented in Mozart/Oz\label{fig:Background:Nondet-Member}}

\end{figure}


It is at this point that the use of Mozart for logic programming differs
most from Prolog. If the $Member$ procedure is simply invoked directly,
it will suspend its execution when the $\mathbf{choice}$ statement
is reached. To resolve the nondeterminism, one must execute the procedure
inside an explicit \emph{search} \emph{object}. These objects are
responsible for exploring the various choicepoints until a non-failing
computation is achieved. They operate by executing the procedure in
a separate \emph{computation space} through which the state of the
underlying computation can be managed \citep{schulte00constraint_services}.

As a demonstration, figure \ref{fig:Background:All-Pairs} uses the
$Member$ procedure to define a procedure $Pairs$, which nondeterministically
selects a pair of elements from a pair of lists. The procedure $AllPairs$
then uses the builtin $Search.base.all$ object to find all solutions
from this procedure, returning a list of all possible pairs from the
two lists. By encapsulating the calls to nondeterministic procedures
inside a search object, $AllPairs$ will not expose any choicepoints
to code that calls it.

Also of note in figure \ref{fig:Background:All-Pairs} is the use
of a \emph{closure} over the procedure $Pair$ to create the one-argument
procedure $FindP$. Search objects work with a one-argument procedure,
which is expected to bind its argument to a result. The dollar symbol
is used to translate a statement (in this case the $\mathbf{proc}$
definition) into an expression. The value that would be bound to the
dollar symbol by the statement becomes the return value of the expression,
so $FindP=proc\,\{\$\, P\}$ is equivalent to $proc\,\{FindP\, P\}$.

%
\begin{figure}[t]
\programinput{listings/background/Pairs.oz}

\caption{Finding all pairs in Mozart/Oz\label{fig:Background:All-Pairs}}

\end{figure}


The power of this decoupled approach to nondeterminism becomes apparent
when defining new search strategies, which can then be used to evaluate
any procedure. For example, it is straightforward to implement breadth-first
or iterative-deepening strategies to replace the standard depth-first
traversal of the $Search.base$ object \citep{schulte00constraint_services}.

Coupled with Mozart's strong support for distributed computing, these
programmable search strategies offer a unique opportunity - it becomes
possible to implement a parallel search object which can automatically
distribute work between several networked machines. Moreover, this
parallel search can be applied without modification to any nondeterministic
procedure. Mozart comes with a built-in $ParallelSearch$ object,
which is described in detail in \citep{schulte00oz_parallel} and
which is our main motivation for the use of Oz in this thesis.

To demonstrate the power of the approach, consider figure \ref{fig:Background:Parallel-All-Pairs},
which describes a parallel-search version of the $AllPairs$ procedure.
In this instance we define $FindP$ as a \emph{functor}, a Mozart
abstraction for code that is portable between machines. This functor
imports the module $MyList$ containing the procedures we defined
earlier, and exports a one-argument procedure $Script$ which will
be executed by the parallel search object. The parallel search object
$Seacher$ launches one instance of Mozart on the machine {}``mango''
and two instances on the machine {}``rambutan'', then is asked to
enumerate all solutions for $FindP$.

%
\begin{figure}[t]
\programinput{listings/background/PPairs.oz}

\caption{Finding all pairs in Mozart/Oz\label{fig:Background:Parallel-All-Pairs}}

\end{figure}


In chapter \ref{ch:mindigolog} we will use this parallel search object
to automatically share the workload of planning a Golog execution
amongst a team of cooperating agents.

As a multi-paradigm programming language with significant research
history, there is much more to Oz than we have described here. However,
these brief examples should be sufficient for a reader well-versed
in Prolog to understand the Oz code used throughout this thesis. For
more information and further examples, consult the general Oz tutorial
\citep{haridi99oz_tutorial} or the specialised tutorial on logic
programming in Oz \citep{lpinoz99}, which are both available online.

\newpage{}


\section{TODO}

Reference and discuss \citep{pinto98sc_observations} in great detail

More on advantages of Golog other other approaches? (e.g. HTN stuff
from MIndiGolog paper?)

Exlicitly define Exogenous actions

Leftover paragraph: Among the advantages offered by the situation
calculus are: straightforward representations of rich domain features
such as concurrent actions and continuous time; effective reasoning
procedures based on regression; flexible task specification using
familiar programming constructs; a mature formalism for epistemic
reasoning; and a straightforward implementation by translation into
a logic program. But as we will highlight throughout this chapter,
the situation calculus also currently suffers several limitations
that make it unsuitable for reasoning about asynchronous multi-agent
domains. It is these limitations that are overcome in this thesis.

