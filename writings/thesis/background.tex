\chapter{Background}\label{ch:lit-review}
%\minitoc

This chapter will largely be an expansion of the literature review from my 2006 progress report, with some new material on first-order logic.

Remaining work: review reference list for quality and completeness.

\section{Distributed Problem Solving}

\begin{itemize}
\item Quickly highlight difference between open multi-agent systems and
distributed problem solving - explain what we're \emph{not} interested in.
\item Overview of DPS formalisms, task representation (HTN, TAEMS, etc...)
\item Briefly introduce HLP approach, discuss advantages.
\item Highlight success of Readylog team, but discuss limitations.
\item Overview of coordination techniques, esp. social laws for coordination without communication
\item Overview of distributed planning techniques
\end{itemize}

\section{First-Order Logic}

\begin{itemize}
\item Basic definitions
\item Theorem proving
\item Shannon Graphs
\end{itemize}

\section{The Situation Calculus}

\begin{itemize}
\item Basic Sitcalc
\item Our own customizations, e.g. "Action Description Predicates"
\item Concurrent Actions, Continuous Time, Natural Actions
\item Reasoning (Regression, Decidable Fragments)
\item Related approaches (Fluent Calculus, Event Calculus)
  \begin{itemize}
  \item justify focusing on sitcalc by highlighting deep links between the formalisms
  \end{itemize}
\item Briefly highlight things we \emph{don't} consider - nondeterministic/probabilistic actions, decision theoretic aspects, etc - and explain why.
\end{itemize}

\section{High-Level Program Execution}

\begin{itemize}
\item Basic Golog, operators, semantics
\item Discuss idea of 'Legal Execution' in depth
\item ConGolog
\item IndiGolog - highlight online/offline distinction and connections to coordination and planning
\item TConGolog - highlight deficiencies
\item Other Gologs (DTGolog, HTNGolog, GTGolog)
  \begin{itemize}
  \item for completeness only, we have already established that we aren't operating in domains appropriate for them
  \item But, take time to highlight rich cross-pollination between works, and that our results similarly have potential for integration with these approaches
  \end{itemize}
\end{itemize}

\section{Epistemic Reasoning}

\begin{itemize}
\item Knowledge, Distributed Knowledge, Common Knowledge
\item Importance of Common Knowledge for Communication
\item Knowledge in the Situation Calculus
  \begin{itemize}
  \item Reasoning about Knowledge
  \item Shortcomings of multi-agent extensions
  \item Decidable fragments, weakening for computational efficiency
  \end{itemize}
\end{itemize}

