

\chapter{Background}

\label{ch:background}

This chapter covers general background material for the thesis and
provides a brief overview of the related literature. We defer more
specific technical details and discussion of related work to the individual
chapters that follow, where it can be presented in the appropriate
context.

Readers familiar with the situation calculus are encouraged to briefly
review this chapter. While it does not present any new results, it
does introduce some novel notation and definitions which will be needed
later in the thesis. They are introduced here to maintain consistency
of the presentation. The introductory material on the Mozart programming
platform may also be helpful.

We begin by introducing the base language of the situation calculus
in Section \ref{sec:Background:The-Situation-Calculus}, illustrated
using examples from the {}``cooking agents'' domain. Section \ref{sec:Background:Golog}
introduces the Golog family of programming languages, which are the
standard formalism for representing complex tasks in the situation
calculus. Reasoning about the knowledge of an agent, or \emph{epistemic
reasoning}, is covered in Section \ref{sec:Background:Epistemic}.
Related formalisms for reasoning about action and change are briefly
discussed in Section \ref{sec:Background:Related-Formalisms}. Finally,
Section \ref{sec:Background:Mozart/Oz} introduces the Mozart programming
system, which will be used to implement our multi-agent Golog variant.
Basic familiarity with formal logic is assumed throughout; readers
requiring background on such material may find a gentle introduction
in \citep{kelly96logic} and a more detailed treatment in \citep{fitting96fol_book}.


\section{The Situation Calculus\label{sec:Background:The-Situation-Calculus}}

The situation calculus is a powerful formalism for describing and
reasoning about dynamic worlds. It was first introduced by \citet{McCHay69sitcalc}
and has since been significantly expanded and formalised \citep{reiter91frameprob,pirri99contributions_sitcalc}.
We use the particular variant due to Reiter et. al. at the University
of Toronto, sometimes called the {}``Toronto school'' or {}``situations-as-histories''
version. The formalisation below is based on the standard definitions
from \citep{levesque98sc_foundations,pirri99contributions_sitcalc,reiter01kia},
but has been slightly generalised to accommodate several existing
extensions to the situation calculus, as well as our own forthcoming
extensions, in a uniform manner.

Readers familiar with the situation calculus should therefore note
some modified notation: the unique names axioms $\Dt_{una}$ are incorporated
into a general background theory $\Dt_{bg}$; the $Poss$ fluent is
subsumed by a general class of \emph{action description predicates}
defined in $\Dt_{ad}$; we parameterise the {}``future situations''
predicate $s\sqsubset s'$ to assert that all intermediate actions
satisfy a given predicate using $s<_{\alpha}s'$; and we use the single-step
variant of the regression operator, with corresponding definitions
of regressable formulae.


\subsection{Notation\label{sec:Background:SC:Notation}}

The language $\Lsit$ of the situation calculus is a many-sorted language
of first-order logic with equality, augmented with a second-order
induction axiom, containing the following disjoint sorts:

\begin{itemize}
\item \emph{\noun{Action}} terms are functions denoting individual instantaneous
events that can cause the state of the world to change; 
\item \noun{Situation} terms are histories of the actions that have occurred
in the world, with the initial situation represented by $S_{0}$ and
successive situations built using the function $do\,:\, Action\times Situation\rightarrow Situation$; 
\item \noun{Object} terms represent any other object in the domain. 
\end{itemize}
\emph{Fluents} are predicates or functions that represent properties
of the world that may change between situations, and so take a situation
term as their final argument. Predicates and functions that do not
take a situation term are called \emph{rigid}. We use the term \emph{primitive
fluent} to describe fluents that are directly affected by actions,
rather than being defined in terms of other fluents. No functions
other than $S_{0}$ and $do$ produce values of sort \noun{Situation}.

For concreteness, let us present some formulae from an example domain
that will be used throughout the thesis. In the {}``cooking agents''
domain a group of robotic chefs inhabit a kitchen containing various
ingredients and utensils, and they must cooperate to prepare a meal.
Some example statements from this domain include {}``Joe does not
have the knife initially'', {}``Jim has the knife after he acquires
it'' and {}``It is only possible to acquire an object if nobody
else has it''. Formally:\begin{gather*}
\neg HasObject(Joe,Knife1,S_{0})\\
HasObject(Jim,Knife1,do(acquire(Jim,Knife1),S_{0}))\\
Poss(acquire(agt,obj),s)\equiv\neg\exists agt_{2}:\, HasObject(agt_{2},obj,s)\end{gather*}


Here $HasObject$ is a primitive fluent, while $Poss$ is defined
in terms of it.\medskip{}


$\Lsit$ contains the standard alphabet of logical connectives, constants
$\top$ and $\ \bot$, countably infinitely many variables of each
sort, countably infinitely many predicates of each arity, etc; for
a complete definition, consult the foundational paper by \citet{pirri99contributions_sitcalc}.
We follow standard naming conventions for the situation calculus:
upper-case roman names indicate constants; lower-case roman names
indicate variables; greek characters indicate meta-variables or formula
templates. All axioms universally close over their free variables
at outermost scope. The notation $\vars{t}$ indicates a vector of
terms of context-appropriate arity and type. The connectives $\wedge$,
$\neg$, $\exists$ are taken as primitive, with $\vee$, $\rightarrow$,
$\equiv$, $\forall$ defined in the usual manner.

In multi-agent domains it is customary to introduce a distinct sort
\noun{Agent} to explicitly represent the agents operating in the world,
and we will do so here. As seen in the example formulae above, the
first argument of each action term gives the performing agent, which
can be accessed by the function $actor(a)$.

Complex properties of the state of the world are represented using
\emph{uniform formulae}. These are basically logical combinations
of fluents referring to a common situation term.

\begin{defnL}
[{Uniform~Terms}] Let $\sigma$ be a fixed situation term,
$r$ an arbitrary rigid function symbol, $f$ an arbitrary fluent
function symbol, and $x$ a variable that is not of sort \noun{Situation}.
Then the terms uniform in $\sigma$ are the smallest set of syntactically-valid
terms satisfying:\[
\tau\,::=x\,|\, r(\vars{\tau})\,|\, f(\vars{\tau},\sigma)\]

\begin{defnL}
[{Uniform~Formulae}] Let $\sigma$ be a fixed situation
term, $R$ an arbitrary rigid predicate, $F$ an arbitrary primitive
fluent predicate, $\tau$ an arbitrary term uniform in $\sigma$,
and $x$ an arbitrary variable that is not of sort \noun{Situation}.
Then the formulae uniform in $\sigma$ are the smallest set of syntactically-valid
formulae satisfying:\[
\phi::=F(\vars{\tau},\sigma)\,|\, R(\vars{\tau})\,|\,\tau_{1}=\tau_{2}\,|\,\phi_{1}\wedge\phi_{2}\,|\,\neg\phi\,|\,\exists x:\phi\]

\end{defnL}
\end{defnL}
We will call a formula \emph{uniform} if it is uniform in some situation.
The important aspect of this definition is that the formula refers
to no situation other than $\sigma$, which appears as the final argument
of all fluents in the formula. In particular, uniform formulae cannot
quantify over situations or compare situation terms, and cannot contain
non-primitive fluents.

The meta-variable $\phi$ is used throughout to refer to an arbitrary
uniform formula. Since they represent some aspect of the state of
the world, it is frequently useful to evaluate uniform formulae at
several different situation terms. The notation $\phi[s']$ represents
a uniform formula with the particular situation $s'$ inserted into
all its fluents. We may also completely suppress the situation term
to simplify the presentation, using $\phi^{-1}$ to represent a uniform
formula with the situation argument removed from all its fluents.
For example, given: \[
\phi=HasObject(Jim,Knife1,s)\wedge HasObject(Joe,Bowl2,s)\]
 Then we have:\begin{gather*}
\phi[s']\,=\, HasObject(Jim,Knife1,s')\wedge HasObject(Joe,Bowl2,s')\\
\phi^{-1}\,=\, HasObject(Jim,Knife1)\wedge HasObject(Joe,Bowl2)\end{gather*}


Note that these are strictly meta-level operations, corresponding
to possibly quite complex sentences from the underlying logic. They
are \emph{not} terms or operators from the logic itself.


\subsection{Axioms\label{sec:Background:SC:Axioms}}

The dynamics of a particular domain are captured by a set of sentences
from $\Lsit$ called a \emph{basic action theory}. Queries about the
behaviour of the world are posed as logical entailment queries relative
to this theory.

\begin{defnL}
[{Basic~Action~Theory}] A basic action theory, denoted
$\Dt$, is a set of situation calculus sentences (of the specific
syntactic form outlined below) describing a particular dynamic world.
It consists of the following disjoint sets: the foundational axioms
of the situation calculus ($\Sigma$); action description axioms defining
preconditions etc for each action ($\Dt_{ad}$); successor state axioms
describing how primitive fluents change between situations ($\Dt_{ssa}$);
axioms describing the value of primitive fluents in the initial situation
($\Dt_{S_{0}}$); and axioms describing the static background facts
of the domain ($\Dt_{bg}$):\[
\Dt=\Sigma\cup\Dt_{ad}\cup\Dt_{ssa}\cup\Dt_{S_{0}}\cup\Dt_{bg}\]

\end{defnL}
These axioms must satisfy some simple consistency criteria to constitute
a valid domain description; see \citep{pirri99contributions_sitcalc}
for the details. This definition is slightly broader than the standard
definitions found in the literature \citep{levesque98sc_foundations,pirri99contributions_sitcalc,reiter01kia}
and is designed to accommodate a variety of extensions to the situation
calculus in a uniform manner.

We assume an arbitrary, but fixed, basic action theory $\Dt$.


\subsubsection{Background Axioms}

The set $\Dt_{bg}$ characterises the static aspects of the domain,
and contains all axioms defining rigid predicates or functions. In
particular, it must contain a set of unique names axioms asserting
that action terms with different types or arguments are in fact different,
e.g.:\begin{gather*}
acquire(agt,obj)\neq release(agt,obj)\\
acquire(agt_{1},obj_{1})=acquire(agt_{2},obj_{2})\,\rightarrow\, agt_{1}=agt_{2}\,\wedge\, obj_{1}=obj_{2}\end{gather*}


It also contains domain closure axioms for the sorts \noun{Action,
Agent} and \noun{Object}, and defines the function $actor(a)$ to
give the agent performing an action. The background axioms are a generalisation
of the set $\Dt_{una}$ commonly found in the literature, which contains
only the unique names axioms.


\subsubsection{Successor State Axioms}

The set $\Dt_{ssa}$ contains one \emph{successor state axiom} for
each primitive fluent in the domain. These axioms provide an elegant
monotonic solution to the frame problem for that fluent \citep{reiter91frameprob}
which has been instrumental to the popularity and utility of the situation
calculus. They have the following general form:\[
F(\vars{x},do(a,s))\,\equiv\,\Phi_{F}(\vars{x},a,s)\]
 Here $\Phi_{F}$ is uniform in $s$. While we will make no assumptions
about the internal structure of $\Phi_{F}$, it typically takes the
form shown below, which may help elucidate the purpose of these axioms:
\[
F(\vars{x},do(a,s))\equiv\Phi_{F}^{+}(\vars{x},a,s)\,\,\vee\,\, F(\vars{x},s)\wedge\neg\Phi_{F}^{-}(\vars{x},a,s)\]


Here $\Phi_{F}^{+}$ and $\Phi_{F}^{-}$ are formulae uniform in $s$,
representing the positive and negative effect axioms for that fluent.
This may be read as {}``$F$ is true after performing $a$ if $a$
made it true, or it was previously true and $a$ did not make it false''.
For example, the dynamics of the $HasObject$ fluent may be specified
using:\begin{multline*}
HasObject(agt,obj,do(a,s))\,\equiv\, a=acquire(agt,obj)\\
\vee\,\, HasObject(agt,obj,s)\wedge a\neq release(agt,obj)\end{multline*}


For functional fluents, $\Dt_{ssa}$ contains a similar axiom to specify
the value $v$ of the fluent after an action has occurred:\[
f(\vars{x},do(a,s))=v\,\equiv\,\Phi_{f}(v,\vars{x},a,s)\]



\subsubsection{Action Description Predicates}

The set $\Dt_{ad}$ generalises the standard \emph{action precondition
axioms} \citep{pirri99contributions_sitcalc} to define fluents that
describe various aspects of the performance of an action, which we
call \emph{action description predicates}. These are the only non-primitive
fluents permitted in a basic action theory. The predicate $Poss(a,s)$
is the canonical example, indicating whether it is possible to perform
an action in a given situation. The set $\Dt_{ad}$ contains a single
axiom of the following form, defining the complete set of preconditions
for the action variable $a$, where $\Pi_{Poss}$ is a formula uniform
in $s$:\[
Poss(a,s)\,\equiv\,\Pi_{Poss}(a,s)\]


Note that this is a slight departure from the standard approach of
\citep{pirri99contributions_sitcalc}, in which the preconditions
for each action type are enumerated individually. The more restrictive
approach presented here embodies a domain-closure assumption on the
\noun{Action} sort. If there are finitely many action types then $\Pi_{Poss}$
is simply the completion of the precondition axioms for each action
type. The single-axiom form is necessary when quantifying over {}``all
possible actions'' and has been widely used in the literature \citep{vassos08progression_future_queries,savelli06sc_quantum_levels}.

In principle, any number of predicates and functions can be defined
in this way; a common example is the sensing-result function $SR(a,s)$
which we will describe in Chapter \ref{ch:observations}. The general
notion of an action description predicate allows us to treat all of
them in a uniform manner. We will use the meta-variable $\alpha$
to represent an arbitrary action description predicate, and allow
the action and situation arguments to be suppressed in a similar way
to situation-suppressed uniform formulae.

In preparation for the coming material on extensions to the situation
calculus in Section \ref{sec:Background:Extensions}, let us introduce
an action description predicate $Legal$ that identifies actions that
can be legally executed in the real world. In the basic situation
calculus, it is simply equivalent to $Poss$:\begin{gather*}
Legal(a,s)\,\equiv\, Poss(a,s)\end{gather*}


As shown by the above, it is often useful to define new action description
predicates in terms of simpler existing ones, rather than directly
in terms of the primitive fluents of the domain. As long as these
definitions are well-founded they can be expanded down to primitive
fluents when constructing the basic action theory.


\subsubsection{Foundational Axioms}

The foundational axioms $\Sigma$ ensure that situations form a branching-time
account of the world state. There is a distinguished situation $S_{0}$
called the \emph{initial situation}. Situations in general form a
tree structure with the initial situation at the root and $do(a,s)$
constructing the successor situation resulting when the action $a$
is performed in situation $s$. All situations thus produced are distinct:\[
do(a_{1},s_{1})=do(a_{2},s_{2})\,\rightarrow\, a_{1}=a_{2}\,\wedge\, s_{1}=s_{2}\]


We abbreviate the performance of several successive actions by writing:\[
do([a_{1},\dots,a_{n}],s)\,\isdef\, do(a_{n},do(\dots,do(a_{1},s)))\]


There is also a second-order induction axiom asserting that all situations
must be constructed in this way, which is needed to prove statements
that universally quantify over situations \citep{Reiter93proving}:\[
\forall P:\,\left[P(S_{0})\wedge\forall s,a:\,\left(P(s)\rightarrow P(do(a,s))\right)\right]\,\rightarrow\,\forall s:\, P(s)\]


The relation $s\sqsubset s'$ indicates that $s'$ is in the future
of $s$ and is defined as follows:\begin{gather*}
\neg(s\sqsubset S_{0})\\
s\sqsubset do(a,s')\equiv s\sqsubseteq s'\end{gather*}


Here $s\sqsubseteq s'$ is the standard abbreviation for $s\sqsubset s'\vee s=s'$.
This notion of {}``in the future of'' can be extended to consider
only those futures in which all actions satisfy a particular action
description predicate. We define as a macro the relation $<_{\alpha}$
for an arbitrary action description predicate $\alpha$, with the
following definition:\[
s<_{\alpha}s'\,\isdef\, s\sqsubset s'\wedge\forall a,s'':\,\left(s\sqsubset do(a,s'')\sqsubseteq s'\rightarrow\alpha[a,s'']\right)\]


It is straightforward to demonstrate that this macro satisfies the
following properties, which are analogous to the definition of $\sqsubset$:\begin{gather*}
\neg\left(s<_{\alpha}S_{0}\right)\\
s<_{\alpha}do(a,s')\equiv s\leq_{\alpha}s'\wedge\alpha[a,s']\end{gather*}


The \emph{legal situations} are those in which every action was legal
to perform in the preceding situation. These are of fundamental importance,
as they are the only situations that could be reached in the real
world:\[
Legal(s)\isdef S_{0}\leq_{Legal}s\]



\subsubsection{Initial State Axioms}

The set $\Dt_{S_{0}}$ describes the actual state of the world before
any actions are performed. It is a collection of sentences uniform
in $S_{0}$ stating what holds in the initial situation. In many domains
the initial state can be completely specified, so $\Dt_{S_{0}}$ is
often in a closed form suitable for efficient automated reasoning.

Note that, unlike \citep{levesque98sc_foundations,pirri99contributions_sitcalc,reiter01kia},
we include static facts about the domain in $\Dt_{bg}$ rather than
$\Dt_{S_{0}}$. This is entirely a cosmetic change to allow us to
talk about these static facts separately from the initial database.


\subsection{Reasoning\label{sec:Background:Reasoning}}

An important feature of the situation calculus is the existence of
effective reasoning procedures for certain types of query. These are
generally based on syntactic manipulation of a query into a form that
is more amenable to reasoning, for example because it can be proven
without using some of the axioms from $\Dt$.


\subsubsection{Types of Reasoning}

In the general case, answering a query about a basic action theory
$\Dt$ is a theorem-proving task in second-order logic (denoted SOL)
due to the induction axiom included in the foundational axioms:\[
\Dt\models_{SOL}\psi\]
 This is clearly problematic for effective automated reasoning, but
fortunately there exist particular syntactic forms for which some
of the axioms in $\Dt$ are not required.

If a query only performs \emph{existential} quantification over situation
terms, it can be answered without the induction axiom (denoted $I$)
and thus using only first-order logic (FOL) \citep{pirri99contributions_sitcalc}:\[
\Dt\models_{SOL}\exists s:\,\psi(s)\,\,\,\,\mathrm{iff}\,\,\,\,\Dt-\{I\}\models_{FOL}\exists s:\,\psi(s)\]


While this is a substantial improvement over requiring a second-order
theorem prover, it is still far from an effective technique. Effective
reasoning requires that the set of axioms be reduced as much as possible.

In their work on state constraints, \citet{Lin94-StateConstraints}
show how to reduce the task of verifying a state constraint to a reasoning
task we call \emph{static domain reasoning}, where only the background
axioms need to be considered:\[
\Dt_{bg}\models_{FOL}\forall s:\,\phi[s]\]


Since the axioms in $\Dt_{bg}$ do not mention situation terms, the
leading quantification in such queries has no effect -- $\phi$ will
be entailed for all $s$ if and only if it is entailed for some $s$.
This is a major improvement because universal quantification over
situation terms usually requires the second-order induction axiom.
Their work has shown that this requirement can be circumvented in
some cases.

Simpler still are queries uniform in the initial situation, which
can be answered using only first-order logic and a limited set of
axioms:\[
\Dt\models_{SOL}\phi[S_{0}]\,\,\,\,\,\mathit{\mathrm{iff}}\,\,\,\,\,\Dt_{S_{0}}\cup\Dt_{bg}\models_{FOL}\phi[S_{0}]\]


We call such reasoning \emph{initial} \emph{situation reasoning}.
Since the axioms $\Dt_{S_{0}}\cup\Dt_{bg}$ often satisfy the closed-world
assumption, provers such as Prolog can be employed to handle this
type of query quite effectively.


\subsubsection{Regression}

The principle tool for effective reasoning in the situation calculus
is the regression meta-operator $\Reg_{\Dt}$, a syntactic manipulation
that encodes the preconditions and effects of actions into the query
itself, meaning fewer axioms are needed for the final reasoning task
\citep{pirri99contributions_sitcalc}. The idea is to reduce a query
about some future situation to a query about the initial situation
only.

There are two styles of regression operator commonly defined in the
literature: the single-pass operator as defined in \citep{pirri99contributions_sitcalc}
which reduces to $S_{0}$ in a single application, the the single-step
operator as defined in \citep{scherl03sc_knowledge} which operates
one action at a time. We use the single-step variant because it is
the more expressive of the two -- while it is straightforward to define
the single-pass operator in terms of the single-step operator, the
reverse is not the case.

Regression is only defined for a certain class of formulae, the \emph{regressable
formulae}.

\begin{defnL}
[{Regressable~Terms}] Let $\sigma$ be an arbitrary situation
term, $x$ an arbitrary variable not of sort situation, $r$ an arbitrary
rigid function and $f$ an arbitrary fluent function. Then the regressable
terms are the smallest set of syntactically-valid terms satisfying:
\[
\nu::=\sigma\,|\, x\,|\, f(\vars{\nu},\sigma)\,|\, r(\vars{\nu})\]

\begin{defnL}
[{Regressable~Formulae}] \label{def:Background:Regressable-Formulae}Let
$\sigma$ be an arbitrary situation term, $x$ an arbitrary variable
not of sort situation, $\nu$ an arbitrary regressable term, $R$
an arbitrary rigid predicate, $F$ an arbitrary primitive fluent predicate,
and $\alpha$ an arbitrary action description predicate. Then the
regressable formulae are the smallest set of syntactically-valid formulae
satisfying: \[
\varphi::=F(\vars{\nu},\sigma)\,|\,\alpha(\vars{\nu},a,\sigma)\,|\, R(\vars{\nu})\,|\,\nu_{1}=\nu_{2}\,|\,\neg\varphi\,|\,\varphi_{1}\wedge\varphi_{2}\,|\,\exists x:\,\varphi\]

\end{defnL}
\end{defnL}
Regressable formulae are more general than uniform formulae. In particular,
they can contain action description predicates and may mention different
situation terms. They cannot, however, quantify over situation terms
or compare situations using the $\sqsubset$ predicate.

The regression operator is then defined using a series of \emph{regression
rules} such as those shown below, which mirror the structural definition
of regressable formulae.

\begin{defnL}
[{Regression~Operator}] Let $R$ be a rigid predicate, $\alpha$
be an action description predicate with axiom $\alpha(\vars{\nu},a,s)\equiv\Pi_{\alpha}(a,s)$
in $\Dt_{ad}$, and $F$ be a primitive fluent with axiom $F(\vars{x},s)\equiv\Phi_{F}(\vars{x},s)$
in $\Dt_{ssa}$. Then the regression of $\phi$, denoted $\Reg_{\Dt}(\phi)$,
is defined according to the following structural rules:\begin{gather*}
\Reg_{\Dt}(\varphi_{1}\wedge\varphi_{2})\,\isdef\,\Reg_{\Dt}(\varphi_{1})\wedge\Reg_{\Dt}(\varphi_{2})\\
\Reg_{\Dt}(\exists x:\,\varphi)\,\isdef\,\exists x:\,\Reg_{\Dt}(\varphi)\\
\Reg_{\Dt}(\neg\varphi)\,\isdef\,\neg\Reg_{\Dt}(\varphi)\\
\Reg_{\Dt}(\alpha(\vars{\nu},a,\sigma))\,\isdef\,\Reg_{\Dt}(\Pi_{\alpha}(\vars{\nu},a,\sigma))\\
\Reg_{\Dt}(F(\vars{\nu},do(a,\sigma)))\,\isdef\,\Phi_{F}(\vars{\nu},a,\sigma)\\
\Reg_{\Dt}(F(\vars{\nu},s))\,\isdef\,F(\vars{\nu},s)\\
\Reg_{\Dt}(F(\vars{\nu},S_{0}))\,\isdef\,F(\vars{\nu},S_{0})\end{gather*}

\end{defnL}
We have omitted some technical details here, such as the handling
of functional fluents; consult \citep{pirri99contributions_sitcalc}
for the details. The key point is that each application of the regression
operator replaces action description predicates with their definitions
from $\Dt_{ad}$ and primitive fluents with their successor state
axioms from $\Dt_{ssa}$, {}``unwinding'' a single action from each
$do(a,\sigma)$ situation term in the query. If the situation term
is not constructed using $do$, it is left unchanged.

Since $\Dt$ is fixed, we will henceforth drop the subscript and simply
write $\Reg$ for the regression operator. When dealing with situation-suppressed
uniform formulae, we will use a two-argument operator $\Reg(\phi,a)$
to indicate the regression of $\phi$ over the action $a$. It should
be read as a shorthand for $\Reg(\phi[do(a,s)])^{-1}$ using the situation-suppression
operator from Section \ref{sec:Background:SC:Notation}.

Let us briefly state some important properties of the regression operator.
First, and most importantly, it preserves equivalence of formulae:

\begin{prop}
Let $\varphi$ be a regressable formula, then $\Dt\,\models\,\varphi\,\equiv\,\Reg(\varphi)$ 
\end{prop}
Any formula uniform in $do(a,s)$ is regressable, and the result is
uniform in $s$:

\begin{prop}
Let $\phi$ be uniform in $do(a,s)$, then $\Reg(\phi)$ is uniform
in $s$ 
\end{prop}
Let $\Reg^{*}$ denote repeated applications of $\Reg$ until the
formula remains unchanged. Such applications can transform a query
about some future situation into a query about the initial situation
only:

\begin{prop}
Let $\phi$ be uniform in \emph{$do([a_{1}\dots a_{n}],S_{0})$,}
then $\Reg^{*}(\phi)$ is uniform in $S_{0}$ 
\end{prop}
This last property is key to effective reasoning in the situation
calculus, as it allows one to answer the \emph{projection problem}.
To determine whether $\phi$ holds in a given future situation, it
suffices to determine whether $\Reg^{*}(\phi)$ holds in the initial
situation. As discussed above, queries uniform in $S_{0}$ are much
easier to answer. The axioms $\Dt_{ad}$ and $\Dt_{ssa}$ are essentially
{}``compiled into the query'' by the $\Reg^{*}$ operator. While
an efficiency gain is not guaranteed, regression has proven a very
effective technique in practice \citep{levesque97golog,pirri99contributions_sitcalc}.


\subsubsection{Decidability}

Even given the use of regression to reduce the number of axioms required,
reasoning still requires first-order logic and is thus only semi-decidable
in general. Practical systems implemented on top of the situation
calculus typically enforce additional restrictions on the domain in
order to gain decidability.

A common restriction is to assume that the \noun{Action} and \noun{Object}
domains are finite. This allows quantification over these variables
to be replaced with finite conjunctions or disjunctions, essentially
{}``propositionalising'' the domain \citep{giacomo99impl_robots,reiter01kia,levesque04krr_book}.
Both static domain and initial situation reasoning can then be performed
using propositional logic, which is decidable. This may also be combined
with special-purpose decision procedures for particular objects in
the domain, such as deciding linear constraints over the integers
or reals \citep{reiter96sc_nat_conc,reiter01kia}.

A similar, but slightly less onerous restriction, is to ensure that
the construction of function terms is well-founded \citep{levesque04krr_book}.
This prevents building the arbitrarily-nested terms from the Herbrand
universe that cause non-termination in first-order theorem provers,
again gaining decidability.

Recent work by \citet{yu07twovar_sitcalc} has shown how to model
some situation calculus domains using to the two-variable fragment
of first-order logic. Since this fragment is decidable in general,
both static domain and initial situation reasoning are decidable in
such domains.


\subsubsection{Inductive Reasoning}

One class of query that cannot be answered effectively using regression
are formulae that universally quantify over situations. Examples of
such queries include verifying state constraints ({}``for all situations,
the constraint is satisfied'') and determining the impossibility
of a goal ({}``for all situations, the goal is not satisfied'').
The difficulty here comes from the induction axiom.

\citet{Reiter93proving} has shown why the induction axiom is necessary
to prove statements that universally quantify over situation terms.
This work demonstrates the use of the axiom in manual proofs, but
offers no procedure for answering such queries automatically. Other
work considering inductive reasoning has focused exclusively on verifying
state constraints \citep{Lin94-StateConstraints,bertossi96automating}.
While it is possible to automate this verification in some cases,
there are currently no general-purpose tools for effectively handling
queries that universally quantify over situation terms.

It is this limitation, more than any other, that has restricted the
situation calculus to synchronous domains. In asynchronous domains
agents must account for potentially arbitrarily-long sequences of
hidden actions, which requires universal quantification over situation
terms. In Chapter \ref{ch:persistence} we develop a new reasoning
technique to help overcome this limitation.


\subsubsection{Progression\label{sec:Background:Progression}}

While regression has proven quite an effective technique in practice,
it has an obvious shortcoming in domains with long histories -- the
computation required to reason about the current state of the world
increases with each action performed.

An alternative approach is \emph{progression}, in which the initial
state of the world $\Dt_{S_{0}}$ is updated with each action performed,
to give a new set of axioms describing the state of the current situation.
Although this increases the upfront complexity when an action is performed,
this work is amortised over many queries about the updated state.
\citet{thielscher04case_for_progression} makes a compelling case
that progression gives better runtime performance in domains with
many actions. Why, then, do we focus only on regression in this thesis?

The theoretical foundations of progression in the situation calculus
were laid out by \citet{reiter97progression} and come with an important
caveat: the progression of a first-order database is not always first-order
definable. This conjecture was recently proven by \citet{vassos08progression_future_queries},
who show that while it is possible to define first-order progressions
of a database that are valid for restricted classes of query, a first-order
progressed database cannot be complete in general. As such, work on
progression in the situation calculus has focused on restricted queries
or restricted databases for which first-order progressions exist \citep{liu05sc_progression_knowledge,vassos07progression,vassos08fo_strong_progression}.
By contrast, the regression operator is sound and complete for answering
a broad range of queries.

In this thesis, we develop formalisms and reasoning techniques for
problems which have not been approached before in the situation calculus.
Our first priority must be a sound and complete reasoning tool, for
which regression is a good match. Advanced techniques such a progression
are considered future work at this stage.


\subsection{Extensions\label{sec:Background:Extensions}}

The base language of the situation calculus may seem simplistic, lacking
many features that would be desirable for modelling rich multi-agent
domains. However, it is possible to significantly enrich the domain
features that can be modelled while maintaining the elegance and simplicity
of the base situation calculus. We now discuss several such extensions
that are important in multi-agent domains.


\subsubsection{Concurrent Actions}

\label{sec:Background:Concurrent-Actions}

In the basic situation calculus only a single action can occur at
any instant. While suitable for most single-agent domains, this limitation
is emphatically not suitable for multi-agent systems -- several actions
can easily occur simultaneously if performed by different agents.
Modelling this \emph{true concurrency} is necessary to avoid problems
with conflicting or incompatible actions. There is also the potential
to utilise concurrency to execute tasks more efficiently. Clearly
a solid account of concurrency is required for reasoning about multi-agent
teams.

The work of \citep{lin92sc_conc,pinto94temporal,reiter96sc_nat_conc}
adds true concurrency to the situation calculus by replacing action
terms with \emph{sets} of actions that are performed simultaneously.
The additional sort \noun{Concurrent} is added to $\Lsit$, and the
appropriate axioms for set theory are added to $\Dt_{bg}$. All functions
and predicates that take an \noun{Action} term are are modified to
take a \noun{Concurrent} term instead. For example, $do(a,s)$ becomes
$do(\{a_{1},a_{2},...\},s)$. Successor state axioms are modified
to test for set membership rather than equality of action terms. For
example, the successor state axiom for $HasObject$ would become:\begin{multline*}
HasObject(agt,obj,do(c,s))\equiv acquire(agt,obj)\in c\\
\vee\,\, HasObject(agt,obj,s)\,\wedge release(agt,obj)\not\in c\end{multline*}


Since it operates solely by replacing formulae with their equivalents,
the regression operator is unchanged by this extension and effective
reasoning is still possible.

There is, however, a subtle complication in axiomatising action description
predicates such as $Poss$: interaction between primitive actions.
A combination of actions is not guaranteed to be possible even if
each of the individual actions are. For example, two agents may not
be able to acquire the same resource at the same time. This is known
as the precondition interaction problem and has undergone extensive
research \citep{pinto94temporal,pinto98interacting_effects,pinto00action_interaction}.
We make no explicit commitment towards a solution for this problem.
Rather, we assume that the axioms in $\Dt_{ad}$ contain the necessary
logic to account for interaction for all action description predicates.

To avoid unintuitive behaviour, we assume that the domain entails
the following consistency requirements for the empty set of actions:

\begin{defnL}
[{Empty~Action~Consistency}] A basic action theory $\Dt$
using concurrent actions must entail the following consistency requirements
for the empty set of actions:\begin{gather*}
\Dt\,\models\,\forall s:\,\neg Legal(\{\},s)\\
\Dt\,\models\,\forall s:\,\phi[do(\{\},s)]\,\equiv\,\phi[s]\end{gather*}

\end{defnL}
Since true concurrency is such an important aspect of multi-agent
systems, we will assume concurrent actions are in use throughout the
rest of the thesis.


\subsubsection{Time}

\label{sec:Background:Time}

An explicit notion of time can make coordination between agents easier,
as joint actions may be performed at a particular time. It also allows
a richer description of the world, particularly in domains such as
the cooking agents where time can play an important part in tasks
to be performed.

The standard approach to time in the situation calculus is that of
\citep{pinto94temporal,pinto95reasoning_time,reiter96sc_nat_conc}.
An additional sort \noun{Timepoint} is introduced, which can be any
appropriately-behaved sequence such as integers or reals. The axiomatisation
of timepoints is added to $\Dt_{bg}$, and each action gains an extra
argument indicating the time at which is was performed. The functions
$time$ and $start$ are introduced to give the performance time of
an action and the start time of a situation respectively. The start
time of the initial situation may be defined arbitrarily, but is typically
taken to be zero.

However, this approach does not integrate cleanly with concurrent
actions: it requires an additional predicate $Coherent$ to ensure
that the performance time is the same for all members in a set of
concurrent actions \citep{reiter96sc_nat_conc}. The legal situations
must be restricted to those in which all actions are coherent.

To avoid this extra complexity, we follow the approach taken in the
related formalism of the fluent calculus \citep{martin03conc_flux}
and attach the temporal component to the set of concurrent actions
itself, rather than to each individual action. A similar approach
is used in \citep{scherl03conc_knowledge} to avoid problems when
combining knowledge and time.

Predicates and functions taking terms of sort \noun{Action} are modified
to take \noun{Concurrent\#Timepoint} pairs, e.g. $do(c,s)$ becomes
$do(c\#t,s)$. The new function $start$ is added to the foundational
axioms with the following definition:\[
start(do(c\#t,s))=t\]


We must ensure that successor situations have later start times than
their preceding situations, by modifying the definition of $Legal$:\[
Legal(c\#t,s)\,\equiv\, Poss(c\#t,s)\wedge start(s)<t\]


Introducing timepoints does not affect the regression operator, but
does increase the complexity of reasoning as $\Dt_{bg}$ now contains
the axioms of number theory. In practice, we limit predicates about
time to express only \emph{linear} relationships, and employ a linear
constraint solver for decidable reasoning over the temporal component.


\subsubsection{Natural Actions\label{sub:Background:Natural-Actions}}

Natural actions are a special class of exogenous actions, those actions
which occur outside of an agent's control \citep{reiter96sc_nat_conc}.
They are classified according to the following requirement: natural
actions must occur if it is possible for them to occur, unless an
earlier action prevents them. For example, a timer will ring at the
time it was set for, unless it is switched off. Such actions are used
to model the predictable behaviour of the environment.

Natural actions are identified by the truth of the predicate $Natural(a)$.
The times at which natural actions may occur are specified by the
$Poss$ predicate. For example, suppose that the fluent $TimerSet(m,s)$
represents the fact that a timer is set to ring in $m$ minutes in
situation $s$. The possibility predicate would entail:\[
Poss(ringTimer\#t,s)\equiv\exists m:\,\left[TimerSet(m,s)\wedge t=start(s)+m\right]\]
 The timer may thus ring only at its predicted time. To enforce the
requirement that natural actions \emph{must} occur whenever possible,
the action description predicate $Legal(c\#t,s)$ is adjusted to ensure
that $c\#t$ is not legal if natural actions could occur at some earlier
time:\begin{multline*}
Legal(c\#t,s)\equiv Poss(c\#t,s)\wedge start(s)<t\\
\wedge\,\,\forall a,t':\,\left[Natural(a)\wedge Poss(\{a\}\#t',s)\,\rightarrow\,\left(a\in c\vee t<t'\right)\right]\end{multline*}
 Thus it is only legal to perform actions $c$ at time $t$ if no
natural actions can occur in $s$ at a time less than $t$.

Following this intuition, the \emph{least natural time point} (or
{}``LNTP'') of a situation is defined as the earliest time at which
a natural action may occur \citep{reiter01kia}. Rather than adding
another axiom, this can be defined using a simple macro: \begin{multline*}
\LNTP(s,t)\isdef\exists a:\left[Natural(a)\wedge Poss(\{a\}\#t,s)\right]\wedge\\
\forall a',t':\left[Natural(a')\wedge Poss(\{a'\}\#t,s)\rightarrow t\leq t'\right]\end{multline*}


We assume that the theory of action avoids certain pathological cases
identified in \citep{reiter01kia}, so that absence of an LNTP implies
that no natural actions are possible. That is to say, we assume the
following is a consequence of the basic action theory:\[
\Dt\,\models\,\left[\exists a,t:\, Natural(a)\wedge Poss(\{a\}\#t,s)\right]\rightarrow\left[\exists t:\,\LNTP(s,t)\right]\]


The LNTP is important when planning in the presence of natural actions
-- one cannot plan to perform some actions at time $t$ if $t$ is
greater than the least natural timepoint of the current situation.
We also define a related concept, the set of \emph{pending natural
actions}, as the set of all natural actions that are possible at the
least natural time point:\[
\PNA(s,c)\isdef\exists t:\,\LNTP(s,t)\wedge\forall a:\left[Natural(a)\wedge Poss(\{a\}\#t,s)\equiv a\in c\right]\]



\subsubsection{Long-Running Tasks}

Although all actions in the situation calculus are instantaneous,
it is still possible to model long-running tasks that have a finite
duration. They are modelled by decomposing them into instantaneous
$beginTask$ and $endTask$ actions, and a fluent $DoingTask$ indicating
that a task is in progress \citep{pinto94temporal}.

In the presence of long-running tasks, a robust account of natural
actions is very important -- the $endTask$ action must be natural to
ensure that any task that is initiated eventually terminates at the
appropriate time.


\subsubsection{Summary}

As can be seen from the discussion above, it is possible to enrich
the situation calculus with some very powerful domain features while
still maintaining the basic structure of the language, and retaining
regression as the principle tool for effective automated reasoning.

While we assume concurrent actions are in use through the rest of
this thesis, we shall only refer explicitly to other rich domain features
-- such as time and natural actions -- when we wish to make a special
point about their treatment. By uniformly using the predicate $Legal$
to identify actions that can legally be performed in the world, rather
than the base $Poss$ predicate, we ensure that our techniques are
applicable regardless of the particular domain features being used.


\section{Golog\label{sec:Background:Golog}}

Golog is a declarative agent programming language that is the standard
approach to specifying complex behaviours in the situation calculus
\citep{levesque97golog}. Testimony to its success are its wide range
of applications and many extensions to provide additional functionality
\citep{giacomo00congolog,giacomo99indigolog,Ferrein2005readylog}.
For simplicity, we use the general name {}``Golog'' to refer to
the standard family of languages based on this technique, including
ConGolog \citep{giacomo00congolog} and IndiGolog \citep{giacomo99indigolog}.


\subsection{Notation}

To program an agent using Golog one specifies a situation calculus
theory of action, and a program consisting of actions from the theory
connected by programming constructs such as if-then-else, while loops,
and nondeterministic choice. Table \ref{tbl:Background:Golog-Operators}
lists the standard operators available in various incarnations of
the language.

%
\begin{table}
\begin{centering}
\begin{tabular}{|c|c|}
\hline 
Operator  & Meaning\tabularnewline
\hline
\hline 
$Nil$  & The empty program\tabularnewline
\hline 
$a$  & Execute action $a$ in the world\tabularnewline
\hline 
$\phi?$  & Proceed if condition $\phi$ is true\tabularnewline
\hline 
$\delta_{1};\delta_{2}$  & Execute $\delta_{1}$followed by $\delta_{2}$\tabularnewline
\hline 
$\delta_{1}|\delta_{2}$  & Execute either $\delta_{1}$ or $\delta_{2}$\tabularnewline
\hline 
$\pi(x,\delta(x))$  & Nondet. select arguments for $\delta$\tabularnewline
\hline 
$\delta*$  & Execute $\delta$ zero or more times\tabularnewline
\hline 
$\mathbf{if}\,\phi\,\mathbf{then}\,\delta_{1}\,\mathbf{else}\,\delta_{2}$  & Exec. $\delta_{1}$ if $\phi$ holds, $\delta_{2}$ otherwise\tabularnewline
\hline 
$\mathbf{while\,}\phi\mathbf{\, do}\,\delta$  & Execute $\delta$ while $\phi$ holds\tabularnewline
\hline 
$\mathbf{proc}P(\overrightarrow{x})\delta(\overrightarrow{x})\mathbf{end}$  & Procedure definition\tabularnewline
\hline 
$\delta_{1}||\delta_{2}$  & Concurrent execution\tabularnewline
\hline 
$\delta_{1}\ll\delta_{2}$  & Prioritised concurrency\tabularnewline
\hline 
$\delta^{||}$  & Concurrent iteration\tabularnewline
\hline 
$\Sigma(\delta)$  & Plan execution offline\tabularnewline
\hline
\end{tabular}
\par\end{centering}

\caption{Operators used in Golog and its descendants \label{tbl:Background:Golog-Operators} }

\end{table}


Readers familiar with dynamic logic will recognise some of these operators,
but others are unique to first-order formalisms such as Golog. Many
Golog operators are nondeterministic and may be executed in several
different ways. It is the task of the agent to plan a deterministic
instantiation of the program, a sequence of actions that can legally
be performed in the world. Such a sequence is called a \emph{legal
execution} of the program.

To get a feel for how these operators can be used, consider some example
programs. Figure \ref{fig:Background:Golog:Washing-Dishes} shows
a simple program for Jim to wash the dishes. It makes use of the nondeterministic
{}``pick'' operator to select and clean a dish that needs washing,
and does so in a loop until no dirty dishes remain. The legal executions
of this program are sequences of $clean(Jim,d)$ actions, one for
each dirty dish in the domain, performed in any order.

%
\begin{figure}
\begin{centering}
\framebox{%
\begin{minipage}[t][1\totalheight]{0.85\columnwidth}%
\begin{flalign*} \mathbf{while}\,\exists d:\, Dirty(d)\,\mathbf{do}\\
 \pi(d,\, clean(Jim,d))\\
 \mathbf{end}\end{flalign*} %
\end{minipage}} 
\par\end{centering}

\caption{A Golog program for washing the dishes\label{fig:Background:Golog:Washing-Dishes}}

\end{figure}


Figure \ref{fig:Background:Golog:MakeSalad} shows a program that
we will return to in subsequent chapters, giving instructions for
how to prepare a simple salad. The procedure $ChopTypeInto$ (not
shown) directs the specified agent to acquire an ingredient of the
specified type, chop it, and place it into the indicated bowl. The
procedure $MakeSalad$ nondeterministically selects an agent to do
this for a lettuce, a carrot, and a tomato. Note the nondeterminism
in this program: the agent assigned to handling each ingredient is
not specified ($\pi$ construct), nor is the order in which they should
be processed ($||$ construct). There is thus considerable scope for
cooperation between agents to effectively carry out this task.

%
\begin{figure}
\begin{centering}
\framebox{%
\begin{minipage}[t][1\totalheight]{0.85\columnwidth}%
\begin{gather*}
\mathbf{proc}\, MakeSalad(dest)\\
\left[\pi(agt,ChopTypeInto(agt,Lettuce,dest))\,||\right.\\
\pi(agt,ChopTypeInto(agt,Carrot,dest))\,||\\
\left.\pi(agt,ChopTypeInto(agt,Tomato,dest))\right]\,;\\
\pi(agt,\left[acquire(agt,dest)\,;\,\right.\\
beginTask(agt,mix(dest,1))\,;\\
endTask(agt,mix(dest,1))\,;\\
\left.\, release(agt,dest)\right])\,\,\mathbf{end}\end{gather*}
 %
\end{minipage}} 
\par\end{centering}

\caption{A Golog program for making a salad\label{fig:Background:Golog:MakeSalad}}

\end{figure}



\subsection{Semantics}

The original semantics of Golog were defined using macro-expansion
\citep{levesque97golog}. The macro $\Do(\delta,s,s')$ was defined
to be true if program $\delta$ could be successfully executed in
situation $s$, leaving the world in situation $s'$. However, these
semantics could not support the concurrent execution of two programs
and were modified with the introduction of ConGolog \citep{giacomo00congolog}
to use two predicates $Trans(\delta,s,\delta',s')$ and $Final(\delta,s)$
which are capable of representing single steps of execution of the
program.

The predicate $Trans(\delta,s,\delta',s')$ holds when executing a
step of program $\delta$ can cause the world to move from situation
$s$ to situation $s'$, after which $\delta'$ remains to be executed.
It thus characterises single steps of computation. The predicate $Final(\delta,s)$
holds when program $\delta$ may legally terminate its execution in
situation $s$. We base our work on the semantics of IndiGolog \citep{giacomo99indigolog},
which builds on ConGolog \citep{giacomo00congolog} and is the most
feature-full of the standard Golog variants. The full semantics are
available in the references, but we present some illustrative examples
below.

The transition rule for a program consisting of a single action is
straightforward -- it transitions by performing the action, provided
it is possible in the current situation. Such a program may not terminate
its execution since the action remains to be performed:\begin{alignat*}{1}
Trans(a,s,\delta',s')\,\equiv\, & \, Poss(a,s)\wedge\delta'=Nil\wedge s'=do(a,s)\\
Final(a,s)\,\equiv\, & \,\bot\end{alignat*}


The transition rule for a test operator proceeds only if the test
is true, leaving the situation unchanged, and likewise cannot terminate
execution until the test has been satisfied:\begin{alignat*}{1}
Trans(?\phi,s,\delta',s')\,\equiv\, & \,\phi[s]\wedge\delta'=Nil\wedge s'=s\\
Final(?\phi,s)\,\equiv\, & \,\bot\end{alignat*}


Now consider a simple nondeterministic operator, the {}``choice''
construct that executes one of two alternate programs:\begin{alignat*}{1}
Trans(\delta_{1}|\delta_{2},s,\delta',s')\,\equiv\, & \, Trans(\delta_{1},s,\delta',s')\,\vee\, Trans(\delta_{2},s,\delta',s')\\
Final(\delta_{1}|\delta_{2},s)\,\equiv\, & \, Final(\delta_{1},s)\,\vee\, Final(\delta_{2},s)\end{alignat*}


It is possible for this operator to transition in two different ways
- by executing a step of execution from the first program, or a step
of execution from the second program. Slightly more complicated, but
of fundamental important in the next chapter, is the semantics of
the concurrency operator:\begin{alignat*}{1}
Trans(\delta_{1}||\delta_{2},s,\delta',s')\,\equiv\, & \,\exists\gamma:\, Trans(\delta_{1},s,\gamma,s')\,\wedge\,\delta'=(\gamma||\delta_{2})\\
 & \,\vee\,\exists\gamma:\, Trans(\delta_{2},s,\gamma,s')\,\wedge\,\delta'=(\delta_{1}||\gamma)\\
Final(\delta_{1}||\delta_{2},s)\,\equiv\, & \, Final(\delta_{1},s)\,\wedge\, Final(\delta_{2},s)\end{alignat*}


This rule specifies the concurrent-execution operator as an \emph{interleaving}
of computation steps. It states that it is possible to single-step
the concurrent execution of $\delta_{1}$ and $\delta_{2}$ by performing
either a step from $\delta_{1}$ or a step from $\delta_{2}$, with
the remainder $\gamma$ left to execute concurrently with the other
program

Clearly there are two notions of concurrency to be considered in the
situation calculus: the possibility of performing several actions
at the same instant (\emph{true concurrency}), and the possibility
of interleaving the execution of several programs (\emph{interleaved
concurrency}). \citet{pinto99tcongolog} have modified ConGolog to
incorporate sets of concurrent actions in an attempt to integrate
these two forms of concurrency. However, their semantics may call
for actions to be performed that are not possible and can also produce
unintuitive program behaviour in some cases. A key aspect of our work
in Chapter \ref{ch:mindigolog} is a robust integration of these two
notions of concurrency.\\


We have omitted many details here that are not relevant to this thesis,
such as the second-order axioms necessary to handle recursive procedure
definitions. We will denote by $\Dt_{golog}$ the standard axioms
defining $Trans$ and $Final$ \citep{giacomo00congolog,giacomo99indigolog}.


\subsection{Execution Planning\label{sec:Background:Golog-Exec-Planning}}

Planning an execution of a Golog program $\delta$ can be reduced
to a theorem proving task as shown in equation (\ref{eqn:Background:golog_execution}).
Here $Trans^{*}$ indicates the standard second-order definition for
the reflexive transitive closure of $Trans$.\begin{equation}
\Dt\cup\Dt_{golog}\models\exists s,\delta':\,\left[Trans^{*}(\delta,S_{0},\delta',s)\wedge Final(\delta',s)\right]\label{eqn:Background:golog_execution}\end{equation}


A constructive proof of this query would produce an instantiation
of $s$, a situation term giving a sequence of actions constituting
a legal execution of the program. These actions are then executed
one-by-one in the world. Since the program remaining after termination
is often not important, the macro $\Do$ is re-defined in terms of
$Trans$ and $Final$ to specify only the resulting situation:\[
\Do(\delta,s,s')\isdef\exists\delta':\, Trans^{*}(\delta,s,\delta',s')\wedge Final(\delta',s')\]


In the original Golog and in ConGolog this forms the entirety of the
execution planning process, as these variants require a full legal
execution to be planned before any actions are performed in the world.
This is referred to as \emph{offline execution}. The Golog execution
algorithm is presented in Algorithm \ref{alg:golog_exec}.

%
\begin{algorithm}[t]
 

\caption{The Golog/ConGolog Execution Algorithm for program $\delta$}


\label{alg:golog_exec} \begin{algorithmic} \STATE Find a situation
$s$ such that:\[
\Dt\cup\Dt_{golog}\models\exists s:\,\Do(\delta,S_{0},s)\]
 \FOR{each action in the resulting situation term} \STATE execute
that action \ENDFOR \end{algorithmic} 
\end{algorithm}


By contrast, IndiGolog allows agents to proceed without planning a
full terminating execution of their program, instead searching for
a legal next step $a$ in the current situation $\sigma$ such that:\[
\Dt\cup\Dt_{golog}\,\models\,\exists a,\delta':\, Trans^{*}(\delta,\sigma,\delta',do(a,\sigma))\]


This next step is then performed immediately, and the process repeats
until a terminating configuration is reached. This is referred to
as \emph{online execution}. The IndiGolog execution algorithm is presented
in Algorithm \ref{alg:indigolog_exec}.

%
\begin{algorithm}[t]
 

\caption{The IndiGolog Execution Algorithm for program $\delta$}


\label{alg:indigolog_exec} \begin{algorithmic} \STATE $\sigma\ \Leftarrow\ S_{0}$
\WHILE{$\Dt\cup\Dt_{golog}\not\models Final(\delta,\sigma)$} \STATE
Find an action $a$ and program $\delta'$ such that: \[
\Dt\cup\Dt_{golog}\models Trans^{*}(\delta,\sigma,\delta',do(a,\sigma))\]
 \STATE Execute the action $a$ \STATE $\sigma\ \Leftarrow\ do(a,\sigma)$
\STATE $\delta\ \Leftarrow\ \delta'$ \ENDWHILE \end{algorithmic} 
\end{algorithm}


In order to incorporate planning into this execution algorithm, IndiGolog
introduces an explicit {}``search'' operator $\Sigma(\delta)$,
which can only make a transition if the program is guaranteed to eventually
terminate successfully:\[
Trans(\Sigma(\delta),s,\delta',s')\equiv\exists s'',\delta'':\, Trans(\delta,s,\delta'',s')\wedge\mathbf{Do}(\delta'',s',s'')\wedge\delta'=\Sigma(\delta'')\]


This approach gives the programmer powerful control over the amount
of non-determinism in the program, and the amount of planning required
to find a legal execution. It also allows the programmer to avoid
planning over sensing actions, which can cause an exponential blowup
in planning complexity. Sensing actions are simply performed outside
the scope of a search operator.


\subsection{Extensions}

There have been a wide range of Golog extensions developed which we
will not consider in this thesis. Among them have been extensions
to include decision-theoretic \citep{boutilier00dtgolog} and game-theoretic
aspects \citep{finzi03gtgolog,finzi05pogtgolog}, additional control
operators such as partially-ordered sequences of actions \citep{son00htn_golog}
and hierarchical task networks \citep{Gabaldon02htn_in_golog,Son04golog+htn+time},
synchronisation between the individual programs of a team of agents
\citep{farinelli07team_golog}, and accounting for continuous change
and event triggering \citep{grosskreutz00ccgolog}.

While we will not consider these Gologs in any detail, we do note
that each has involved relatively modest extensions to
the underlying situation calculus theory and/or the semantics of the
Golog operators, and as a result there has been rich cross-pollination
between these different works. We therefore hope that our work may
in turn be combined with some of these extensions to provide an even
richer formalism.


\section{Epistemic Reasoning\label{sec:Background:Epistemic}}

Reasoning about the knowledge of an agent and the combined knowledge
of a group of agents, referred to in general as \emph{epistemic reasoning},
is a fundamental aspect of reasoning and planning in multi-agent domains.
An excellent explanation of the importance of epistemic reasoning
is the classic paper by \citet{halpern90knowledge_distrib}.


\subsection{Epistemic Reasoning in General}

The standard semantics of epistemic reasoning are given by the Kripke
structures of modal logic, and are based on the idea of \emph{possible
worlds}. Intuitively, if an agent is unsure about the state of the
world, then there must be several candidate states of the world that
it considers possible. The agent is then said to \emph{know} a proposition
if it is true in all worlds considered possible. In modal epistemic
logics this is typically written as $[K_{agt}]\phi$, but we will
consistently use the standard situation calculus notation of $\Knows(agt,\phi,s)$.

While this works well for a static knowledge base, there is also the
question of how each agent's knowledge changes over time as actions
are performed in the world. Various formal systems have been devised
to represent the interaction between knowledge and action, including
those of \citet{fagin95}, \citet{parikh85dist_knowledge} and \citet{baltag98pa_ck}.
Recent work on the foundations of epistemic dynamic logic has shown
these different perspectives to be essentially equivalent \citep{vanBentham06tree_of_knowledge,pacuit07history_structures},
and they can be briefly summarised as follows.

A system is considered to have a set of possible \emph{events} which
could occur at any given instant, and the system \emph{state} at any
time is determined by the sequence of events that have occurred. For
each agent there is some subset of these events that it is capable
of observing, and it thus has a restricted local \emph{view} of the
state of the system. From the agent's perspective, the system may
be in any one of the states that would be compatible with its current
local view. If some proposition holds in all such states, then the
agent \emph{knows} that proposition.

We will not comment on these related formalisms for epistemic reasoning
in any further detail, except to note that the approach we develop
in Chapters \ref{ch:knowledge} and \ref{ch:cknowledge} deliberately
parallels the intuitions and notation of this wider field.


\subsection{Epistemic Reasoning in the Situation Calculus}

Epistemic reasoning was first introduced to the situation calculus
by \citet{moore80know_act}, and formalised extensively by \citet{scherl03sc_knowledge}
whose paper is now the canonical reference. Their work has been extended
to include concurrent actions \citep{scherl03conc_knowledge} and
multiple agents \citep{shapiro98specifying_ma_systems}. Our work
further extends these techniques.

The semantics of knowledge are based on a reification of the {}``possible
worlds'' semantics of modal logic, using situation terms rather than
abstract worlds. A special fluent $K(agt,s',s)$ is used to indicate
that {}``in situation $s$, the agent $agt$ considers the alternate
situation $s'$ to be possible''. The macro $\Knows$ is then defined
as a shorthand for the standard possible-worlds definition of knowledge,
stating that an agent knows $\phi$ when $\phi$ is true in all situations
considered possible: \begin{equation}
\Knows(agt,\phi,s)\isdef\forall s':\, K(agt,s',s)\rightarrow\phi[s']\label{eqn:Knowledge:knows_def}\end{equation}


Readers familiar with modal logic will recognise $K(agt,s',s)$ as
the situation calculus analogue of the modal reachability relation
$K_{agt}$, and the macro $\Knows$ as the equivalent of the modal
box operator $[K_{agt}]\phi$. The definitions of uniform and regressable
formulae are updated to permit statements of the form $\Knows(agt,\phi,\sigma)$.

In preparation for developments later in the thesis, we break slightly
with standard situation calculus notation and introduce an additional
fluent $K_{0}(agt,s',s)$ to model the \emph{initial} epistemic uncertainty
of the agents, with the corresponding macro $\KnowsZ(agt,\phi,s)$
defined in the obvious way. In synchronous domains these are always
equivalent to $K$ and $\Knows$, but we introduce them here to maintain
consistency of the presentation.

To model initial epistemic uncertainty, the foundational axioms $\Sigma$
are modified to permit multiple initial situations, identified by
the predicate $Init(s)$. $S_{0}$ then represents the \emph{actual}
initial situation, while other initial situations represent alternatives
that the agents consider possible. We have the following straightforward
extensions of the foundational axioms presented in Section \ref{sec:Background:SC:Axioms}:

No situation precedes an initial situation:\[
Init(s)\,\rightarrow\,\neg(s'\sqsubset s)\]


Situation terms in general are always \emph{rooted} at a particular
initial situation:\begin{gather*}
Init(s)\,\rightarrow\, root(s)=s\\
root(do(c,s))=root(s)\end{gather*}


The definition of legal situations permits any root situation, not
just $S_{0}$:\[
Legal(s)\,\isdef\, root(s)\leq_{Legal}s\]


Finally, the initial epistemic uncertainty of the agents is restricted
to initial situations only, so that they initially know that no actions
have been performed:\[
K_{0}(agt,s',s)\,\rightarrow Init(s)\wedge Init(s')\]


With this infrastructure in place, the initial situation axioms $\Dt_{S_{0}}$
can now contain sentences of the form $\KnowsZ(agt,\phi,S_{0})$ specifying
what the agents initially know.

Since we intend to replace the standard axioms of knowledge later
in the thesis, we assume that the dynamics of the $K$ fluent are
specified in a separate set of axioms $\Dt_{K}$, which must be included
when reasoning about knowledge queries. While we defer a detailed
discussion of the dynamics of $K$ until Chapter \ref{ch:knowledge},
note at this stage that its behaviour is specified by a successor
state axiom like the following:\[
K(agt,do(a,s'),do(a,s))\equiv K(agt,s',s)\wedge Legal(a,s')\]


That is, each agent's knowledge is updated to account for every action
that occurs. The situations considered possible \emph{after} an action
occurrence are the legal successors of situations considered possible
\emph{before} that action. As various types of knowledge-producing
action are introduced to the situation calculus, the successor state
axiom for $K$ is modified to encode the intended semantics \citep{shapiro98specifying_ma_systems,Lesperance99sitcalc_approach,shapiro01casl_feat_inter,Petrick06thesis,shapiro07sc_goal_change}.

In multi-agent settings, one must also consider group-level knowledge.
We briefly review the various group-level epistemic modalities commonly
found in the literature; an excellent overview and discussion can
be found in \citep{halpern90knowledge_distrib}. Let $G$ be a finite
group of agents. The basic group-level modality is {}``everyone knows
$\phi$'', which is defined as:\[
\EKnows(G,\phi,s)\isdef\,\forall agt\in G:\,\Knows(agt,\phi,s)\]


Since $G$ is a finite set, this can be written equivalently as a
finite conjunction:\[
\mathbf{EKnows}(G,\phi,s)\,\isdef\,\bigwedge_{agt\in G}\,\mathbf{Knows}(agt,\phi,s)\]


To assert more complete knowledge by members of the group, one can
say {}``everyone knows that everyone knows $\phi$'' by nesting
$\mathbf{EKnows}$ operators. In general we have {}``everyone knows
to depth $n$'' defined by:\begin{gather*}
\mathbf{EKnows}^{1}(G,\phi,s)\,\isdef\,\mathbf{EKnows}(G,\phi,s)\\
\mathbf{EKnows}^{n}(G,\phi,s)\,\isdef\,\mathbf{EKnows}(G,\mathbf{EKnows}^{n-1}(G,\phi),s)\end{gather*}


The higher the value of $n$, the stronger an assertion is made about
the knowledge of the group. The strongest group-level modality is
{}``it is common knowledge that $\phi$''. Intuitively this indicates
that everyone knows $\phi$, everyone knows that everyone knows $\phi$,
and so on ad infinitum. Formally, it can be defined as the infinite
conjunction:\[
\mathbf{CKnows}(G,\phi,s)\,\isdef\,\bigwedge_{n\in\mathbb{N}}\mathbf{EKnows}^{n}(G,\phi,s)\]


Equivalently, it can be defined as a fixpoint or transitive closure
of the $\EKnows$ relation. Common knowledge is an extremely powerful
form of knowledge that has deep implications for coordinated group
behaviour. For example, in the famous {}``Coordinated Attack'' problem,
the proof that the generals cannot coordinate an attack depends heavily
on their inability to obtain common knowledge \citep{halpern90knowledge_distrib}.

As we shall see in Chapter \ref{ch:cknowledge}, it is surprisingly
difficult to reason effectively about common knowledge, as it is not
directly amenable to a standard regression rule. Our work is the first
to provide an effective reasoning procedure for common knowledge in
the situation calculus.


\section{Related Formalisms}

\label{sec:Background:Related-Formalisms}

There are a range of related formalisms for reasoning about knowledge,
action and change, which we do not directly consider in this thesis.
Most closely related to the situation calculus are the fluent calculus
of \citet{thielscher98fluent_calculus} and the event calculus of
\citet{kowalski86event_calculus}.

The fluent calculus is based explicitly on the use of progression
for solving the projection problem, and so maintains an explicit representation
of the state of the world which is updated as actions are performed.
It can be derived from a restricted variant of the situation calculus
by transforming successor state axioms into state update axioms that
explicitly add and remove fluents from the state \citep{thielscher99fluentcalc_from_sitcalc}.
Notably, it is relatively straightforward to interpret Golog programs
on top of the fluent calculus \citep{thielscher05golog_in_flux}.
As discussed in Section \ref{sec:Background:Progression}, it would
be interesting to translate our regression-based ideas into a progression-based
formalism such as this, but we do not consider it in this thesis.

The event calculus is slightly further removed, in that it contains
a single linear timeline rather than the branching time structure
of the situation calculus. This makes it more suitable for representing
some domains and posing some queries, but less suitable for others;
a detailed comparison can be found in \citep{kowalski97reconcile_sitcalc_evtcalc,belleghem97sitcalc_evtcalc}.
Like the situation calculus, it has found significant applications
in a logical approach to planning and agent control \citep{shanahan00ec_planner}.
One particular strength of the event calculus is in planning with
partially-ordered sequences of events, highlighted by a reimplementation
of Golog using the event calculus that supports partially-ordered
plans \citep{pereira04ec_golog}. We add a similar ability to the
situation calculus in this thesis.

There is also the family of approaches known as {}``dynamic epistemic
logic'', which are based on modal logic \citep{baltag98pa_ck,vanBenthem06lcc,vanBentham06tree_of_knowledge}
While these formalisms are typically propositional rather than first-order,
and focus more on reasoning about knowledge and communication than
on modelling a changing dynamic world, there are still strong similarities
with the situation calculus \citep{vanbentham07ml_sitcalc}. In Chapter
\ref{ch:cknowledge} we will adapt and extend some reasoning techniques
from these formalisms to model common knowledge in the situation calculus.

There have been several attempts to combine the various action formalisms
into a unifying theory of action, including \citep{belleghem95combine_sitcalc_evtcalc,kowalski97reconcile_sitcalc_evtcalc,thielscher06reconcile_sc_fc,thielscher07unifying_action_calculus},
but there is yet to emerge a clear standard in this regard. In the
meantime, we find the notation and meta-theory of the situation calculus
particularly suitable for expressing our main ideas, and find the
Golog programming language to be a particularly powerful and flexible
approach to specifying agent behaviour and programming shared tasks.

It is our hope that the strong underlying similarities between the
major action formalisms will allow the ideas presented in this thesis
to find some application or resonance beyond the specifics of the
situation calculus.


\section{Mozart/Oz\label{sec:Background:Mozart/Oz}}

One of the main advantages of the situation calculus and Golog are
their straightforward implementation as a logic program. As the dominant
implementation of the logic programming paradigm, Prolog is typically
used for such implementations. In this thesis we use Mozart, a multi-paradigm
programming system with some unique features that are particularly
suited to our work.

The Mozart system \citep{vanroy99mozart} is an implementation of
the Oz programming language \citep{vanRoyHaridi04ctm} with strong
support for logic programming and distributed computing. While a full
explanation of its features is well outside the scope of this thesis,
we provide a short introduction to the subset of its features we will
be using -- in particular, doing Prolog-style logic programming in
Oz. Familiarity with logic programming in the style of Prolog is assumed.

Terms, variables and unification in Oz work similarly to Prolog, although
arguments in compound terms are separated by whitespace rather than
a comma. Predicates are implemented as ordinary procedures, so all
clauses for a predicate must be contained in a single procedure. Figure
\ref{fig:Background:Naive-List-Reverse} shows an Oz implementation
of a classic Prolog example predicate, naive list reverse.

\newpage{}Some things to note about this example include:

\begin{itemize}
\item The syntax for procedure definition is $\mathbf{proc}\,\{Name\, Arg\,\dots\,\}\,\, Body\,\,\mathbf{end}$ 
\item The syntax for procedure calls is $\{Name\, Arg\,\dots\,\}$ 
\item The $\mathbf{case}$ statement is used to pattern-match the contents
of a variable 
\item Local variables must be explicitly introduced using the keyword $\mathbf{in}$ 
\item Mozart separates functionality into modules, such as $List$ 
\end{itemize}
%
\begin{figure}[t]
\programinput{listings/background/Reverse.oz}

\caption{Naive List Reverse implemented in Mozart/Oz\label{fig:Background:Naive-List-Reverse}}

\end{figure}


Procedures in Oz are deterministic by default, and there is no default
search strategy for exploring different alternatives. Instead, Oz
provides independent facilities for creating choicepoints and for
exploring procedures that contain choicepoints. The result is a much
more flexible, although sometimes syntactically more cumbersome, approach
to logic programming \citep{lpinoz99}.

The creation of choice points is explicit in Oz, and performed using
the $\mathbf{choice}$ keyword. To demonstrate, consider another classic
Prolog example: the nondeterministic list member predicate shown in
Figure \ref{fig:Background:Nondet-Member}. In the case of the empty
list, $Member$ simply fails. For a non-empty list, $Member$ explicitly
creates a \emph{choice point} with two options -- either bind $E$
to the head of the list, or bind $E$ to a member of the tail of the
list.

%
\begin{figure}[t]
\programinput{listings/background/Member.oz}

\caption{Nondeterministic List Member implemented in Mozart/Oz\label{fig:Background:Nondet-Member}}

\end{figure}


It is at this point that the use of Mozart for logic programming differs
most from Prolog. If the $Member$ procedure is invoked directly,
it will suspend its execution when the $\mathbf{choice}$ statement
is reached. To resolve the nondeterminism, one must execute the procedure
inside an explicit \emph{search} \emph{object}. These objects are
responsible for exploring the various choicepoints until a non-failing
computation is achieved. They operate by executing the procedure in
a separate \emph{computation space} through which the state of the
underlying computation can be managed \citep{schulte00constraint_services}.

As a demonstration, Figure \ref{fig:Background:All-Pairs} uses the
$Member$ procedure to define a procedure $Pairs$, which nondeterministically
selects a pair of elements from a pair of lists. The procedure $AllPairs$
then uses the builtin $Search.base.all$ object to find all solutions
from this procedure, returning a list of all possible pairs from the
two lists. By encapsulating the calls to nondeterministic procedures
inside a search object, $AllPairs$ will not expose any choicepoints
to code that calls it.

%
\begin{figure}[t]
\programinput{listings/background/Pairs.oz}

\caption{Finding all pairs in Mozart/Oz\label{fig:Background:All-Pairs}}

\end{figure}


Also of note in Figure \ref{fig:Background:All-Pairs} is the use
of a \emph{closure} over the procedure $Pair$ to create the one-argument
procedure $FindP$. Search objects work with a one-argument procedure,
which is expected to bind its argument to a result. The dollar symbol
is used to translate a statement (in this case the $\mathbf{proc}$
definition) into an expression. The value that would be bound to the
dollar symbol by the statement becomes the return value of the expression,
so $FindP=proc\,\{\$\, P\}$ is equivalent to $proc\,\{FindP\, P\}$.

The power of this decoupled approach to nondeterminism and search
becomes apparent when defining new search strategies, which can then
be used to evaluate any procedure. For example, it is straightforward
to implement breadth-first or iterative-deepening strategies to replace
the standard depth-first traversal of the $Search.base$ object \citep{schulte00constraint_services}.

Coupled with Mozart's strong support for distributed computing, these
programmable search strategies offer a unique opportunity -- it becomes
possible to implement a parallel search object which can automatically
distribute work between several networked machines. Moreover, this
parallel search can be applied without modification to any nondeterministic
procedure. Mozart comes with a built-in $ParallelSearch$ object,
which is described in detail in \citep{schulte00oz_parallel} and
which is our main motivation for the use of Oz in this thesis.

To demonstrate the power of the approach, consider Figure \ref{fig:Background:Parallel-All-Pairs},
which describes a parallel-search version of the $AllPairs$ procedure.
In this instance we define $FindP$ as a \emph{functor}, an Oz abstraction
for code that is portable between machines. This functor imports the
module $MyList$ containing the procedures we defined earlier, and
exports a one-argument procedure $Script$ which will be executed
by the parallel search object. The parallel search object $Seacher$
launches one instance of Mozart on the machine {}``mango'' and two
instances on the machine {}``rambutan'', then is asked to enumerate
all solutions for $FindP$.

%
\begin{figure}[!t]
\programinput{listings/background/PPairs.oz}

\caption{Finding all pairs in parallel in Mozart/Oz\label{fig:Background:Parallel-All-Pairs}}

\end{figure}


In Chapter \ref{ch:mindigolog} we will use this parallel search object
to automatically share the workload of planning a Golog execution
amongst a team of cooperating agents.

As a multi-paradigm programming language with significant research
history, there is much more to Oz than we have described here. However,
these brief examples should be sufficient for a reader well-versed
in Prolog to understand the Oz code used throughout this thesis. For
more information and further examples, consult the general Oz tutorial
\citep{haridi99oz_tutorial} or the specialised tutorial on logic
programming in Oz \citep{lpinoz99}, which are both available online.

