 


\chapter{Background}

\label{ch:background} 

\begin{itemize}
\item Quickly highlight difference between open multi-agent systems and
distributed problem solving - explain what we're \emph{not} interested
in. 
\item Overview of DPS formalisms, task representation (HTN, TAEMS, etc...) 
\item Briefly introduce HLP approach, discuss advantages. 
\item Highlight success of Readylog team, but discuss limitations. 
\item Overview of coordination techniques, esp. social laws for coordination
without communication 
\item Overview of distributed planning techniques \cite{desjardins99survey_dist_planning} 
\end{itemize}

\section{The Situation Calculus}

The situation calculus is a formalism for reasoning about action and
change based on first-order logic. A solid understanding of formal
logic is assumed throughout this thesis; see \cite{fitting96fol_book}
for a comprehensive treatment. It was first introduced by McCarthy
and Hayes \cite{McCHay69sitcalc} and has since been significantly
expanded and formalized \cite{reiter91frameprob,pirri99contributions_sitcalc,levesque98sc_foundations}.
The notation and conventions used in this thesis are derived from
\cite{levesque98sc_foundations}.

The situation calculus is built on three fundamental notions: 

\begin{itemize}
\item An \noun{action} is an instantaneous event that causes the state of
the world to change 
\item A \noun{situation} is a history of all the actions that have occurred
in the world, thus determining the state of the world 
\item A \noun{fluent} is a particular aspect of the state of the world 
\end{itemize}
Actions and situations are represented by function terms in the logic,
while fluents are predicates or functions that are restricted to taking
a situation term as their final argument. These notions will be formally
defined in subsequent paragraphs.

As originally conceived, the situation calculus describes domains
in which there is a single agent who has complete control over the
world and can perform only a single action at a time. To provide a
formalism expressive enough for the righ multi-agent domains targeted
in this thesis, several important extensions to the situation calculus
must be incorporated: 

\begin{itemize}
\item Multiple agents acting in the world, by having the first argument
of each action term identify the agent performing the action as in
\cite{shapiro98specifying_ma_systems,shapiro01casl_feat_inter}. 
\item Multiple actions occurring at the same instant, by using sets of individual
actions to build up situation terms as in \cite{lin92sc_conc,pinto94temporal,reiter96sc_nat_conc} 
\item Actions with a finite duration, by breaking them into explicit start
and end actions as in \cite{pinto94temporal} 
\item An explicit notion of time, as in \cite{pinto94temporal,reiter96sc_nat_conc} 
\end{itemize}
The language of the situation calculus, $\mathcal{L}_{sitcalc}$,
is a language of multi-sorted second-order logic with equality.

\begin{itemize}
\item Basic Sitcalc: origins with McCarthy \cite{McCHay69sitcalc}, formalizations
by Reiter et al \cite{reiter01kia,pirri99contributions_sitcalc,levesque98sc_foundations} 
\item incorporate \cite{pinto99ramification,pinto98interacting_effects}
and the unfortunately unpublished \cite{pinto00action_interaction} 
\item Our own customizations, e.g. \char`\"{}Action Description Predicates\char`\"{}
\cite{kelly07sc_persistence} 
\item Concurrent Actions, Continuous Time, Natural Actions \cite{pinto94temporal,reiter96sc_nat_conc} 
\item Reasoning (Regression, Decidable Fragments) 
\item Related approaches (Fluent Calculus, Event Calculus) 

\begin{itemize}
\item justify focusing on sitcalc by highlighting deep links between the
formalisms 
\end{itemize}
\item Briefly highlight things we \emph{don't} consider - nondeterministic/probabilistic
actions, decision theoretic aspects, etc - and explain why. 
\item maybe also \char`\"{}stratified definitions\char`\"{} from \cite{pinto94temporal} 
\end{itemize}

\section{High-Level Program Execution}

\begin{itemize}
\item Basic Golog, operators, semantics \cite{levesque97golog} 
\item Highlight the vast body of similar approaches, e.g. Dynamic Logic 
\item So why golog? - full power of sitcalc, answer extraction for planning. 
\item Discuss idea of 'Legal Execution' in depth 
\item ConGolog \cite{giacomo00congolog}. Also some discussion of related
semantics, such as CCS 
\item IndiGolog - highlight online/offline distinction and connections to
coordination and planning \cite{giacomo99indigolog} 
\item TConGolog - highlight deficiencies \cite{pinto99tcongolog} 
\item Other Gologs (DTGolog, HTNGolog, GTGolog) 

\begin{itemize}
\item for completeness only, we have already established that we aren't
operating in domains appropriate for them 
\item But, take time to highlight rich cross-pollination between works,
and that our results similarly have potential for integration with
these approaches 
\end{itemize}
\end{itemize}

\section{Epistemic Reasoning}

\begin{itemize}
\item Knowledge, Distributed Knowledge, Common Knowledge \cite{halpern90knowledge_distrib,fagin95} 
\item Importance of Common Knowledge for Coordination 
\item Knowledge in the Situation Calculus 

\begin{itemize}
\item Reasoning about Knowledge \cite{moore80know_act,scherl03sc_knowledge} 
\item Use in a multi-agent setting (??what's the best reference) 
\item Concurrency and time \cite{scherl03conc_knowledge} 
\item Shortcomings of multi-agent extensions 
\item Decidable fragments, weakening for computational efficiency 
\end{itemize}
\item Epistemic feasibility of plans: \cite{giacomo04sem_delib_indigolog,Lesperance01epi_feas_casl} 
\end{itemize}

\section{The Oz Programming Language}

\begin{itemize}
\item Basic introduction \cite{vanroy99mozart} 
\item Oz for Logic Programming \cite{vanroy03mozart_logic} 
\item Distributed Logi Programming: Parallel Search \cite{schulte00oz_parallel} 
\item include some simple example programs 
\end{itemize}
