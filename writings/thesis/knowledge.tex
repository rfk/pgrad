

\chapter{Knowledge with Hidden Actions}

\label{ch:knowledge}

This chapter develops a formalism for individual knowledge in the
face of hidden actions.

First, existing accounts of epistemic reasoning in the situation calculus
require that whenever an action occurs, all agents \emph{know} that
an action has occurred. This demands a level of synchronicity that
is unreasonable in many multi-agent domains. But if the restriction
is lifted, each agent's knowledge must account for arbitrarily-long
sequences of hidden actions. This requires a second-order induction
axiom which precludes the use of regression for effective automated
reasoning. It also means that agents cannot reason about their own
knowledge, as they may not have enough information to formulate an
appropriate query.


\section{Background}

The foundational axioms $\Sigma$ define a special fluent $K_{0}(agt,s',s)$
that is used to model the initial epistemic uncertainty of the agents,
with $\Dt_{S_{0}}$ containing sentences of the form $\KnowsZ(agt,\phi,S_{0})$
to specify what is initially known%
\footnote{The standard account does not require a separate $K_{0}$ fluent,
as evidenced by equation \eqref{eq:k_s0_standard}. It is required
when incorporating hidden actions, so we introduce it now to maintain
consistency. %
}:\begin{gather*}
K_{0}(agt,s',s)\,\rightarrow Init(s)\wedge Init(s')\\
\KnowsZ(agt,\phi,s)\isdef\forall s':\, K_{0}(agt,s',s')\rightarrow\phi[s']\end{gather*}


The action description function $SR(a,s)$ specifies the sensing result
returned by action $a$ when executed in situation $s$. For non-sensing
actions the value of $SR$ is set to an arbitrary constant.

The dynamics of knowledge are then specified by an additional set
of axioms, which we will denote $\Dt_{K}$. In the standard account
of knowledge $\Dt_{K}$ contains the following axioms:\begin{equation}
Init(s)\rightarrow\left(K(agt,s',s)\,\equiv\, K_{0}(agt,s',s)\right)\label{eq:k_s0_standard}\end{equation}
 \begin{align}
K(agt,s'',do(c,s))\equiv\exists s': & \, s''=do(c,s')\,\wedge K(agt,s',s)\wedge Poss(c,s')\nonumber \\
 & \wedge\,\forall a\in c:\,\left(actor(a)=agt\,\rightarrow\, SR(a,s)=SR(a,s')\right)\label{eq:k_ssa_standard}\end{align}


Equation \eqref{eq:k_s0_standard} ensures that the agents begin with
their knowledge as specified by $\Dt_{S_{0}}$. Equation \eqref{eq:k_ssa_standard}
takes the form of a standard successor state axiom for the $K$ fluent.
It ensures that $s''$ is considered a possible alternative to $do(c,s)$
when $s''$ is the result of doing the same actions $c$ in a situation
$s'$ that is considered a possible alternative to $s$. It must furthermore
have been possible to perform those actions in $s'$, and the sensing
results must match for each action that was carried out by the agent
in question. Thus an agent's knowledge after the occurrence of an
action is completely determined by the combination of its knowledge
before the action, and the sensing results from the action.

\medskip{}


\begin{defn}
We will denote by $\Dt_{K}^{std}$ the axioms of the standard account
of knowledge due to \citet{scherl03sc_knowledge}, as detailed in
equations (\ref{eq:k_s0_standard},\ref{eq:k_ssa_standard}) above. 
\end{defn}
While powerful, this knowledge-representation formalism has an important
limitation: it is fundamentally \emph{synchronous.} Each agent is
assumed to have full knowledge of all actions that have occurred -
in other words, all actions are public. While suitable for some domains,
there are clearly many multi-agent domains where achieving total awareness
of actions would be infeasible. A major contribution of this paper
is a more flexible formalism for knowledge that can be applied to
a much wider range of domains.

A key contribution of \citet{scherl03sc_knowledge} was showing how
to apply the regression operator to formulae containing the $\mathbf{Knows}$
macro, allowing it to be treated syntactically as if it were a primitive
fluent. This means that epistemic queries can be approached using
standard reasoning techniques of the situation calculus. Although
we have changed the notation somewhat to foreshadow the techniques
we will develop in Section \ref{sec:Obs-Knowledge}, their definition
operates as follows. First, define the \emph{results} of a concurrent
action to be the set of $action\#result$ pairs for all primitive
actions perform by the agent in question:\[
\mathbf{res}(agt,c,s)\isdef\{a\#SR(a,s)\,\,|\,\, a\in c\,\wedge\, actor(a)=agt\}\]


This definition is then used to formulate a regression rule as follows:\begin{multline}
\Reg(\Knows(agt,\phi,do(c,s))\isdef\\
\exists y:\, y=\mathbf{res}(agt,c,s)\,\wedge\Knows(agt,\left[Poss(c)\wedge\mathbf{res}(agt,c)=y\right]\rightarrow\Reg(\phi[do(c,s)]),s)\label{eq:reg_k_std}\end{multline}


This works by collecting the sensing results from each action performed
by the agent into the set $y$, then ensuring matching sensing results
in every situation considered possible. It expresses the knowledge
of the agent after a concurrent action in terms of what it knew before
the action, along with the information returned by the action. This
technique relies heavily on the fact that all actions are public,
since it requires every agent's knowledge to be updated in response
to every action.

Repeated applications of $\Reg$ can thus transform a knowledge query
into one that is uniform in the initial situation. While it would
be valid to then expand the $\Knows$ macros and handle the query
using first-order logic, in practice the reasoning procedure would
leave $\Knows$ intact and use a specialised prover based on modal
logic.

It is possible to formulate an alternate successor state axiom for
knowledge that does not assume all actions are public, such as that
of \citet{Lesperance99sitcalc_approach}. Such formulations invariably
require universal quantification over situation terms, to account
for arbitrarily-long sequences of hidden actions. This is incompatible
with regression rules like the above, and these formulations offer
no reasoning procedure other than general second-order theorem proving.
By utilising a new reasoning technique called the {}``persistence
condition'' to reason about universal quantification over situation
terms, our work is the first to provide an account of knowledge with
hidden actions while maintaining regression in the style presented
above as an effective reasoning tool.


\section{Intuitions}

Existing accounts of knowledge in the situation calculus directly
express the dynamics of knowledge update in terms of the actions that
have occurred in the world, as seen in Section \ref{sub:Knowledge-and-Sensing}.
This works well when agents can be assumed to have full knowledge
of the actions that have been performed, but quickly becomes cumbersome
when trying to allow for hidden actions. In this section we first
develop a principled axiomatisation of the \emph{observability} of
actions, then build a powerful yet succinct axiomatisation of knowledge
upon it.

The basic idea is as follows: each occurrence of an action results
in an agent making a set of \emph{observations}. Every situation then
corresponds to a local \emph{view} for that agent: the sequence of
all its observations, excluding cases where the set of observations
was empty. An agent knows something if it is true in all situations
matching its current view. Decoupling knowledge in this manner makes
it easy to express various degrees of observability of actions, from
public actions through to actions that are completely hidden.

To maintain the elegant solution to the frame problem for knowledge
developed by \citet{scherl03sc_knowledge}, the regression rule of
knowledge must be modified to account for arbitrarily long sequences
of hidden actions. We present a new regression rule for knowledge
that uses the persistence condition meta-operator to augment standard
regression and perform the necessary inductive reasoning.

The direct coupling between knowledge and action also has undesirable
implications for situated agents reasoning about their own knowledge.
As a consequence of using regression to handle knowledge queries,
one can only reason about knowledge if one has a rooted situation
term, as the required query is:\[
\Dt_{S_{0}}\cup\Dt_{una}\,\models\,\Reg^{*}(\Knows(agt,\phi,do([c_{1},\dots,c_{n}],S_{0}))\]


In asynchronous domains with hidden actions, where agents are not
necessarily aware how many actions have been performed, the agents
cannot use this formulation to reason about their own knowledge. Since
knowledge is directly coupled to actions, and hence to situation terms,
they cannot construct the appropriate query.

Further demonstrating the power of our approach, we show how the new
regression rules can be applied using an agent's individual view,
rather than requiring a full situation term. Agents can thus use our
techniques to reason about their own knowledge using only their local
information, making the formalism suitable both for reasoning \emph{about},
and reasoning \emph{in}, rich multi-agent domains.


\section{Observations}

To remove the direct coupling between knowledge and actions, we introduce
an explicit notion of \emph{observations}. These are internal notifications
that agents receive when an action has occurred. By expressing knowledge
in terms of observations rather than actions, we ensure that agents
can always reason about their own knowledge based on their own internal
history of observations.

\begin{defnL}
[{Observations}] An observation is a notification event received
by an agent, making it aware of some change in the state of the world.
When an agent receives such a notification, we say that it {}``observed'',
{}``made'' or {}``perceived'' that observation. 
\end{defnL}
Since the state of the world may only change in response to some action,
observations can only be made as a result of some action. For simplicity
we assume that agents perceive observations instantaneously, i.e.
in the same instant as the actions that cause them.

Since {}``observation'' is quite a loaded term, it is worth re-iterating
this point: observations are instantaneous \emph{events} generated
internally by each agent in response to actions occurring in the world.
We make no commitment as to how these events are generated, preferring
a clean delineation between the task of observing change and the dynamics
of knowledge update based on those observations. As with the work
of \citep{scherl03sc_knowledge}, we consider change-awareness to
be the responsibility of a lower-level component of the agent's control
software.

To make this idea concrete, let us introduce an additional sort \noun{Observations}
to the language of the situation calculus, for the moment without
any particular commitment towards what this sort will contain. We
then add to $\Dt_{ad}$ the action description function:\[
Obs(agt,c,s)=o\]
 This function returns a set of observations, and should be read as
{}``when actions $c$ are performed in situation $s$, agent $agt$
will perceive $o$''. Using a set of observations allows an agent
to perceive any number of observations in response to an action occurrence
- perhaps several observations, perhaps none. When $Obs(agt,c,s)$
is the empty set (denoted $\{\}$) then the agent makes no observations
and the actions $c$ are completely hidden.

The concept of a \emph{view} follows naturally - it is the sequence
of all the observations made by an agent as the world has evolved.

\begin{defnL}
[{Views}] An agent's view in a given situation $\mathrm{s}$
is the corresponding sequence of observations made by the agent as
a result of each action in $\mathrm{s}$, excluding those actions
for which no observations were made. 
\end{defnL}
We introduce another sort \noun{Views} consisting of sequences of
sets of observations, with $\epsilon$ being the empty sequence and
the functional fluent $View$ giving the history of observations associated
with a particular situation:\begin{align}
Init(s)\,\rightarrow & \, View(agt,s)=\epsilon\nonumber \\
Obs(agt,a,s)=\{\}\,\rightarrow & \, View(agt,do(a,s))=View(agt,s)\nonumber \\
Obs(agt,a,s)\neq\{\}\,\rightarrow & \, View(agt,do(a,s))=Obs(agt,a,s)\cdot View(agt,s)\label{eq:view_defn}\end{align}


Observations and views can be seen as localised analogues of actions
and situations respectively. An action is a global event that causes
the state of the world to change, while an observation is an internal
event that causes an agent's knowledge of the state of the world to
change. Similarly, a situation represents a complete, global history
of all the actions that have occurred in the world, while a view is
an agent's local history of all the observations it has made. The
situation is an omniscient perspective on the world, the view a local
perspective. As we shall see, this distinction is fundamental to developing
a truly general multi-agent semantics for knowledge.


\section{Knowledge and Observation\label{sub:Knowledge-and-Observation}}

It is a basic tenet of epistemic reasoning that an agent's knowledge
at any particular time must depend solely on its local history: the
knowledge that it started out with combined with the observations
it has made since then \citep{halpern90knowledge_distrib}. Given
an explicit account of the observations made by each agent, the required
semantics of the $K$ relation are clear: $K(agt,s',s)$ must hold
whenever $s'$ is legal, both $s$ and $s'$ would result in the same
view for the agent, and $s$ and $s'$ are rooted at $K$-related
initial situations:\begin{equation}
K(agt,s',s)\equiv K(root(s'),root(s))\wedge Legal(s')\wedge View(agt,s')=View(agt,s)\label{eq:k-desired}\end{equation}


While a wonderfully succinct definition of how knowledge should behave,
this formulation cannot be used directly in a basic action theory.
The dynamics of fluent change must be specified by a successor state
axiom, so we must formulate a successor state axiom for the $K$ fluent
which enforces the above equivalence.

For notational convenience, let us first introduce an action description
predicate $\PbU$ (for {}``possible but unobservable'') indicating
that the actions $c$ are possible in $s$, but no observations will
be made by $agt$ if they are performed:\begin{equation}
\PbU(agt,c,s)\,\equiv\, Poss(c,s)\wedge Obs(agt,c,s)=\{\}\label{eqn:PbU_defn}\end{equation}


By stating that $s\leq_{\PbU(agt)}s'$ we assert that $agt$ would
make no observations were the world to move from situation $s$ to
$s'$. This means that its view in both situations would be identical,
so if it considers $s$ possible then it must also consider $s'$
possible. Following this intuition, we propose the following successor
state axiom to capture the desired dynamics of the knowledge fluent:\begin{align}
K(agt,s'',do(c,s))\equiv & \left[\, Obs(agt,c,s)=\{\}\rightarrow K(agt,s'',s)\,\right]\nonumber \\
\wedge & \left[\, Obs(agt,c,s)\neq\{\}\rightarrow\exists c',s':\, Obs(agt,c',s')=Obs(agt,c,s)\right.\nonumber \\
 & \left.\,\,\,\,\,\,\,\,\wedge\, Poss(c',s')\wedge K(agt,s',s)\wedge do(c',s')\leq_{\PbU(agt)}s''\,\right]\label{eqn:new_k_ssa}\end{align}


If $c$ was totally unobservable, the agent's state of knowledge does
not change. Otherwise, it considers possible any legal successor to
a possible alternate situation $s'$ that can be brought about by
an action $c'$ yielding identical observations. It also considers
possible any future of such a situation in which is would make no
further observations. To reiterate: unlike the standard successor
state axiom from equation \eqref{eq:k_ssa_standard}, our new formalism
requires agents to consider any possible future situation in which
they would make no further observations.

It remains to specify $K$ in the initial situation. The relation
$K_{0}$ defines knowledge before \emph{any} actions have occurred,
but the agents must consider the possibility that some hidden actions
have occurred. In other words, we must include situations where $root(s)\leq_{\PbU(agt)}s$
in the $K$-relation for initial situations. We therefore propose
the following axiom:\begin{gather}
Init(s)\rightarrow\left[K(agt,s'',s)\equiv\exists s'\,.\, K_{0}(agt,s',s)\wedge s'\leq_{\PbU(agt)}s'')\right]\label{eqn:new_k_s0}\end{gather}


\begin{defn}
We will denote by $\Dt_{K}^{obs}$ the axioms for our new observation-based
semantics for knowledge, as detailed in equations (\ref{eq:view_defn},\ref{eqn:new_k_ssa},\ref{eqn:new_k_s0})
above. 
\end{defn}
These axioms suffice to ensure that knowledge behaves as we require:
two situations will be related by $K(agt,s',s)$ if and only if they
result in identical views for that agent, $s'$ is legal, and their
root situations were initially related.

\begin{thm}
\label{thm:k_obs_equiv} For any agent $agt$ and situations $s$
and $s''$:\[
\Dt\cup\Dt_{K}^{obs}\models K(agt,s'',s)\equiv K(root(s''),root(s))\wedge Legal(s'')\wedge View(agt,s'')=View(agt,s)\]

\end{thm}
\begin{proofsketch}
For the \emph{if} direction we establish each of the three conjuncts
individually. The $root$ case is trivial since equation (\ref{eqn:new_k_s0})
always expresses $K(s'',do(c,s))$ in terms of $K(s',s)$. The $Legal$
case relies on the fact that $\PbU$ implies $Poss$, while the $View$
case relies on the fact that $s\leq_{\PbU}s'\rightarrow View(s)=View(s')$.
For the \emph{only-if} direction we show how to construct an $s'$
satisfying the $\exists s'$ parts of equations (\ref{eqn:new_k_ssa},\ref{eqn:new_k_s0}). 
\end{proofsketch}
Using this new formulation, an agent's knowledge is completely decoupled
from the global notion of actions, instead depending only on the local
information that it has observed. It remains to specify precisely
what the \noun{Observations} sort contains, and how the $Obs()$ function
behaves. This will be the focus of the next subsection.


\section{Axiomatising Observations}

Let us begin by considering again the standard account of knowledge
due to \citep{scherl03sc_knowledge}. Its basic assumption that {}``all
agents are aware of all actions'' may be rephrased as {}``when an
action occurs, all agents will observe that action''. Allowing the
\noun{Observations} sort to contain \noun{Action} terms, this assumption
is akin to the following axiom for the $Obs()$ function:\begin{equation}
a\in Obs(agt,c,s)\equiv a\in c\label{eq:ax_obs_std1}\end{equation}


What about sensing information? We can extend the \noun{Observations}
sort to contain $Action\#Result$ pairs and axiomatise like so:\begin{equation}
a\#r\in Obs(agt,c,s)\equiv a\in c\wedge SR(a,s)=r\wedge actor(a)=agt\label{eq:ax_obs_std2}\end{equation}


Using these definitions, our new account of knowledge will behave
identically to the standard account:

\begin{thm}
Suppose $\Dt_{ad}$ contains equations (\ref{eq:ax_obs_std1},\ref{eq:ax_obs_std2})
as definitions of the $Obs()$ function, then for any situation terms
$s$ and $s'$:\[
\Dt\cup\Dt_{K}^{std}\models K(agt,s',s)\,\,\,\,\mathrm{iff}\,\,\,\,\Dt\cup\Dt_{K}^{obs}\models K(agt,s',s)\]

\end{thm}
\begin{proof}
Equations (\ref{eq:ax_obs_std1},\ref{eq:ax_obs_std2}) mean $Obs(agt,c,s)$
cannot be empty for $c\neq\{\}$, so $s=s'$ iff $s\leq_{\PbU(agt)}s'$.
Equations (\ref{eqn:new_k_ssa},\ref{eqn:new_k_s0}) then amount to
simple transformations of equations (\ref{eq:k_s0_standard},\ref{eq:k_ssa_standard})
respectively, meaning that $K$ behaves the same under both theories. 
\end{proof}
We now discuss one straightforward way to generalise this for partial
observability of actions. A new action description predicate $CanObs(agt,a,s)$
is used to indicate that agent $agt$ would observe action $a$ being
performed in situation $s$. If $CanObs(agt,a,s)$ is false, then
that action will be hidden. We can then formulate the $Obs()$ function
according to:\[
a\in Obs(agt,c,s)\equiv a\in c\wedge CanObs(agt,a,s)\]


There is also a subtle limitation in the standard handling of sensing
actions: only the agent performing a sensing action can be aware of
its result. Such a restriction is common but certainly not universal.
For example, if an agent waiting for a train activates a speaker to
determine when it will arrive, the result of this sensing action would
provide information to any other agent within earshot. To generalise
the formalism, an analogous predicate $CanSense(agt,a,s)$ is used
to indicate when sensing information is available to an agent. We
then include bare action terms in an agent's observations when it
observes the action but not its result, and \emph{$Action\#Result$}
terms when it also senses the result:\begin{gather*}
a\in Obs(agt,c,s)\equiv a\in c\wedge CanObs(agt,a,s)\wedge\neg CanSense(agt,a,s)\\
a\#r\in Obs(agt,c,s)\equiv a\in c\wedge SR(a,s)=r\wedge CanObs(agt,a,s)\wedge CanSense(agt,a,s)\end{gather*}


We feel that this formulation provides a good balance between simplicity
and expressiveness; it allows the observability of actions to vary
according to the state of the world, but provides agents with a complete
description of each action that they are capable of observing. In
Section \ref{sub:An-Illustrative-Example} we show how to use this
formalism to model a domain in which agents can only observe actions
performed in the same room as them. To show the flexibility of the
observation-based approach, Section \ref{sub:Observing-Effects} outlines
a more powerful axiomatisation of observations in which agents may
know that some action has occurred without necessarily knowing precisely
what the action was.


\section{Regression}

The final aspect of our new account of knowledge is to extend the
techniques for effective reasoning in the situation calculus to handle
the modified formalism. The appearance of $\leq_{\PbU(agt)}$ in equation
(\ref{eqn:new_k_ssa}) means that our new successor state axiom universally
quantifies over situations, so standard regression techniques cannot
be used. We must appeal to the persistence condition meta-operator
introduced in Section \ref{sub:Property-Persistence} to transform
this quantification into a uniform formula, so that regression can
be applied.

We propose the following as the regression rule for $\Knows$ under
our formalism:\begin{align}
\Reg(\Knows(agt,\phi,do(c,s))\isdef\,\, & \exists o:\, Obs(agt,c,s)=o\nonumber \\
 & \wedge\,\left[o=\{\}\,\rightarrow\,\Knows(agt,\phi,s)\right]\nonumber \\
 & \wedge\,\left[o\neq\{\}\,\rightarrow\,\Knows(agt,\forall c':\, Obs(agt,c')=o\right.\nonumber \\
 & \,\,\,\,\,\,\,\,\,\,\,\wedge Poss(c')\rightarrow\left.\Reg(\Pst(\phi,\PbU(agt)),c'),s)\right]\label{eqn:R_do_c_s}\end{align}


Note the similarity to the standard regression rule for knowledge
in equation \eqref{eq:reg_k_std}. New in our version are: the replacement
of the $\mathbf{res}$ macro with a flexible definition of what the
agent has observed; explicit handling of the case when the agent makes
no observations; and use of the persistence condition to account for
arbitrarily-long sequences of hidden actions.

As required for a regression rule, equation \eqref{eqn:R_do_c_s}
reduces a knowledge query at $do(c,s)$ to a knowledge query at $s$.
It is also intuitively appealing: to know that $\phi$ holds, the
agent must know that in all situations that agree with its observations,
$\phi$ cannot become false without it making an observation - this
is the meaning of $\Pst(\phi,\PbU(agt))$ in the above.

We must also specify the regression of $\Knows$ in the initial situation,
as equation (\ref{eqn:new_k_s0}) also uses the $\leq_{\PbU(agt)}$
ordering. This clause produces an expression in $\KnowsZ$ at $S_{0}$,
meaning that it can be handled by epistemic reasoning about the initial
situation only:\begin{equation}
\Reg(\Knows(agt,\phi,S_{0}))\,\isdef\,\KnowsZ(agt,\Pst(\phi,\PbU(agt)),S_{0})\label{eqn:R_s0}\end{equation}


Using these regression rules, we can handle knowledge queries in our
formalism using standard techniques for effective reasoning in the
situation calculus.

\begin{thm}
\label{thm:Reg_Knows}Given a basic action theory $\Dt$ and a uniform
formula $\phi$:\[
\Dt\cup\Dt_{K}^{obs}\,\models\,\phi[do(c,s)]\equiv\Reg(\phi,c)[s]\]

\end{thm}
\begin{proofsketch}
We need only consider the case of $\phi=\Knows(agt,\phi,s)$. In the
$do(c,s)$ case, we proceed by expanding the definition for $\Knows$
using our new successor state axiom for $K$, collecting sub-formulae
that match the form of the $\Knows$ macro, and using regression and
the persistence condition to render the resulting knowledge expressions
uniform in $s$. In the base case, we apply the persistence condition
to an expansion of $\Knows$ at the initial situation to produce the
desired result. 
\end{proofsketch}
While this reasoning method is suitable for modelling and simulation
purposes, it would be unreasonable for a situated agent to ask {}``do
I know $\phi$ in the current situation?'' using the situation calculus
query $\Dt\models\mathbf{Knows}(agt,\phi,s)$, as it cannot be expected
to have the full current situation $s$. However, it will have its
current view $v$. We define knowledge with respect to a view as follows:\[
\mathbf{Knows}(agt,\phi,v)\,\isdef\,\forall s:\, View(agt,s)=v\wedge root(s)=S_{0}\rightarrow\mathbf{Knows}(agt,\phi,s)\]


It is a straightforward consequence of Theorem \ref{thm:k_obs_equiv}
that this form of knowledge is equivalent to knowledge based on a
situation term having that view and rooted at $S_{0}$. Modifying
the regression rules in equations (\ref{eqn:R_do_c_s},\ref{eqn:R_s0})
to handle formulae of this form is actually simpler than for regression
over situations, as there are no empty observations in a view. The
result is:\begin{align*}
\Reg(\mathbf{Knows}(agt,\phi,o\cdot v))\isdef\,\, & \mathbf{Knows}(agt,\forall c:\, Obs(agt,c)=o\\
 & \,\,\,\,\,\,\,\,\wedge Poss(c)\rightarrow\Reg(\Pst(\phi,\PbU(agt)),c),v)\\
\Reg(\mathbf{Knows}(agt,\phi,\epsilon))\isdef\,\, & \KnowsZ(agt,\Pst(\phi,\PbU(agt)),S_{0})\end{align*}


Using regression in this way, an agent can reduce the query $\Knows(agt,\phi,v)$
to an equivalent query about its knowledge in the initial situation.
Agents can thus reason about their own knowledge using only their
local information. Our work makes it possible to include a situation
calculus model in the implementation of a real-world multi-agent system,
even when agents have only partial awareness of the actions being
performed.

It is worth re-iterating that our regression rules are no longer straightforward
syntactic transformations - rather, they involve a fixpoint calculation
to generate $\Pst(\phi,\PbU(agt))$. Can this really be considered
an effective reasoning technique? Our previous work on the persistence
condition meta-operator \citep{kelly07sc_persistence} discusses the
advantages of this approach in detail. The primary advantage is that
this form of reasoning can be performed at all, as the alternative
is general second-order theorem proving.

Of course, the ultimate proof is in the implementation. We have implemented
a preliminary version of our technique and used it to verify the examples
found in the following section. The code is available from the author's
website%
\footnote{Currently at http://www.csse.unimelb.edu.au/\textasciitilde{}rfk/pknows/%
}.


\section{An Illustrative Example\label{sub:An-Illustrative-Example}}

Consider again the example domain of the party invitation. The fluents
of interest are the location of the party (the function $loc$) and
whether each agent is in the room (the predicate $InRoom$). The action
$read$ reads the invitation and returns the location of the party,
while the non-sensing actions $enter$ and $leave$ cause the agents
to move in/out of the room. The $read$ action is only observed by
agents who are in the room. This domain can be summarised by the following
axioms:\begin{gather*}
loc(S_{0})=C\\
loc(do(c,s))=l\equiv loc(s)=l\end{gather*}
 \begin{gather*}
InRoom(Ann,S_{0})\equiv InRoom(Bob,S_{0})\equiv true\\
InRoom(agt,do(c,s))\equiv enter(agt)\in c\,\vee\, InRoom(agt,s)\wedge leave(agt)\notin c\\
Poss(enter(agt),s)\equiv\neg InRoom(agt,s)\\
Poss(leave(agt),s)\equiv InRoom(agt,s)\end{gather*}
 \begin{gather*}
Poss(read(agt),s)\equiv InRoom(agt,s)\\
SR(read(agt),s)=r\equiv r=loc(s)\end{gather*}
 \begin{gather*}
\forall agt,l:\,\neg\KnowsZ(agt,loc=l,S_{0})\\
\forall agt_{1},agt_{2},l:\,\KnowsZ(agt_{1},\neg\KnowsZ(agt_{2},loc=l),S_{0})\\
\forall agt:\,\KnowsZ(agt,InRoom(Ann)\wedge InRoom(Bob),S_{0})\end{gather*}
 \begin{gather*}
CanObs(agt,leave(agt'),s)\equiv CanObs(agt,enter(agt'),s)\equiv true\\
CanSense(agt,leave(agt'),s)\equiv CanSense(agt,enter(agt'),s)\equiv false\\
CanObs(agt,read(agt'),s)\equiv InRoom(agt',s)\\
CanSense(agt,read(agt'),s)\equiv agt=agt'\end{gather*}


\medskip{}


The following are examples of knowledge queries that can be posed
in our formalism, a brief explanation of their outcome, and a demonstration
of how they can be answered using our new regression rules. Each has
been verified by the preliminary implementation of our reasoning engine.

\begin{example}
Initially, Ann doesn't know where the party is:\[
\Dt\cup\Dt_{K}^{obs}\models\neg\exists l:\,\mathbf{Knows}(Ann,loc=l,S_{0})\]

\end{example}
It is given that $\neg\exists l:\,\KnowsZ(Ann,loc=l,S_{0})$, and
the only way for Ann to learn such information is by performing a
$read$ action. Since she would always observe such an action, she
cannot have learnt the party's location as a result of hidden actions
and the example is entailed. Formally:\begin{align*}
\Reg(\neg\exists l:\,\mathbf{Knows}(Ann,loc=l,S_{0}))\,\Rightarrow\,\,\,\, & \neg\exists l:\,\KnowsZ(Ann,\Pst(loc=l,PbU(Ann)),S_{0})\\
\Pst(loc=l,PbU(Ann))\,\Rightarrow\,\,\,\, & loc=l\end{align*}


So the query reduces to:\[
\neg\exists l:\,\KnowsZ(Ann,loc=l,S_{0})\]


Which is entailed by the domain.\\
 \\


\begin{example}
After reading the invitation, Bob will know where the party is:\[
\Dt\cup\Dt_{K}^{obs}\models\Knows(Bob,loc=C,do(\{read(Bob)\},S_{0}))\]

\end{example}
The sensing results of the $read$ action inform Bob of the location
of the party. Since this location cannot change after any sequence
of hidden actions, he can be sure of the party's location. Formally,
using the fact that $Obs(Bob,\{read(Bob)\},s)=\{read(Bob)\#loc(s)\}$:\begin{multline*}
\Reg(\Knows(Bob,loc=C,do(\{read(Bob)\},S_{0})))\,\Rightarrow\\
\exists o:\, Obs(Bob,\{read(Bob)\},S_{0})=o\,\wedge\,\\
\Knows(Bob,\forall c':\, Poss(c')\wedge Obs(Bob,c')=o\rightarrow\Reg(\Pst(loc=C,PbU(Bob)),c'),S_{0})\end{multline*}


Since $\Reg(\Pst(loc=C,PbU(Bob)),c')\Rightarrow loc=c$ and $loc(S_{0})=C$,
this simplifies to:\[
\Knows(Bob,\forall c':\, Poss(c')\wedge Obs(Bob,c')=\{read(Bob)\#C\}\rightarrow loc=C,S_{0})\]


Since the only possible value of $c'$ that satisfies the antecedent
is $\{read(Bob)\}$, we can insert the definitions of $Poss$ and
$Obs$ to obtain:\[
\Knows(Bob,InRoom(Bob)\wedge loc=C\,\rightarrow\, loc=C,S_{0})\]


This tautology is clearly entailed by the domain.\\


\begin{example}
Initially, Bob knows that Ann doesn't know where the party is:\[
\Dt\cup\Dt_{K}^{obs}\models\mathbf{Knows}(Bob,\neg\exists l:\,\Knows(Ann,loc=l),S_{0})\]

\end{example}
Ann could learn the location of the party by performing the $read$
action, but since Bob is in the room he would observe this action
taking place. Since he has not observed it, he can conclude that Ann
does not know the location of the party. Formally:\begin{multline*}
\Reg(\mathbf{Knows}(Bob,\neg\exists l:\,\Knows(Ann,loc=l),S_{0}))\,\Rightarrow\\
\KnowsZ(Bob,\Pst(\neg\exists l:\,\Knows(Ann,loc=l),PbU(Bob)),S_{0})\end{multline*}
 \begin{multline*}
\Pst(\neg\exists l:\,\Knows(Ann,loc=l),PbU(Bob))\,\Rightarrow\\
\neg\exists l:\,\Knows(Ann,loc=l)\wedge\left(InRoom(Bob)\,\vee\,\neg InRoom(Ann)\right)\end{multline*}


So the query reduces to:\[
\KnowsZ(Bob,\neg\exists l:\,\Knows(Ann,loc=l)\wedge\left(InRoom(Bob)\,\vee\,\neg InRoom(Ann)\right),S_{0})\]


Which is entailed by the domain.\\
 \\


\begin{example}
After leaving, Bob won't know that Ann doesn't know where the party
is:\[
\Dt\cup\Dt_{K}^{obs}\models\mathbf{\neg Knows}(Bob,\neg\exists l:\,\Knows(Ann,loc=l),do(\{leave(Bob)\},S_{0}))\]

\end{example}
Once Bob leaves the room, he would be unable to observe Ann reading
the invitation. He must therefore consider it possible that she has
read it, and may know the location of the party. Formally, we can
use $Obs(Bob,\{leave(Bob)\})=\{leave(Bob)\}$ to regress the outer
expression as follows:\begin{multline*}
\Reg(\neg\Knows(Bob,\phi,do(\{leave(Bob)\},S_{0})))\,\Rightarrow\\
\neg\Knows(Bob,InRoom(Bob)\rightarrow\Reg(\Pst(\phi,PbU(Bob)),\{leave(Bob)\}),S_{0})\end{multline*}


For the inner expression, we have from the previous example:\begin{multline*}
\Pst(\neg\exists l:\,\Knows(Ann,loc=l),PbU(Bob))\,\Rightarrow\\
\neg\exists l:\,\Knows(Ann,loc=l)\wedge\left(InRoom(Bob)\,\vee\,\neg InRoom(Ann)\right)\end{multline*}


This expression is key: for Bob to know $\neg\exists l:\Knows(Ann,loc=l)$,
he must also know either that he is in the room (and will thus observe
the $read(Ann)$ action if it occurs) or that Ann is not in the room
(so the $read(Ann)$ action will not be possible). Otherwise, Ann
could learn the location of the party without him making any further
observations.

When we regress over the action $\{leave(Bob)\}$ then $InRoom(Bob)$
is made false:\begin{multline*}
\Reg(\Pst(\neg\exists l:\,\Knows(Ann,loc=l),PbU(Bob)),\{leave(Bob)\})\,\Rightarrow\\
\neg\exists l:\,\Knows(Ann,loc=l)\wedge\left(false\,\vee\,\neg InRoom(Ann)\right)\end{multline*}


And the entire expression can be simplified to:\[
\neg\Knows(Bob,InRoom(Bob)\rightarrow\neg\exists l:\,\Knows(Ann,loc=l)\wedge\neg InRoom(Ann),S_{0})\]


Since Ann is known to be in the room, this expression will be entailed
by the domain.


\section{Observing the Effects of Actions\label{sub:Observing-Effects}}

In many domains it would be infeasible for an agent to observe a particular
action occurring, but it may observe some of the effects of that action.
For example, suppose that an agent monitors the state of a light in
its environment, such that it notices it changing from dark to light.
While it knows that \emph{some} action must have occurred to produce
that effect, it may not be sure precisely what action took place (e.g.
precisely \emph{who} turned on the light).

This can be modelled by further extending the \noun{Observations}
sort. Suppose that the observation term $f_{\phi}$ indicates that
a particular property of the world $\phi$ has changed from false
to true and (for simplicity) that this information would be available
to all agents. The following could be used to include this information
in an agent's observations:\[
f_{\phi}\in Obs(agt,c,s)\equiv\neg\phi[s]\wedge\Reg(\phi,c)\]


Note that since $Obs()$ is an action description predicate, we must
use regression to ensure that the right-hand side of this definition
is uniform in $s$. Expanding on the example of the light, we might
have an axiom like this:\[
lightCameOn\in Obs(agt,c,s)\equiv\neg lightIsOn(s)\wedge\exists agt':\, turnLightOn(agt')\in c\]


When the light is switched on, each agent's observation set will contain
$lightCameOn$, and they will know that this change has occurred without
necessarily knowing the specific action responsible for the change.
That this powerful new ability is a straightforward extension of our
approach highlights the flexibility and robustness of the observation-based
semantics.


\section{TODO: Approximate Epistemic Reasoning}

\begin{itemize}
\item Restrict formulae allowed inside $\Knows$ to atomic literals, in
style of \citet{demolombe00tractable_sc_belief}. 
\item Leverage encoding develope by \citet{petrick02knowledge_equivalence} 
\end{itemize}

\section{Discussion}

In this section, we have first developed a principled axiomatisation
of the observability of actions, using the notion of observations
and views as analogues of actions and situations that are localised
to an individual agent. This terminology has been deliberately chosen
to match similar concepts in other formalisations of knowledge, such
as the well-known treatise of \citet{halpern90knowledge_distrib}.
By reifying these concepts as terms in the logic, we are able to give
a succinct definition of the dynamics of the knowledge fluent and
prove that its behaviour matches our intuitive expectations.

As an example of why this is important, consider one of the few existing
formulations of knowledge in the situation calculus that allows for
hidden actions, that of \citep{Lesperance99sitcalc_approach}. Their
successor state axiom for the $K$ fluent is as follows:\begin{align*}
K(agt,s'',do(a,s))\equiv\,\,\, & \exists s':\, K(agt,s',s)\\
 & \wedge\,(actor(a)\neq agt\,\rightarrow\, s'\leq_{actor(a)\neq agt}s''))\\
 & \wedge\,(actor(a)=agt\,\rightarrow\,\exists s^{*}:\,\left[s'\leq_{actor(a)=agt}s^{*}\wedge\right.\\
 & \,\,\,\,\,\,\,\,\,\,\,\,\left.s''=do(a,s^{*})\wedge Poss(a,s^{*})\wedge sr(a,s)=sr(a,s^{*}))\right]\end{align*}


In this case agents are only aware of the actions that they themselves
perform, and they consider possible an arbitrary sequence of hidden
actions preceding each action of their own. However, this formulation
has a subtle problem: an agent's knowledge can change in response
to actions performed by others. Suppose that $agt$ has just performed
action $a_{1}$, so the world is in situation $do(a_{1},s)$. Another
agent then performs the action $a_{2}$, leaving the world in situation
$do(a_{2},do(a_{1},s))$. Since it is not aware of the occurrence
of $a_{2}$, the knowledge of $agt$ should be unchanged between these
two situations. This is not the case under the formulation of \citeauthor{Lesperance99sitcalc_approach}.
By explicitly formalising the notion of a view, our framework avoids
such problems.

A further advantage of our explicit axiomatisation of observations
is in establishing properties of the knowledge relation. A major theorem
of \citet{scherl03sc_knowledge} states that if the $K$-relation
is reflexive, symmetric or transitive at the initial situation, then
it has that property at every situation. In our formulation these
are all simple corollaries of Theorem \ref{thm:k_obs_equiv}, based
on the reflexive, symmetric and transitive nature of the equality
symbol.

We have demonstrated that our formalism is expressive enough to capture
the standard account of knowledge based on public actions, as well
as more complex formulations where the observability of actions depends
on the state of the world. We have also demonstrated that despite
allowing for arbitrarily-long sequences of hidden actions, our formalism
still permits automated reasoning for handling knowledge queries,
including a preliminary implementation of such a reasoning system.

Of course, the effectiveness of automated reasoning is now highly
dependent on the effectiveness of calculating the persistence condition.
Since this is a fixpoint calculation, it can be computationally expensive
and even undecidable in very complex domains. By factoring out the
necessary inductive reasoning into a separate operator, it can now
be studied and improved in isolation. We have already identified several
classes of basic action theory in which the persistence condition
can be calculated quite effectively, and our investigations in this
area are ongoing; see the reference for details \citep{kelly07sc_persistence}.

Finally, we have shown that a simple modification to our regression
rules allows a situated agent to reason about its own knowledge using
only its local view, rather than requiring a full situation term.
Our new observation-based semantics thus provides a powerful account
of knowledge suitable both for reasoning \emph{about}, and for reasoning
\emph{in}, asynchronous multi-agent domains.

\label{ch:references} \bibliographystyle{plainnat} \bibliographystyle{plainnat}
\bibliography{../../library/references}

