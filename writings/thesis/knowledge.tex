

\chapter{Knowledge with Hidden Actions}

\label{ch:knowledge}

This chapter develops a new theory of knowledge in the situation calculus
that directly leverages the agent-local perspective developed in Chapter
\ref{ch:observations}.

Existing accounts of epistemic reasoning in the situation calculus
require that whenever an action occurs, all agents \emph{know} that
an action has occurred. This demands a level of synchronicity that
is unreasonable in many multi-agent domains. In asynchronous domains,
each agent's knowledge must instead account for arbitrarily-long sequences
of \emph{hidden actions}. This requires a second-order induction axiom
which precludes the use of regression for effective automated reasoning.
It also make it difficult for agents to reason about their own knowledge,
as they may not have enough information to formulate an appropriate
query.

To overcome this limitation we combine two of the contributions developed
in preceding chapters - the explicit representation of an agent's
local perspective from Chapter \ref{ch:observations}, and the persistence
condition meta-operator from Chapter \ref{ch:persistence} - to formulate
an account of knowledge in the situation calculus that can faithfully
represent the hidden actions inherent in asynchronous domains while
maintaining a regression rule for effective automated reasoning.

We begin by developing an axiomatisation of knowledge based explicitly
on each agent's local view. This axiomatisation is shown to be sound,
and to faithfully preserve important properties of the agent's epistemic
state through the occurrence of actions. Moreover, our formulation
is \emph{elaboration tolerant}, automatically preserving these properties
in the face of more complex information-producing actions, such as
guarded sensing actions and speech acts, that can invalidate these
properties if not axiomatised carefully.

To formulate a regression rule for knowledge, we appeal to the persistence
condition operator to factor out the inductive reasoning required
for dealing with hidden actions. We propose a new regression rule
that is sound and complete with respect to our axiomatisation of knowledge,
and demonstrate how it can used by a situated agent to reason about
its own knowledge in a straightforward manner.

The end result is a significantly more general and robust theory of
knowledge in the situation calculus that still permits an effective
reasoning procedure.

The chapter proceeds as follows: after introducing more details background
material on epistemic reasoning in the situation calculus in Section
\ref{sec:Knowledge:Background}, we develop the axioms for our new
observation-based account of knowledge in Section \ref{sec:Knowledge:Observation}.
Section \ref{sec:Knowledge:Properties} explores the properties of
our formalism with reference to the standard account. Section \ref{sec:Knowledge:Regression}
develops a regression rule for our formalism using the persistence
condition operator, while Section \ref{sec:Knowledge:Example} gives
a brief example of its use for reasoning about a partially-observable
domain. Section \ref{sec:Knowledge:Advances} links our work to a
related formalism based on knowing fluent literals, while Section
\ref{sec:Knowledge:Discussion} concludes with a general discussion
and summary.


\section{Background\label{sec:Knowledge:Background}}

Recall from Section \ref{sec:Background:Epistemic} that the dynamics
of knowledge in the situation calculus are specific using an additional
set of axioms $\Dt_{K}$, which define the behaviour of a special
knowledge-fluent $K$. In the standard account of knowledge, based
on the work of \citet{scherl03sc_knowledge} and incorporating concurrent
actions and multiple agents \citep{shapiro98specifying_ma_systems,scherl03conc_knowledge},
the axioms in $\Dt_{K}$ are the following:\begin{equation}
Init(s)\rightarrow\left(K(agt,s',s)\,\equiv\, K_{0}(agt,s',s)\right)\label{eq:k_s0_standard}\end{equation}
 \begin{multline}
K(agt,s'',do(c,s))\equiv\exists s':\, s''=do(c,s')\,\wedge K(agt,s',s)\\
\wedge Legal(c,s')\wedge\,\forall a\in c:\,\left(actor(a)=agt\,\rightarrow\, SR(a,s)=SR(a,s')\right)\label{eq:k_ssa_standard}\end{multline}


Equation \eqref{eq:k_s0_standard} ensures that the agents begin with
their knowledge as specified by the initial knowledge fluent $K_{0}$.
In fact, the work of \citep{scherl03sc_knowledge,shapiro98specifying_ma_systems,scherl03conc_knowledge}
does not use an explicit $K_{0}$ fluent and instead specifies initial
knowledge directly using $K$. The introduction of $K_{0}$ in this
case is purely cosmetic, but it will make comparisons with our new
formalism easier.

Equation \eqref{eq:k_ssa_standard} takes the form of a standard successor
state axiom for the $K$ fluent. It ensures that $s''$ is considered
a possible alternative to $do(c,s)$ when $s''$ is the result of
doing the same actions $c$ in a situation $s'$ that is considered
a possible alternative to $s$. It must furthermore have been possible
to perform those actions in $s'$, and the sensing results must match
for each action that was carried out by the agent in question. Thus
an agent's knowledge after the occurrence of an action is completely
determined by the combination of its knowledge before the action,
and the sensing results from the action.

\medskip{}


\begin{defn}
We will denote by $\Dt_{K}^{std}$ the axioms of the standard account
of knowledge due to \citep{scherl03sc_knowledge,scherl03conc_knowledge},
as detailed in equations (\ref{eq:k_s0_standard},\ref{eq:k_ssa_standard})
above. 
\end{defn}
While powerful, this knowledge-representation formalism has an important
limitation: it is fundamentally \emph{synchronous.} Each agent is
assumed to have full knowledge of all actions that have occurred -
in other words, all actions are public. While suitable for some domains,
there are clearly many multi-agent domains where achieving total awareness
of actions would be infeasible. A major contribution of this chapter
is a more flexible formalism for knowledge that can be applied to
a much wider range of domains.


\subsection{Reasoning about Knowledge}

A key contribution of \citet{scherl03sc_knowledge} was showing how
to apply the regression operator to formulae containing the $\mathbf{Knows}$
macro, allowing it to be treated syntactically as if it were a primitive
fluent. This means that epistemic queries can be approached using
standard reasoning techniques of the situation calculus. Although
we have changed the notation somewhat to foreshadow the techniques
we will develop in Section \ref{sec:Knowledge:Regression}, their
definition operates as follows. First, define the \emph{results} of
a concurrent action to be the set of $action\#result$ pairs for all
primitive actions performed by the agent in question:\[
\mathbf{res}(agt,c,s)\isdef\{a\#SR(a,s)\,\,|\,\, a\in c\,\wedge\, actor(a)=agt\}\]


This definition is then used to formulate a regression rule as follows:\begin{multline}
\Reg(\Knows(agt,\phi,do(c,s))\isdef\,\,\,\,\,\,\,\,\,\,\exists y:\, y=\mathbf{res}(agt,c,s)\,\\
\wedge\Knows(agt,\left[Poss(c)\wedge\mathbf{res}(agt,c)=y\right]\rightarrow\Reg(\phi[do(c,s)]),s)\label{eq:reg_k_std}\end{multline}


This works by collecting the sensing results from each action performed
by the agent into the set $y$, then ensuring matching sensing results
in every situation considered possible. It expresses the knowledge
of the agent after a concurrent action in terms of what it knew before
the action, along with the information returned by the action. This
technique relies heavily on the fact that all actions are public,
since it requires every agent's knowledge to be updated in response
to every action.

As with the non-epistemic case, repeated applications of $\Reg$ can
transform a knowledge query into one that is uniform in the initial
situation. While it would be valid to then expand the $\Knows$ macros
and handle the query using first-order logic, in practice the reasoning
procedure would leave $\Knows$ intact and use a specialised prover
based on modal logic.


\subsection{Accessibility Properties\label{sec:Knowledge:Accessibility-Properties}}

A fundamental aspect of epistemic reasoning is identifying certain
properties of the $K$ relation - in modal logic terminology, the
agent's \emph{accessibility} relation - that guarantee certain desired
axioms of knowledge. The most commonly-used accessibility restrictions
are:

\begin{description}
\item [{{{{{{Reflexivity:}}}}}}] $K(agt,s,s)$ 
\item [{{{{{{Transitivity:}}}}}}] $K(agt,s_{2},s_{1})\wedge K(agt,s_{3},s_{2})\rightarrow K(agt,s_{3},s_{1})$ 
\item [{{{{{{Euclidean:}}}}}}] $K(agt,s_{2},s_{1})\wedge K(agt,s_{3},s_{1})\rightarrow K(agt,s_{3},s_{2})$ 
\item [{{{{{{Symmetry:}}}}}}] $K(agt,s_{2},s_{1})\rightarrow K(agt,s_{1},s_{2})$ 
\end{description}
The first three of these properties are directly equivalent to asserting
three important axioms about the knowledge operator:

\begin{description}
\item [{{{{{{Correct~Knowledge:}}}}}}] $\Knows(agt,\phi)\rightarrow\phi$ 
\item [{{{{{{Positive~Introspection:}}}}}}] $\Knows(agt,\phi)\rightarrow\Knows(agt,\Knows(agt,\phi))$ 
\item [{{{{{{Negative~Introspection:}}}}}}] $\neg\Knows(agt,\phi)\rightarrow\Knows(agt,\neg\Knows(agt,\phi))$ 
\end{description}
In particular, the restriction of $K$ to be reflexive ensures that
the logic is one of \emph{knowledge} rather than \emph{belief} --
the agents cannot know statements that are not true in the real world.

One of the major theorems of \citep{scherl03sc_knowledge} is that
if one or more of these properties hold for the $K$ relation in the
initial situation, then they will hold for the $K$ relation in all
future situations. This means that the above axioms of knowledge are
preserved through the occurrence of actions, an important confirmation
that the semantics of knowledge have been axiomatised correctly; the
occurrence of an action should not, for example, remove the agent's
ability to introspect its own knowledge.


\subsection{Extending Knowledge Theories\label{sec:Knowledge:Extending}}

As we briefly discussed in Section \ref{sec:Observations:Background},
different kinds of information-generating actions are typically modelled
by directly modifying the successor state axiom for $K$ \citep{shapiro98specifying_ma_systems,Lesperance99sitcalc_approach,shapiro01casl_feat_inter,Petrick06thesis,shapiro07sc_goal_change}.
The axiom typically grows from the form presented in equation \eqref{eq:k_ssa_standard}
to look something like the following:\begin{gather*}
K(agt,s'',do(a,s))\equiv\exists s':\, s''=do(a,s)\wedge K(agt,s',s)\wedge\\
a=A_{1}(\vars{x}_{1})\,\rightarrow\,\Phi_{1}(\vars{x}_{1},s,s')\\
\dots\\
a=A_{n}(\vars{x}_{n})\,\rightarrow\,\Phi_{n}(\vars{x}_{1},s,s')\end{gather*}


Here each formula $\Phi_{i}$ encodes the particular semantics of
action $A_{i}$, typically restricting $s'$ to satisfy some formula
if and only if it is satisfied at $s$. Unfortunately, these extensions
do not in general maintain the important theorems of the standard
account of knowledge \citep{scherl03sc_knowledge}; there is no guarantee
that the modified axioms will preserve accessibility properties of
$K$, or will permit a regression rule for $\Knows$.

For example, \citet{Petrick06thesis} extends the approach of \citep{scherl03sc_knowledge}
to include \emph{guarded} sensing actions. These actions cause the
agent to learn that some formula $\phi$ holds, but only if an additional
guard formula $\psi$ also holds in the world. They are included in
the axiom for $K$ as follows:\[
a=sense_{i}\,\rightarrow\,\left[\psi_{i}(s)\,\rightarrow\,\phi_{i}(s)\equiv\phi_{i}(s')\right]\]


It is straightforward to demonstrate that the modified successor state
axiom is no longer guaranteed to preserve the accessibility properties
identified in Section \eqref{sec:Knowledge:Accessibility-Properties},
although it is possible to syntactically restrict $\psi$ to regain
this important result \citep{Petrick06thesis}.

As another example, consider the alternate successor state axiom for
$K$ proposed by \citet{Lesperance99sitcalc_approach}, in one of
the few existing works in the situation calculus that does not assume
a synchronous domain:\begin{align*}
K(agt,s'',do(a,s))\equiv\,\,\, & \exists s':\, K(agt,s',s)\\
 & \wedge\,(actor(a)\neq agt\,\rightarrow\, s'\leq_{actor(a)\neq agt}s''))\\
 & \wedge\,(actor(a)=agt\,\rightarrow\,\exists s^{*}:\,\left[s'\leq_{actor(a)\neq agt}s^{*}\wedge\right.\\
 & \,\,\,\,\,\,\,\,\,\,\,\,\left.s''=do(a,s^{*})\wedge Poss(a,s^{*})\wedge sr(a,s)=sr(a,s^{*}))\right]\end{align*}


In order to account for the actions of other agents being completely
hidden, this axiom must universally quantify over situation terms.
It is therefore incompatible with the standard regression rule for
knowledge, and \citep{Lesperance99sitcalc_approach} offers no reasoning
procedure other than general second-order theorem proving.

The standard approach to formalising the local information available
to each agent, by directly modifying the successor state axiom for
knowledge, is not elaboration tolerant -- it is easy to invalidate
important properties of the formalism. As we shall demonstrate in
this chapter, by basing our approach on an explicit, separately-axiomatised
account of the local perspective of each agent, our formulation robustly
maintains these properties in the face of theory elaboration.


\section{Knowledge and Observation\label{sec:Knowledge:Observation}}

In this section we develop the axioms for our new formulation of knowledge,
which is based on the explicit account of each agent's local perspective
that we developed in Chapter \ref{ch:observations}. We begin from
one of the basic tenets of epistemic reasoning, as described by \citep{halpern90knowledge_distrib}
-- that an agent's knowledge at any particular time must depend solely
on its local history: the knowledge that it started out with combined
with the observations it has made since then .

Given an explicit account of the observations made by each agent,
the required semantics of the $K$ relation are clear: $K(agt,s',s)$
must hold whenever $s'$ is legal, both $s$ and $s'$ would result
in the same view for the agent, and $s$ and $s'$ are rooted at $K_{0}$-related
initial situations:\begin{equation}
K(agt,s',s)\equiv Legal(s')\wedge View(agt,s')=View(agt,s)\wedge K_{0}(root(s'),root(s))\label{eq:k-desired}\end{equation}


In essence, this is a direct encoding into the situation calculus
of the definitions of knowledge from the classic epistemic reasoning
literature \citep{parikh85dist_knowledge,halpern90knowledge_distrib,fagin95}.

While a wonderfully succinct definition of how knowledge should behave,
this formulation cannot be used directly in a basic action theory.
The dynamics of fluent change must be specified by a successor state
axiom, so we must formulate a successor state axiom for the $K$ fluent
which enforces the above equivalence.

For notational convenience, let us first introduce an action description
predicate $\LbU$ (for {}``legal but unobservable'') indicating
that the actions $c$ are legally performed in $s$, but no observations
will be made by $agt$ if they occur:\begin{equation}
\LbU(agt,c,s)\,\equiv\, Legal(c,s)\wedge Obs(agt,c,s)=\{\}\label{eqn:LbU_defn}\end{equation}


By stating that $s\leq_{\LbU(agt)}s'$, we assert that $agt$ would
make no observations were the world to move from situation $s$ to
$s'$. This means that the agent's view in both situations would be
identical, so if it considers $s$ possible then it must also consider
$s'$ possible. Following this intuition, we propose the following
successor state axiom to capture the desired dynamics of the knowledge
fluent:\begin{align}
K(agt,s'',do(c,s))\equiv & \left[\, Obs(agt,c,s)=\{\}\rightarrow K(agt,s'',s)\,\right]\nonumber \\
\wedge & \left[\, Obs(agt,c,s)\neq\{\}\rightarrow\exists c',s':\, Obs(agt,c',s')=Obs(agt,c,s)\right.\nonumber \\
 & \left.\,\,\,\,\,\,\,\,\wedge\, Legal(c',s')\wedge K(agt,s',s)\wedge do(c',s')\leq_{\LbU(agt)}s''\,\right]\label{eqn:new_k_ssa}\end{align}


If $c$ was totally unobservable, the agent's state of knowledge does
not change. Otherwise, it considers possible any legal successor to
a possible alternate situation $s'$ that can be brought about by
an action $c'$ yielding identical observations. It also considers
possible any future of such a situation in which is would make no
further observations. To reiterate: unlike the standard successor
state axiom from equation \eqref{eq:k_ssa_standard}, our new formalism
requires agents to consider any possible future situation in which
they would make no further observations.

It remains to specify $K$ in the initial situation. The relation
$K_{0}$ defines knowledge before \emph{any} actions have occurred,
but the agents must consider the possibility that some hidden actions
have occurred. In other words, we must include situations where $root(s)\leq_{\LbU(agt)}s$
in the $K$-relation for initial situations. We therefore propose
the following axiom:\begin{gather}
Init(s)\rightarrow\left[K(agt,s'',s)\equiv\exists s':\, K_{0}(agt,s',s)\wedge s'\leq_{\LbU(agt)}s'')\right]\label{eqn:new_k_s0}\end{gather}


\begin{defn}
We will denote by $\Dt_{K}^{obs}$ the axioms for our new observation-based
semantics for knowledge, as detailed in equations (\ref{eqn:new_k_ssa},\ref{eqn:new_k_s0})
above. 
\end{defn}
These axioms suffice to ensure that knowledge behaves as we require:
two situations will be related by $K(agt,s',s)$ if and only if they
result in identical views for that agent, $s'$ is legal, and their
root situations were initially related.

\begin{thm}
\label{thm:k_obs_equiv} For any agent $agt$ and situations $s$
and $s''$:\begin{multline*}
\Dt\cup\Dt_{K}^{obs}\models K(agt,s'',s)\equiv\\
Legal(s'')\wedge View(agt,s'')=View(agt,s)\wedge K_{0}(root(s''),root(s))\end{multline*}

\end{thm}
\begin{proofsketch}
For the \emph{if} direction we establish each of the three conjuncts
individually. The $root$ case is trivial since equation (\ref{eqn:new_k_ssa})
always expresses $K(s'',do(c,s))$ in terms of $K(s',s)$, while equation
(\ref{eqn:new_k_s0}) relates $K$ for initial situations back to
$K_{0}$. The $Legal$ case relies on the fact that $\LbU$ implies
$Legal$, while the $View$ case relies on the fact that $s\leq_{\LbU}s'\rightarrow View(s)=View(s')$.
For the \emph{only-if} direction we show how to construct an $s'$
satisfying the $\exists s'$ parts of equations (\ref{eqn:new_k_ssa},\ref{eqn:new_k_s0}). 
\end{proofsketch}
Using this new formulation, an agent's knowledge is completely decoupled
from the global notion of actions, instead depending only on the local
information that it has observed. Of course, this must be combined
with a specific axiomatisation of how the $Obs$ function behaves.
Any of the axiomatisations demonstrated in Chapter \ref{ch:observations}
can be used, and our account of knowledge can be included unmodified.

As a demonstration of the correctness of their axioms, \citet{scherl03sc_knowledge}
prove five properties of their formalism: that knowledge-producing
actions have only knowledge-producing effects; that unknown fluents
remain unknown by default; that knowledge incorporates the results
of sensing actions; that known fluents remain known by default; and
that agents have knowledge of the effects of their actions.

However, the intuition behind these properties depends heavily on
the assumption of a single agent and the separation of actions into
two classes: knowledge-producing actions that only return sensing
information, and ordinary actions that only affect the state of the
world. In asynchronous multi-agent domains, these restrictions cannot
be meaningfully applied.

For example, it is entirely possible that a knowledge-producing action
and an ordinary action are performed concurrently by two different
agents, so the results of a sensing action might be immediately made
invalid. Moreover, suppose that an agent performs an action to make
a formula $\phi$ true, but there is a series of hidden actions that
could subsequently make $\phi$ false. The agent cannot meaningfully
claim to know $\phi$, since it could become false without updating
the local view of that agent.

The proofs used in \citep{scherl03sc_knowledge} all hinge on showing
that the situations $K$-related to $do(a,s)$ are precisely the {}``correct''
ones, where correctness if formulated in terms of the preconditions
and effects of $a$. We claim that in our formulation, the {}``correct''
situations to be related to $do(a,s)$ are precisely those with the
same view, and the validity of Theorem \ref{thm:k_obs_equiv} provides
sufficient justification for the correctness of our knowledge axioms.


\section{Properties of Knowledge\label{sec:Knowledge:Properties}}

With the basic axioms in place, let us study some properties of our
formalism in greater detail. We begin by comparing it to the standard
account of knowledge due to \citet{scherl03sc_knowledge}. Its basic
assumption that {}``all agents are aware of all actions'' is captured
in our observation-based formulation using equations (\ref{eq:Observations:ObsStd1},\ref{eq:Observations:ObsStd2})
from Chapter \ref{ch:observations}, which we repeat here for convenience:\begin{gather*}
a\in Obs(agt,c,s)\,\equiv\, a\in c\\
a\#r\in Obs(agt,c,s)\,\equiv\, a\in c\wedge actor(a)=agt\wedge SR(a,s)=r\end{gather*}


That is, an agent observes all actions that occur, and additionally
observes the sensing results of all actions that it performs. If these
definitions are used, our new account of knowledge will behave identically
to the standard account:

\begin{thm}
Suppose $\Dt_{ad}$ contains equations (\ref{eq:Observations:ObsStd1},\ref{eq:Observations:ObsStd2})
as definitions of the $Obs()$ function, then for any legal situation
terms $\sigma$ and $\sigma'$:\[
\Dt\cup\Dt_{K}^{std}\models K(agt,\sigma',\sigma)\,\,\,\,\mathrm{iff}\,\,\,\,\Dt\cup\Dt_{K}^{obs}\models K(agt,\sigma',\sigma)\]

\end{thm}
\begin{proof}
Equations (\ref{eq:Observations:ObsStd1},\ref{eq:Observations:ObsStd2})
mean $Obs(agt,c,s)$ cannot be empty for $c\neq\{\}$, so $s=s'$
iff $s\leq_{\LbU(agt)}s'$. Since we restrict our attention to legal
situations, we can substitute $\bot$ for $Obs(agt,c,s)=\{\}$ and
$\top$ for $Obs(agt,c,s)\neq\{\}$ into equations (\ref{eqn:new_k_ssa},\ref{eqn:new_k_s0})
to obtain the following:\begin{multline*}
K(agt,s'',do(c,s))\equiv\left[\bot\rightarrow K(agt,s'',s)\,\right]\\
\wedge\left[\,\top\rightarrow\exists c',s':\, Obs(agt,c',s')=Obs(agt,c,s)\right.\\
\left.\,\,\,\,\,\,\,\,\wedge\, Legal(c',s')\wedge K(agt,s',s)\wedge do(c',s')=s''\,\right]\end{multline*}
 \[
Init(s)\rightarrow\equiv\left[K(agt,s'',s)\equiv\exists s':\, K_{0}(agt,s',s)\wedge s'=s'')\right]\]


Which further simplify to:\begin{multline}
K(agt,s'',do(c,s))\equiv\exists c',s':\, Obs(agt,c',s')=Obs(agt,c,s)\\
\wedge\, Legal(c',s')\wedge K(agt,s',s)\wedge do(c',s')=s''\label{eq:Kobs-Kstd-link1}\end{multline}
 \begin{equation}
Init(s)\rightarrow\equiv\left[K(agt,s'',s)\equiv\, K_{0}(agt,s'',s)\right]\label{eq:Kobs-Kstd-link2}\end{equation}


Using equations (\ref{eq:Observations:ObsStd1},\ref{eq:Observations:ObsStd2}),
it is straightforward to show that:\begin{multline*}
Obs(agt,c',s')=Obs(agt,c,s)\equiv\\
c=c'\wedge\forall a\in c:\, actor(a)=agt\rightarrow\left[SR(a,s)=SR(a,s')\right]\end{multline*}
 Equations (\ref{eq:Kobs-Kstd-link1},\ref{eq:Kobs-Kstd-link2}) are
therefore equivalent to equations (\ref{eq:k_s0_standard},\ref{eq:k_ssa_standard})
from $\Dt_{K}^{std}$, meaning that $K$ behaves the same under both
theories. 
\end{proof}
Having established that our account subsumes the standard {}``public
actions'' account of knowledge, we can also show that it maintains
many of its desirable properties in the general case. One of the fundamental
results in \citep{scherl03sc_knowledge} is that if the initial knowledge
relation $K_{0}$ is reflexive, symmetric, transitive or Euclidean,
then the $K$ relation has these properties for any situation. In
our formalism, such preservation of accessibility properties follows
immediately from Theorem \ref{thm:k_obs_equiv} and the reflexive,
symmetric, transitive and Euclidean nature of the equality operator.

\begin{thm}
If the $K_{0}$ relation is restricted to be reflexive, transitive,
symmetric or Euclidean, then the $K$ relation defined by $\Dt_{K}^{obs}$
will satisfying the same restrictions at every legal situation. \label{thm:Knowledge:Accessibility} 
\end{thm}
\begin{proof}
Each follows directly from Theorem \ref{thm:k_obs_equiv} and the
properties of equality. We will take the transitive case as an example;
other cases are similar.

Suppose that $K_{0}$ is transitive, and we have legal situations
$s_{1}$, $s_{2}$, $s_{3}$ such that $K(agt,s_{2},s_{1})$ and $K(agt,s_{3},s_{2})$.
Then by Theorem \ref{thm:k_obs_equiv} we have the following:\begin{gather*}
K_{0}(root(s_{2}),root(s_{1}))\\
K_{0}(root(s_{3}),root(s_{2}))\\
View(agt,s_{1})=View(agt,s_{2})\\
View(agt,s_{2})=View(agt,s_{3})\end{gather*}
 From the transitivity of $K_{0}$ we can conclude that $K_{0}(root(s_{3}),root(s_{1}))$.
From the transitivity of equality we can conclude that $View(agt,s_{1})=View(agt,s_{3})$.
Since $s_{3}$ is restricted to be legal, we have enough to satisfy
the RHS of the equivalence in Theorem \ref{thm:k_obs_equiv}, so $K(agt,s_{3},s_{1})$
and $K$ is therefore transitive. 
\end{proof}
That these properties hold regardless of the axiomatisation of $Obs$
is a compelling argument in favour of our approach. As discussed in
Section \ref{sec:Knowledge:Extending}, certain kinds of sensing action
can easily invalidate these properties if not axiomatised carefully.
It is therefore worth considering such cases in more detail.

The problematic sensing actions identified in \citep{Petrick06thesis}
are \emph{guarded} sensing actions, which update $K(agt,s',s)$ according
to the following axiom:\begin{gather*}
K(agt,s'',do(a,s))\equiv\exists s':\, s''=do(a,s)\wedge K(agt,s',s)\wedge\dots\\
a=sense_{\phi,\psi}\,\rightarrow\,\left[\psi(s)\,\rightarrow\,\phi(s)\equiv\phi_{}(s')\right]\end{gather*}


The difficultly here is that although the agent will learn $\phi$
if the guard $\psi$ is true, it will not necessarily \emph{know whether}
it has learnt this. If the action fails to produce the desired sensing
result, the agent's knowledge is not updated to reflect that the guard
was false. This means symmetry of the $K$ relation may not be preserved.

To ensure that symmetry is preserved through action, it is necessary
to axiomatise such sensing actions in such a way that the status of
the guard formula itself also becomes known. While this can be achieved
by syntactically restricting the formulae, as in \citep{Petrick06thesis},
our approach of explicitly representing the observations made by each
agent avoids the problem automatically.

Consider how guarded sensing actions can be axiomatised using explicit
observations, as discussed in Section \ref{sec:Observations:Guarded-Sensing}
and repeated below for convenience:\begin{gather*}
sense_{\phi,\psi}\#T\in Obs(agt,c,s)\,\equiv\, sense_{\phi,\psi}\in c\,\wedge\, actor(sense_{\phi})=agt\wedge\psi(s)\wedge\phi(s)\\
sense_{\phi,\psi}\#F\in Obs(agt,c,s)\,\equiv\, sense_{\phi,\psi}\in c\,\wedge\, actor(sense_{\phi})=agt\wedge\psi(s)\wedge\neg\phi(s)\\
sense_{\phi,\psi}\in Obs(agt,c,s)\,\equiv\, sense_{\phi,\psi}\in c\,\wedge\, actor(sense_{\phi})=agt\wedge\neg\psi(s)\end{gather*}


An agent using our formalism can therefore conclude, by virtue of
not receiving a sensing result from $sense_{\phi,\psi}$, that the
guard condition must not hold. This is sufficient to maintain symmetry
of the knowledge accessibility relation as guaranteed by Theorem \ref{thm:Knowledge:Accessibility}.

Our formalism is thus a proper generalisation of the standard account
of knowledge in the situation calculus. It is also an \emph{elaboration
tolerant} generalisation, automatically maintaining important properties
of the axiomatisation as more complex models of sensing and observability
are introduced. To demonstrate the power gained by such generalisation,
Section \ref{sec:Knowledge:Example} shows how to use our formalism
to model a domain in which agents can only observe actions performed
in the same room as them.


\section{Regression\label{sec:Knowledge:Regression}}

The final aspect of our new account of knowledge is to extend the
techniques for effective reasoning in the situation calculus to handle
the modified formalism -- that is, to develop a regression rule for
$\Knows$.

The appearance of $\leq_{\LbU(agt)}$ in equation (\ref{eqn:new_k_ssa})
means that our new successor state axiom universally quantifies over
situations, so the regression technique of \citep{scherl03sc_knowledge}
cannot be used. We must appeal to the persistence condition meta-operator
introduced in Section \ref{sub:Property-Persistence} to handle the
inductive component of this reasoning, by transforming the quantification
into a uniform formula so that standard regression techniques can
be applied.

We propose the following as the regression rule for $\Knows$ under
our formalism:\begin{align}
\Reg(\Knows(agt,\phi,do(c,s))\isdef\,\, & \exists o:\, Obs(agt,c,s)=o\nonumber \\
 & \wedge\,\left[o=\{\}\,\rightarrow\,\Knows(agt,\phi,s)\right]\nonumber \\
 & \wedge\,\left[o\neq\{\}\,\rightarrow\,\Knows(agt,\forall c':\, Obs(agt,c')=o\right.\nonumber \\
 & \,\,\,\,\,\,\,\,\,\,\,\wedge Legal(c')\rightarrow\left.\Reg(\Pst(\phi,\LbU(agt)),c'),s)\right]\label{eqn:R_do_c_s}\end{align}


Note the similarity to the standard regression rule for knowledge
in equation \eqref{eq:reg_k_std}. New in our version are: the replacement
of the $\mathbf{res}$ macro with an explicit, flexible definition
of what the agent has observed; explicit handling of the case when
the agent makes no observations; and use of the persistence condition
to account for arbitrarily-long sequences of hidden actions.

As required for a regression rule, equation \eqref{eqn:R_do_c_s}
reduces a knowledge query at $do(c,s)$ to a knowledge query at $s$.
It is also intuitively appealing: to know that $\phi$ holds, the
agent must know that in all situations that agree with its observations,
$\phi$ cannot become false without it making some further observation
-- this is the meaning of $\Pst(\phi,\LbU(agt))$ in the above, to
express the agent's knowledge that {}``if $\phi$ were to be made
false, I would be informed of it''.

We must also specify a regression rule for $\Knows$ in the initial
situation, as equation (\ref{eqn:new_k_s0}) also uses the $\leq_{\LbU(agt)}$
ordering. This clause produces an expression in $\KnowsZ$ at $S_{0}$,
meaning that it can be handled by epistemic reasoning about the initial
situation only:\begin{equation}
\Reg(\Knows(agt,\phi,S_{0}))\,\isdef\,\KnowsZ(agt,\Reg(\Pst(\phi,\LbU(agt))[S_{0}])^{-1},S_{0})\label{eqn:R_s0}\end{equation}


The use of $\Pst$ here is similar to its use in the previous regression
rule. The use of $\Reg(\phi[S_{0}])^{-1}$ is required to transform
nested knowledge formulae into nested \emph{initial} knowledge formulae.
For example:\[
\Knows(A,\Knows(B,\phi),S_{0})\,\Rightarrow\,\KnowsZ(A,\KnowsZ(B,\phi),S_{0})\]
 When the enclosed formula $\phi$ does not contain nested knowledge
macros, regressing it at $S_{0}$ and then suppressing the situation
term will leave it unchanged.

\begin{thm}
\label{thm:Reg_Knows}Given a basic action theory $\Dt$ and a uniform
formula $\phi$:\[
\Dt\cup\Dt_{K}^{obs}\,\models\,\Knows(agt,\phi,s)\equiv\Reg(\Knows(agt,\phi,s))\]

\end{thm}
\begin{proofsketch}
In the $do(c,s)$ case, we proceed by expanding the definition for
$\Knows$ using our new successor state axiom for $K$, collecting
sub-formulae that match the form of the $\Knows$ macro, and using
regression and the persistence condition to render the resulting expressions
uniform in $s$. In the base case, we apply the persistence condition
to an expansion of $\Knows$ at $S_{0}$ to produce the desired result. 
\end{proofsketch}
With these regression rules, we can this handle knowledge queries
in our formalism using standard techniques for effective reasoning
in the situation calculus.

While this reasoning method is suitable for modelling and simulation
purposes, it would be unreasonable for a situated agent to ask {}``do
I know $\phi$ in the current situation?'' using the situation calculus
query $\Dt\models\mathbf{Knows}(agt,\phi,\sigma)$, as it cannot be
expected to have the full current situation $\sigma$. It will however
have its current view $v$ and can construct a query like the following:\[
\Dt\cup\Dt_{K}\models\forall s:\, View(agt,s)=v\wedge root(s)=S_{0}\rightarrow\mathbf{Knows}(agt,\phi,s)\]


Such a query universally quantifies over situations and so cannot
be handled using regression. It is also not in a form amenable to
the persistence condition operator, so the agent has no means of effectively
answering such a query.

However, we should expect from Theorem \ref{thm:k_obs_equiv} that
this quantification over situations is unnecessary -- after all, all
situations with the same view for that agent should result in the
same knowledge. Let us explicitly define knowledge with respect to
a view as follows:\[
\mathbf{Knows}(agt,\phi,v)\,\isdef\,\forall s:\, View(agt,s)=v\wedge root(s)=S_{0}\rightarrow\mathbf{Knows}(agt,\phi,s)\]


We can then modify the regression rules in equations (\ref{eqn:R_do_c_s},\ref{eqn:R_s0})
to work directly on formulae of this form. The resulting rules are
actually simpler than for regression over situations, as there are
no empty observations in a view. The result is:\begin{align}
\Reg(\mathbf{Knows}(agt,\phi,o\cdot v))\isdef\,\, & \mathbf{Knows}(agt,\forall c:\, Obs(agt,c)=o\nonumber \\
 & \,\,\,\,\,\,\,\,\wedge Legal(c)\rightarrow\Reg(\Pst(\phi,\LbU(agt)),c),v)\label{eq:Knowledge:Reg_V_O}\\
\Reg(\mathbf{Knows}(agt,\phi,\epsilon))\isdef\,\, & \KnowsZ(agt,\Pst(\phi,\LbU(agt)),S_{0})\label{eq:Knowldge:Reg_V_E}\end{align}
 Using regression in this way, an agent can reduce the query $\Knows(agt,\phi,v)$
to an equivalent query about its knowledge in the initial situation.

\begin{thm}
\label{thm:Reg_KnowsO}Given a basic action theory $\Dt$ and a uniform
formula $\phi$:\[
\Dt\cup\Dt_{K}^{obs}\,\models\,\Knows(agt,\phi,v)\equiv\Reg(\Knows(agt,\phi,v))\]

\end{thm}
\begin{proofsketch}
The proof hinges on a simple corollary of Theorem \ref{thm:k_obs_equiv}
that situations with the same root and same view entail the same knowledge:\begin{multline*}
\Dt\cup\Dt_{K}^{obs}\models\forall s,s',s'':\, root(s)=root(s')\wedge View(s)=View(s')\\
\wedge K(agt,s'',s)\rightarrow K(agt,s'',s')\end{multline*}
 We can then proceed by induction over views. For both the $\epsilon$
and $o\cdot v$ cases we split on whether there exists a situation
having that view. If no such situation exists, we show that the regression
rules (\ref{eq:Knowledge:Reg_V_O}, \ref{eq:Knowldge:Reg_V_E}) generate
a formula that is vacuously true, as an invalid view cases anything
to be known. If such a solution does exist, we select an arbitrary
witness and demonstrate that rules (\ref{eq:Knowledge:Reg_V_O}, \ref{eq:Knowldge:Reg_V_E})
generate an equivalent formula to rules (\ref{eqn:R_do_c_s},\ref{eqn:R_s0})
using that witness. By the above corollary, this is enough to establish
equivalence for any such such situation. 
\end{proofsketch}
Our formalism thus allows agents to reason effectively about their
own knowledge using only their local information, even in asynchronous
domains where they are unable to construct the current situation term.

It is worth re-iterating that our regression rules are no longer straightforward
syntactic transformations - rather, they involve a fixpoint calculation
to generate $\Pst(\phi,\LbU(agt))$. Can this really be considered
an effective reasoning technique? The previous work on the persistence
condition meta-operator discussed the advantages of this approach
in detail. The primary advantage is that this form of reasoning can
be performed at all, as the alternative is general second-order theorem
proving.

Of course, the ultimate proof is in the implementation. We have implemented
a preliminary version of our technique and used it to verify the examples
found in the following section. As with all code produced for this
thesis, it is available online at \url{http://www.rfk.id.au/software/projects/thesis/}.

We close this section with a formal statement of a small but important
point: the persistence condition is not required when reasoning in
synchronous domains. It is straightforward to show that $\Pst(\phi,\LbU(agt))$
in synchronous domains is always equivalent to $\phi$. The regression
rules in equations (\ref{eqn:R_do_c_s},\ref{eqn:R_s0}) then reduce
to purely syntactical manipulations. We thus do not introduce unnecessary
complications for domains in which effective reasoning procedures
already exist, while extending the reach of our formalism into richer
domains where some inductive reasoning is required.

\begin{thm}
Let $\Dt_{sync}$ be a synchronous basic action theory, then:\[
\Dt_{sync}\models\forall s,agt:\,\phi[s]\,\equiv\,\Pst(\phi,\LbU(agt))[s]\]

\end{thm}
\begin{proof}
By definition, we have:\[
\Dt_{sync}\,\models\,\forall agt,c,s:\, Legal(c,s)\,\rightarrow Obs(agt,c,s)\neq\{\}\]


Recall from equation (\ref{eqn:LbU_defn}) that:\[
\LbU(agt,c,s)\,\equiv\, Legal(c,s)\wedge Obs(agt,c,s)=\{\}\]


So clearly:\[
\Dt_{sync}\,\models\,\forall agt,c,s:\,\LbU(agt,c,s)\equiv\bot\]


The definition of $\Pst^{1}(\phi,\LbU(agt))$ will then produce:\[
\Pst^{1}(\phi,\LbU(agt))\,\equiv\,\phi\wedge\forall c:\,\bot\rightarrow\Reg(\phi,c)\,\equiv\,\phi\]


The calculation of $\Pst$ thus terminates immediately at the first
iteration, giving $\Pst(\phi,\LbU(agt))$ equal to $\Pst^{1}(\phi,\LbU(agt))$,
which is equivalent to $\phi$ as desired. 
\end{proof}

\section{An Illustrative Example\label{sec:Knowledge:Example}}

We now give a brief demonstration of our formalism in action, using
it to model the {}``party invitation'' domain outlined in Chapter
\ref{ch:intro}. We adopt an explicit axiomatisation of partial observability
based on the $CanObs$/$CanSense$ predicates introduced in Section
\ref{sec:Observations:CanObs}.

The fluents of interest in this domain are the location of the party
(the function $loc$) and whether each agent is in the room (the predicate
$InRoom$). The action $read$ reads the invitation and returns the
location of the party, while the non-sensing actions $enter$ and
$leave$ cause the agents to move in/out of the room. The $read$
action is only observed by agents who are in the room. This domain
can be summarised by the following axioms:\begin{gather*}
loc(S_{0})=C\\
loc(do(c,s))=l\equiv loc(s)=l\end{gather*}
 \begin{gather*}
InRoom(Alice,S_{0})\equiv InRoom(Bob,S_{0})\equiv\top\\
InRoom(agt,do(c,s))\equiv enter(agt)\in c\,\vee\, InRoom(agt,s)\wedge leave(agt)\notin c\\
Poss(enter(agt),s)\equiv\neg InRoom(agt,s)\\
Poss(leave(agt),s)\equiv InRoom(agt,s)\end{gather*}
 \begin{gather*}
Poss(read(agt),s)\equiv InRoom(agt,s)\\
SR(read(agt),s)=r\equiv r=loc(s)\end{gather*}
 \begin{gather*}
\forall agt,l:\,\neg\KnowsZ(agt,loc=l,S_{0})\\
\forall agt_{1},agt_{2},l:\,\KnowsZ(agt_{1},\neg\KnowsZ(agt_{2},loc=l),S_{0})\\
\forall agt:\,\KnowsZ(agt,InRoom(Alice)\wedge InRoom(Bob),S_{0})\end{gather*}
 \begin{gather*}
CanObs(agt,leave(agt'),s)\equiv CanObs(agt,enter(agt'),s)\equiv\top\\
CanSense(agt,leave(agt'),s)\equiv CanSense(agt,enter(agt'),s)\equiv\bot\\
CanObs(agt,read(agt'),s)\equiv InRoom(agt',s)\\
CanSense(agt,read(agt'),s)\equiv agt=agt'\end{gather*}


The following are examples of knowledge queries that can be posed
in our formalism, a brief explanation of their outcome, and a demonstration
of how they can be answered using our new regression rules. Each has
been verified by the preliminary implementation of our reasoning engine.

\begin{example}
Initially, Alice doesn't know where the party is:\[
\Dt\cup\Dt_{K}^{obs}\models\neg\exists l:\,\mathbf{Knows}(A,loc=l,S_{0})\]

\end{example}
It is given that $\neg\exists l:\,\KnowsZ(A,loc=l,S_{0})$, and the
only way for her to learn such information is by performing a $read$
action. Since she would always observe such an action, she cannot
have learnt the party's location as a result of hidden actions, and
the example is entailed. Formally:\begin{align*}
\Pst(loc=l,\LbU(A))\,\Rightarrow\,\,\,\, & loc=l\\
\Reg(\neg\exists l:\,\mathbf{Knows}(A,loc=l,S_{0}))\,\Rightarrow\,\,\,\, & \neg\exists l:\,\KnowsZ(A,\Reg((loc=l)[S_{0}])^{-1},S_{0})\end{align*}


Note that the calculation of this persistence condition is trivial
since the location of the party cannot change. The nested regression
of $loc=l$ at $S_{0}$ leave the formula unchanged. The query thus
regresses to:\[
\neg\exists l:\,\KnowsZ(A,loc=l,S_{0})\]


This is entailed by the domain.

\begin{example}
After reading the invitation, Bob will know where the party is:\[
\Dt\cup\Dt_{K}^{obs}\models\Knows(B,loc=C,do(\{read(B)\},S_{0}))\]

\end{example}
The sensing results of the $read$ action inform Bob of the location
of the party. Since this location cannot change after any sequence
of hidden actions, he can be sure of the party's location. Formally,
using the fact that $Obs(B,\{read(B)\},s)=\{read(B)\#loc(s)\}$:\begin{multline*}
\Reg(\Knows(B,loc=C,do(\{read(B)\},S_{0})))\,\Rightarrow\\
\exists o:\, o=Obs(B,\{read(B)\},S_{0})\,\wedge\,\\
\Knows(B,\forall c':\, Legal(c')\wedge Obs(B,c')=o\rightarrow\Reg(\Pst(loc=C,\LbU(B)),c'),S_{0})\end{multline*}


Since $\Reg(\Pst(loc=C,\LbU(Bob)),c')$ reduces to $loc=C$, and $loc(S_{0})=C$
is given, this simplifies to:\[
\Knows(B,\forall c':\, Legal(c')\wedge Obs(B,c')=\{read(B)\#C\}\rightarrow loc=C,S_{0})\]


Since the only possible value of $c'$ that satisfies the antecedent
is $\{read(B)\}$, we can insert the definitions of $Legal$ and $Obs$
to obtain:\[
\Knows(B,InRoom(B)\wedge loc=C\,\rightarrow\, loc=C,S_{0})\]


This tautology is clearly entailed by the domain.

\begin{example}
Initially, Bob knows that Alice doesn't know where the party is:\[
\Dt\cup\Dt_{K}^{obs}\models\mathbf{Knows}(B,\neg\exists l:\,\Knows(A,loc=l),S_{0})\]

\end{example}
Alice could learn the location of the party by performing the $read$
action, but since Bob is in the room he would observe this action
taking place. Since he has not observed it, he can conclude that Alice
does not know the location of the party. Formally:\begin{multline*}
\Reg(\mathbf{Knows}(B,\neg\exists l:\,\Knows(A,loc=l),S_{0}))\,\Rightarrow\\
\KnowsZ(B,\Reg(\Pst(\neg\exists l:\,\Knows(A,loc=l),\LbU(B))[S_{0}])^{-1},S_{0})\end{multline*}
 \begin{multline*}
\Pst(\neg\exists l:\,\Knows(A,loc=l),\LbU(Bob))\,\Rightarrow\\
\neg\exists l:\,\Knows(A,loc=l)\wedge\left(InRoom(B)\,\vee\,\neg InRoom(A)\right)\end{multline*}


From the previous examples we know that:\[
\Reg(\neg\exists l:\,\mathbf{Knows}(A,loc=l,S_{0}))\,\Rightarrow\,\neg\exists l:\,\KnowsZ(A,loc=l,S_{0})\]


So the entire query regresses to:\[
\KnowsZ(B,\neg\exists l:\,\KnowsZ(A,loc=l)\wedge\left(InRoom(B)\,\vee\,\neg InRoom(A)\right),S_{0})\]


This is entailed by the domain.

\begin{example}
After leaving the room, Bob won't know that Alice doesn't know the
location of the party:\[
\Dt\cup\Dt_{K}^{obs}\models\mathbf{\neg Knows}(B,\neg\exists l:\,\Knows(A,loc=l),do(\{leave(B)\},S_{0}))\]

\end{example}
Once Bob leaves the room, he would be unable to observe Alice reading
the invitation. He must therefore consider it possible that she has
read it, and may know the location of the party. Formally, we can
use $Obs(B,\{leave(B)\})=\{leave(B)\}$ to regress the outer expression
as follows:\begin{multline*}
\Reg(\neg\Knows(B,\phi,do(\{leave(B)\},S_{0})))\,\Rightarrow\\
\neg\Knows(B,InRoom(B)\rightarrow\Reg(\Pst(\phi,LbU(B)),\{leave(B)\}),S_{0})\end{multline*}


For the inner expression, we have from the previous example:\begin{multline*}
\Pst(\neg\exists l:\,\Knows(A,loc=l),LbU(B))\,\Rightarrow\\
\neg\exists l:\,\Knows(A,loc=l)\wedge\left(InRoom(B)\,\vee\,\neg InRoom(A)\right)\end{multline*}


This expression is key: for Bob to know $\neg\exists l:\Knows(A,loc=l)$,
he must also know either that he is in the room (and will thus observe
the $read(A)$ action if it occurs) or that Alice is not in the room
(so the $read(A)$ action will not be possible). Otherwise, Alice
could learn the location of the party without him making any further
observations.

When we regress over the action $\{leave(B)\}$ then $InRoom(B)$
is made false:\begin{multline*}
\Reg(\Pst(\neg\exists l:\,\Knows(A,loc=l),LbU(B)),\{leave(B)\})\,\Rightarrow\\
\neg\exists l:\,\Knows(A,loc=l)\wedge\left(\bot\,\vee\,\neg InRoom(A)\right)\end{multline*}


And the entire expression can be simplified to:\[
\neg\Knows(B,InRoom(B)\rightarrow\neg\exists l:\,\Knows(A,loc=l)\wedge\neg InRoom(A),S_{0})\]


Since Alice is known to be in the room, this expression will be entailed
by the domain.


\section{Advanced Applications\label{sec:Knowledge:Advances}}

In this section we briefly consider how our work could integrate with
some other recent developments in the situation calculus literature.
While this is far from a detailed treatment, it does demonstrate the
potential for interesting future work utilising our formalism.


\subsection{Approximate Epistemic Reasoning}

From the examples in the previous section, it is clear that agents
may need to perform significant amounts of reasoning to answer knowledge
queries about arbitrary formulae. This is on top of the already significant
task of performing possible-worlds reasoning in the initial situation
\citep{Petrick06thesis}. An interesting alternative to making reasoning
about knowledge more tractable is the formalism of \citet{demolombe00tractable_sc_belief},
in which knowledge is limited to be about fluent literals only.

The basic idea is to introduce, for each fluent $F$ in the domain,
two explicit fluents $K_{agt}^{+}F$ and $K_{agt}^{-}F$ to explicitly
represent {}``$agt$ knows $F$'' and {}``$agt$ knows $\neg F$''
respectively. By formulating ordinary successor state axioms for these
fluents, literal-level knowledge can be reasoned about using standard
regression and does not require a possible-worlds relation. However,
this approach cannot represent indeterminate disjunctive knowledge
such as {}``$agt$ knows $F$ or $G$''.

The \citeauthor{demolombe00tractable_sc_belief} approach has been
formally related to the standard \citeauthor{scherl03sc_knowledge}
approach by \citet{petrick02knowledge_equivalence}. They show there
is an equivalence between the two approaches when an agent's knowledge
is restricted to be \emph{disjunctive}, so that the following holds:\[
\Knows(agt,\phi_{1}\vee\phi_{2},s)\rightarrow\Knows(agt,\phi_{1},s)\vee\Knows(agt,\phi_{2,}s)\]


In \citep{Petrick06thesis} this equivalence is extended to cover
existential quantification by restricting knowledge to also satisfy
the following:\[
\Knows(agt,\exists x:\,\phi(x),s)\rightarrow\exists x:\,\Knows(agt,\phi(x),s)\]
 These disjunctive properties of knowledge are \emph{not} entailed
by a general possible-worlds style theorem in the tradition of \citep{scherl03sc_knowledge},
although there are restrictions that can be placed on the theory in
order to enforce them \citep{petrick08cartesian_situations,Petrick06thesis}.

While we do not consider maintenance of these disjunctive knowledge
properties in any detail, we do note that they also permit a sound
\emph{approximation} of knowledge that can be reasoned about more
tractably than the standard possible-worlds account. Following the
style of \citep{petrick02knowledge_equivalence} we could provide
the following definitions:\begin{gather*}
\Knows_{A}(agt,\phi_{1}\wedge\phi_{2},s)\isdef\Knows_{A}(agt,\phi_{1},s)\wedge\Knows_{A}(agt,\phi_{2},s)\\
\Knows_{A}(agt,\neg(\phi_{1}\wedge\phi_{2}),s)\isdef\Knows_{A}(agt,\neg\phi_{1},s)\vee\Knows_{A}(agt,\neg\phi_{2},s)\\
\Knows_{A}(agt,\forall x:\,\phi(x),s)\isdef\forall x:\,\Knows_{A}(agt,\phi(x),s)\\
\Knows_{A}(agt,\neg\forall x:\,\phi(x),s)\isdef\exists x:\,\Knows_{A}(agt,\neg\phi(x),s)\\
\Knows_{A}(agt,\neg\neg\phi,s)\isdef\Knows_{A}(agt,\phi,s)\\
\Knows_{A}(agt,F,s)\isdef\Knows(agt,F,s)\\
\Knows_{A}(agt,\neg F,s)\isdef\Knows(agt,\neg F,s)\end{gather*}


A knowledge query is split across the logical operators until we are
left with only knowledge of fluent literals, which is then handled
using our formalism. If we assume a finite number of fluents, then
we can use our regression rule for knowledge to \emph{pre-calculate}
an explicit successor state axiom for $\Knows_{A}(agt,F,s)$ and $\Knows_{A}(agt,\neg F,s)$,
allowing them to be treated as primitive fluents.

Unlike the approach of \citep{demolombe00tractable_sc_belief} in
which the knowledge literal fluents must be axiomatised separately
from the actual fluents they describe, our approach allows a successor
state axiom for literal-level knowledge to be derived from the dynamics
of the domain. Allow all persistence condition calculations could
be performed once, offline, and then used directly for approximate
reasoning about the knowledge of an agent.


\subsection{Knowledge under a Protocol}

The knowledge formalism we have developed here is \emph{permissive},
in that it assumes the world could potentially evolve via any legal
sequence of actions. In the wider field of epistemic reasoning, it
is common to constrain the world to evolve according to some \emph{protocol}
\citep{halpern90knowledge_distrib,fagin95,vanBentham06tree_of_knowledge}.
One then speaks of an agent's knowledge under a given protocol.

As discussed in \citep{vanBentham06tree_of_knowledge}, permissive
formulations of knowledge can easily be extended to support a \emph{local}
protocol, where the allowable next actions can be determined based
on the current state of the world. Our use of $Legal$ in the axioms
for knowledge could easily be replaced with predicates axiomatising
actions that are, for example, $Permissable$ or $Motivated$. But
the expessiveness of the situation calculus also presents an intruiging
possibility to extend this approach.

The most natural language for expressing a protocol in the situation
calculus is Golog, so one may wish to reason about an agent's knowledge
assuming the world evolves as specified by the Golog program $\delta$:\[
K_{p}(agt,s',s)\isdef K(agt,s',s)\wedge\exists s'',\delta':\, Init(s'')\wedge Trans^{*}(\delta,s'',\delta',s')\]


Such knowledge would be queried like so:

\[
\Dt\cup\Dt_{golog}\cup\Dt_{K}^{obs}\,\models\,\Knows_{p}(agt,\delta,\phi,\sigma)\]


The recent work of \citet{fritz08congolog_sin_trans} has demonstrated
that the details of a given ConGolog program $\delta$ can be \emph{compiled
into} a theory of action $\Dt$, producing a new theory $\Dt^{\delta}$
in which the only legal situations are those that form part of a legal
execution of $\delta$:\begin{gather*}
\Dt\cup\Dt_{golog}\,\models\,\exists\delta':\, Trans^{*}(\delta,S_{0},\delta',\sigma)\\
\mathrm{iff}\\
\Dt^{\delta}\,\models\, Legal(\sigma)\end{gather*}


So to investigate the knowledge of an agent under a protocol, we could
use this compilation to get a query which can be handled using the
formalism:\begin{gather*}
\Dt\cup\Dt_{golog}\cup\Dt_{K}^{obs}\,\models\,\Knows_{p}(agt,\delta,\phi,\sigma)\\
\mathrm{iff}\\
\Dt^{\delta}\cup\Dt_{K}^{obs}\,\models\,\Knows(agt,\phi,\sigma)\end{gather*}


The details are not quite so straightforward, as the compilation procedure
of \citep{fritz08congolog_sin_trans} introduces some auxilliary actions
and fluents that should be hidden from the agent's knowledge. However,
it does offer an intruiging possibility for future work.


\section{Discussion\label{sec:Knowledge:Discussion}}

In this chapter, we have used a principled axiomatisation of the observability
of actions to explicitly define an agent's knowledge in terms of its
local view. By reifying observations and views as terms in the logic,
we are able to give a succinct definition of the dynamics of the knowledge
fluent and prove that its behaviour matches our intuitive expectations.
Moreover, it maintains its important theorems and the availability
of a regression rule as more complex kinds of information-producing
action are introduced.

As an example of the problems that can arise when trying to characterise
knowledge using an implicit representation of an agent's local perspective,
consider again one of the few existing formulations of knowledge in
the situation calculus that allows for hidden actions, that of \citep{Lesperance99sitcalc_approach}.
Their successor state axiom for the $K$ fluent is repeated below:\begin{align*}
K(agt,s'',do(a,s))\equiv\,\,\, & \exists s':\, K(agt,s',s)\\
 & \wedge\,(actor(a)\neq agt\,\rightarrow\, s'\leq_{actor(a)\neq agt}s''))\\
 & \wedge\,(actor(a)=agt\,\rightarrow\,\exists s^{*}:\,\left[s'\leq_{actor(a)\neq agt}s^{*}\wedge\right.\\
 & \,\,\,\,\,\,\,\,\,\,\,\,\left.s''=do(a,s^{*})\wedge Poss(a,s^{*})\wedge sr(a,s)=sr(a,s^{*}))\right]\end{align*}


While the axiom seems intuitively plausible, it has a subtle problem:
an agent's knowledge can change in response to actions performed by
others. Suppose that $agt$ has just performed action $a_{1}$, so
the world is in situation $do(a_{1},s)$. Another agent then performs
the action $a_{2}$, leaving the world in situation $do(a_{2},do(a_{1},s))$.
Since it is not aware of the occurrence of $a_{2}$, the knowledge
of $agt$ should be unchanged between these two situations. This is
not the case under the formulation of \citep{Lesperance99sitcalc_approach},
which introduces arbitrarily-long sequences of hidden actions into
the \emph{past} of the possible situation $s''$. Based on our explicit
formalisation of the agent's local view, our axiom includes hidden
actions in the \emph{future} of $s''$ and avoids this unintuitive
behaviour.

A further advantage of our explicit axiomatisation of observations
is in establishing properties of the knowledge relation. A major theorem
of \citep{scherl03sc_knowledge} states that if the $K$-relation
is reflexive, symmetric or transitive at the initial situation, then
it has that property at every situation. In our formulation these
are all simple corollaries of Theorem \ref{thm:k_obs_equiv} and the
reflexive, symmetric and transitive nature of the equality symbol.\\


Our insistence on allowing for {}``any possible future actions''
may seem like it would leave the agents with too little knowledge.
Indeed, it is easy to construct cases in which agents can never know
the value of certain fluents. We argue that this restriction is necessary
in asynchronous domains -- if definition \ref{defn:Observations:View}
is accepted as the definition of a view, and equation \eqref{eq:k-desired}
is accepted as a definition for how knowledge should behave, then
the agent must consider any possible sequence of hidden actions.

Note, however, that the sequences of actions considered must not only
be legal, but unobservable as well. As shown in our example domain,
if the agents have good observability of parts of the domain, they
can acquire significant amounts of knowledge because there will be
few hidden actions.

One approach to further taming these hidden future situations is to
assume that agents always know the current time, or some bound on
the current time. They can use this information in conjunction with
their local view to determine what they know, given that the current
time is $\tau$:\[
\Dt\cup\Dt_{K}^{obs}\,\models\,\Knows(agt,(start<\tau)\rightarrow\phi,v)\]


Since this query is uniform, it can be approached directly using our
regression rules. If all actions take some finite duration, $\tau$
will effectively bound the length of hidden action sequences that
need to be considered. Not only will this permit the agents to gain
knowledge of more fluents, but it will also guarantee the termination
of the persistence condition calculation.

Another option would be to abandon the requirement that the agents
\emph{know} particular fluents and formulate a logic of \emph{belief}.
Belief-based formalisms have been constructed for the situation calculus
using a possible-worlds approach similar to that used for knowledge
\citep{shapiro00sc_belief}. In such formalisms agents are allow to
be mistaken, and so do not need to consider all possible ways the
world might evolve. Indeed, \citet{shapiro04sc_belief_exog} have
shown how an agent can hypothesise the occurrence of hidden actions
when it discovers that its beliefs are mistaken.

We do not consider belief-based systems in this thesis, but note that
the use of a similar possible-worlds semantics means that our explicit
notions of observations and views could also provide benefits for
such formalisms, as well as for the formulation of other modalities
(such as goals \citep{shapiro07sc_goal_change}) in the situation
calculus.\\


We have demonstrated that our account of knowledge is expressive enough
to capture the standard account of knowledge based on public actions,
as well as more complex formulations where the observability of actions
depends on the state of the world. We have also demonstrated that
despite allowing for arbitrarily-long sequences of hidden actions,
our formalism still permits automated reasoning for handling knowledge
queries, and we have build a preliminary implementation of such a
reasoning system.

Of course, the effectiveness of automated reasoning is now highly
dependent on the effectiveness of calculating the persistence condition.
Since this is a fixpoint calculation, it can be computationally expensive
and even undecidable in very complex domains. But by factoring out
the necessary inductive reasoning into a separate operator, it can
now be studied and improved in isolation.

Finally, we have shown that a simple modification to our regression
rules allows a situated agent to reason directly about its own knowledge
using only its local view, rather than constructing a query that universally
quantifies over all situations compatible with its view. Our new observation-based
semantics thus provides a powerful account of knowledge suitable both
for reasoning \emph{about}, and for reasoning \emph{in}, asynchronous
multi-agent domains.

