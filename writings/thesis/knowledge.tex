

\chapter{Knowledge with Hidden Actions}

\label{ch:knowledge}

This chapter develops a new theory of knowledge in the situation calculus
that directly leverages the agent-local perspective developed in Chapter
\ref{ch:observations}.

Existing accounts of epistemic reasoning in the situation calculus
require that whenever an action occurs, all agents \emph{know} that
an action has occurred. This demands a level of synchronicity that
is unreasonable in many multi-agent domains. In asynchronous domains,
each agent's knowledge must instead account for arbitrarily-long sequences
of \emph{hidden actions}. This requires a second-order induction axiom
which precludes the use of regression for effective automated reasoning.
It also make it difficult for agents to reason about their own knowledge,
as they may not have enough information to formulate an appropriate
query.

To overcome this limitation we combine two of the contributions developed
in preceding chapters - the explicit representation of an agent's
local perspective from Chapter \ref{ch:observations}, and the persistence
condition meta-operator from Chapter \ref{ch:persistence} - to formulate
an account of knowledge in the situation calculus that can faithfully
represent the hidden actions inherent in asynchronous domains while
maintaining a regression rule for effective automated reasoning.

We begin by developing an axiomatisation of knowledge based explicitly
on each agent's local view. This axiomatisation is shown to be sound,
and to faithfully maintain important properties of the agent's epistemic
state (such reflexivity and transitivity) through the occurrence of
actions. Moreover, our formulation is \emph{elaboration tolerant},
automatically maintaining these properties in the face of more complex
information-producing actions, such as guarded sensing actions and
speech acts, that can invalidate these properties if not axiomatised
carefully.

To formulate a regression rule for knowledge, we appeal to the persistence
condition operator to factor out the inductive reasoning required
for dealing with hidden actions. We propose a new regression rule
that is sound and complete with respect to our axiomatisation of knowledge,
and demonstrate how it can used by a situated agent to reason about
its own knowledge in a straightforward manner.

The end result is a significantly more general and robust theory of
knowledge in the situation calculus that still permits an effective
reasoning procedure.

The chapter proceeds as follows: after introducing more details background
material on epistemic reasoning in the situation calculus in Section
\ref{sec:Knowledge:Background}, we develop the axioms for our new
observation-based account of knowledge in Section \ref{sec:Knowledge:Observation}.
Section \ref{sec:Knowledge:Properties} explores the properties of
our formalism with reference to the standard account. Section \ref{sec:Knowledge:Regression}
develops a regression rule for our formalism using the persistence
condition operator, while Section \ref{sec:Knowledge:Example} gives
a brief example of its use for reasoning about a partially-observable
domain. Section \ref{sec:Knowledge:Approximate} links our work to
a related formalism based on knowing fluent literals, while Section
\ref{sec:Knowledge:Discussion} concludes with a general discussion
and summary.


\section{Background\label{sec:Knowledge:Background}}

Recall from Section \ref{sec:Background:Epistemic} that the dynamics
of knowledge in the situation calculus are specific using an additional
set of axioms $\Dt_{K}$, which define the behavior of a special knowledge-fluent
$K$. In the standard account of knowledge, based on the work of \citet{scherl03sc_knowledge}
and incorporating concurrent actions and multple agents \citep{shapiro98specifying_ma_systems,scherl03conc_knowledge},
the axioms in $\Dt_{K}$ are the following:\begin{equation}
Init(s)\rightarrow\left(K(agt,s',s)\,\equiv\, K_{0}(agt,s',s)\right)\label{eq:k_s0_standard}\end{equation}
 \begin{multline}
K(agt,s'',do(c,s))\equiv\exists s':\, s''=do(c,s')\,\wedge K(agt,s',s)\\
\wedge Legal(c,s')\wedge\,\forall a\in c:\,\left(actor(a)=agt\,\rightarrow\, SR(a,s)=SR(a,s')\right)\label{eq:k_ssa_standard}\end{multline}


Equation \eqref{eq:k_s0_standard} ensures that the agents begin with
their knowledge as specified by the initial knowledge fluent $K_{0}$.
Equation \eqref{eq:k_ssa_standard} takes the form of a standard successor
state axiom for the $K$ fluent. It ensures that $s''$ is considered
a possible alternative to $do(c,s)$ when $s''$ is the result of
doing the same actions $c$ in a situation $s'$ that is considered
a possible alternative to $s$. It must furthermore have been possible
to perform those actions in $s'$, and the sensing results must match
for each action that was carried out by the agent in question. Thus
an agent's knowledge after the occurrence of an action is completely
determined by the combination of its knowledge before the action,
and the sensing results from the action.

\medskip{}


\begin{defn}
We will denote by $\Dt_{K}^{std}$ the axioms of the standard account
of knowledge due to \citep{scherl03sc_knowledge,scherl03conc_knowledge},
as detailed in equations (\ref{eq:k_s0_standard},\ref{eq:k_ssa_standard})
above. 
\end{defn}
While powerful, this knowledge-representation formalism has an important
limitation: it is fundamentally \emph{synchronous.} Each agent is
assumed to have full knowledge of all actions that have occurred -
in other words, all actions are public. While suitable for some domains,
there are clearly many multi-agent domains where achieving total awareness
of actions would be infeasible. A major contribution of this chapter
is a more flexible formalism for knowledge that can be applied to
a much wider range of domains.


\subsection{Accessability Properties}

TODO: this section


\subsection{Reasoning about Knowledge}

A key contribution of \citet{scherl03sc_knowledge} was showing how
to apply the regression operator to formulae containing the $\mathbf{Knows}$
macro, allowing it to be treated syntactically as if it were a primitive
fluent. This means that epistemic queries can be approached using
standard reasoning techniques of the situation calculus. Although
we have changed the notation somewhat to foreshadow the techniques
we will develop in Section \ref{sec:Obs-Knowledge}, their definition
operates as follows. First, define the \emph{results} of a concurrent
action to be the set of $action\#result$ pairs for all primitive
actions performed by the agent in question:\[
\mathbf{res}(agt,c,s)\isdef\{a\#SR(a,s)\,\,|\,\, a\in c\,\wedge\, actor(a)=agt\}\]


This definition is then used to formulate a regression rule as follows:\begin{multline}
\Reg(\Knows(agt,\phi,do(c,s))\isdef\,\,\,\,\,\,\,\,\,\,\exists y:\, y=\mathbf{res}(agt,c,s)\,\\
\wedge\Knows(agt,\left[Poss(c)\wedge\mathbf{res}(agt,c)=y\right]\rightarrow\Reg(\phi[do(c,s)]),s)\label{eq:reg_k_std}\end{multline}


This works by collecting the sensing results from each action performed
by the agent into the set $y$, then ensuring matching sensing results
in every situation considered possible. It expresses the knowledge
of the agent after a concurrent action in terms of what it knew before
the action, along with the information returned by the action. This
technique relies heavily on the fact that all actions are public,
since it requires every agent's knowledge to be updated in response
to every action.

As with the non-epistemic case, repeated applications of $\Reg$ can
transform a knowledge query into one that is uniform in the initial
situation. While it would be valid to then expand the $\Knows$ macros
and handle the query using first-order logic, in practice the reasoning
procedure would leave $\Knows$ intact and use a specialised prover
based on modal logic.


\subsection{Extending Knowledge Theories}

As we briefly discussed in Section \ref{sec:Observations:Background},
different kinds of information-generating actions are typically modelled
by directly modifying the successor state axiom for $K$. Unfortunately,
this does not in general preserve the results of \citep{scherl03sc_knowledge};
there is no guarantee that the modified axioms will preserve accessability
properties of $K$, or will permit a regression rule for knowledge.

For example, \citet{Petrick06thesis} extends the approach of \citep{scherl03sc_knowledge}
to include \emph{guarded} sensing actions. These actions cause the
agent to learn that some formula $\phi$ holds, but only if an additional
guard formula $\psi$ also holds in the world. They are axiomatised
like so:\begin{multline*}
K(agt,s'',do(a,s))\equiv\exists s':\, s''=do(a,s)\wedge K(agt,s',s)\wedge\dots\\
a=sense_{i}\,\rightarrow\,\left[\psi_{i}(s)\,\rightarrow\,\phi_{i}(s)\equiv\phi_{i}(s')\right]\end{multline*}


It is straightforward to demonstrate that this successor state axiom
is no longer guaranteed to preserve the accessibility properties of
Section TODO in general, although it is possible to syntactically
restrict $\psi$ and $\phi$ to regain this important property \citet{Petrick06thesis}.

As another example, consider the alternate successor state axiom for
$K$ proposed by \citet{Lesperance99sitcalc_approach}, in one of
the few existing works in the situation calculus that does not assume
synchroncity:

TODO: put it here

In order to account for hidden actions, this axiom must universally
quantify over situation terms. This new formulation is therefore incompatible
with the standard regression rule for knowledge, and \citep{Lesperance99sitcalc_approach}
offers no reasoning procedure other than general second-order theorem
proving.

The standard approach to formalising the local information available
to each agent, by directly modifying the successor state axiom for
knowledge, is not elaboration tolerant -- it is easy to destroy important
properties of the formalism. As we shall demonstrate in this chapter,
by basing our approach on an explicit, separately-axiomatied account
of the local perspective of each agent, our formulation robustly maintains
these properties in the face of theory elaboration.


\subsection{Literal Knowledge Fluents}

\begin{itemize}
\item Restrict formulae allowed inside $\Knows$ to atomic literals, in
style of \citet{demolombe00tractable_sc_belief}. 
\item Leverage encoding develope by \citet{petrick02knowledge_equivalence} 
\end{itemize}

\section{Knowledge and Observation\label{sec:Knowledge:Observation}}

In this section we develop the axioms for our new formulation of knowledge,
which is based on the explicit account of each agent's local perspective
that we developed in Chapter \ref{ch:observations}. We begin from
one of the basic tenets of epistemic reasoning, as described by \citep{halpern90knowledge_distrib}
-- that an agent's knowledge at any particular time must depend solely
on its local history: the knowledge that it started out with combined
with the observations it has made since then .

Given an explicit account of the observations made by each agent,
the required semantics of the $K$ relation are clear: $K(agt,s',s)$
must hold whenever $s'$ is legal, both $s$ and $s'$ would result
in the same view for the agent, and $s$ and $s'$ are rooted at $K_{0}$-related
initial situations:\begin{equation}
K(agt,s',s)\equiv Legal(s')\wedge View(agt,s')=View(agt,s)\wedge K_{0}(root(s'),root(s))\label{eq:k-desired}\end{equation}


In essence, this is a direct encoding into the situation calculus
of the definitions of knowledge from the classic epistemic reasoning
literature \citep{parikh85dist_knowledge,halpern90knowledge_distrib,fagin95}.

While a wonderfully succinct definition of how knowledge should behave,
this formulation cannot be used directly in a basic action theory.
The dynamics of fluent change must be specified by a successor state
axiom, so we must formulate a successor state axiom for the $K$ fluent
which enforces the above equivalence.

For notational convenience, let us first introduce an action description
predicate $\PbU$ (for {}``legal but unobservable'') indicating
that the actions $c$ are legally performed in $s$, but no observations
will be made by $agt$ if they occur:\begin{equation}
\PbU(agt,c,s)\,\equiv\, Legal(c,s)\wedge Obs(agt,c,s)=\{\}\label{eqn:PbU_defn}\end{equation}


By stating that $s\leq_{\PbU(agt)}s'$, we assert that $agt$ would
make no observations were the world to move from situation $s$ to
$s'$. This means that the agent's view in both situations would be
identical, so if it considers $s$ possible then it must also consider
$s'$ possible. Following this intuition, we propose the following
successor state axiom to capture the desired dynamics of the knowledge
fluent:\begin{align}
K(agt,s'',do(c,s))\equiv & \left[\, Obs(agt,c,s)=\{\}\rightarrow K(agt,s'',s)\,\right]\nonumber \\
\wedge & \left[\, Obs(agt,c,s)\neq\{\}\rightarrow\exists c',s':\, Obs(agt,c',s')=Obs(agt,c,s)\right.\nonumber \\
 & \left.\,\,\,\,\,\,\,\,\wedge\, Poss(c',s')\wedge K(agt,s',s)\wedge do(c',s')\leq_{\PbU(agt)}s''\,\right]\label{eqn:new_k_ssa}\end{align}


If $c$ was totally unobservable, the agent's state of knowledge does
not change. Otherwise, it considers possible any legal successor to
a possible alternate situation $s'$ that can be brought about by
an action $c'$ yielding identical observations. It also considers
possible any future of such a situation in which is would make no
further observations. To reiterate: unlike the standard successor
state axiom from equation \eqref{eq:k_ssa_standard}, our new formalism
requires agents to consider any possible future situation in which
they would make no further observations.

It remains to specify $K$ in the initial situation. The relation
$K_{0}$ defines knowledge before \emph{any} actions have occurred,
but the agents must consider the possibility that some hidden actions
have occurred. In other words, we must include situations where $root(s)\leq_{\PbU(agt)}s$
in the $K$-relation for initial situations. We therefore propose
the following axiom:\begin{gather}
Init(s)\rightarrow\left[K(agt,s'',s)\equiv\exists s'\,.\, K_{0}(agt,s',s)\wedge s'\leq_{\PbU(agt)}s'')\right]\label{eqn:new_k_s0}\end{gather}


\begin{defn}
We will denote by $\Dt_{K}^{obs}$ the axioms for our new observation-based
semantics for knowledge, as detailed in equations (\ref{eqn:new_k_ssa},\ref{eqn:new_k_s0})
above. 
\end{defn}
These axioms suffice to ensure that knowledge behaves as we require:
two situations will be related by $K(agt,s',s)$ if and only if they
result in identical views for that agent, $s'$ is legal, and their
root situations were initially related.

\begin{thm}
\label{thm:k_obs_equiv} For any agent $agt$ and situations $s$
and $s''$:\begin{multline*}
\Dt\cup\Dt_{K}^{obs}\models K(agt,s'',s)\equiv\\
Legal(s'')\wedge View(agt,s'')=View(agt,s)\wedge K_{0}(root(s''),root(s))\end{multline*}

\end{thm}
\begin{proofsketch}
For the \emph{if} direction we establish each of the three conjuncts
individually. The $root$ case is trivial since equation (\ref{eqn:new_k_ssa})
always expresses $K(s'',do(c,s))$ in terms of $K(s',s)$, while equation
(\ref{eqn:new_k_s0}) related $K$ for initial situations back to
$K_{0}$. The $Legal$ case relies on the fact that $\PbU$ implies
$Legal$, while the $View$ case relies on the fact that $s\leq_{\PbU}s'\rightarrow View(s)=View(s')$.
For the \emph{only-if} direction we show how to construct an $s'$
satisfying the $\exists s'$ parts of equations (\ref{eqn:new_k_ssa},\ref{eqn:new_k_s0}). 
\end{proofsketch}
Using this new formulation, an agent's knowledge is completely decoupled
from the global notion of actions, instead depending only on the local
information that it has observed. Of course, this must be coupled
with a specific axiomatiation of how the $Obs$ function behaves.
Any of the axiomatisations demonstrated in Chapter \ref{ch:observations}
can be used, and our account of knowledge can be used unmodified.

As a demonstration of the correctness of their axioms, \citet{scherl03sc_knowledge}
prove five properties of their formalism: that knowledge-producing
actions have only knowledge-producing effects; that unknown fluents
remain unknown by default; that knowledge incorporates the results
of sensing actions; that known fluents remain known by default; and
that agents have knowledge of the effects of their actions.

However, the intuition behind these properties depends heavily on
the assumption of a single agent and the separation of actions into
two classes: knowledge-producting actions that only return sensing
information, and ordinary actions that only affect the state of the
world. In asynchronous multi-agent domains, these restrictions cannot
be meaningfully applied.

For example, it is entirely possible that a knowledge-producing action
and an ordinary action are performed concurrently by two different
agents, so the results of a sensing action might be immediatey made
invalid. Moreover, suppose that an agent performs an action to make
a formula $\phi$ true, but there is a series of hidden actions that
could subsequently make $\phi$ false. The agent cannot meaningfully
claim to know $\phi$, since it could become false updating the local
view of that agent.

We claim that, given an appropriate axiomatisation of how actions
are related to observations, the validity of Theorem \ref{thm:k_obs_equiv}
is sufficient justification for the correctness of our knowledge axioms.


\section{Properties of Knowledge\label{sec:Knowledge:Properties}}

With the basic axioms in place, let us study some properties of our
formalism in greater detail. We begin by comparing it to the standard
account of knowledge due to \citet{scherl03sc_knowledge}. Its basic
assumption that {}``all agents are aware of all actions'' is captured
in our observation-based formulation using equations (\ref{eq:Observations:ObsStd1},\ref{eq:Observations:ObsStd2})
from Chapter \ref{ch:observations}, which we repeat here for convenience:\begin{gather*}
a\in Obs(agt,c,s)\,\equiv\, a\in c\\
a\#r\in Obs(agt,c,s)\,\equiv\, a\in c\wedge actor(a)=agt\wedge SR(a,s)=r\end{gather*}


That is, an agent observes all actions that occur, and additionall
observed the sensing results of all actions that it performs. If these
definitions are used, our new account of knowledge will behave identically
to the standard account:

\begin{thm}
Suppose $\Dt_{ad}$ contains equations (\ref{eq:Observations:ObsStd1},\ref{eq:Observations:ObsStd2})
as definitions of the $Obs()$ function, then for any situation terms
$\sigma$ and $\sigma'$:\[
\Dt\cup\Dt_{K}^{std}\models K(agt,\sigma',\sigma)\,\,\,\,\mathrm{iff}\,\,\,\,\Dt\cup\Dt_{K}^{obs}\models K(agt,\sigma',\sigma)\]

\end{thm}
\begin{proof}
Equations (\ref{eq:Observations:ObsStd1},\ref{eq:Observations:ObsStd2})
mean $Obs(agt,c,s)$ cannot be empty for $c\neq\{\}$, so $s=s'$
iff $s\leq_{\PbU(agt)}s'$. Substituting the equivalences from (\ref{eq:Observations:ObsStd1},\ref{eq:Observations:ObsStd2})
into equations (\ref{eqn:new_k_ssa},\ref{eqn:new_k_s0}), these then
amount to simple re-arrangements of equations (\ref{eq:k_s0_standard},\ref{eq:k_ssa_standard})
respectively, meaning that $K$ behaves the same under both theories. 
\end{proof}
Having established that our account subsumes the standard {}``public
actions'' account of knowledge, we can also show that it maintains
many of its desirable properties in the general case. One of the fundamental
results in \citep{scherl03sc_knowledge} is that if the initial knowledge
relation $K_{0}$ is reflexive, symmetric, transitive or Euclidean,
then the $K$ relation has these properties for any situation. In
our formalism, these properties follow immediately from Theorem \ref{thm:k_obs_equiv}
and the reflexive, symmetric, transitive and Euclidean nature of the
equality operator.

\begin{thm}
If the $K_{0}$ relation is restricted to be reflexive, transitive,
symmetric or Euclidean, then the $K$ relation at every legal situation
will satisfy the same properties. 
\end{thm}
\begin{proof}
Each follows directly from Theorem \ref{thm:k_obs_equiv} and the
properties of equality. We will take the transitive case as an example,
the other cases are similar.

Suppose that $K_{0}$ is transitive, and we have legal situations
$s_{1}$, $s_{2}$, $s_{3}$ such that $K(agt,s_{2},s_{1})$ and $K(agt,s_{3},s_{2})$.
Then by Theorem \ref{thm:k_obs_equiv} we have the following:\begin{gather*}
K_{0}(root(s_{2}),root(s_{1}))\\
K_{0}(root(s_{3}),root(s_{2}))\\
View(agt,s_{1})=View(agt,s_{2})\\
View(agt,s_{2})=View(agt,s_{3})\end{gather*}
 From the transitivity of $K_{0}$ we can conclude that $K_{0}(root(s_{3}),root(s_{1}))$.
From the transitivity of equality we can conclude that $View(agt,s_{1})=View(agt,s_{3})$.
Since $s_{3}$ is restricted to be legal, we have enough to satisfy
the RHS of the equivalence in Theorem \ref{thm:k_obs_equiv}, and
can conlude that $K(agt,s_{3},s_{1})$ and $K$ is therefore transitive. 
\end{proof}
That these properties hold regardless of the axiomatisation of $Obs$
is a compelling argument in favour of our approach. As noted by \citet{Petrick06thesis},
certain kinds of sensing action can easily invalidate these properties
if not axiomatised carefully. It is therefore worth considering such
cases in more detail.

The problematic sensing actions in \citep{Petrick06thesis} are \emph{guarded}
sensing actions, which update $K(agt,s',s)$ according to the following
axiom: \begin{multline*}
K(agt,s'',do(a,s))\equiv\exists s':\, s''=do(a,s)\wedge K(agt,s',s)\wedge\dots\\
a=sense_{i}\,\rightarrow\,\left[\psi_{i}(s)\,\rightarrow\,\phi_{i}(s)\equiv\phi_{i}(s')\right]\end{multline*}


The difficultly here is that although the agent will learn $\phi$
if the guard $\psi$ is true, it will not necessarily \emph{know whether}
it has learned this. If the action fails to produce the desired sensing
result, the agent's knowledge is not updated to reflect that the guard
was false. This means symmetry of the $K$ relation may not be preserved.

To ensure that symmetry is preserved through action, it is necessary
to axiomatise such sensing actions in such a way that the the guard
formula itself also becomes known. While this can be achived by syntactically
restricting the formulae, as in \citep{Petrick06thesis}, our approach
of explicitly representing the observations made by each agent avoids
the problem automatically -- if the sensing information is not included
in the agent's observations, it can conclude that the guard condition
must have been false.

Our formalism is thus a proper generalisation of the standard account
of knowledge in the situation calculus. It is also an \emph{elaboration
tolerant} generalisation, automatically maintaining important properties
of the axiomatisation as more complex models of sensing and observability
are introduced. To demonstrate the power gained by such generalisation,
Section \ref{sec:Knowledge:Example} shows how to use our formalism
to model a domain in which agents can only observe actions performed
in the same room as them.


\section{Regression\label{sec:Knowledge:Regression}}

The final aspect of our new account of knowledge is to extend the
techniques for effective reasoning in the situation calculus to handle
the modified formalism. The appearance of $\leq_{\PbU(agt)}$ in equation
(\ref{eqn:new_k_ssa}) means that our new successor state axiom universally
quantifies over situations, so standard regression techniques cannot
be used. We must appeal to the persistence condition meta-operator
introduced in Section \ref{sub:Property-Persistence} to transform
this quantification into a uniform formula, so that regression can
be applied.

We propose the following as the regression rule for $\Knows$ under
our formalism:\begin{align}
\Reg(\Knows(agt,\phi,do(c,s))\isdef\,\, & \exists o:\, Obs(agt,c,s)=o\nonumber \\
 & \wedge\,\left[o=\{\}\,\rightarrow\,\Knows(agt,\phi,s)\right]\nonumber \\
 & \wedge\,\left[o\neq\{\}\,\rightarrow\,\Knows(agt,\forall c':\, Obs(agt,c')=o\right.\nonumber \\
 & \,\,\,\,\,\,\,\,\,\,\,\wedge Legal(c')\rightarrow\left.\Reg(\Pst(\phi,\PbU(agt)),c'),s)\right]\label{eqn:R_do_c_s}\end{align}


Note the similarity to the standard regression rule for knowledge
in equation \eqref{eq:reg_k_std}. New in our version are: the replacement
of the $\mathbf{res}$ macro with an explicit, flexibly definition
of what the agent has observed; explicit handling of the case when
the agent makes no observations; and use of the persistence condition
to account for arbitrarily-long sequences of hidden actions.

As required for a regression rule, equation \eqref{eqn:R_do_c_s}
reduces a knowledge query at $do(c,s)$ to a knowledge query at $s$.
It is also intuitively appealing: to know that $\phi$ holds, the
agent must know that in all situations that agree with its observations,
$\phi$ cannot become false without it making an observation - this
is the meaning of $\Pst(\phi,\PbU(agt))$ in the above.

We must also specify the regression of $\Knows$ in the initial situation,
as equation (\ref{eqn:new_k_s0}) also uses the $\leq_{\PbU(agt)}$
ordering. This clause produces an expression in $\KnowsZ$ at $S_{0}$,
meaning that it can be handled by epistemic reasoning about the initial
situation only:\begin{equation}
\Reg(\Knows(agt,\phi,S_{0}))\,\isdef\,\KnowsZ(agt,\Pst(\phi,\PbU(agt)),S_{0})\label{eqn:R_s0}\end{equation}


Using these regression rules, we can handle knowledge queries in our
formalism using standard techniques for effective reasoning in the
situation calculus.

\begin{thm}
\label{thm:Reg_Knows}Given a basic action theory $\Dt$ and a uniform
formula $\phi$:\[
\Dt\cup\Dt_{K}^{obs}\,\models\,\Knows(agt,\phi,s)\equiv\Reg(\Knows(agt,\phi,s))\]

\end{thm}
\begin{proofsketch}
In the $do(c,s)$ case, we proceed by expanding the definition for
$\Knows$ using our new successor state axiom for $K$, collecting
sub-formulae that match the form of the $\Knows$ macro, and using
regression and the persistence condition to render the resulting knowledge
expressions uniform in $s$. In the base case, we apply the persistence
condition to an expansion of $\Knows$ at $S_{0}$ to produce the
desired result. 
\end{proofsketch}
While this reasoning method is suitable for modelling and simulation
purposes, it would be unreasonable for a situated agent to ask {}``do
I know $\phi$ in the current situation?'' using the situation calculus
query $\Dt\models\mathbf{Knows}(agt,\phi,\sigma)$, as it cannot be
expected to have the full current situation $\sigma$. It will however
have its current view $v$ and can constructa query like the following:\[
\Dt\cup\Dt_{K}\models\forall s:\, View(agt,s)=v\wedge root(s)=S_{0}\rightarrow\mathbf{Knows}(agt,\phi,s)\]


Such a query universally quantifies over situations and so cannot
be handled using regression. It is also not in a form amenable to
the persistence condition operator, so the agent has no means of effectively
answering this query.

We should expect from Theorem \ref{thm:k_obs_equiv} that this quantification
over situations is unnecessary -- after all, all situations with the
same view for that agent should result in the same knowledge. Let
us explicitly define knowledge with respect to a view as follows:\[
\mathbf{Knows}(agt,\phi,v)\,\isdef\,\forall s:\, View(agt,s)=v\wedge root(s)=S_{0}\rightarrow\mathbf{Knows}(agt,\phi,s)\]


We can then modify the regression rules in equations (\ref{eqn:R_do_c_s},\ref{eqn:R_s0})
to work directly on formulae of this form. The resulting rules are
actually simpler than for regression over situations, as there are
no empty observations in a view. The result is:\begin{align*}
\Reg(\mathbf{Knows}(agt,\phi,o\cdot v))\isdef\,\, & \mathbf{Knows}(agt,\forall c:\, Obs(agt,c)=o\\
 & \,\,\,\,\,\,\,\,\wedge Legal(c)\rightarrow\Reg(\Pst(\phi,\PbU(agt)),c),v)\\
\Reg(\mathbf{Knows}(agt,\phi,\epsilon))\isdef\,\, & \KnowsZ(agt,\Pst(\phi,\PbU(agt)),S_{0})\end{align*}
 Using regression in this way, an agent can reduce the query $\Knows(agt,\phi,v)$
to an equivalent query about its knowledge in the initial situation.

\begin{thm}
\label{thm:Reg_KnowsO}Given a basic action theory $\Dt$ and a uniform
formula $\phi$:\[
\Dt\cup\Dt_{K}^{obs}\,\models\,\Knows(agt,\phi,v)\equiv\Reg(\Knows(agt,\phi,v))\]

\end{thm}
\begin{proof}
TODO 
\end{proof}
Agents can thus reason effectively about their own knowledge using
only their local information. Our work makes it possible to include
a situation calculus model in the implementation of a real-world multi-agent
system, even when agents have only partial awareness of the actions
being performed.

It is worth re-iterating that our regression rules are no longer straightforward
syntactic transformations - rather, they involve a fixpoint calculation
to generate $\Pst(\phi,\PbU(agt))$. Can this really be considered
an effective reasoning technique? The previous work on the persistence
condition meta-operator discused the advantages of this approach in
detail. The primary advantage is that this form of reasoning can be
performed at all, as the alternative is general second-order theorem
proving.

Of course, the ultimate proof is in the implementation. We have implemented
a preliminary version of our technique and used it to verify the examples
found in the following section. TODO: clean up implementation, put
on website.

We close this section with a formal statement of a point which is
hopefully obvious: the persistence condition is not required when
reasoning in syncrhonous domains. It is straightforward to show that
$\Pst(\phi,\PbU(agt))$ in syncrhonous domains is always equivalent
to $\phi$. And the regression rule reduces to being a purely syntactical
manipulation. We thus do not introduce unnecessary complications for
domains in which effective reasoning procedures already exist, but
extend the reach of our formalism in richer domains where some inductive
reasonig is required.

\begin{thm}
Let $\Dt_{sync}$ be a synchronous basic action theory, then for any
uniform formula $\phi$:\[
\Dt_{sync}\models\forall s,agt:\,\phi[s]\,\equiv\,\Pst(\phi,\PbU(agt))[s]\]

\end{thm}
\begin{proof}
By definition, we have:\[
\Dt_{sync}\,\models\,\forall agt,c,s:\, Legal(c,s)\,\rightarrow Obs(agt,c,s)\neq\{\}\]

\end{proof}
Recall from equation (\ref{eqn:PbU_defn}) that:\[
\PbU(agt,c,s)\,\equiv\, Legal(c,s)\wedge Obs(agt,c,s)=\{\}\]


So clearly:\[
\Dt_{sync}\,\models\,\forall agt,c,s:\,\PbU(agt,c,s)\equiv\bot\]


The definition of $\Pst^{1}(\phi,\PbU(agt))$ will then produce:\[
\Pst^{1}(\phi,\PbU(agt))\,\equiv\,\phi\wedge\forall c:\,\bot\rightarrow\Reg(\phi,c)\,\equiv\,\phi\]


The calculation of $\Pst$ thus terminate immediately at the first
iteration, giving $\Pst(\phi,\PbU(agt))$ equal to $\Pst^{1}(\phi,\PbU(agt))$,
which is equiavalent to $\phi$ as desired.


\section{An Illustrative Example\label{sec:Knowledge:Example}}

We now give a brief demonstration of our formalism in action, using
it to model the {}``party invitation'' domain outlined in Chapter
\ref{ch:intro}. We adopt an explicit axiomatisation of partially
obervability based on the $CanObs$/$CanSense$ predicates introduced
in TODO.

The fluents of interest in this domain are the location of the party
(the function $loc$) and whether each agent is in the room (the predicate
$InRoom$). The action $read$ reads the invitation and returns the
location of the party, while the non-sensing actions $enter$ and
$leave$ cause the agents to move in/out of the room. The $read$
action is only observed by agents who are in the room. This domain
can be summarised by the following axioms:\begin{gather*}
loc(S_{0})=C\\
loc(do(c,s))=l\equiv loc(s)=l\end{gather*}
 \begin{gather*}
InRoom(Alice,S_{0})\equiv InRoom(Bob,S_{0})\equiv true\\
InRoom(agt,do(c,s))\equiv enter(agt)\in c\,\vee\, InRoom(agt,s)\wedge leave(agt)\notin c\\
Poss(enter(agt),s)\equiv\neg InRoom(agt,s)\\
Poss(leave(agt),s)\equiv InRoom(agt,s)\end{gather*}
 \begin{gather*}
Poss(read(agt),s)\equiv InRoom(agt,s)\\
SR(read(agt),s)=r\equiv r=loc(s)\end{gather*}
 \begin{gather*}
\forall agt,l:\,\neg\KnowsZ(agt,loc=l,S_{0})\\
\forall agt_{1},agt_{2},l:\,\KnowsZ(agt_{1},\neg\KnowsZ(agt_{2},loc=l),S_{0})\\
\forall agt:\,\KnowsZ(agt,InRoom(Alice)\wedge InRoom(Bob),S_{0})\end{gather*}
 \begin{gather*}
CanObs(agt,leave(agt'),s)\equiv CanObs(agt,enter(agt'),s)\equiv true\\
CanSense(agt,leave(agt'),s)\equiv CanSense(agt,enter(agt'),s)\equiv false\\
CanObs(agt,read(agt'),s)\equiv InRoom(agt',s)\\
CanSense(agt,read(agt'),s)\equiv agt=agt'\end{gather*}


\medskip{}


The following are examples of knowledge queries that can be posed
in our formalism, a brief explanation of their outcome, and a demonstration
of how they can be answered using our new regression rules. Each has
been verified by the preliminary implementation of our reasoning engine.

\begin{example}
Initially, Alice doesn't know where the party is:\[
\Dt\cup\Dt_{K}^{obs}\models\neg\exists l:\,\mathbf{Knows}(Alice,loc=l,S_{0})\]

\end{example}
It is given that $\neg\exists l:\,\KnowsZ(Alice,loc=l,S_{0})$, and
the only way for her to learn such information is by performing a
$read$ action. Since she would always observe such an action, she
cannot have learnt the party's location as a result of hidden actions,
and the example is entailed. Formally:\begin{align*}
\Reg(\neg\exists l:\,\mathbf{Knows}(Alice,loc=l,S_{0}))\,\Rightarrow\,\,\,\, & \neg\exists l:\,\KnowsZ(Alice,\Pst(loc=l,PbU(Alice)),S_{0})\\
\Pst(loc=l,PbU(Alice))\,\Rightarrow\,\,\,\, & loc=l\end{align*}


So the query reduces to:\[
\neg\exists l:\,\KnowsZ(Alice,loc=l,S_{0})\]


Which is entailed by the domain.\\
 \\


\begin{example}
After reading the invitation, Bob will know where the party is:\[
\Dt\cup\Dt_{K}^{obs}\models\Knows(Bob,loc=C,do(\{read(Bob)\},S_{0}))\]

\end{example}
The sensing results of the $read$ action inform Bob of the location
of the party. Since this location cannot change after any sequence
of hidden actions, he can be sure of the party's location. Formally,
using the fact that $Obs(Bob,\{read(Bob)\},s)=\{read(Bob)\#loc(s)\}$:\begin{multline*}
\Reg(\Knows(Bob,loc=C,do(\{read(Bob)\},S_{0})))\,\Rightarrow\\
\exists o:\, Obs(Bob,\{read(Bob)\},S_{0})=o\,\wedge\,\\
\Knows(Bob,\forall c':\, Poss(c')\wedge Obs(Bob,c')=o\rightarrow\Reg(\Pst(loc=C,PbU(Bob)),c'),S_{0})\end{multline*}


Since $\Reg(\Pst(loc=C,PbU(Bob)),c')\Rightarrow loc=c$ and $loc(S_{0})=C$,
this simplifies to:\[
\Knows(Bob,\forall c':\, Poss(c')\wedge Obs(Bob,c')=\{read(Bob)\#C\}\rightarrow loc=C,S_{0})\]


Since the only possible value of $c'$ that satisfies the antecedent
is $\{read(Bob)\}$, we can insert the definitions of $Poss$ and
$Obs$ to obtain:\[
\Knows(Bob,InRoom(Bob)\wedge loc=C\,\rightarrow\, loc=C,S_{0})\]


This tautology is clearly entailed by the domain.\\


\begin{example}
Initially, Bob knows that Alice doesn't know where the party is:\[
\Dt\cup\Dt_{K}^{obs}\models\mathbf{Knows}(Bob,\neg\exists l:\,\Knows(Alice,loc=l),S_{0})\]

\end{example}
Alice could learn the location of the party by performing the $read$
action, but since Bob is in the room he would observe this action
taking place. Since he has not observed it, he can conclude that Alice
does not know the location of the party. Formally:\begin{multline*}
\Reg(\mathbf{Knows}(Bob,\neg\exists l:\,\Knows(Alice,loc=l),S_{0}))\,\Rightarrow\\
\KnowsZ(Bob,\Pst(\neg\exists l:\,\Knows(Alice,loc=l),PbU(Bob)),S_{0})\end{multline*}
 \begin{multline*}
\Pst(\neg\exists l:\,\Knows(Alice,loc=l),PbU(Bob))\,\Rightarrow\\
\neg\exists l:\,\Knows(Alice,loc=l)\wedge\left(InRoom(Bob)\,\vee\,\neg InRoom(Alice)\right)\end{multline*}


So the query reduces to:\[
\KnowsZ(Bob,\neg\exists l:\,\Knows(Alice,loc=l)\wedge\left(InRoom(Bob)\,\vee\,\neg InRoom(Alice)\right),S_{0})\]


Which is entailed by the domain.\\
 \\


\begin{example}
After leaving the room, Bob won't know that Alice doesn't know the
location of the party:\[
\Dt\cup\Dt_{K}^{obs}\models\mathbf{\neg Knows}(Bob,\neg\exists l:\,\Knows(Alice,loc=l),do(\{leave(Bob)\},S_{0}))\]

\end{example}
Once Bob leaves the room, he would be unable to observe Alice reading
the invitation. He must therefore consider it possible that she has
read it, and may know the location of the party. Formally, we can
use $Obs(Bob,\{leave(Bob)\})=\{leave(Bob)\}$ to regress the outer
expression as follows:\begin{multline*}
\Reg(\neg\Knows(Bob,\phi,do(\{leave(Bob)\},S_{0})))\,\Rightarrow\\
\neg\Knows(Bob,InRoom(Bob)\rightarrow\Reg(\Pst(\phi,PbU(Bob)),\{leave(Bob)\}),S_{0})\end{multline*}


For the inner expression, we have from the previous example:\begin{multline*}
\Pst(\neg\exists l:\,\Knows(Alice,loc=l),PbU(Bob))\,\Rightarrow\\
\neg\exists l:\,\Knows(Alice,loc=l)\wedge\left(InRoom(Bob)\,\vee\,\neg InRoom(Alice)\right)\end{multline*}


This expression is key: for Bob to know $\neg\exists l:\Knows(Alice,loc=l)$,
he must also know either that he is in the room (and will thus observe
the $read(Alice)$ action if it occurs) or that Alice is not in the
room (so the $read(Alice)$ action will not be possible). Otherwise,
Alice could learn the location of the party without him making any
further observations.

When we regress over the action $\{leave(Bob)\}$ then $InRoom(Bob)$
is made false:\begin{multline*}
\Reg(\Pst(\neg\exists l:\,\Knows(Alice,loc=l),PbU(Bob)),\{leave(Bob)\})\,\Rightarrow\\
\neg\exists l:\,\Knows(Alice,loc=l)\wedge\left(false\,\vee\,\neg InRoom(Alice)\right)\end{multline*}


And the entire expression can be simplified to:\[
\neg\Knows(Bob,InRoom(Bob)\rightarrow\neg\exists l:\,\Knows(Alice,loc=l)\wedge\neg InRoom(Alice),S_{0})\]


Since Alice is known to be in the room, this expression will be entailed
by the domain.


\section{TODO: Approximate Epistemic Reasoning\label{sec:Knowledge:Approximate}}

\begin{itemize}
\item Restrict formulae allowed inside $\Knows$ to atomic literals, in
style of \citet{demolombe00tractable_sc_belief}. 
\item Leverage encoding develope by \citet{petrick02knowledge_equivalence} 
\end{itemize}

\section{Discussion\label{sec:Knowledge:Discussion}}

In this section, we have first developed a principled axiomatisation
of the observability of actions, using the notion of observations
and views as analogues of actions and situations that are localised
to an individual agent. This terminology has been deliberately chosen
to match similar concepts in other formalisations of knowledge, such
as the well-known treatise of \citet{halpern90knowledge_distrib}.
By reifying these concepts as terms in the logic, we are able to give
a succinct definition of the dynamics of the knowledge fluent and
prove that its behaviour matches our intuitive expectations.

As an example of why this is important, consider one of the few existing
formulations of knowledge in the situation calculus that allows for
hidden actions, that of \citep{Lesperance99sitcalc_approach}. Their
successor state axiom for the $K$ fluent is as follows:\begin{align*}
K(agt,s'',do(a,s))\equiv\,\,\, & \exists s':\, K(agt,s',s)\\
 & \wedge\,(actor(a)\neq agt\,\rightarrow\, s'\leq_{actor(a)\neq agt}s''))\\
 & \wedge\,(actor(a)=agt\,\rightarrow\,\exists s^{*}:\,\left[s'\leq_{actor(a)=agt}s^{*}\wedge\right.\\
 & \,\,\,\,\,\,\,\,\,\,\,\,\left.s''=do(a,s^{*})\wedge Poss(a,s^{*})\wedge sr(a,s)=sr(a,s^{*}))\right]\end{align*}


In this case agents are only aware of the actions that they themselves
perform, and they consider possible an arbitrary sequence of hidden
actions preceding each action of their own. However, this formulation
has a subtle problem: an agent's knowledge can change in response
to actions performed by others. Suppose that $agt$ has just performed
action $a_{1}$, so the world is in situation $do(a_{1},s)$. Another
agent then performs the action $a_{2}$, leaving the world in situation
$do(a_{2},do(a_{1},s))$. Since it is not aware of the occurrence
of $a_{2}$, the knowledge of $agt$ should be unchanged between these
two situations. This is not the case under the formulation of \citeauthor{Lesperance99sitcalc_approach}.
By explicitly formalising the notion of a view, our framework avoids
such problems.

A further advantage of our explicit axiomatisation of observations
is in establishing properties of the knowledge relation. A major theorem
of \citet{scherl03sc_knowledge} states that if the $K$-relation
is reflexive, symmetric or transitive at the initial situation, then
it has that property at every situation. In our formulation these
are all simple corollaries of Theorem \ref{thm:k_obs_equiv}, based
on the reflexive, symmetric and transitive nature of the equality
symbol.

TODO: belief.When reasoning about knowledge, it is generally assumed
that the \emph{actual} state of the world is among those considered
possible, so that an agent's knowledge is always true:\[
\Knows(agt,\phi,s)\,\rightarrow\,\phi[s]\]


If this is not the case, one is performing \emph{doxastic} or belief-based
reasoning.

TODO: refs here.

We have demonstrated that our formalism is expressive enough to capture
the standard account of knowledge based on public actions, as well
as more complex formulations where the observability of actions depends
on the state of the world. We have also demonstrated that despite
allowing for arbitrarily-long sequences of hidden actions, our formalism
still permits automated reasoning for handling knowledge queries,
including a preliminary implementation of such a reasoning system.

Of course, the effectiveness of automated reasoning is now highly
dependent on the effectiveness of calculating the persistence condition.
Since this is a fixpoint calculation, it can be computationally expensive
and even undecidable in very complex domains. By factoring out the
necessary inductive reasoning into a separate operator, it can now
be studied and improved in isolation. We have already identified several
classes of basic action theory in which the persistence condition
can be calculated quite effectively, and our investigations in this
area are ongoing; see the reference for details \citep{kelly07sc_persistence}.

Finally, we have shown that a simple modification to our regression
rules allows a situated agent to reason about its own knowledge using
only its local view, rather than requiring a full situation term.
Our new observation-based semantics thus provides a powerful account
of knowledge suitable both for reasoning \emph{about}, and for reasoning
\emph{in}, asynchronous multi-agent domains.

