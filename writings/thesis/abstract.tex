


\chapter*{Abstract}

This thesis develops three powerful extensions to the situation
calculus for reasoning about teams of agents in asynchronous, multi-agent
domains. We explicitly represent the local \emph{observations} made
by each agent when an action occurs, constructing: a theory of individual-level
knowledge in which agents can account for arbitrarily-long sequences
of hidden actions; a formalism for group-level epistemic modalities
that permits a regression rule for common knowledge; and techniques
for planning the joint execution of a shared task using a partially-ordered
sequence of actions.

Existing accounts of knowledge in the situation calculus assume each
agent knows how many actions have been performed, demanding a degree
of synchronicity among agents. In asynchronous domains an agent's
knowledge must allow for arbitrarily-long sequences of hidden actions,
which cannot be handled by existing reasoning techniques. We develop
a new operator called the \emph{persistence condition} to augment
the standard regression operator, and use it to build a new account
of individual-level knowledge suitable for asynchronous domains. Our
formalism lets agents reason about their own knowledge using only
their local information.

Common knowledge is traditionally modelled using a second-order axiom
that precludes regression as a reasoning technique. We use dynamic
logic to formulate a group-level epistemic language that
subsumes the common knowledge modality, and develop a general
regression rule for this language. This allows common knowledge to
be reasoned about effectively using standard regression techniques.

Execution planning in the situation calculus involves building situation
terms, which are fully-ordered sequences of the actions to be performed.
We develop a variant of prime event structures, which we call \emph{joint
executions,} to facilitate planning for cooperative execution in asynchronous
domains. These structures replace the use of situation terms in the
planning process, permitting partial ordering of actions that are
independent while automatically managing inter-agent synchronisation.

We both motivate and demonstrate our new techniques by developing
\emph{MIndiGolog}, a multi-agent variant of the Golog programming
language. Our implementation uses distributed logic programming to
share the planning workload, allowing a team of agents to cooperatively
plan and perform the execution of a shared Golog program.

The end result is a more robust and flexible account of knowledge
and action in the situation calculus, suitable both reasoning \emph{about},
and reasoning \emph{in,} asynchronous multi-agent domains.

