


\chapter*{Abstract}

This thesis develops three powerful extensions to the situation calculus
to facilitate reasoning about teams of agents in rich, asynchronous
multi-agent domains: a formalism for knowledge that accounts for potentially
arbitrarily-long sequences of hidden actions; support for reasoning
about common knowledge and other group-level epistemic modalities;
and techniques for planning the joint execution of a shared task using
a partially-ordered sequence of actions rather than a fully-ordered
situation term. Each new feature is accompanied by an effective reasoning
procedure based on regression, the standard technique for effective
reasoning in the situation calculus.

Existing accounts of knowledge in the situation calculus require the
domain to be synchronous. To support knowledge in the face of hidden
actions, we explicitly reify the \emph{observations} made by each
agent as the world evolves, and express an agent's knowledge in terms
of what it has observed. To deal effectively with arbitrarily-long
sequences of hidden actions, we augment regression with a new operator
called the \emph{persistence condition}, which uses a restricted fixpoint
calculation handle queries that universally quantify over situation
terms.

Common knowledge is traditionally modelled using a second-order axiom,
and thus cannot be reasoned about using regression. By developing
a more expressive epistemic language, we are able to formulate a regression
rule for common knowledge and so enable effective reasoning about
group-level knowledge.

Execution planning in the situation calculus involves building situation
terms, which are fully-ordered sequences of the actions to be performed.
We develop \emph{joint executions} to plan for cooperative execution
in asynchronous domains. These structures replace the use of raw situation
terms in the planning process, and permit partial ordering of actions
that are independent, reducing the synchronisation required among
agents.

We simultaneously motivate and demonstrate our new techniques by implementing
a multi-agent variant of the Golog programming language which we call
\emph{MIndiGolog}. Our implementation allows a team of agents to cooperatively
plan and perform the execution of a shared program.

The end result is a more robust and flexible account of knowledge
and action in the situation calculus, suitable for rich asynchronous
multi-agent domains.

