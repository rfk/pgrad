

\chapter*{Abstract}

This thesis develops several powerful extensions to the situation
calculus for reasoning about multi-agent teams in asynchronous domains.
We enrich the situation calculus with an explicit representation of
the local perspective of each agent, and then construct: a partially-ordered
representation of actions for planning the joint execution of a shared
task; a theory of individual-level knowledge that allows agents to
consider arbitrarily-long sequences of hidden actions; and a formalism
for group-level epistemic modalities that permits a regression rule
for common knowledge.

Execution planning in the situation calculus typically involves building
a fully-ordered sequence of the actions to be performed, requiring
constant synchronisation between agents if the plan is to be carried
out cooperatively. We develop a partially-ordered representation of
actions which we call a \emph{joint execution.} These structures allow
independent actions to be performed independently while ensuring that,
when necessary, synchronisation can be achieved based on the local
observations of each agent. Joint executions can be reasoned about
using standard situation calculus techniques, allowing them to easily
replace raw situation terms during planning.

Existing accounts of knowledge in the situation calculus assume that
everyone always knows how many actions have occurred, demanding significant
synchronicity among agents. In asynchronous domains agents must instead
consider arbitrarily-long sequences of hidden actions, which cannot
be reasoned about effectively using existing techniques. We develop
a new reasoning technique called the \emph{persistence condition}
operator to augment the standard regression operator, and use it to
build a new account of individual-level knowledge that correctly accounts
for hidden actions while retaining an effective reasoning procedure.
Our formalism directly allows agents to reason about their own knowledge
using only their local information.

Common knowledge is traditionally modelled using an explicit second-order
axiom, which precludes regression as an effective reasoning technique.
We formulate a more powerful language of group-level knowledge using
an epistemic interpretation of dynamic logic, and develop a regression
rule that is sound and complete for this language. As a consequence,
our formalism allows common knowledge to be reasoned about effectively
using standard regression techniques.

The end result is a more robust and flexible account of knowledge
and action in the situation calculus, suitable both reasoning \emph{about},
and reasoning \emph{in,} asynchronous multi-agent domains.

