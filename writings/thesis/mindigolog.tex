

\chapter{MIndiGolog}

\label{ch:mindigolog}

This chapter develops a Golog variant designed specifically for execution
in multi-agent domains. The aim is to build a distributed problem
solving system based on cooperative execution of a shared Golog program.
As this chapter will demonstrate, the existing features of the situation
calculus and Golog go a long way towards achieving this goal, but
ultimately fall short in rich multi-agent domains.

We begin by integrating several existing extensions to the situation
calculus into the semantics of Golog to better represent the dynamics
of a multi-agent team. Key among these is true concurrency of actions,
which we combine with the interleaved concurrency of ConGolog to give
a flexible account of concurrent execution. An explicit notion of
time is incorporated to enrich the world model and to assist in coordination
between agents. The concept of natural actions is also tightly integrated
into the language, to allow agents to predict the behaviour of their
teammates and environment. We name the resulting language {}``MIndiGolog''
for {}``Multi-Agent IndiGolog''.

The new features added to the situation calculus in the rest of the
thesis allow us to overcome these difficulties, and we will revisit
MIndiGolog at the end of chapter TODO. MIndiGolog thus simultaneous
serves as a motivation for, and a demonstration of, the techniques
developed throughout this thesis.

The chapter proceeds as follows: TODO.


\section{Background}


\section{MIndiGolog Semantics\label{sec:MIndiGolog}}

We have integrated three extensions to the situation calculus with
the semantics of IndiGolog to better model the dynamics of a multi-agent
team. These extensions allow agents to represent time, concurrently-occurring
actions, and natural actions in a robust way.


\subsection{Time}

TODO: The semantics of IndiGolog trivially accommodate this change.


\subsection{Concurrency}

While it is straightforward to modify the IndiGolog $Trans$ rule
for primitive actions to accept sets of concurrent actions, there
are deeper implications for the concurrency operator. This is implemented
by accepting a transition from either of the two programs as a transition
for the pair \citet{giacomo00congolog}. In the presence of true concurrency,
this is insufficient. Suppose program $\delta_{1}$ may be transitioned
by performing actions $c_{1}$, and $\delta_{2}$ may be transitioned
by performing actions $c_{2}$. As noted in \citet{pinto99tcongolog},
it should be possible to exploit true concurrency by performing both
simultaneously, i.e. $c_{1}\cup c_{2}$. However, this introduces
several complications that \citet{pinto99tcongolog} does not address.

First, precondition interaction means that $c_{1}\cup c_{2}$ may
not be possible even if the individual actions are. The transition
clause must ensure that the combination of the two sets of actions
is possible. Another issue arises when two programs can legitimately
be transitioned by executing the same action. Consider the following
programs which add ingredients to a bowl:\begin{multline*}
\delta_{1}=placeIn(Thomas,Flour,Bowl)\,;\\
\shoveright{placeIn(Thomas,Sugar,Bowl)}\\
\shoveleft{\delta_{2}=placeIn(Thomas,Flour,Bowl)\,;}\\
placeIn(Thomas,Egg,Bowl)\end{multline*}
 Executing $\delta_{1}||\delta_{2}$ should result in the bowl containing
two units of flour, one unit of sugar and an egg. However, an individual
transition for both programs is $c_{1}=c_{2}=placeIn(Thomas,Flour,Bowl)$.
Naively executing $c_{1}\cup c_{2}$ to transition both programs would
add only one unit of flour. Alternately, consider two programs waiting
for a timer to ring:\begin{gather*}
\delta_{1}=ringTimer\,;\, acquire(Thomas,Bowl)\\
\delta_{2}=ringTimer\,;\, acquire(Richard,Board)\end{gather*}
 Both programs should be allowed to proceed with a single occurrence
of the $ringTimer$ action, because it is an aspect of the environment.
To avoid unintuitive (and potentially dangerous) behavior, concurrent
execution must not be allowed to transition both programs using the
same \emph{agent-initiated} action. If an agent-initiated action may
be safely skipped, it can be enclosed in an appropriate if-then-else
or choice construct.

Taking these factors into account, we develop the improved transition
rule for concurrency in equation (\ref{eqn:trans_conc_new}). The
first two lines are the original interleaved concurrency clause from
ConGolog, while the remainder characterises the above considerations
for true concurrency. This robust combination allows the language
to more accurately reflect the concurrency present in multi-agent
teams. As with ConGolog and IndiGolog, our semantics make no attempt
to maximise concurrency or otherwise differentiate between potential
executions, they only state which transitions can be legally performed.\begin{multline}
Trans(\delta_{1}||\delta_{2},s,\delta',s')\equiv\\
\shoveright{\exists\gamma\,.\, Trans(\delta_{1},s,\gamma,s')\wedge\delta'=(\gamma||\delta_{2})}\\
\shoveright{\vee\exists\gamma\,.\, Trans(\delta_{2},s,\gamma,s')\wedge\delta'=(\delta_{1}||\gamma)}\\
\shoveright{\vee\exists c_{1},c_{2},\gamma_{1},\gamma_{2},t\,.\, Trans(\delta_{1},s,\gamma_{1},do(c_{1},t,s))}\\
\wedge Trans(\delta_{2},s,\gamma_{2},do(c_{2},t,s))\wedge Poss(c_{1}\cup c_{2},t,s)\\
\shoveright{\wedge\forall a.\left[a\in c_{1}\wedge a\in c_{2}\rightarrow Natural(a)\right]}\\
\wedge\delta'=(\gamma_{1}||\gamma_{2})\wedge s'=do(c_{1}\cup c_{2},t,s)\label{eqn:trans_conc_new}\end{multline}



\subsection{Natural Actions}

The formalism of \citet{reiter96sc_nat_conc} is adopted, with simple
modifications for our handling of time. While planning with natural
actions has previously been done in Golog \citet{pirri00planning_nat_acts},
the programmer was required to explicitly check for any possible natural
actions and ensure that they appear in the execution. We significantly
lower the burden on the programmer by guaranteeing that all legal
program executions result in legal situations. MIndiGolog agents will
plan for the occurrence of natural actions without having them explicitly
mentioned in the program. They may optionally be included in the program,
instructing the agents to wait for the action to occur before proceeding.

This is achieved using a new $Trans$ clause for the case of a single
action $c$, as shown in equation (\ref{eqn:trans_prim_new}). If
$s$ has an LNTP $t_{n}$ and corresponding set of natural actions
$c_{n}$, a transition can be made in three ways: perform $c$ at
a time before $t_{n}$ (fourth line), perform it along with the natural
actions at $t_{n}$ (fifth line), or wait for the natural actions
to occur (sixth line). If there is no LNTP, then $c$ may be performed
at any time greater than $start(s)$.\begin{multline}
Trans(c,s,\delta',s')\equiv\\
\exists t,t_{n},c_{n}\,.\, Lntp(s,t_{n})\wedge t\geq start(s)\wedge\\
\forall a.\left[Natural(a)\wedge Poss(a,t_{n},s)\equiv a\in c_{n}\right]\wedge\\
\left[t<t_{n}\wedge Poss(c,t,s)\wedge s'=do(c,t,s)\wedge\delta'=Nil\right.\\
\vee Poss(c\cup c_{n},t_{n},s)\wedge s'=do(c\cup c_{n},t,s)\wedge\delta'=Nil\\
\vee\left.s'=do(c_{n},t_{n},s)\wedge\delta'=c\right]\\
\vee\,\neg\exists t_{n}\,.\, Lntp(s,t_{n})\wedge\exists t\,.\, Poss(c,t,s)\wedge\\
t\geq start(s)\wedge s'=do(c,t,s)\wedge\delta'=Nil\label{eqn:trans_prim_new}\end{multline}


The occurrence of natural actions may also cause test conditions within
the program to become satisfied, so a new $Trans$ clause for $\phi?$
is also required as shown in equation (\ref{eqn:trans_cond_new}).
This permits a program consisting of a single test condition to make
a transition if the condition is satisfied, or if a natural action
occurs:%
\footnote{$\phi[s]$ should be read as {}``$\phi$ holds in situation $s$''%
} \begin{multline}
Trans(\phi?,s,\delta',s')\,\,\equiv\,\,\phi[s]\wedge\delta'=Nil\wedge s'=s\\
\shoveright{\vee\,\exists t_{n},c_{n}\,.\, Lntp(s,t_{n})\wedge\delta'=\phi?\wedge s'=do(c_{n},t_{n}s)}\\
\wedge\,\forall a.\left[Natural(a)\wedge Poss(a,t_{n},s)\rightarrow a\in c_{n}\right]\label{eqn:trans_cond_new}\end{multline}


A MIndiGolog execution will thus contain all natural actions that
will occur, regardless of whether they were considered explicitly
by the programmer.

Contrast this with the standard handling of exogenous events in IndiGolog,
which is achieved by executing the main program concurrently with
a program that generates exogenous actions:\[
\delta_{main}\,||\,\left(\pi(a)(Exog(a)?\,;\, a)\right)^{*}\]
 This allows the program to make a legal transition regardless of
what exogenous actions occur - an approach suitable for dealing with
arbitrary exogenous actions which may occur at any time, but suboptimal
for handling \emph{predictable} exogenous actions. Our approach allows
the agents to directly predict the natural actions that will occur
and automatically include them in a planned execution.


\subsection{Legality of the Semantics}

Let a \emph{MIndiGolog Theory of Action} be a theory of action in
the situation calculus enhanced with time, true concurrency and natural
actions, augmented with the predicates $Trans$ and $Final$ from
IndiGolog, modified according to equations (\ref{eqn:trans_conc_new}),
(\ref{eqn:trans_prim_new}) and (\ref{eqn:trans_cond_new}). All legal
executions of a MIndiGolog program derived from such a theory of action
produce legal situations.

\begin{lemma}
Let $\mathcal{D}$ be a MIndiGolog theory of action. Then:\label{thm:trans_legal}\[
\mathcal{D}\models\forall s,s',\delta,\delta'.Legal(s)\wedge Trans(\delta,s,\delta',s')\rightarrow Legal(s')\]

\end{lemma}
\begin{proof}
By induction on the structure of $\delta$. That the theorem holds
for the modified $Trans$ clauses of equations (\ref{eqn:trans_conc_new}),
(\ref{eqn:trans_prim_new}) and (\ref{eqn:trans_cond_new}) is straightforward,
and no other clause constructs new situation terms. 
\end{proof}
\begin{thm}
Let $\mathcal{D}$ be a MIndiGolog theory of action. Then: \[
\mathcal{D}\models\forall s',\delta,\delta'\,.\, Trans*(\delta,S_{0},\delta',s')\rightarrow Legal(s')\]
 Thus, all legal executions of a MIndiGolog program produce legal
situations. 
\end{thm}
\begin{proof}
From lemma \ref{thm:trans_legal}, the legality of $S_{0}$, and the
properties of transitive closure. 
\end{proof}
Like standard IndiGolog, MIndiGolog will only call for actions to
be performed if it is actually possible to perform them. MIndiGolog
has the added advantage of effortlessly incorporating any natural
actions that may occur.


\section{Example Execution}

The effect of our new semantics can be seen in figure \ref{cap:example_trace},
which shows one possible legal execution of the $MakeSalad$ program
using the new semantics of MIndiGolog%
\footnote{A Prolog implementation of the MIndiGolog semantics, from which this
trace was obtained, is available at http://www.csse.unimelb.edu.au/\textasciitilde{}rfk/golog/%
} in a domain with thee agents. Note the occurrence of several actions
within each \textbf{do} statement, demonstrating the integration of
true concurrency into the language. Note also the incorporation of
natural $endTask$ actions into the program, and the explicit occurrence
time for each action. (in this trace the occurrence times have been
instantiated to their earliest possible value).

%
\begin{figure*}[t]
 \centering \subfigure[One possible execution of the $MakeSalad$ program with three agents,
using MIndiGolog. Note the occurrence of multiple
actions within each do statement.]{ \label{cap:example_trace} \framebox{\makebox[1\textwidth]{%
\begin{minipage}[t][1\totalheight]{0.5\textwidth}%
{\small \verbatiminput{listings/mindigolog/output_makeSalad_1.txt}}%
\end{minipage}%
\begin{minipage}[t][1\totalheight]{0.45\textwidth}%
{\small \verbatiminput{listings/mindigolog/output_makeSalad_2.txt}}%
\end{minipage}}} } \subfigure[One possible execution of the $MakeSalad$ program with three
agents, using IndiGolog. Note that only one agent acts at
each step.]{ \label{cap:indigolog_trace} \framebox{\makebox[1\textwidth]{%
\begin{minipage}[t][1\totalheight]{0.5\textwidth}%
{\small \verbatiminput{listings/mindigolog/output_makeSalad_ig_1.txt}}%
\end{minipage}%
\begin{minipage}[t][1\totalheight]{0.45\textwidth}%
{\small \verbatiminput{listings/mindigolog/output_makeSalad_ig_2.txt}}%
\end{minipage}}} }

\caption{Example executions of the $MakeSalad$ program}

\end{figure*}


For comparison purposes, we added a temporal element and inserted
explicit $endTask$ actions into $MakeSalad$ to allow a legal execution
to be found using the standard IndiGolog semantics. Such an execution
is shown in figure \ref{cap:indigolog_trace}. Note that the lack
of true concurrency means only a single agent can act at each step,
leaving the other two agents completely idle. The ability to reduce
idle time by performing actions concurrency is clearly an advantage
for distributed problem solving applications.


\section{Distributed Execution Planning}

As the existing languages based on Golog have demonstrated, an interpreter
can be conveniently constructed using a logic programming language
such a Prolog. We have followed the style of \citet{giacomo00congolog,giacomo99indigolog}
to build interpreters for our language in both Prolog and Oz \citet{vanroy03mozart_logic},
a multi-paradigm programming language with strong support for distributed
computing.

One powerful feature of Oz is the ability to use several networked
computers to search for solutions to a logic program in parallel.
Since the task of planning a MIndiGolog execution is encoded as a
logic program, this immediately allows a team of agents to distribute
the execution planning workload. Below we briefly summarize our implementation.

MIndiGolog programs are represented in Oz as record terms (which are
similar to Prolog data terms) with the name of the record representing
an operator and its features the arguments. Actions are also encoded
as records. As in Prolog, uppercase terms in Oz represent variables.
For example, the program:\[
\pi(agt)\left[acquire(agt,Bowl);acquire(agt,Knife)\right]\]


is represented as follows:

{\small \verbatiminput{listings/goloz_ex_prog.oz} }{\small \par}

The predicates $Trans$ and $Final$ have a straightforward encoding
as Oz procedures, using the \textbf{case} statement to encode each
individual clause using pattern matching, and the \textbf{choice}
statement to explicitly introduce choice points:

{\small \verbatiminput{listings/goloz_trans.oz} }{\small \par}

{\small \verbatiminput{listings/goloz_final.oz} }{\small \par}

A procedure $Do(\delta,s,s')\equiv Trans*(\delta,s,\delta',s')\wedge Final(\delta',s')$
is defined that determines a legal execution \textbf{\emph{Sp}} for
a given program \textbf{\emph{D}}:

{\small \verbatiminput{listings/goloz_do.oz} }{\small \par}

This \textbf{\emph{Do}} procedure can then be passed to the parallel
search functionality to plan a program execution. Here {}``agent1''
and {}``agent2'' are the DNS names of agents in the team, and {}``Goloz''
is an Oz functor (basically, a portable piece of code) that exports
the \textbf{\emph{Do}} procedure defined above:

{\small \verbatiminput{listings/goloz_do_parallel.oz} }{\small \par}

When this code is run, it will utilize the computational resources
of both agents to plan a legal execution of a given MIndiGolog program.
This requires that the same information is available to each agent,
which restricts the technique to fully-observable domains. We are
currently developing an algorithm for cooperative execution of MIndiGolog
programs that utilises such distribution of the planning workload.


\section{Discussion\label{sec:MIndiGolog:Discussion}}


\subsection{Achievements}

That Golog shows promise for multi-agent teams is evidenced by the
success of \citet{Ferrein2005readylog} with a RoboCup soccer team
executing a shared Golog program. However, the semantics of their
Golog variant {}``ReadyLog'' remain largely single-agent and do
not address concerns such as: the possibility of performing actions
concurrently and the coordination of concurrent actions; differing
knowledge or beliefs between team members; sharing the computational
workload of planning; and predicting the behavior of teams members
and the environment in the face of many concurrently-executing tasks.
MIndiGolog overcomes some of these limitations, while our ongoing
work on cooperative execution will address the others.

As stated earlier, there has been much promising work on distributed
problem solving systems using the Hierarchical Task Networks formalism
(\citet{tambe97flexible_teamwork,decker95designing_coordination,grosz99planning_together},
among others). We believe high-level program execution to have several
clear advantages over HTN, in particular the ready availability of
controlled nondeterminism. Combined with familiar programming constructs
such as loops and if-then-else, it provides a very powerful formalism
for expressing complex behaviors and tasks \citet{Gabaldon02htn_in_golog,son00htn_golog}.
Golog also benefits from a logic of action rich enough to capture
many challenging aspects of multi-agent domains (such as time and
concurrency) while remaining computationally feasible.

Our work integrates several important extensions to the situation
calculus and Golog to better model the dynamics of multi-agent teams.
Specifically, MIndiGolog combines true and interleaved concurrency,
an explicit account of time, and seamless integration of natural actions.
It defines legal executions of high-level programs that are suitable
for cooperative execution by a multi-agent team.

Since the semantics of MIndiGolog are based on first-order logic,
existing techniques for distributed logic programming can be used
to share the execution-planning workload between agents. In fully-observable
domains, the parallel search capabilities of Oz can be used directly.

This paper thus represents a first step towards providing the advantages
of Golog (such as controlled nondeterminism, powerful programming
constructs, and a rich logic of action) for task specification for
multi-agent teams. While significant work remains to be done to produce
a full distributed problem solving system, our current implementations
of MIndiGolog in Prolog and Oz, particularly combined with distributed
logic programming techniques, have already proven useful for programming
the behaviour of multi-agent teams in fully observable domains.


\subsection{Limitations}

TODO: limitations

\begin{itemize}
\item All actions known to all agents. 
\item No support for sensing actions (e.g. is cake cooked?) 
\item DKnows = Knows = EKnows = CKnows 
\item Basically signpost the rest of the thesis 
\end{itemize}
