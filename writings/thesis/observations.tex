

\chapter{Observations and Views}

\label{ch:observations}

This chapter develops an explicit formalisation of the local perspective
of each agent, representing it using concrete terms in the logic,
so that we can approach reasoning and planning in asynchronous domains
in a systematic way.

Existing work on multi-agent domains in the situation calculus has
left this agent-local perspective largely implicit; for example, it
is customary to introduce different kinds of sensing or communication
actions by directly modifying the axioms that define the dynamics
of knowledge. We choose instead to reify the local perspective of
each agent by explicitly talking about what it has observed, independent
of how this information will be used by the rest of the action theory.

The basic idea is as follows: each occurrence of an action results
in an agent making a set of \emph{observations}. Every situation then
corresponds to a local \emph{view} for that agent: the sequence of
all its observations, excluding cases where the set of observations
was empty. These form agent-local analogues to standard action and
situation terms, which represent the global state of the world. Allowing
the set of observations to be empty lets us model truly asynchronous
domains, in which an agent's local view is not always updated when
the state of the world is changed.

By having views as explicit terms in the logic, we are then in a position
to ensure that agents only reason and act based on their local information.
Having factored out the precise details of each agent's local view,
we can develop reasoning techniques and tools that can be applied
in a variety of different domains, rather than depending on any particular
details of how actions are perceived by each agent.

To demonstrate the appeal of this decoupling, we show how a variety
of domain dynamics can be modelled using our approach. The techniques
we develop using this foundation of observations and views - our planning
formalism using joint executions, our account of knowledge with hidden
actions, our regression rule for common knowledge - can be used unmodified
in any of these domains.

The chapter begins with additional background material in Section
\ref{sec:Observations:Background}, discussing how an agent's local
information is traditionally modelled in the situation calculus. We
then define the notion of observations and views in Section \ref{sec:Observations:Definitions},
and discuss how they can be specified within the structure of a basic
action theory. Section \ref{sec:Observations:Axiomatising-simple}
discusses how some common domain dynamics can be modelled using observations
and views, while Section \ref{sec:Observations:Axiomatising-extended}
demonstrates the power and flexibility of the approach by axiomatising
more complex domains that have not previously been approached in the
situation calculus. Section \ref{sec:Observations:Reasoning} discusses
how agents can reason using their local view, and Section \ref{sec:Observations:Discussion}
concludes with some general discussion.


\section{Background\label{sec:Observations:Background}}

In many single-agent applications of the situation calculus, there
is no need to consider the local perspective of the agent -- the agent
has complete knowledge and is the only entity acting in the world,
its local information is precisely equivalent to the global information
captured by the current situation term.

If the agent has incomplete knowledge about the state of the world,
it may need to perform \emph{sensing actions} to obtain additional
information. To represent such actions, a new sort \noun{Result }is
added to $\Lsit$, along with an action description function $SR(a,s)=r$
that specifies the result returned by each action. The agent's local
perspective on the world is then given by a \emph{history}, a sequence
of $a\#r$ pairs giving each action performed and its corresponding
sensing result.

When sensing actions are used in IndiGolog \citep{giacomo99indigolog},
the agent must plan its execution using this history rather than a
raw situation term. This is accomplished without any modifications
to the underlying theory of action, by handling the history as a purely
meta-level structure and modifying the way queries are posed.

First, a pair of macros are defined to convert a history into proper
sentences of the situation calculus that capture the information it
contains. The macro $\mathbf{end}$ gives the situation term corresponding
to a history, while the macro $\mathbf{Sensed}$ produces a formula
asserting that each action produced the given sensing result. Let
$\epsilon$ be the empty history, then the definitions are:\begin{alignat*}{1}
\mathbf{end}[\epsilon] & \isdef S_{0}\\
\mathbf{end}[h\cdot(a\#r)] & \isdef do(a,\mathbf{end}[h])\\
\mathbf{Sensed}[\epsilon] & \isdef\top\\
\mathbf{Sensed}[h\cdot(a\#r)] & \isdef\mathbf{Sensed}[h]\wedge SR(a,\mathbf{end}[h])=r\end{alignat*}


Then, instead of asking whether a query holds at the current situation
$s$:

\[
\Dt\,\models\,\phi[s]\]


The agent asks whether the query holds given its current history $h$:\[
\Dt\,\models\mathbf{Sensed}[h]\,\rightarrow\,\phi[\mathbf{end}[h]]\]


This approach works well for a single agent, but we are aware of no
works extending this meta-level handling of histories to the multi-agent
case.\\


While the meta-level approach of \citep{giacomo99indigolog} allows
an agent to reason based on its local perspective, it is cumbersome
for reasoning \emph{about} that local perspective. To determine whether
an agent \emph{knows} that a formula holds in a given situation, we
must explicitly specify the agent's history of sensing results for
that situation, which may not be available until run-time. This meta-level
approach is not suitable for rich epistemic reasoning, where we may
want to reason offline about what the agent (or a group of agents)
will or will not know after a series actions.

This kind of reasoning requires an explicit representation of an agent's
knowledge, as described in Section \ref{sec:Background:Epistemic}.
We will review such epistemic reasoning in more detail in Chapter
\ref{ch:knowledge}, where we extend existing approaches to handle
asynchronous domains based on the formalism developed in this chapter.
For now we briefly discuss its use in axiomatising the local perspective
of each agent.

The effect of sensing results on an agent's knowledge is axiomatised
in \citep{scherl03sc_knowledge} by directly specifying it in the
successor state axiom for the knowledge fluent $K$. Agents discard
situations that do not agree with the obtained sensing result:\[
K(agt,do(a,s'),do(a,s))\,\equiv\, K(agt,s',s)\wedge SR(a,s')=SR(a,s)\]
 As a variety of richer domains have been modelled on top of this
formalism, their particular accounts of the local information available
to each agent have been specified by progressively modifying this
successor state axiom.

For example, when multiple agents are introduced, the successor state
axiom for $K$ is modified to ensure that an agent knows the results
produced by its own actions, but not by the actions of others \citep{shapiro98specifying_ma_systems}.
When communication actions are introduced, the successor state axiom
for $K$ is modified to ensure that the agent's knowledge is updated
to include the communicated information \citep{shapiro01casl_feat_inter,shapiro07sc_goal_change}.
When private actions are introduced, the successor state axiom for
knowledge is modified to account for sequences of private actions
\citep{Lesperance99sitcalc_approach}. When concurrent actions and
time are introduced, the successor state axiom for $K$ is modified
to ensure that the agent knows how much time has passed since the
last action, while not inadvertently learning the real value of the
current time unless this was already known \citep{scherl03conc_knowledge}.

In these works there is no explicit representation of the local perspective
of each agent -- rather, the information each agent receives from
an action is specified only in terms of its effect on the agent's
knowledge. The formalism developed in this chapter will allow us to
decouple the dynamics of knowledge from the specific details of how
each action affects the agent's local perspective. As we will show
in Chapter \ref{ch:knowledge}, this can produce a much more general
and robust formalism for knowledge.

The observation-based approach will also allow us to work directly
with the local information available to each agent without needing
to make explicit statements about knowledge. For example, when planning
the cooperative execution of a task, we can formulate a reactive plan
in which each agent can act based directly on its local observations
without having to introspective reason about what it knows.\\


A related approach to ours is the work by \citet{pinto98sc_observations}
on axiomatising narratives in the situation calculus. Here the term
{}``observation'' is used in a more general sense to mean some partial
information about the state of the world, such as an action occurring
or a fluent holding at a particular time. These are asserted using
predicates such as $occurs(a,t)$ and $holds(F,t)$, and situations
are defined as \emph{proper} if they respect the asserted $occurs$
and $holds$ facts. Although the focus of \citep{pinto98sc_observations}
is on reasoning from a single omniscient perspective, it could easily
be extended to reason about the local perspective of multiple agents.

The crucial difference between this work and the approach presented
in this chapter is that we provide an explicit axiomatisation not
just of \emph{observations} but of \emph{observability.} We provide
a complete account of what each agent would observe if any particular
action occurred in any particular state of the world. By virtue of
not having made particular observations, agents in our formalism can
conclude that certain actions cannot have occurred. By contrast, the
use of $occurs$ and $holds$ in \citep{pinto98sc_observations} specifies
only what \emph{must} have happened, not what \emph{cannot} have happened.
This distinction will play an important role for effective reasoning
for our formalism.


\section{Definitions\label{sec:Observations:Definitions}}

In this section we formally define an explicit representation of the
local information available to each agent, and do so in a manner that
is independent of how that information will eventually be used. Our
approach can be seen as a generalisation of the history-based approach
of \citep{giacomo99indigolog}, explicitly representing the information
available to each agent. However, we encode this information as terms
in the logic rather than in the meta-level reasoning machinery. This
allows us to use this explicit local perspective in more general ways.

We begin by defining the notion of an \emph{observation}, which is
fundamental to the entire approach\emph{.} At the simplest level,
this is an internal notification that an agent receives when some
action has occurred.

\begin{defnL}
[{Observations}] An observation is a notification event received
by an agent, making it aware of some change in the state of the world.
When an agent receives such a notification, we say that it {}``observed'',
{}``made'' or {}``perceived'' that observation. 
\end{defnL}
Since {}``observation'' is quite a loaded term, it is worth re-iterating
this point: our observations are instantaneous \emph{events} generated
internally by each agent in response to actions occurring in the world.
We make no commitment as to how these events are generated, preferring
a clean delineation between the task of observing change and the dynamics
of knowledge update based on those observations.

Since the state of the world may only change in response to some action,
observations can only be made as a result of some action. For simplicity
we assume that agents perceive observations instantaneously, i.e.
in the same instant as the actions that cause them, but see Section
\ref{sec:Observations:Delayed} for a suggestion on how delayed observations
can be modelled.

To make this idea concrete, let us introduce an additional sort \noun{Observation}
to the language $\Lsit$, for the moment without any particular commitment
towards what this sort will contain. We then add an action description
function of the following form to $\Dt_{ad}$:\[
Obs(agt,c,s)=o\]
 This function returns a set of observations, and should be read as
{}``when actions $c$ are performed in situation $s$, agent $agt$
will perceive observations $o$''. Using a set of observations allows
an agent to perceive any number of observations in response to an
action occurrence -- perhaps several observations, perhaps none. When
$Obs(agt,c,s)$ is empty, the agent makes no observations and the
actions $c$ are completely hidden.

The concept of a \emph{view} follows naturally - it is the sequence
of all the observations made by an agent as the world has evolved.

\begin{defnL}
[{Views}] An agent's view in a given situation $\mathrm{s}$
is the corresponding sequence of observations made by the agent as
a result of each action in $\mathrm{s}$, excluding those actions
for which no observations were made. \label{defn:Observations:View} 
\end{defnL}
We introduce another sort \noun{View} consisting of sequences of sets
of observations, with $\epsilon$ being the empty sequence and the
functional fluent $View$ giving the history of observations associated
with a particular situation. Since these definitions will not change
from one domain to another, they are added to the foundational axioms:\begin{align}
Init(s)\,\rightarrow & \, View(agt,s)=\epsilon\nonumber \\
Obs(agt,c,s)=\{\}\,\rightarrow & \, View(agt,do(c,s))=View(agt,s)\nonumber \\
Obs(agt,c,s)\neq\{\}\,\rightarrow & \, View(agt,do(c,s))=Obs(agt,c,s)\cdot View(agt,s)\label{eq:view_defn}\end{align}


Observations and views can be seen as localised analogues of actions
and situations respectively. An action is a global event that causes
the state of the world to change, while an observation is an internal
event that causes an agent's knowledge of the state of the world to
change. Similarly, a situation represents a complete, global history
of all the actions that have occurred in the world, while a view is
an agent's local history of all the observations it has made. The
situation is an omniscient perspective on the world, the view a local
perspective. This distinction will be fundamental to the new techniques
we develop throughout this thesis.

To provide a global account of the results returned by each action,
we can define a \emph{history} in a similar way to IndiGolog, but
represented explicitly as a term in the language. First, we define
the \emph{outcome} of an action as a mapping from agents to the observations
they made. To represent this mapping we use a set of \noun{Agent\#Observation}
pairs:

\begin{defnL}
[{Outcomes}] The outcome of an action $c$ is the set of
$agt\#Obs(agt,c)$ pairs generated by that action:\[
Outcome(c,s)=y\,\equiv\,\forall agt,o:\, agt\#o\in y\,\equiv\, Obs(agt,c,s)=o\]

\end{defnL}
Then we can build the global history of a situation as a sequence
of outcomes:

\begin{defnL}
[{Histories}] The history of a situation $s$ is the sequence
of outcomes corresponding to each action in $s$:\begin{gather*}
Init(s)\,\rightarrow\, History(s)=\epsilon\\
History(do(c,s))=h\,\equiv\, h=Outcome(c,s)\cdot History(s)\end{gather*}

\end{defnL}
Histories will be useful for planning the cooperative execution of
a shared task, when agents must explicitly reason about both the global
state of the system and the local perspective of each agent. To this
end, we introduce some suggestive notation for accessing the individual
observations in an outcome:\[
Outcome(c,s)[agt]=o\,\equiv\, agt\#o\in Outcome(c,s)\]


To ensure that these definitions operate in an intuitively correct
way, we also need a simple consistency constraint. Just as the empty
set of actions is assumed to never be legal, so we should assume that
it generates no observations -- clearly the agents cannot observe
anything if no action has taken place. Formally, we impose the following
consistency requirement on basic actions theories containing $Obs$:

\begin{defnL}
[{Observation~Causality~Requirement}] A basic action theory
$\Dt$ specifying the $Obs$ function must ensure that agents do not
perceive observations that are not caused by some action:\[
\Dt\,\models\,\forall agt,s:\, Obs(agt,\{\},s)=\{\}\]

\end{defnL}
The key distinguishing features of our approach is that the agent's
view excludes cases where $Obs(agt,c,s)$ is empty, so the agent may
not have enough information to determine how many actions have been
performed in the world. As discussed in Chapter \ref{ch:intro}, this
property is fundamental to modelling truly asynchronous domains. Mirroring
the terminology of \citep{vanBentham06tree_of_knowledge}, we can
explicitly define what it means for a domain to be \emph{synchronous}
in the situation calculus.

\begin{defnL}
[{Synchronous~Action~Theory}] A basic action theory $\Dt$
is synchronous if every agent observes something whenever an action
occurs:\label{def:Synchronous-Action-Theory}\[
\Dt\,\models\,\forall agt,c,s:\, Legal(c,s)\,\rightarrow\, Obs(agt,c,s)\neq\{\}\]

\end{defnL}
As we shall see, such domains make reasoning from the local perspective
of an agent significantly easier, as they do not need to consider
arbitrarily-long sequences of hidden actions. Before proceeding with
some example definitions of $Obs$, let us briefly foreshadow how
observations and views will be used in the coming chapters.

In Chapter \ref{ch:jointexec}, we will define a partially-ordered
branching action structure to be generated as the output of the MIndiGolog
execution planning process. This structure, called a \emph{joint execution,}
represents many possible situations that are legal executions of the
program. Since agents can only be expected to act based on their local
information, we will require that if $s$ and $s'$ are two situations
that could be reached while performing a joint execution, and $View(agt,s)=View(agt,s')$,
then the agent's next action in both situations must be the same.
Moreover, if the joint execution requires an agent to execute some
action $a_{2}$ after another action $a_{1}$, we will require that
$Obs(agt,a_{1},s)$ is not empty, so that it will have some local
observation to trigger the performance of $a_{2}$. These restrictions
ensure that the joint execution can actually be carried out by the
agents.

In Chapter \ref{ch:knowledge}, we formalise the intuition that an
agent's knowledge should be based only on its local information. So
if the agent believes that the world might be in situation $s$, then
it must also consider possible any other situation $s'$ such that
$View(agt,s)=View(agt,s')$. By decoupling the axiomatisation of knowledge
from the specific details of how each action affects the agent's local
information, we develop a very general formalism that can be applied
without modification in a wide variety of domains.


\section{Axiomatising Observations\label{sec:Observations:Axiomatising-simple}}

We now show how observations and views can be used to model a variety
of common domain dynamics from the situation calculus literature.
We argue that these axiomatisations intuitively capture the {}``correct''
information in each case, but defer a formal comparison between our
approach and existing axiomatisations until we have developed our
theory of knowledge in Chapter \ref{ch:knowledge}.


\subsection{Public Actions}

By far the most common assumption about the observability of actions
is that {}``all actions are public'', which can be rephrased as
{}``when an action occurs, all agents will observe that action''.
Letting the \noun{Observation }sort contain \noun{Action }terms, this
can be captured using the following axiom in the definition of $Obs$:\begin{equation}
a\in Obs(agt,c,s)\,\equiv\, a\in c\label{eq:Observations:ObsStd1}\end{equation}


When sensing actions are included, it is typically assumed that only
the performing agent has access to the sensing results. This can be
modelled by extending the \noun{Observation} sort to contain \noun{Action\#Result}
pairs, and including the following in the definition for $Obs$:\begin{equation}
a\#r\in Obs(agt,c,s)\,\equiv\, a\in c\wedge actor(a)=agt\wedge SR(a,s)=r\label{eq:Observations:ObsStd2}\end{equation}


Note that since $Obs$ is an action description function, technically
we must specify it using a single axiom as described in Section \ref{sec:Background:SC:Axioms}.
For the sake of clarity we specify these two cases independently,
and assume that the final axiom defining $Obs$ takes the completion
of the individual cases in the standard way.

It should be clear that these definitions capture the intuition behind
this most common model of action observability. When we develop our
new axiomatisation of knowledge in Chapter \ref{ch:knowledge}, we
will demonstrate that these definitions provide an equivalent account
to the standard knowledge axioms of \citet{scherl03sc_knowledge}.

This approach clearly leads to synchronous domains, since it entails that
$Obs(agt,c,s)$ if and only if $c={}$.


\subsection{Private Actions}

Another common model for action observability is to assume that {}``all
actions are private'', which can be rephrased as {}``when an action
occurs, only the performing agent will observe it''. This can be
modelled by simple dropping the public-observability axiom from equation
\ref{eq:Observations:ObsStd1}, leaving the following definition of
$Obs$:\[
a\#r\in Obs(agt,c,s)\,\equiv\, a\in c\wedge actor(a)=agt\wedge SR(a,s)=r\]


As noted in \citep{Lesperance99sitcalc_approach}, this approach means
that agents need to consider arbitrarily-long sequences of hidden
actions which may or may not have occurred, and thus forego regression
as an effective reasoning technique. By explicitly formalising this
situation, we will be in a position provide the first formal account
of effective reasoning in such asynchronous domains.


\subsection{Guarded Sensing Actions\label{sec:Observations:Guarded-Sensing}}

While the standard approach to sensing actions has the result $SR(a,s)$
returned unconditionally, it is also possible to model actions that
return sensing information only when some additional conditions hold
in the environment, for example as used by \citet{Petrick06thesis}.
These can be modelled in our framework by adding the guard conditions
$\Psi$ to the definition of $Obs$:\[
a\#r\in Obs(agt,c,s)\,\equiv\, a\in c\wedge actor(a)=agt\wedge SR(a,s)=r\wedge\Psi(a,s)\]


For example, an action $sense_{\phi,\psi}$ that senses the truth
of some formula $\phi$ when the guard condition $\psi$ is true would
require the following to be entailed by the definition:\begin{gather*}
sense_{\phi,\psi}\#T\in Obs(agt,c,s)\,\equiv\, sense_{\phi,\psi}\in c\,\wedge\,\phi(s)\wedge\psi(s)\\
sense_{\phi,\psi}\#F\in Obs(agt,c,s)\,\equiv\, sense_{\phi,\psi}\in c\,\wedge\,\neg\phi(s)\wedge\psi(s)\\
sense_{\phi,\psi}\in Obs(agt,c,s)\,\equiv\, sense_{\phi,\psi}\in c\,\wedge\,\neg\psi(s)\end{gather*}


As noted in \citep{Petrick06thesis}, guarded sensing actions can
create difficulties when axiomatising the $K$ fluent. The approach
we develop in Chapter \ref{ch:knowledge} will show that by explicitly
representing the information returned by the action, rather than defining
it implicitly in the axiom for $K$, these difficulties are avoided.


\subsection{Speech Acts}

Communication in the situation calculus is traditionally modelled
using explicit communicative actions or {}``speech acts'' \citep{shapiro01casl_feat_inter,shapiro07sc_goal_change}.
These actions are axiomatised as per standard actions, but special-case
handling is introduced in the axioms for knowledge in order to model
their communicative effects.

Instantaneous communication is modelled using actions such as $inform$,
where $inform(agt_{s},agt_{r},\phi)$ means the sender $agt_{s}$
informs the receiver $agt_{r}$ of the truth of some formula $\phi$.
If we assert that only truthful speech acts are allowed, and all actions
are publicly observable, then this requires no further axiomatisation:\[
Poss(inform(agt_{s},agt_{r},\phi),s)\,\equiv\,\phi[s]\]


The $inform$ action will be included in each agent's observations
whenever it occurs, from which the agent can conclude that it was
possible and thus that the contained formula holds in the world.

However, this simple approach can lead to third-party agents being
aware of what was communicated, which is often not desirable. In \citep{shapiro01casl_feat_inter}
encrypted speech acts are introduced to overcome this limitation,
ensuring that only the intended recipient of a message is able to
access its contents by performing a special \emph{decrypt} action.
While it would be straightforward to copy this approach in our formalism,
the problem it was introduced to solve no longer exists; we can directly
limit the accessibility of the message contents to the receiving agent
without introducing another action:\begin{gather*}
inform(s,r)\in Obs(agt,c,s)\,\equiv\,\exists m:\, inform(s,r,m)\in c\\
inform(s,r,m)\in Obs(agt,c,s)\,\equiv\, inform(s,r,m)\in c\wedge\left(agt=r\vee agt=s\right)\end{gather*}


Here all agents will observe that the communication occurred, but
only the sender and recipient will know the contents of the message.

Non-instantaneous communication can be modelled using a message queue
for each agent with separate $send$ and $check$ actions \citep{Lesperance99sitcalc_approach}.
The $send$ action adds a message to the queue, while the $check$
action returns the details of pending messages as its sensing result.
Since this approach uses the standard sensing-result machinery, it
requires no special axiomatisation in our framework.


\section{New Axiomatisations\label{sec:Observations:Axiomatising-extended}}

From the above examples, it should be clear that our formalism can
capture the information available to each agent under a variety of
domain dynamics already modelled in the situation calculus. We now
demonstrate some new axiomatisations of domains that have not previously
been explored in the situation calculus.


\subsection{Explicit Observability Axioms\label{sec:Observations:CanObs}}

Our approach offers a straightforward way to explore the middle ground
between the two extremes of {}``public actions'' and {}``private
actions'' discussed in the previous section. To axiomatise general
\emph{partial observability} of actions, we introduce a new action
description predicate $CanObs(agt,a,s)$ that defines the conditions
under which agent $agt$ would observe action $a$ being performed
in situation $s$. If $CanObs(agt,a,s)$ is false, then that action
will be hidden. We can then define $Obs$ as follows:\[
a\in Obs(agt,c,s)\equiv a\in c\wedge CanObs(agt,a,s)\]


This permits a great deal of flexibility in the axiomatisation. Consider
a domain in which the agents inhabit several different rooms, and
are aware of all the actions performed in the same room as themselves:\[
CanObs(agt,a,s)\equiv InSameRoom(agt,actor(a),s)\]


It is also possible to allow partial observability of sensing results
using an analogous predicate $CanSense(agt,a,s)$ and the following
definition of $Obs$:\begin{multline*}
a\in Obs(agt,c,s)\equiv a\in c\wedge CanObs(agt,a,s)\wedge\neg CanSense(agt,a,s)\\
\shoveleft{a\#r\in Obs(agt,c,s)\equiv a\in c\wedge SR(a,s)=r}\\
\wedge CanObs(agt,a,s)\wedge CanSense(agt,a,s)\end{multline*}


For example, consider an agent waiting for a train who activates a
speaker to determine when it will arrive. The results of this sensing
action would provide information to any other agent within earshot:\[
CanSense(agt,activateSpeaker(agt_{2}),s)\equiv CloseToSpeaker(agt)\]


We feel that this formulation provides a good balance between simplicity
and expressiveness; it allows the observability of actions to vary
according to the state of the world, but provides agents with a complete
description of each action that they are capable of observing.


\subsection{Observability Interaction}

Reasoning about observability of concurrent actions raises the potential
for \emph{observability interaction}, in which some actions produce
different observations when they are performed concurrently with another
action. Like the precondition interaction problem for $Poss$ discussed
in Section \ref{sec:Background:Concurrent-Actions}, we assume that
the axiom defining $Obs$ contains the appropriate logic to handle
such interaction. A simple axiomatisation might have actions being
{}``masked'' by the co-occurrence of another action, and would appear
like so:\[
a\in Obs(agt,c,s)\equiv a\in c\wedge CanObs(agt,a,s)\wedge\neg\exists a'\in c:\, Masks(a',a,s)\]


The important point is that, given an explicit account of the local
perspective of each agent, such interaction can be axiomatised independently
of the rest of the action theory.


\subsection{Observing the Effects of Actions}

In many domains it would be infeasible for an agent to observe all
of the details of a particular action when it occurs, but it may observe
some of the effects of that action. For example, suppose that an agent
monitors the state of a light in its environment, such that it notices
it changing from dark to light. While it knows that \emph{some} action
must have occurred to produce that effect, it may not be sure precisely
what action took place (e.g. precisely \emph{who} turned on the light).
This can be modelled by further extending the \noun{Observation} sort
to contain a special {}``effect observation'' term $lightCameOn$,
and axiomatising like so:\[
lightCameOn\in Obs(agt,c,s)\equiv\neg lightIsOn(s)\wedge\exists agt':\, turnLightOn(agt')\in c\]


When the light is switched on, each agent's observation set will contain
the term $lightCameOn$, and they will be able to deduce that this
change has occurred without necessarily knowing the specific action
responsible for the change. This is similar to the {}``fluent change''
actions proposed by \citet{degiacomo98execution_monitoring}, but
embedded in the theory itself rather than generated by the agent when
it discovers that it must update its beliefs.


\subsection{Delayed Communication\label{sec:Observations:Delayed}}

Delayed communication can be modelled using separate $send$ and $recv$
actions. However, unlike the use of explicit communication channels
discussed on the previous section, we do not want the receiving agent
to have to poll the message queue. Rather, the $recv$ action should
occur automatically some time after the $send$ action.

This is easily modelled by making $recv$ a natural action. The $send/recv$
pair can then be axiomatised mirroring the standard account of long-running
tasks in the situation calculus. A fluent $PendMsg(s,r,m,t)$ indicates
that some message is pending and will be delivered at time $t$. We
have:\begin{gather*}
natural(recv(agt_{s},agt_{r},m))\\
send(agt_{s},agt_{r},m)\in Obs(agt,c,s)\equiv send(agt_{s},agt_{r},m)\in c\wedge agt=agt_{s}\\
recv(agt_{s},agt_{r},m)\in Obs(agt,c,s)\equiv recv(agt_{s},agt_{r},m)\in c\wedge agt=agt_{r}\\
Poss(recv(agt_{s},agt_{r},m)\#t,s)\equiv PendMsg(agt_{s},agt_{r},m,t,s)\end{gather*}
 \begin{multline*}
PendMsg(s,r,m,t_{m},do(c\#t,s))\,\equiv send(s,r,m)\in c\wedge t_{m}=t+delay(s,r,m,s)\\
\vee PendMsg(s,s,m,t_{m},s)\wedge\left(recv(s,r,m)\not\in c\vee t\neq t_{m}\right)\end{multline*}


A $send$ action thus causes the message to become pending, with its
delivery time determined by the functional fluent $delay$. Once the
delay time has elapsed, the natural action $recv$ will be triggered
and the message delivered. The $send$ and $recv$ messages are observed
only by the sender and receiver respectively. If the agents have incomplete
knowledge about the $delay$ function, this can easily model domains
in which the message delay is unpredictable or even unbounded.


\section{Reasoning From Observations\label{sec:Observations:Reasoning}}

With these definitions in place, we can now give a principled account
of what it means for an agent to reason using its local information.
Recall that in the single-agent setting of IndiGolog \citep{giacomo99indigolog}
a pair of macros is used to construct a query of the following form
given the agent's current history $h$:\[
\Dt\,\models\,\mathbf{Sensed}[h]\,\rightarrow\,\phi[\mathbf{end}[h]]\]


This depends crucially on the assumption that all actions are publicly
observable, so that the macro $\mathbf{end}$ can construct the precise
situation term corresponding to a given history. The resulting query
is in a form that can be answered effectively using the standard regression
operator.

We can pose a similar query using the definition of a global history
in our framework, but we must account for the fact that many situations
could potentially have the same history, requiring quantification
over situation terms:\[
\Dt\,\models\,\forall s:\, History(s)=h\,\rightarrow\,\phi[s]\]


Due to the use of quantification, this query cannot be answered using
regression. However, we know from the definition of $History$ that
all situation terms sharing a given history will contain the same
number of actions. Since the agent knows that $n$ have been performed
in a given history, it can pose the query in a form suitable for effective
reasoning via regression:\[
\Dt\,\models\,\forall c_{1},\dots,c_{n}:\, History(do([c_{1},\dots,c_{n}],S_{0}))=h\,\rightarrow\phi[do([c_{1},\dots,c_{n}],S_{0})]\]


Agents could use such a query to reason about some hypothetical future
state of the world, for example for the purposes of planning. They
could not, however, answer queries about the current state of their
world in this manner, since they will not have the current history.
 They must reason based on only their current view $v$:\[
\Dt\,\models\,\forall s:\, View(agt,s)=v\,\rightarrow\,\phi[s]\]


Here the agent encounters a problem -- this is a much more difficult
query in general. Since it cannot tell how many actions have occurred
based on its local view, it cannot re-write this query into a form
suitable for regression. The agent must instead perform second-order
theorem proving, using the induction axiom over situations, in order
to reason based on its local view. The situation calculus currently
offers no tools for effective reasoning about such queries.

However, suppose the domain is synchronous. Then combining Definition
\ref{def:Synchronous-Action-Theory} with equation \eqref{eq:view_defn},
we can prove that all situations matching a given view will contain
the same number of actions. The agent can therefore construct a query
like the following to reason about the the world using standard regression
techniques:\[
\Dt\,\models\,\forall c_{1},\dots,c_{n}:\, View(agt,do([c_{1},\dots,c_{n}],S_{0}))=v\,\rightarrow\phi[do([c_{1},\dots,c_{n}],S_{0})]\]


Thus in synchronous domains, existing reasoning techniques of the
situation calculus can be used by an agent to reason from its own
local perspective in much the same way as in the single-agent case.
In asynchronous domains, the induction axiom is required and no effective
reasoning procedures currently exist. This restriction, more than
any other, has limited the use of the situation calculus for modelling
asynchronous multi-agent domains.

As we shall see in the coming chapters, for some applications we can
permit the agents to reason using the global history rather than their
local observations. For richer epistemic reasoning, we will require
a technique capable of handling the inductive component of this reasoning.


\section{Discussion\label{sec:Observations:Discussion}}

In this chapter we have constructed an explicit representation of
the local perspective of each agent, in terms of \emph{observations}
and \emph{views}. This terminology has been deliberately chosen to
mirror that used in other formalisms where representing this local
perspective is the norm, such as \citep{parikh85dist_knowledge,halpern90knowledge_distrib}.
As the examples in Sections \ref{sec:Observations:Axiomatising-simple}
and \ref{sec:Observations:Axiomatising-extended} have demonstrated,
this approach is able to capture a very wide variety of domain dynamics
in a flexible way.

Some of our axioms in Sections \ref{sec:Observations:Axiomatising-simple}
and \ref{sec:Observations:Axiomatising-extended} may seem rather
ad-hoc, but we claim they are no more or less ad-hoc than the many
adjustments made to the axioms defining the knowledge fluent $K$
to accommodate different kinds of information-producing action \citep{shapiro98specifying_ma_systems,Lesperance99sitcalc_approach,shapiro01casl_feat_inter,Petrick06thesis,shapiro07sc_goal_change}.
The difference is that these adjustments can now be made separately
from the rest of the theory, rather than in the fundamental axiom
for reasoning about knowledge. This makes our formalism significantly
more elaboration tolerant, a point we will return to in Chapter \ref{ch:knowledge}.
It also means that for certain applications, we can reason about an
agent's local view without the overhead of performing explicit epistemic
reasoning.

Of course, we also pay a price for this extra expressive power: representational complexity.
The theory of action must contain an explicit axiomatisation of the
\noun{Observation} sort and of our new $Obs$ function. There is something
of a tradition in the situation calculus of doing as much as possible
at the meta-level, adding to the theory itself only when necessary.
As we will demonstrate in the remainder of this thesis, the advantages
provided by our explicit representation of each agent's local perspective
outweigh the added complexity it introduces to the theory of action.

From the perspective of the rest of the thesis, the key contribution
of this chapter is to provide a uniform representation formalism.
The domain-specific observability dynamics can now be specified independently
from the rest of the theory. By {}``factoring out'' the details
in this way, we are in a position to construct formalisms and reasoning
techniques that do not make any assumptions about action observability.
In particular, we can represent and reason about asynchronous domains.

