

\chapter{Observations and Views}

\label{ch:observations}

This chapter develops an explicit representation of the local perspective
of each agent, so that we can approach reasoning and planning in asynchronous
domains in a systematic way. Existing work on multi-agent domains
in the situation calculus has left this the local perspective largely
implicit; for example, it is customary to introduce different kinds
of sensing or communicating actions by directly modify the axioms
that define the dynamics of knowledge \citep{scherl03sc_knowledge,Lesperance99sitcalc_approach}.
We choose instead to reify the local perspective of each agent by
explicitly talking about what it has observed.

The basic idea is as follows: each occurrence of an action results
in an agent making a set of \emph{observations}. Every situation then
corresponds to a local \emph{view} for that agent: the sequence of
all its observations, excluding cases where the set of observations
was empty. These form agent-local analogues to standard action and
situation terms, which represent the global state of the world. Allowing
the set of observations to be empty lets us model truly asynchronous
domains, in which an agent's local view is not always updated when
the state of the world is changed.

By having views as explicit terms in the logic, we are then in a position
to ensure that agents only reason and act based on their local information.
Having factored out the precise details of each agent's local view,
we can develop reasoning techniques and tools that can be applied
in a variety of different domains. The techniques developed in subsequent
chapters can operate directly on observations and views, rather than
depending on any particular details of how each action is perceived
by each agent.

To demonstrate the appeal of this decoupling, we show how a variety
of domain dynamics can be modelled using our approach. The techniques
we develop using this foundation of observations and views - our planning
formalism using joint executions, our account of knowledge with hidden
actions, our regression rule for common knowledge - can be used unmodified
in any of these domains.

The chapter begins with some additional background material in Section
\ref{sec:Observations:Background}, discussing how the local information
available to each agent is traditionally modelled in the situation
calculus. We then define the notion of observations and views in Section
\ref{sec:Observations:Definitions}, and discuss how they can be specified
within the structure of a basic action theory. Section \ref{sec:Observations:Axiomatising-simple}
discusses how some common domain dynamics can be modelled using observations
and views, while Section \ref{sec:Observations:Axiomatising-extended}
demonstrates the power and flexibility of the approach by axiomatising
more complex domains that have not previously been approached in the
situation calculus. We conclude with some discussion in Section \ref{sec:Observations:Discussion}.


\section{Background\label{sec:Observations:Background}}

In many single-agent applications of the situation calculus, there
is no need to consider the local perspective of the agent -- since
the agent has complete knowledge and is the only entity acting in
the world, its local information is precisely equivalent to the global
information about the state of the world. Its local perspective is
simply the current situation term.

If the agent has incomplete knowledge about the state of the world,
it may need to perform \emph{sensing actions} to obtain additional
information. To represent such actions, a new sort \noun{Result }is
added to $\mathcal{L}_{sitcalc}$, along with an action description
function $SR(a,s)=r$ axiomatising the result returned by each action.
The agent's local perspective on the world is then given by a \emph{history},
a sequence of $a\#r$ pairs giving each action performed and its corresponding
sensing result.

When sensing actions are used in IndiGolog \citep{giacomo99indigolog},
the agent must plan its execution using this history rather than a
raw situation term. This is accomplished without any modifications
to the underlying theory of action, by handling the history as a purely
meta-level structure and modifying the way queries are posed.

First, a pair of macros are defined to convert a history into proper
sentences of the situation calculus that capture the information it
contains. The macro $\mathbf{end}$ gives the situation term corresponding
to a history, while the macro $\mathbf{Sensed}$ produces a formula
asserting that each action produced the given sensing result. Let
$\epsilon$ be the empty history, then the definitions are:\begin{alignat*}{1}
\mathbf{end}[\epsilon] & \isdef S_{0}\\
\mathbf{end}[h\cdot(a\#r)] & \isdef do(a,\mathbf{end}[h])\\
\mathbf{Sensed}[\epsilon] & \isdef\top\\
\mathbf{Sensed}[h\cdot(a\#r)] & \isdef\mathbf{Sensed}[h]\wedge SR(a,\mathbf{end}[h])=r\end{alignat*}


Then, instead of asking whether a query holds at the current situation
$s$:

\[
\Dt\,\models\,\phi[s]\]


The agent asks whether the query holds given its current history $h$:\[
\Dt\,\models\mathbf{Sensed}[h]\,\rightarrow\,\phi[\mathbf{end}[h]]\]


This approach works well for a single agent, but we are aware of no
works extending this meta-level handling of histories to the multi-agent
case.

While the history-based approach of \citep{giacomo99indigolog} allows
an agent to reason based on its local perspective, it is cumbersome
for reasoning \emph{about} that local perspective. To determine whether
an agent \emph{knows} that a formula holds in a given situation, we
must explicitly specify the agent's history of sensing results for
that situation, which may not be available until run-time. This meta-level
approach is not suitable for rich epistemic reasoning, where we may
want to reason offline about what the agent (or a group of agents)
will or will not know after a series actions.

This kind of reasoning requires an explicit representation of an agent's
knowledge, which has been developed for the situation calculus in
\citep{scherl03sc_knowledge}. We will review such epistemic reasoning
in more detail in chapter \ref{ch:knowledge}, where we extend their
approach to handle asynchronous domains based on the formalism developed
in this chapter. For now we briefly discuss its use in axiomatising
the local perspective of each agent.

The effect of sensing results on an agent's knowledge is axiomatised
in \citep{scherl03sc_knowledge} by directly specifying it in the
successor state axiom for a special {}``knowledge fluent'' $K$.
As a variety of richer domains have been modelled using this formalism,
their particular accounts of the local information available to each
agent have been specified by progressively modifying this successor
state axiom.

For example, when multiple agents are introduced, the successor state
axiom for $K$ is modified to ensure that an agent knows the results
produced by its own actions, but not by the actions of others \citep{shapiro98specifying_ma_systems}.
When communication actions are introduced, the successor state axiom
for $K$ is modified to ensure that the agent's knowledge is updated
to include the communicated information \citep{shapiro98specifying_ma_systems}.
When hidden actions are introduced, the successor state axiom for
knowledge is modified to account for sequences of hidden actions \citep{Lesperance99sitcalc_approach}.
When concurrent actions and time are introduced, the successor state
axiom for $K$ is modified to ensure that the agent knows how much
time has passed since the last action, while ensuring that it does
not inadvertently learn the real value of the current time unless
this was already known \citep{scherl03conc_knowledge}.

In these works there is no explicit representation of the local perspective
of each agent -- rather, the information each agent receives from
an action is specified only in terms of its effect on the agent's
knowledge. The formalism developed in this chapter will allow us to
decouple the dynamics of knowledge from the specific details of how
each action affects the agent's local perspective. As we will show
in chapter \ref{ch:knowledge}, this can produce a much more general
and robust formalism for knowledge.\\


In this chapter we explicitly represent the local information available
to each agent, and do so in a manner that is independent of how that
information will be used. Our approach can be seen as a generalisation
of the history-based approach of \citep{giacomo99indigolog}, explicitly
representing the information available to each agent. However, we
encode this information as terms in the logic rather than in the meta-level
reasoning machinery. This allows us to use this explicit local perspective
in more general ways, such as to perform explicit epistemic reasoning.


\section{Definitions\label{sec:Observations:Definitions}}

We begin by defining the notion of an \emph{observation}, which is
fundamental to the entire approach\emph{.} At the simpled level, this
is an internal notification that an agent receives when some action
has occurred.

\begin{defnL}
[{Observations}] An observation is a notification event received
by an agent, making it aware of some change in the state of the world.
When an agent receives such a notification, we say that it {}``observed'',
{}``made'' or {}``perceived'' that observation. 
\end{defnL}
Since {}``observation'' is quite a loaded term, it is worth re-iterating
this point: observations are instantaneous \emph{events} generated
internally by each agent in response to actions occurring in the world.
We make no commitment as to how these events are generated, preferring
a clean delineation between the task of observing change and the dynamics
of knowledge update based on those observations.

The state of the world may only change in response to some action,
so observations can only be made as a result of some action. For simplicity
we assume that agents perceive observations instantaneously, i.e.
in the same instant as the actions that cause them.

To make this idea concrete, let us introduce an additional sort \noun{Observation}
to the language $\mathcal{L}_{sitcalc}$, for the moment without any
particular commitment towards what this sort will contain. We then
add an action description function of the following form to $\Dt_{ad}$:\[
Obs(agt,c,s)=o\]
 This function returns a set of observations, and should be read as
{}``when actions $c$ are performed in situation $s$, agent $agt$
will perceive $o$''. Using a set of observations allows an agent
to perceive any number of observations in response to an action occurrence
- perhaps several observations, perhaps none. When $Obs(agt,c,s)$
is the empty set, the agent makes no observations and the actions
$c$ are completely hidden.

The concept of a \emph{view} follows naturally - it is the sequence
of all the observations made by an agent as the world has evolved.

\begin{defnL}
[{Views}] An agent's view in a given situation $\mathrm{s}$
is the corresponding sequence of observations made by the agent as
a result of each action in $\mathrm{s}$, excluding those actions
for which no observations were made. 
\end{defnL}
We introduce another sort \noun{View} consisting of sequences of sets
of observations, with $\epsilon$ being the empty sequence and the
functional fluent $View$ giving the history of observations associated
with a particular situation. Since these definitions will not change
from one domain to another, they are added to the foundational axioms
$\Sigma$:\begin{align}
Init(s)\,\rightarrow & \, View(agt,s)=\epsilon\nonumber \\
Obs(agt,c,s)=\{\}\,\rightarrow & \, View(agt,do(c,s))=View(agt,s)\nonumber \\
Obs(agt,c,s)\neq\{\}\,\rightarrow & \, View(agt,do(c,s))=Obs(agt,c,s)\cdot View(agt,s)\label{eq:view_defn}\end{align}


Observations and views can be seen as localised analogues of actions
and situations respectively. An action is a global event that causes
the state of the world to change, while an observation is an internal
event that causes an agent's knowledge of the state of the world to
change. Similarly, a situation represents a complete, global history
of all the actions that have occurred in the world, while a view is
an agent's local history of all the observations it has made. The
situation is an omniscient perspective on the world, the view a local
perspective. This distinction will be fundamental to the new techniques
we develop throughout this thesis.\\


The key point that distinguishes our approach is that the agent's
view excludes cases where $Obs(agt,c,s)$ is empty. Thus, the agent
may not have enough information to determine how many actions have
been performed in the world. As discussed in the Chapter \ref{ch:intro},
this property is fundamental to modelling truly asynchronous domains.
Mirroring the terminology of \citep{vanBentham06tree_of_knowledge},
we can explicitly define what it means for a domain to be \emph{synchronous}
in the situation calculus.

\begin{defnL}
[{Synchronous~Action~Theory:}] A basic action theory $\Dt$
is synchronous if every agent observes something whenever an action
occurs:\[
\Dt\,\models\,\forall agt,c,s:\, Legal(c,s)\,\rightarrow Obs(agt,c,s)\neq\{\}\]

\end{defnL}
Before proceeding with some example axiomatisations of the $Obs$
function, let us briefly foreshadow how observations and views will
be used in the coming chapters.

In chapter \ref{ch:jointexec}, we will define a partially-ordered
branching action structure to be generated as the output of the MIndiGolog
execution planning process. This structure, called a \emph{joint execution,}
represents many possible situations that are legal executions of the
program. Since agents can only be expected to act based on their local
information, we will require that if $s$ and $s'$ are two situations
that could be reached following a joint execution, and $View(agt,s)=View(agt,s')$,
then the agent's next action in both situations must be the same.
Moreover, if the joint execution requires an agent to execute some
action $a_{2}$ after another action $a_{1}$, we will require that
$Obs(agt,a_{1},s)$ is not empty. These restrictions ensure that the
joint execution can actually be performed by the agents.

In chapter \ref{ch:knowledge}, we formalise the notion that an agent's
knowledge should be based only on its local information. So if the
agent believes that the world might be in situation $s$, then it
must also consider possible any other situation $s'$ such that $View(agt,s)=View(agt,s')$.
By decoupling the axiomatisation of knowledge from the specific details
of how each action affects the agent's local information, we develop
a very general formalism that can be applied without modification
in a wide variety of domains.

TODO: Reference and discuss \citep{pinto98sc_observations} in great
detail


\section{Axiomatising Observations\label{sec:Observations:Axiomatising-simple}}

We now proceed to show how observations and views can be used to model
a variety of common cases from the situation calculus literature.
We argue that these axiomatisations intuitively capture the {}``right''
information in each case, but defer a formal comparison between our
approach and existing axiomatisations until we have developed our
theory of knowledge in Chapter \ref{ch:knowledge}.


\subsection{Public Actions}

By far the most common assumption about the observability of actions
is that {}``all actions are public'', which can be rephrased as
{}``when an action occurs, all agents will observe that action''.
Letting the \noun{Observation }sort contain \noun{Action }terms, this
can be captured using the following axiom in the definition of $Obs$:\begin{equation}
a\in Obs(agt,c,s)\,\equiv\, a\in c\label{eq:Observations:ObsStd1}\end{equation}


When sensing actions are included, it is typically assumed that only
the performing agent has access to the sensing results. This can be
modelled by extending the \noun{Observation} sort to contain \noun{Action\#Result}
pairs, and including the following in the definition for $Obs$:\begin{equation}
a\#r\in Obs(agt,c,s)\,\equiv\, a\in c\wedge actor(a)=agt\wedge SR(a,s)=r\label{eq:Observations:ObsStd2}\end{equation}


It should be clear that these definitions capture the intuition behind
this most common model of action observability. When we develop our
new axiomatisation of knowledge in Chapter \ref{ch:knowledge}, we
will demonstrate that these definitions provide an equivalent account
to the standard knowledge axioms defined in \citep{scherl03sc_knowledge}.

This approach clearly leads to synchronous domains, since it entails
the following:\[
Poss(c,s)\rightarrow Obs(agt,c,s)\neq\{\}\]



\subsection{Private Actions}

Another common model for action observability is to assume that {}``all
actions are private'', which can be rephrased as {}``when an action
occurs, only the performing agent will observe it''. This can be
modelled using the following axiom for $Obs$:\[
a\#r\in Obs(agt,c,s)\,\equiv\, a\in c\wedge actor(a)=agt\wedge SR(a,s)=r\]


As noted in \citep{Lesperance99sitcalc_approach}, this approach means
that agents need to consider arbitrarily-long sequences of hidden
actions which may or may not have occurred. By explicitly formalising
this situation, we are in a position provide the first formal account
of effective reasoning in such asynchronous domains.


\subsection{Communicative Actions}

Communication in the situation calculus is traditionally modelled
using explicit communicative actions or {}``speech acts''. {[}TODO:
refs] These actions are axiomatised as per standard actions, but special-case
handling is introduced in the axioms for knowledge in order to model
their communicative effects.

Instantaneous communication is modelled using actions such as $inform(agt_{s},agt_{r},\phi)$,
where the sender $agt_{s}$ informs the receiver $agt_{r}$ of the
truth of some formula $\phi$. If we assert that only truthful speech
acts are allowed, this requires no further axiomatisation:\[
Poss(inform(agt_{s},agt_{r},\phi),s)\,\equiv\,\phi[s]\]


However, actions are public, this simple approach can lead to other
agents being aware of what was communicated. In \citep{shapiro01casl_feat_inter}
encrypted speech acts are introduced, to ensure that only the intended
recipient of a message is able to access its contents by performing
a special \emph{decrypt} action. While it would be straightforward
to copy this approach in our formalism, it is also unnecessary - we
can directly limit the accessibility of the message contents to the
receiving agent:\begin{gather*}
inform(s,r)\in Obs(agt,c,s)\,\equiv\,\exists m:\, inform(s,r,m)\in c\\
inform(s,r,m)\in Obs(agt,c,s)\,\equiv\, inform(s,r,m)\in c\wedge\left(agt=r\vee agt=s\right)\end{gather*}


Non-instantaneous communication can be modelled using a message queue
for each agent {[}TODO:ref] with separate $send$ and $recieve$ actions.
The $send$ action adds a message to the queue, while the $recieve$
action returns the contents of a pending message as its sensing result.
Since this approach used the standard sensing-result machinery, it
requires no special axiomatisation in our framework.


\section{New Axiomatisations\label{sec:Observations:Axiomatising-extended}}

From the above examples, it should be clear that our formalism can
capture the information available to each agent under a variety of
domain dynamics already modelled in the situation calculus. We now
demonstrate some new axiomatisations of domains that have not previously
been explored in the situation calculus.


\subsection{Explicit Observability Axioms}

Our approach offers a straightforward way to explore the middle ground
between the two extremes of {}``public actions'' and {}``private
actions'' discussed in the previous section. To axiomatise general
\emph{partial observability} of actions, we introduce a new action
description predicate $CanObs(agt,a,s)$ that defined the conditions
under which agent $agt$ would observe action $a$ being performed
in situation $s$. If $CanObs(agt,a,s)$ is false, then that action
will be hidden. We then use the following in the definition of $Obs$:\[
a\in Obs(agt,c,s)\equiv a\in c\wedge CanObs(agt,a,s)\]


This permits a great deal of flexibility in the axiomatisation. Consider
a domain in which the agents inhabit several different rooms, and
are aware of all the actions performed in the same room as themselves:\[
CanObs(agt,a,s)\equiv InSameRoom(agt,actor(a),s)\]


It is also possible to allow partial observability of sensing results
using an analogous predicate $CanSense(agt,a,s)$ and the following
definition of $Obs$:\begin{multline*}
a\in Obs(agt,c,s)\equiv a\in c\wedge CanObs(agt,a,s)\wedge\neg CanSense(agt,a,s)\\
\shoveleft{a\#r\in Obs(agt,c,s)\equiv a\in c\wedge SR(a,s)=r}\\
\wedge CanObs(agt,a,s)\wedge CanSense(agt,a,s)\end{multline*}


For example, consider an agent waiting for a train who activates a
speaker to determine when it will arrive. The results of this sensing
action would provide information to any other agent within earshot:\[
CanSense(agt,activateSpeaker(agt_{2}),s)\equiv CloseToSpeaker(agt)\]


We feel that this formulation provides a good balance between simplicity
and expressiveness; it allows the observability of actions to vary
according to the state of the world, but provides agents with a complete
description of each action that they are capable of observing.


\subsection{Observability Interaction}

Reasoning about observability of concurrent actions raises the potential
for \emph{observability interaction}, in which some actions produce
different observations when they are performed concurrently with another
action. Like the precondition interaction problem for $Poss$ discussed
briefly in Section \ref{sec:Background:Concurrent-Actions}, we assume
that the axiom defining $Obs$ contains the appropriate logic to handle
such interaction. A simple axiomatisation might have actions being
{}``masked'' by the co-occurrence of another action, and would appear
like so:\[
a\in Obs(agt,c,s)\equiv a\in c\wedge CanObs(agt,a,s)\wedge\neg\exists a'\in c:\, Masks(a',a,s)\]


The important point is that, given an explicit account of the local
perspective of each agent, such interaction can be axiomatised independently
of the rest of the action theory.


\subsection{Observing the Effects of Actions}

In many domains it would be infeasible for an agent to observe all
of the details of a particular action when it occurs, but it may observe
some of the effects of that action. For example, suppose that an agent
monitors the state of a light in its environment, such that it notices
it changing from dark to light. While it knows that \emph{some} action
must have occurred to produce that effect, it may not be sure precisely
what action took place (e.g. precisely \emph{who} turned on the light).
This can be modelled by further extending the \noun{Observation} sort
to contain a special {}``effect observation'' term $lightCameOn$,
and axiomatising like so:\[
lightCameOn\in Obs(agt,c,s)\equiv\neg lightIsOn(s)\wedge\exists agt':\, turnLightOn(agt')\in c\]


When the light is switched on, each agent's observation set will contain
the term $lightCameOn$, and they will be able to deduce that this
change has occurred without necessarily knowing the specific action
responsible for the change. In Chapter \ref{ch:knowledge} we will
show that an agent can indeed conclude that it {}``knows that the
light came on'' when it perceives a $listCameOn$ observation.


\subsection{Delayed Communication}

Delayed communication can be modelled using separate $send$ and $recv$
actions. However, unlike the use of explicit communication channels
discussed on the previous section, we do not want the receiving agent
to constantly poll the message queue. Rather, the $recv$ action should
occur automatically some time after the $send$ action.

This is easily modelled by making $recv$ a natural action. The $send/recv$
pair then be axiomatised mirroring the standard account of long-running
tasks in the situation calculus. A fluent $PendMsg(s,r,m,t)$ indicates
that some message is pending and will be delivered at time $t$. We
have:\begin{gather*}
natural(recv(agt_{s},agt_{r},m))\\
send(agt_{s},agt_{r},m)\in Obs(agt,c,s)\equiv send(agt_{s},agt_{r},m)\in c\wedge agt=agt_{s}\\
recv(agt_{s},agt_{r},m)\in Obs(agt,c,s)\equiv recv(agt_{s},agt_{r},m)\in c\wedge agt=agt_{r}\\
Poss(recv(agt_{s},agt_{r},m)\#t,s)\equiv PendMsg(agt_{s},agt_{r},m,t,s)\end{gather*}
 \begin{multline*}
PendMsg(s,r,m,t_{m},do(c\#t,s))\,\equiv send(s,r,m)\in c\wedge t_{m}=t+delay(s,r,m,s)\\
\vee PendMsg(s,s,m,t_{m},s)\wedge\left(recv(s,r,m)\not\in c\vee t\neq t_{m}\right)\end{multline*}


A $send$ action thus causes the message to become pending, with its
delivery time determined by the functional fluent $delay$. Once the
delay time has elapsed, the natural action $recv$ will be triggered
and the message delivered. The $send$ and $recv$ messages are observed
only by the sender and receiver respectively. If the agents have incomplete
knowledge about the $delay$ function, this can easily model domains
in which the message delay is unpredictable or even unbounded.


\section{Discussion\label{sec:Observations:Discussion}}

In this chapter we have constructed an explicit representation of
the local perspective of each agent, in terms of \emph{observations}
and \emph{views}. This terminology has been deliberately chosen to
mirror that uses in other formalisms where representing this local
perspective is the norm, such as the classic treatise of \citet{halpern90knowledge_distrib}.
As the examples in Sections \ref{sec:Observations:Axiomatising-simple}
and \ref{sec:Observations:Axiomatising-extended} have demonstrated,
our formalism is able to capture a very wide variety of domain dynamics.

From the perspective of the rest of the thesis, the key contribution
of this chapter is in providing a uniform representation. This allows
the specific observability dynamics of a domain to be specified independently
from the rest of the theory. By {}``factoring out'' the details
in this way, we are now in a position to construct formalisms and
reasoning techniques that do not depend on any assumptions about the
domain dynamics. In particular, we need not assume that the domain
is synchronous.

We conclude this chapter by linking our formalism back to the reasoning
techniques based on single-agent histories that are used to handle
sensing results in IndiGolog \citep{giacomo99indigolog}. Recall that
this approach uses macros to construct a query of the following form:

\[
\Dt\,\models\,\mathbf{Sensed}[h]\,\rightarrow\,\phi[\mathbf{end}[h]]\]


This depends crucially on the assumption that the domain is synchronous
and publicly observable, so that the macro $\mathbf{end}$ can construct
the precise situation term corresponding to the history $h$. Since
in our framework there may be many situations sharing the same view,
the equivalent query using the agent's view $v$ would be:\[
\Dt\,\models\,\forall s:\, View(agt,s)=v\,\rightarrow\,\phi[s]\]


This is a much more difficult query in general, since it is not in
a form that can be handled by the regression operator. Indeed, since
it universally quantifies over situation terms, answering this query
requires the use of the second-order induction axiom for situations.

However, suppose the domain is synchronous, so that the agent knows
how many actions have occurred. In this case it can construct a query
that does not quantify over situation terms. Let $n$ be the number
of items in the agent's current view $v$, then it can query the current
state of the world like so:\[
\Dt\,\models\,\forall c_{1},\dots,c_{n}:\, View(agt,do([c_{1},\dots,c_{n}],S_{0}))=v\,\rightarrow\phi[do([c_{1},\dots,c_{n}],S_{0})]\]


This query is in a form amenable to effective reasoning by the regression
operator.

Thus in synchronous domains, existing reasoning techniques of the
situation calculus can be used by an agent to reason from its own
local perspective. In asynchronous domains, the induction axiom is
required and no effective reasoning procedures currently exist.

TODO: what can we \emph{not} represent using this formalism? E.g.
universal sensing actions, where we learn whether F(x) for all x?

