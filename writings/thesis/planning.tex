\chapter{Planning and Program Execution}\label{ch:planning}
%\minitoc

\section{What is Planning?}
\begin{itemize}
\item In this setting, it is the process of \emph{resolving nondeterminism}
\item Planning should produce a program for the agent to follow, that does
not itself require deliberation to perform. \cite{levesque96what_is_planning,giacomo04sem_delib_indigolog}
\item Must account for outcomes of sensing actions - some sort of branch/case statement
\item Must be epistemically feasible (show formalisation from \cite{giacomo04sem_delib_indigolog})
\end{itemize}

\section{What is Team Planning?}
\begin{itemize}
\item Produce a program for each agent
\item Programs must include the necessary coordination actions/conditions
\item We will need explicit syntax for synchronisation - "waiting for an observation"
\item Any interleaving of the programs must be a valid legal execution
\item similar idea in strips: \cite{boutilier01partialorder_conc}
\end{itemize}

\section{Resolving Non-Determinism}
\begin{itemize}
\item Hard Problem: removing only as much non-determinism as necessary
\item Easier Problem: re-inserting non-determinism where possible
\item Idea:  plan by first \emph{completely} resolving nondeterminism, but
maintain auxiliary information that allows actions to be nondeterministically 
interleaved where a legal execution is maintained.
\end{itemize}

\section{Distributed Planning}
\begin{itemize}
\item To obtain coordination without communication, must use common knowledge.
\item Each agent can reason about CKnows(IsPlan($\delta$,$\delta_N$)) to come up with a common plan, which they can then execute
\item Or, they can communicate to share the planning workload by using distributed knowledge DKnows(IsPlan($\delta$,$\delta_N$))
\end{itemize}

\section{Online Execution: Knowing what to do}
\begin{itemize}
\item Agents may be unsure exactly what remains to be executed
\item Encoding the program as a fluent, so it can be dealt with epistemically
\item S.S.A. based on transition semantics
\item Using natural actions to transition test conditions
\end{itemize}

\section{Coordination using Social Laws}
\begin{itemize}
\item Use an ordering to specify which actions are prefered
\item Define the notion of a "safe" action
\item Perform a joint action when its safety is common knowledge among actors
\item Communicating to determine safety of actions
\item ?? Ways to increase efficiency of this approach ??
\end{itemize}

\section{Managing the Search Operator}
\begin{itemize}
\item beginplan() and endplan() actions observable by all
\item planning in the face of change - "restart" actions incorporated into planning procedure
\end{itemize}

