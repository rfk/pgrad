
%+++DEFINE CLASS & PACKAGES+++
\documentclass[a4paper,british,twoside,12pt,openany]{book}
\usepackage[Lenny]{fncychap}
\usepackage{graphics}
\usepackage{graphicx}
\usepackage{subfigure}
\usepackage{amsmath}
\usepackage{amssymb}
\usepackage{amsthm}
\usepackage{verbatim}
\usepackage{algorithm}
\usepackage{algorithmic}
\usepackage{tikz}
\usepackage{bibentry}
\usepackage[numbers]{natbib}
\usepackage{setspace}   %enables use of \doublespace \onehalfspace \singlespace
\usepackage{fancyhdr}   %customises header and allows chap, sec or page number

%===DEFINE PAGE===
\topmargin      -1.8cm              %top of page margin
\headheight     14.5pt              %running head height
\parskip        2mm                 %paragraph spacing
\oddsidemargin  12mm                %left margin of odd page =28mm +oddsidemargin
\evensidemargin 1mm                 %left margin of even page =29mm +evensidemargin
\textheight     250mm               %height of text box on page
\textwidth      145mm               %width of text across the page =textwidth -5mm
\parindent      8mm                 %paragraph indentation width
\pagestyle{fancy}                   
    \fancyhead{}                    %Reset fancy fields
    \fancyhead[LO]{R. F. Kelly}      %Other position is [RE]
    \fancyhead[RO]{\rightmark}      %rightmark is the section name
    \fancyhead[LE]{\leftmark}       %leftmark is the chapter name
    \renewcommand{\headrulewidth}{0.5mm}
        \headsep=   1.3cm             %Text separation from header
    \renewcommand{\footrulewidth}{0.5mm}
    \cfoot{\thepage}
        \footskip=  1.3cm             %Text separation from footer

\renewcommand\bibname{References} 

%===Redefine default figure placement to h (for 'here') ===
\makeatletter
\def\fps@figure{htbp} %default is {tbp}
\makeatother

%===DEFINE PERSONALISED COMMANDS===
% Instead of typing out commonly used long commands/items
% that are long you can define your own commands to use.
% Notes: (1)\ensuremath is used in case you want to use the
% command while already in a math environment
% e.g. $\theta =30\degr +45\degr$
% (2) Latex wont put a space between this and the next word
% so I put it in manually and define another command name with
% a s suffix when I specifically don't need a space e.g. at the
% end of a sentence. This can cause issues though as Latex
% wont break the words joined by the tilde apart for line splitting.
%\newcommand{\sq}{\ensuremath{^2~}}
%\newcommand{\sqs}{\ensuremath{^2}}
%\newcommand{\cb}{\ensuremath{^3~}}
%\newcommand{\cbs}{\ensuremath{^3}}

% Pinched from LyX
\newcommand{\noun}[1]{\textsc{#1}}

% Environment Definitions
\newenvironment{proofsketch}{\begin{proof}[Proof Sketch]}{\end{proof}}
\newtheorem{thm}{Theorem}
\newtheorem{lemma}{Lemma}
\newtheorem{prop}{Proposition}
\newtheorem{example}{Example}
\newtheorem{defn}{Definition}
\newtheorem{defnL}[defn]{Definition}

\newtheoremstyle{thmext}{\topsep}{\topsep}{\itshape}{}{\bfseries}{.}{ }{\thmname{#1}\thmnote{ #3}}
\theoremstyle{thmext}
\newtheorem*{thmext}{Theorem}
\theoremstyle{plain}
\theoremstyle{thmext}
\newtheorem*{lemmaext}{Lemma}
\theoremstyle{plain}

