

\chapter{Introduction}

\label{ch:intro}

The situation calculus, along with the programming language Golog
that is built upon it, is a powerful formalism for reasoning about
dynamic worlds and specifying the behaviour of autonomous agents.
It combines a rich language for expressing domain features with techniques
for effective reasoning and a straightforward implementation using
logic programming. But while powerful, the situation calculus currently
suffers some major limitations that make it unsuitable for reasoning
about asynchronous multi-agent domains.

We begin by noting that there is a significant body of work on modelling
and implementing multi-agent systems in the situation calculus, including:
the cognitive agent specification language and verification environment
\citep{shapiro02casl}; theories of coordination and ability \citep{ghaderi07sc_joint_ability};
entries in AI competitions such as robot soccer \citep{Ferrein2005readylog}
and robot rescue \citep{farinelli07team_golog}; reasoning about the
epistemic feasibility of plans \citep{Lesperance01epi_feas_casl};
analysing multi-player games \citep{delgrande01sitcalc_cleudo}; and
the cooperative execution of Golog programs \citep{Ferrein2005readylog,kelly06hlp_dps}.
This literature shows the power and flexibility of the situation calculus,
but also highlights three current weaknesses when working with rich
multi-agent domains.

First, each of these works has been limited to \emph{synchronous}
domains -- domains in which each agent's local perspective on the
world is updated in lock-step with the global perspective available
to the system designer. This restriction ensures that, while an agent
may not know the full state of the world, it will always know precisely
how many actions have been performed. The agents can therefore perform
effective automated reasoning using \emph{regression}, a process that
depends on systematically removing action terms from a query. In many
cases this synchronicity restriction is enforced using a blanket assumption
that that all actions are publicly observable.

Second, the fundamental unit of reasoning, and the output of the Golog
execution planning process, is the \emph{situation}: a complete, ordered
history of all actions that are to be performed. Having to execute
a totally-ordered sequence of actions is far from ideal in a multi-agent
setting, as it requires constant synchronisation between the agents.
While this may be acceptable for systems that are already restricted
to synchronous domains, it does not take advantage of the concurrency
inherent in a multi-agent team. In asynchronous domains the required
synchronisation of actions may be impossible to achieve, and a partially-ordered
representation must be used instead. But there have been no formal
accounts of such a representation in the existing situation calculus
literature.

Third, the situation calculus lacks a comprehensive treatment of group-level
epistemic modalities such as \emph{common knowledge}, which are fundamental
to coordination in multi-agent domains. While common knowledge can
easily be modelled using an explicit second-order axiom, this precludes
the use of regression for effective automated reasoning. In synchronous
domains agents can often coordinate their activities without reasoning
explicitly about common knowledge, but in more general multi-agent
domains the lack of an effective reasoning procedure for common knowledge
can be a serious shortcoming.\\


In each of these cases, the problem is not in modelling asynchronous
multi-agent domains, but in combining such rich domain models with
effective reasoning techniques. The standard technique for effective
reasoning in the situation calculus, and the core of the Golog execution
planning procedure, is \emph{regression.} It operates by systematically
removing action terms from a query until it is in a form that is easy
to answer. The traditional restriction of the situation calculus to
synchronous domains allows all agents to know the number of actions
performed, so they can directly use existing regression-based techniques
for reasoning and planning.

This thesis lifts that restriction by developing reasoning and planning
techniques for \emph{asynchronous} domains -- domains in which the
state of the world can change without updating the local perspective
of each agent. We begin by explicitly representing this local perspective:
whenever an action occurs, each agent makes a corresponding set of
\emph{observations} that are local to that agent. We allow the set
of observations to be empty, formalising the case where asynchronicity
means some actions are completely hidden from some agents.

On top of this seemingly simple extension, we construct a planning
process based on partially-ordered action sequences, a new technique
for effective inductive reasoning to augment standard regression,
and a principled axiomatisation of both individual and group-level
knowledge coupled with an effective reasoning procedure. These contributions
greatly extend the reach of the situation calculus, enabling its use
for specifying, simulating, and implementing more realistic multi-agent
systems.


\section{Motivation}

To make things more concrete, let us introduce two motivating examples
that will be used throughout the thesis, along with an overview of
the challenges they pose to existing techniques from the situation
calculus literature.


\subsection*{Example 1: The Cooking Agents}

Cathy is hosting a dinner party. A brilliant engineer but a mediocre
cook, she has built a team of robotic chefs to help her prepare the
meal, and must now program them to carry out their duties. She needs
a powerful formalism with which the agents can plan their actions,
and a programming language flexible enough to specify the major steps
in each recipe while leaving the precise details of execution for
the agents to plan amongst themselves. Moreover, she wants to specify
the tasks to be performed as a single shared program, and have the
agents automatically distribute the work amongst themselves.\\


The situation calculus offers a compelling approach for this example
domain: each recipe can be represented as a Golog program, and the
team of agents can cooperate to plan and perform the concurrent execution
of these shared programs. However, existing Golog implementations
generate raw situation terms as the output of their planning process.
These are linear, fully-ordered sequences of the actions to be performed,
and require constant synchronisation if the agents are to execute
the plan cooperatively.

In synchronous domains this is not a problem, as the agents will always
know how many steps of execution have already been performed, and
thus what actions should be performed next. But it is still desirable
to take advantage of the natural concurrency present when planning
for a team of agents. The Golog planning process must be modified
to reason about and allow such concurrency.

In asynchronous domains, an agent may not necessarily know how far
execution has progressed, and may thus be unsure when or if to perform
its next action. The planning process should account for this by only
calling for agents to perform an action if they will know, based on
their local information, that the action should be performed. But
the situation calculus currently has no means of representing this
kind of partially-ordered action structure, or of determining whether
it is executable.

Rather than demand that Cathy equip her robots with a global synchronisation
mechanism of some kind, we will extend the situation calculus and
the semantics of Golog execution planning to better handle concurrency
and inter-agent synchronisation in asynchronous, partially-observable
domains.


\subsection*{Example 2: The Party Invitation}

Alice and Bob have heard about Cathy's party but don't know where
it is, and have just received an invitation telling them the location.
Having suffered decades of trouble with their communications, they
like to reason about each other's knowledge by directly observing
each other's actions. If Alice reads the invitation, she will know
the location of the party. But will Bob know that she knows this?
What if he temporarily leaves the room, meaning Alice is able to read
the invitation in secret? And most importantly, can they achieve common
knowledge of the party's location in order to coordinate their travel
plans for the evening?\\


The situation calculus permits an elegant axiomatisation of this domain,
but its standard account of knowledge uses regression for effective
reasoning. Existing regression techniques cannot handle two important
aspects of this example.

First, the standard account of knowledge requires that whenever an
action occurs, all agents \emph{know} that an action has occurred.
In domains such as this example, where some actions may be completely
hidden, each agent's knowledge must account for arbitrarily-long sequences
of hidden actions. This requires a second-order induction axiom which
precludes the use of regression for effective automated reasoning.

Second, the situation calculus lacks a comprehensive treatment of
group-level epistemic modalities such as common knowledge. Common
knowledge is typically introduced using an explicit second-order definition,
again precluding the use of regression for effective automated reasoning.

Rather than demand that Alice and Bob perform open-ended second-order
theorem proving, we will develop a new technique for inductive reasoning
in the situation calculus, use it to formalise an account of knowledge
in the face of hidden actions, and provide a new formulation of group-level
epistemic modalities which can capture a regression rule for common
knowledge.


\section{Contributions}

Our contributions hinge on extending the situation calculus to explicitly
represent the local \emph{observations} made by each agent when an
action is performed. We introduce a function $Obs(agt,a,s)$ that
returns the set of observations made by agent $agt$ when action $a$
is performed in situation $s$. From this, we associate with each
situation $s$ an agent-local view, denoted $View(agt,s)$, which
is the sequence of observations made by the agent in that situation.
Crucially, the agent's view excludes cases where $Obs(agt,a,s)$ is
empty, indicating actions that were completely hidden.

With a principled axiomatisation of the local perspective of each
agent in hand, we construct new formalisms and techniques for effective
use of the situation calculus in asynchronous multi-agent domains.
Specifically, we provide:

\begin{itemize}
\item A new multi-agent variant of Golog, with semantics suitable for cooperative
execution by a team of agents, and a planning procedure that builds
a partially-ordered sequence of actions rather than a raw situation
term. The planning process ensures synchronisation is always possible
based on agent's local views. 
\item An implementation of this Golog variant using the Mozart programming
platform, taking advantage of its support for distributed logic programming
to automatically share the planning workload amongst all members of
the team. 
\item A new procedure for effective inductive reasoning about a restricted
form of query, using a meta-level fixpoint calculation on top of the
standard regression operator. This allows the second-order aspects
of our axiomatisation to be {}``factored out'' of the reasoning
process so that proper regression rules can be formulated. 
\item A new formalism for individual-level knowledge based explicitly on
the agent's local view, with accompanying regression rules that use
our new technique for inductive reasoning to handle arbitrarily-long
sequences of hidden actions. 
\item A comprehensive treatment of group-level epistemic modalities such
as common knowledge, using an epistemic interpretation of dynamic
logic to capture a proper regression rule for common knowledge. 
\end{itemize}
While our contributions significantly extend the capabilities of the
situation calculus, they also stay firmly within the bounds of existing
situation calculus theory and practice. The new concepts are axiomatised
in a way that is compatible with standard basic action theories, as
well as with common extensions such as concurrent actions and continous
time. Our implementation of a Golog execution planner, although written
using a different programming language and supporting much richer
domains, is structurally very simlar to existing Golog systems.

We are therefore confident that our results can be integrated smoothly
with existing theories and systems based on the situation calculus,
providing a solid foundation for extending their reach into both reasoning
about, and planning in, asynchronous multi-agent domains. \newpage{}


\section{Scope}

Before proceeding with the main body of the thesis, it is worth pausing
to clarify the precise scope of our investigation and hopefully avoid
any confusion over details of our terminology and contributions.


\subsection{Asynchronicity}

The term {}``asynchronous domain'' has come to mean slightly different
things in different research fields, and is potentially quite confusing.
Precisely what aspects of the domain are asynchronous? Is it a property
of the environment, the agents themselves, or their communication
mechanisms? Two common definitions of asynchronous domains are those
having unbounded message delivery times \citep{fischer85distributed_consensus},
and those having delayed message delivery and no global clock \citep{halpern90knowledge_distrib}.

The definition used in this thesis is more primitive than these, and
corresponds to the account given in \citep{vanBentham06tree_of_knowledge}:
a domain is \emph{synchronous} if at any time, all agents know precisely
how many actions have been performed. The internal state of each agent
is thus updated in lock-step with the global state of the world. By
contrast, in asynchronous domains agents must allow for potentially
arbitrarily-many hidden actions which may or may not have occurred.

Where a more specific definition of asynchronicity is used, as in
\citep{fischer85distributed_consensus,halpern90knowledge_distrib},
the potential for such hidden actions is implicitly assumed. The situation
calculus typically assumes the exact opposite, limiting itself to
strictly synchronous domains. We therefore argue that modelling richer
notions of asynchronous communication in the situation calculus first
requires a robust account of the kind of foundational asynchronicity
treated in this thesis. While we do not specifically investigate asynchronous
communication in the style of \citep{fischer85distributed_consensus,halpern90knowledge_distrib},
Chapter \ref{ch:observations} does briefly discuss how it can be
modelled within our framework.


\subsection{Common Knowledge}

One of our contributions may, at first glance, seem to suggest that
our definition of asynchronicity is flawed: our work on reasoning
about common knowledge. Based on the famous paper of \citet{halpern90knowledge_distrib},
it has become something of a \char`\"{}grand theorem\char`\"{} in
epistemic reasoning that common knowledge cannot be obtained in asynchronous
domains. So how can this thesis devote an entire chapter to reasoning
about common knowledge in such domains?

The results of \citep{halpern90knowledge_distrib} apply specifically
to obtaining common knowledge via asynchronous \emph{communication}.
They show that what is required for common knowledge is not synchronicity,
per se, but \emph{simultaneity} -- the co-presence of the agents at
the occurrence of a common event. Asynchronous communication is not
a common event, and hence it cannot generate common knowledge.

Our approach, in essence, offers an explicit axiomatisation of this
notion of simultaneity. By reasoning about their observations and
the observations of others, the agents can deduce whether the occurrence
of an action is simultaneously observed and thus whether it can contribute
to common knowledge. Indeed, it is straightforward to model systems
with no simultaneous events in our framework, and correspondingly
straightforward to demonstrate that the agents cannot obtain common
knowledge in such systems. Our contributions thus complement the standard
results on asynchronicity and common knowledge, capturing them in
the situation calculus and providing regression-based reasoning procedures
with which they can be explored.


\subsection{The Situation Calculus}

As the title suggests, this thesis focuses exclusively on reasoning
about multi-agent systems in the situation calculus. There are of
course a range of related formalisms for reasoning about knowledge,
action and change, which we overview briefly in Chapter \ref{ch:background}
but otherwise do not directly consider.

Most closely related to the situation calculus are the fluent calculus
of \citet{thielscher98fluent_calculus} and the event calculus of
\citet{kowalski86event_calculus}. There is also the family of approaches
known as {}``dynamic epistemic logic'' \citep{baltag98pa_ck,vanBenthem06lcc,vanBentham06tree_of_knowledge},
which are based on modal logic and from which we draw some inspiration
for our work in later chapters.

While there have been many attempts to combine the various action
formalisms into a unifying theory of action, there is yet to emerge
a clear standard in this regard. In the meantime, we find the notation
and meta-theory of the situation calculus particularly suitable for
expressing our main ideas, and find the Golog programming language
to be a particularly powerful and flexible approach to specifying
agent behaviour and programming shared tasks.

It is our hope that the strong underlying similarities between the
major action formalisms will allow the ideas presented in this thesis
to find some application or resonance beyond the specifics of the
situation calculus.\newpage{}


\section{Thesis Outline}

The thesis proceeds as follows:

\begin{itemize}
\item Chapter \ref{ch:background} presents the necessary background material
on the situation calculus, Golog, and the Mozart programming system.
More detailed reviews of the relevant literature are included in each
subsequent chapter. 
\item Chapter \ref{ch:mindigolog} introduces \emph{MIndiGolog}, a Golog
variant suitable for planning the cooperative execution of a shared
task\emph{.} We demonstrate an implementation using distributed logic
programming to share the planning workload, and discuss why current
situation calculus techniques limit it to synchronous domains. 
\item Chapter \ref{ch:observations} formalises the notion of a {}``local
perspective'' by reifying the \emph{observations} made by each agent
as the world evolves. We show that this formalism generalises existing
approaches in which this perspective is modelled implicitly. 
\item Chapter \ref{ch:jointexec} develops \emph{joint executions}, a restricted
kind of event structure where synchronisation is based on observations,
and show how to use them as a data structure in planning the execution
of a shared MIndiGolog program. 
\item Chapter \ref{ch:persistence} develops a new technique for effective
inductive reasoning, capable of handling a limited form of query that
universally quantifies over situation terms. Dubbed the \emph{Persistence
Condition} operator, it uses a restricted fixpoint calculation to
replace a second-order induction axiom. 
\item Chapter \ref{ch:knowledge} develops a formalism for individual knowledge
in the face of hidden actions, by specifying an agent's knowledge
in terms of its local view. The persistence condition operator is
used to augment the traditional regression rule for knowledge to account
for arbitrarily-long sequences of hidden actions. 
\item Chapter \ref{ch:cknowledge} introduces common knowledge by using
the syntax of dynamic logic to formulate a more expressive epistemic
language than existing situation calculus theories. We formulate a
regression rule for these complex epistemic modalities and formally
relate them to our account of individual knowledge. 
\item Chapter \ref{ch:conclusion} concludes with a summary of our achievements and our plans for ongoing work based on these results.
\item Appendix \ref{ch:implementation} discusses our various implementations in
more detail. 
\item Appendix \ref{ch:proofs} provides the detailed proofs for those theorems
where only a proof sketch is provided in the main thesis body. 
\end{itemize}
