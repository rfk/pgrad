\chapter{Introduction} %\minitoc


This thesis develops techniques for the cooperative execution of a
shared program by a team of agents in rich multi-agent domain. To
make this more concrete, let us begin with a short motivating example
which will be used throughout the thesis:

\begin{quote}
You have just taken possession of a team of robotic chefs. It is your
task to program the chefs to prepare a delicious meal. 
\end{quote}
\begin{itemize}
\item motivating example: cooking agents 
\item introduce HLP paradigm, briefly argue in its favour 
\item major limitation: focused on single-agent systems 
\item the \char`\"{}MIndiGolog Vision\char`\"{}: cooperative execution of
a HLP 
\item secondary aim: general-purpose tools for HLP in multi-agent domains 
\item achievements in this thesis: 

\begin{itemize}
\item MIndiGolog: HLP semantics suitable for multi-agent domains 
\item New reasoning technique for univsersally quantified queries 
\item Robustly multi-agent account of knowledge, common knowledge 
\item Semantics and techniques for cooperative planning of a legal execution 
\item Techniques for online execution using \char`\"{}social laws\char`\"{}
coordination 
\item Implementations in Oz with distributed execution planning 
\end{itemize}
\end{itemize}

