

\chapter{Introduction}

The situation calculus is a powerful formalism for reasoning about
dynamic worlds and specifying the behavior of autonomous agents. Among
its features are: TODO enumerate sitcalc goodness.

However, the situation calculus also has some limitations that have
restricted its use in multi-agent domains.

\begin{quote}
You have just taken possession of a team of robotic chefs. It is your
task to program the chefs to prepare a delicious meal. 
\end{quote}
\begin{itemize}
\item motivating example: cooking agents 
\item introduce HLP paradigm, briefly argue in its favour 
\item major limitation: focused on single-agent systems 
\item the \char`\"{}MIndiGolog Vision\char`\"{}: cooperative execution of
a HLP 
\item situation calculus needs extending to represent rich MA domains 
\item achievements in this thesis:

\begin{itemize}
\item MIndiGolog: HLP semantics suitable for multi-agent domains 
\item New reasoning technique for univsersally quantified queries 
\item Robustly multi-agent account of knowledge, common knowledge 
\item Semantics and techniques for cooperative planning of a legal execution 
\item Implementations in Oz with distributed execution planning 
\end{itemize}
\end{itemize}
