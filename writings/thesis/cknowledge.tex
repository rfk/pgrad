


\chapter{Complex Epistemic Modalities}

\label{ch:cknowledge}

So far we have constructed a powerful new account of \emph{individual}
knowledge in rich multi-agent domains. To be truly useful in a multi-agent
setting, our formalism must also support reasoning about group-level
knowledge and, in particular, about common knowledge. The primary
motivation for this section is a formal treatment of common knowledge
within our framework, but as we shall see, this requires significant
technical machinery capable of handling much more general epistemic
modalities.


\section{Group-Level Epistemic Modalities}

We briefly review the various group-level epistemic modalities commonly
found in the literature; an excellent overview and discussion can
be found in the work of \citet{halpern90knowledge_distrib}. Let $G$
be a finite group of agents. The basic group-level modality is {}``everyone
knows $\phi$'', which is defined as:\[
\EKnows(G,\phi,s)\isdef\,\forall agt\in G:\,\Knows(agt,\phi,s)\]


Since $G$ is a finite set, this can be written equivalently as a
finite conjunction:\[
\mathbf{EKnows}(G,\phi,s)\,\isdef\,\bigwedge_{agt\in G}\,\mathbf{Knows}(agt,\phi,s)\]


To assert more complete knowledge by members of the group, one can
say {}``everyone knows that everyone knows $\phi$'' by nesting
$\mathbf{EKnows}$ operators. In general:\begin{gather*}
\mathbf{EKnows}^{1}(G,\phi,s)\,\isdef\,\mathbf{EKnows}(G,\phi,s)\\
\mathbf{EKnows}^{n}(G,\phi,s)\,\isdef\,\mathbf{EKnows}(G,\mathbf{EKnows}^{n-1}(G,\phi),s)\end{gather*}


The higher the value of $n$, the stronger an assertion is made about
the knowledge of the group. The strongest group-level modality is
{}``it is common knowledge that $\phi$''. Intuitively this indicates
that everyone knows $\phi$, everyone knows that everyone knows $\phi$,
and so on ad infinitum. Formally, it can be defined as the infinite
conjunction:\[
\mathbf{CKnows}(G,\phi,s)\,\isdef\,\bigwedge_{n\in\mathbb{N}}\mathbf{EKnows}^{n}(G,\phi,s)\]


Equivalently, it can be defined as a fixpoint or transitive closure
of the $\EKnows$ relation. Common knowledge is an extremely powerful
form of knowledge that has deep implications for coordinated group
behaviour. For example, in the famous {}``Coordinated Attack'' problem,
the proof that the generals cannot coordinate an attack depends heavily
on their inability to obtain common knowledge \citep{halpern90knowledge_distrib}.


\section{Reasoning about Common Knowledge}

Existing treatments of common knowledge in the situation calculus
and related literature specify it as the transitive closure of $\EKnows$
using an explicit second-order axiom \citep{davis05fo_ma_theory,ghaderi07sc_joint_ability}.
While logically sound, this approach forgoes the use of regression
as an effective reasoning technique. Indeed, reasoning in such formalisms
requires a second-order theorem prover.

This difficulty in effectively handling common knowledge can be attributed
to a famous expressivity result from the related field of dynamic
epistemic logic:

\begin{quote}
Epistemic logic with actions and common knowledge is more expressive
than epistemic logic with common knowledge alone \citep{baltag98pa_ck} 
\end{quote}
In our terminology: given a formula $\CKnows(G,\phi,do(c,s))$, it
is impossible in general to find an equivalent formula $\CKnows(G,\psi,s)$.
It is thus impossible to formulate a regression rule for common knowledge
using only the $\Knows$ and $\CKnows$ operators.

Given the deep similarities between the situation calculus and dynamic
epistemic logic \citep{vanbentham07ml_sitcalc}, we can be confident
that this expressivity limitation also holds in the situation calculus.
To see why, consider again the successor state axiom for the $K$
fluent, which has the simplified general form:\[
K(do(c',s'),do(c,s))\,\equiv\, K(s',s)\wedge\Phi_{K}(c',s')\]


We could construct an analogous fluent $E$ that captures the $\EKnows$
relation, with a successor state axiom of the general form:\[
E(do(c',s'),do(c,s))\,\equiv\, E(s',s)\wedge\Phi_{E}(c',s')\]


Now consider constructing such a fluent for the $\EKnows^{2}$ relation.
The general form for its successor state axiom would be:\begin{align*}
E^{2}(do(c',s'),do(c,s)) & \equiv\exists c'',s'':\, E(do(c'',s''),do(c,s))\wedge E(do(c',s'),do(c'',s''))\\
 & \Rightarrow\exists c'',s'':\, E(s'',s)\wedge E(s',s'')\wedge\Phi_{E}(c',s')\wedge\Phi_{E}(c'',s'')\\
 & \Rightarrow\exists c'',s'':\, E^{2}(s',s)\wedge\Phi_{E}(c',s')\wedge\Phi_{E}(c'',s'')\end{align*}


This successor state axiom must make assertions not only about $s$
and $s'$, but also about the hypothesised intermediate situation
$s''$. Extending this reasoning, a successor state axiom for common
knowledge would be required make assertions not only about $s$ and
$s'$, but all of the intermediate situations in the transitive closure.
However, the macro $\CKnows$ can only make assertions about the final
situation reached in the transitive closure, not about the path leading
to it. It is thus not expressive enough to formulate a proper regression
rule.\\


To overcome this expressiveness limitation, we follow the recent promising
work of \citep{vanBenthem06lcc}, who use two important new ideas
to produce a regression rule for common knowledge in their logic LCC:

\begin{itemize}
\item Form more expressive epistemic modalities using the syntax of dynamic
logic, interpreted over the epistemic frame of the agents. 
\item Apply regression within the modality as well as to the enclosed formula. 
\end{itemize}
We apply these ideas to perform epistemic reasoning in the situation
calculus, allowing common knowledge to be handled using regression.
While the development naturally parallels that of LCC, the much richer
ontology of the situation calculus means there are also substantial
differences. In particular:

\begin{itemize}
\item LCC is \emph{propositional}: actions do not take arguments, there
are finitely many actions, and no quantification is required. 
\item LCC is \emph{synchronous}: reasoning is performed by regressing one
action at a time, without the {}``all possible futures'' approach
needed to handle hidden actions. 
\end{itemize}
By contrast, our formalism must capture first-order preconditions
and effects, quantifying-in and de-dicto/de-re, and arbitrary sets
of concurrent actions, while incorporating our new technique for handling
hidden actions and remaining compatible with other extensions to the
situation calculus.

The remainder of this section proceeds as follows. In Section \ref{sub:Epistemic-Paths}
we define a variant of first-order dynamic logic for use as an epistemic
path language, which is encoded in the situation calculus using the
macro $\KDo(\pi,s,s')$. This macro is analogous to the fluent $K(agt,s',s)$
from the previous section, but expresses more complex epistemic relationships
between situations.

Section \ref{sub:Synchronous-Epistemic-Fluent} develops a \emph{synchronous}
account of complex epistemic modalities. The macro $\PKnowsZ(\pi,\phi,s)$
expresses knowledge using an epistemic path $\pi$ under the assumption
that there have been no hidden actions. We develop a regression rule
using a new operator $\Trn$ to perform regression inside the epistemic
path, transforming $\PKnowsZ(\pi,\phi,do(c,s))$ into an equivalent
expression $\forall c':\,\PKnowsZ(\Trn(\pi,c,c'),\Reg(\phi,c'),s)$.

Section \ref{sub:Introducing-Hidden-Actions} introduces hidden actions
by explicitly representing them with an empty action set. We simulate
agents reasoning about arbitrarily-long sequences of hidden actions
by inserting arbitrarily many empty action terms between each real
action in a situation, and show how a fixpoint construction can reason
about such modified situations. Section \ref{sub:The-Link-with-IK}
then relates this fixpoint construction back to the results of Section
\ref{sec:Obs-Knowledge} by showing that in the case of a single agent,
it precisely matches the fixpoint generated by the persistence condition
operator in equation \eqref{eqn:R_do_c_s}.

The end result is a powerful account of complex epistemic modalities
constructed almost entirely in the meta-level reasoning machinery
of the situation calculus. We need only a single new successor state
axiom, given below.


\section{Synchronous Knowledge\label{sub:Syncrhonous-Knowledge}}

To begin, we must define the \emph{synchronous} knowledge of an individual
agent in an arbitrary situation $s$. This is the agent's knowledge
when it assumes that no hidden actions have occurred, and so it is
not required to do any {}``all possible futures'' style reasoning.
We extend the fluent $K_{0}(agt,s',s)$ which is already used to represent
synchronous knowledge in the initial situation. The axiom set $\Dt_{K}^{obs}$
gains the following successor state axiom for $K_{0}$:\begin{align}
K_{0}(agt,s'',do(c,s))\,\equiv\,\, & \exists s',c':\,\left(s''=do(c',s')\wedge Poss(c',s')\,\,\vee\,\, s''=s'\wedge c'=\{\}\right)\nonumber \\
 & \,\,\,\,\,\wedge Obs(agt,c,s)=Obs(agt,c',s')\wedge K_{0}(agt,s',s)\label{eq:K0_ssa}\end{align}


Given synchronicity, this axiom is a simple modification of the standard
successor state axiom for knowledge from equation \eqref{eq:k_ssa_standard}.
The only complication is in handling the empty action $\{\}$. When
\textbf{$Obs(agt,c,s)=\{\}$} then $c'$ is allowed to be $\{\}$,
and the agent considers it possible that no actions were actually
performed (i.e. $s''=s'$). Thus the number of actions in $s$ puts
an upper bound on the number of actions that the agent thinks might
have occurred.

Beginning with the assumption of synchronicity allows us to focus
first on increasing the expressiveness of the epistemic language.
Once this have been achieved, we will generalise the formalism to
asynchronous domains.


\section{Epistemic Paths\label{sub:Epistemic-Paths}}

In this section, we adopt the language of dynamic logic to express
complex epistemic modalities. To deal gracefully with the many first-order
aspects of the situation calculus we use a variant of \emph{first-order
dynamic logic,} adapted from the dynamic term-modal logic of \citet{kooi07dyn_termmodal_logic}
but with some simplifications.

First, we must specify the syntax of our epistemic path language.
We will use $\pi$ to denote an arbitrary epistemic path expression.

\begin{defnL}
[{{[}{Epistemic~Path}]}] Let $agt$ be an \noun{Agent }term, $\phi$
a uniform formula and $x$ a variable name, then the epistemic path
terms $\pi$ are the smallest set matching the following structural
rules:\[
\pi::=agt\,|\,?\phi\,|\,\pi_{1};\pi_{2}\,|\,\pi_{1}\cup\pi_{2}\,|\,\pi^{*}\,|\,\exists x\]

\end{defnL}
\smallskip{}


The test ($?$), sequence ($;$), choice ($\cup$) and transitive
closure ($^{*}$) operators are standard in dynamic logic, although
test formulae may now contain variables that must be interpreted.
The operator $\exists x$ allows the value of a variable to change
during path traversal, by non-deterministically re-binding $x$ to
some value.

The semantics of this epistemic path language are defined at the meta-level
as a series of macro expansions. The full details of this development
can be found in Appendix \ref{sec:Encoding-Dynamic-Logic}, and would
complicate the presentation here. Instead we give a simplified presentation
that treats variable bindings as concrete terms in the logic. Formulae
of first-order dynamic logic are interpreted relative to both a {}``current
world'' and a {}``current variable binding'' \citep{kooi07dyn_termmodal_logic}.
Variable bindings are represented below by a first-order substitution
$\mu$, with $\mu(\phi)$ applying the substitution to the variables
in $\phi$ and $\mu[x/z]$ setting the value of variable $x$ to the
term $z$. The semantics operate over pairs $(\mu,s)$.

\begin{defnL}
[{{[}{Epistemic~Path~Semantics}]}] \label{def:KDo}A situation
$s'$ is reachable from situation $s$ via epistemic path $\pi$,
denoted $\KDo(\pi,s,s')$, according to the following definitions.
These semantics are encoded using macro expansion as detailed in Appendix
\ref{sec:Encoding-Dynamic-Logic}.\[
\KDo(\pi,s,s')\,\,\equiv\,\,\exists\mu,\mu':\,\KTrans(\pi,\mu,s,\mu',s')\]
 \begin{gather*}
\KTrans(agt,\mu,s,\mu',s')\,\equiv\,\mu'=\mu\,\wedge\, K_{0}(agt,s',s)\\
\KTrans(?\phi,\mu,s,\mu',s')\,\equiv\, s'=s\,\wedge\,\mu'=\mu\,\wedge\,\mu(\phi)[s]\\
\KTrans(\pi_{1};\pi_{2},\mu,s,\mu',s')\,\equiv\,\exists\mu'',s'':\,\KTrans(\pi_{1},\mu,s,\mu'',s'')\,\wedge\,\KTrans(\pi_{2},\mu'',s'',\mu',s')\\
\KTrans(\pi_{1}\cup\pi_{2},\mu,s,\mu',s')\,\equiv\,\KTrans(\pi_{1},\mu,s,\mu',s')\,\vee\,\KTrans(\pi_{2},\mu,s,\mu',s')\\
\KTrans(\exists x,\mu,s,\mu',s')\,\equiv\,\exists z:\, s'=s\,\wedge\,\mu'=\mu[x/z]\\
\forall P:\,\left[\mathbf{refl}\,\wedge\,\mathbf{tran}\,\wedge\,\mathbf{cont}\right]\rightarrow\left[P(\mu,s,\mu',s')\rightarrow\KTrans(\pi^{*},\mu,s,\mu',s')\right]\end{gather*}

\end{defnL}
Where we use the following abbreviations (standing for {}``reflexive'',
{}``transitive'' and {}``contains'' respectively) to specify that
$\pi^{*}$ is the reflexive transitive closure of $\pi$:\begin{align*}
\mathbf{refl}\isdef & \,\, P(\mu,s,\mu,s)\\
\mathbf{tran\isdef} & \,\,\forall\mu_{1},s_{1},\mu_{2},s_{2}:\, P(\mu_{1},s_{1},\mu,s)\wedge\KTrans(\pi,\mu_{2},s_{2},\mu_{1},s_{1})\,\rightarrow\, P(\mu_{2},s_{2},\mu,s)\\
\mathbf{cont}\isdef & \,\,\forall\mu',s':\,\KTrans(\pi,\mu',s',\mu,s)\,\rightarrow\, P(\mu',s',\mu,s)\end{align*}


Let us re-iterate: these are \emph{not} axioms to be included in our
basic action theory, but are intended only to demonstrate the semantics
of the epistemic path language and the macro $\KDo$. Paths do not
appear in situation calculus terms, but are handled by macro-expansion
of $\KDo(\pi,s,s')$ into second-order sentences of the situation
calculus.


\section{A Synchronous Epistemic Fluent\label{sub:Synchronous-Epistemic-Fluent}}

At this point it's worth reviewing again the purpose of this path
language. Despite utilising the syntax of dynamic logic, it is \emph{not}
related to actions in any way. Rather it expresses complex \emph{epistemic}
paths, and is interpreted over the epistemic frame generated by the
agents' knowledge relations. We will be introducing a new macro $\PKnows(\pi,\phi,s)$
(read this as {}``Path-Knows'') to express knowledge using these
epistemic paths. To make this clear, here is how some different kinds
of knowledge would be expressed using the standard account of knowledge,
and how we intend to express them using epistemic paths:\begin{align}
\Knows(A,\phi,s) & \,\,\equiv\,\,\PKnows(A,\phi,s)\nonumber \\
\Knows(A,\Knows(B,\phi),s) & \,\,\equiv\,\,\PKnows(A;B,\phi,s)\nonumber \\
\Knows(A,\phi)\wedge\Knows(B,\phi,s) & \,\,\equiv\,\,\PKnows(A\cup B,\phi,s)\nonumber \\
\EKnows(G,\phi,s) & \,\,\equiv\,\,\PKnows(\bigcup_{a\in G}a,\phi,s)\nonumber \\
\CKnows(G,\phi,s) & \,\,\equiv\,\,\PKnows((\bigcup_{a\in G}a)^{*},\phi,s)\label{eq:pknows_identities}\end{align}


In this section, we develop a synchronous version $\PKnowsZ(\pi,\phi,s)$
of our path-knowledge operator, building on the synchronous $K_{0}$
relation defined earlier. Its definition is a straightforward analogue
of the individual-level $\Knows$ macro:\[
\PKnowsZ(\pi,\phi,s)\isdef\forall s':\,\KDo(\pi,s,s')\,\rightarrow\,\phi[s']\]


But this macro expands to a complicated second-order formula in the
base language of the situation calculus. As with the case of the basic
$\Knows$ macro, we need to treat $\PKnowsZ$ syntactically as a primitive
fluent. This means we need a regression rule for such expressions.
It is here that we incorporate the second key idea from LCC - use
of a syntactic transform to encode the effects of actions within epistemic
paths as well as in primitive formulae. Mirroring LCC, we introduce
the meta-operator $\Trn$ for this purpose.

Let us consider the required operation of $\Trn$ by analogy with
the standard regression operator $\Reg$. One can think of regression
as a {}``pre-encoding'' of the effects of an action: $\phi$ will
hold in $do(c,s)$ if and only if $\Reg(\phi,c)$ holds in $s$. The
path regressor $\Trn$ needs to lift this idea to epistemic paths
as follows: there is a $\pi$-path from $do(c,s)$ to $do(c',s')$
if and only if there is a $\Trn(\pi,c,c')$-path from $s$ to $s'$.

In order to accomplish this task of pre-encoding the effects of actions,
the path regressor will need to make various assertions about the
action that is to be performed in each situation traversed by the
path. It uses a fresh variable to track this {}``current action''
in the regressed path. The basic operation of $\Trn$ is as follows:

\begin{itemize}
\item Introduce a fresh variable $x$ to hold the action to be performed
in the current situation; 
\item at the beginning of the path, bind $x$ to the known action $c$; 
\item at the end of the path, assert that $x$ is the known action $c'$;
and 
\item when the path moves to a new situation, select a new action using
$\exists x$. 
\end{itemize}
This is accomplished with an auxiliary operator $\TrnA(\pi,x)$, which
translates $\pi$ under the assumption that variable $x$ contains
the action to be performed in the current situation.

\begin{defnL}
[{{[}{Epistemic~Path~Regressor\label{def:EpistemicPathRegression}}]}] The
epistemic path regressor $\Trn(\pi,c,c')$ operates according to the
definitions below, where $x$ and $z$ are fresh variables not appearing
in $\pi$:\begin{align*}
\Trn(\pi,c,c')\,\isdef\,\, & \exists x\,;\,?x=c\,;\,\TrnA(\pi,x)\,;\,?x=c'\\
\\\TrnA(agt,x)\,\isdef\,\, & \exists z\,;\,?Obs(agt,x)=z\,;\, agt\,;\,\exists x\,;\,?Poss(x)\vee x=\{\}\,;\,?Obs(agt,x)=z\\
\TrnA(?\phi,x)\isdef\,\, & ?\Reg(\phi,x)\\
\TrnA(\exists y,x)\isdef\,\, & \exists y\\
\begin{array}{c}
\TrnA(\pi_{1};\pi_{2},x)\isdef\,\,\end{array} & \TrnA(\pi_{1},x)\,;\,\TrnA(\pi_{2},x)\\
\TrnA(\pi_{1}\cup\pi_{2},x)\isdef\,\, & \TrnA(\pi_{1},x)\cup\TrnA(\pi_{2},x)\\
\TrnA(\pi^{*},x)\isdef\,\, & \TrnA(\pi,x)^{*}\end{align*}

\end{defnL}
Most of these clauses are straightforward, but note how the clause
for an individual agent term encodes the successor-state axiom for
$K_{0}$. The following theorem states that these definitions behave
has desired, respecting the semantics of epistemic paths:

\begin{thm}
\label{thm:Trn-respects-epi-paths}For any epistemic path $\pi$:\begin{multline*}
\Dt\cup\Dt_{K}^{obs}\models\,\KDo(\pi,do(c,s),s'')\,\equiv\\
\exists c',s':\,\left(s''=do(c',s')\,\vee\, s''=s'\wedge c'=\{\}\right)\wedge\KDo(\Trn(\pi,c,c'),s,s')\end{multline*}

\end{thm}
\begin{proofsketch}
The proof proceeds by cases, covering each path operator in turn.
The base cases $agt$, $?\phi$ and $\exists y$ follow from Definition
\ref{def:KDo} and the successor state axiom for $K_{0}$ in equation
\eqref{eq:K0_ssa}. The inductive cases are straightforward as $\Trn_{a}$
is simply pushed inside each operator. 
\end{proofsketch}
Given that $\Trn$ correctly regresses our epistemic path language,
we are free to use it to define the regression of a complex epistemic
modality. We define the regression of a $\PKnowsZ$ expression as
follows:\[
\Reg(\PKnowsZ(\pi,\phi,do(c,s)))\isdef\forall c':\,\PKnowsZ(\Trn(\pi,c,c'),\Reg(\phi,c'),s)\]


Note that this rule must universally quantify over action terms $c'$
in order to account for different actions producing the same observation.
Such quantification is also found in the rule for individual knowledge
from equation \eqref{eqn:R_do_c_s}, although in the current instance
it has been taken outside the scope of the knowledge macro.

\begin{thm}
\label{thm:Reg_PKnowsZ}For any epistemic path $\pi$, uniform formula
$\phi$ and action $c$:\begin{gather*}
\Dt\cup\Dt_{K}^{obs}\,\models\,\PKnowsZ(\pi,\phi,do(c,s))\,\equiv\,\forall c':\,\PKnowsZ(\Trn(\pi,c,c'),\Reg(\phi,c'),s)\end{gather*}

\end{thm}
\begin{proofsketch}
The mechanics of this proof mirror that of Theorem \ref{thm:Reg_Knows}:
we expand the $\PKnowsZ$ macro, apply Theorem \ref{thm:Trn-respects-epi-paths}
as a successor state axiom for $\KDo$, re-arrange to eliminate existential
quantifiers, then collect terms back into forms that match $\PKnowsZ$ 
\end{proofsketch}

\section{Introducing Hidden Actions\label{sub:Introducing-Hidden-Actions}}

We now have a powerful account of multi-agent knowledge for \emph{synchronous}
domains, but it remains to generalise this to \emph{asynchronous}
domains by incorporating support for arbitrarily-long sequences of
hidden actions. We continue to operate at the meta-level, developing
support for hidden actions directly in the rules governing the regression
operator.

The idea is to use the empty action term $\{\}$ to explicitly represent
the notion that {}``nothing happens''. We simulate agents reasoning
about hypothetical futures in which they make no more observations
by inserting these empty actions between each regular action in a
situation term.

\begin{defn}
Let $\mathcal{E}^{n}(s)$ be $s$ with $n$ empty actions inserted
between each action, as follows:\begin{align*}
\mathcal{E}^{0}(s)\,\isdef\,\, & s\\
\mathcal{E}^{1}(S_{0})\,\isdef\,\, & do(\{\},S_{0})\\
\mathcal{E}^{1}(do(c,s))\,\isdef\,\, & do(\{\},do(c,\mathcal{E}^{1}(s)))\\
\mathcal{E}^{n}(s)\,\isdef\,\, & \mathcal{E}^{1}(\mathcal{E}^{n-1}(s))\end{align*}

\end{defn}
The intuition here is that we want $\PKnows(\pi,\phi)$ to hold if
$\PKnowsZ(\pi,\phi)$ holds after allowing for any number of empty
actions. Formally, we want to use the following infinite conjunction
as the definition of $\PKnows(\pi,\phi,s)$:\[
\PKnows(\pi,\phi,s)\isdef\bigwedge_{n\in\mathbb{N}}\PKnowsZ(\pi,\phi,\mathcal{E}^{n}(s))\]


Of course we cannot use this definition directly for automated reasoning,
any more than we can use an infinitary definition of $\CKnows$. Instead
we will construct a series of regression rules that simulate the infinite
conjunction. This requires the following results:

\begin{thm}
\label{thm:En_impl_En-1}For any epistemic path $\pi$: \[
\Dt\cup\Dt_{K}^{obs}\models\,\PKnowsZ(\pi,\phi,\mathcal{E}^{1}(s))\rightarrow\PKnowsZ(\pi,\phi,s)\]

\end{thm}
\begin{proofsketch}
By a case analysis on the definition of $\Trn$, we determine that
that path $\Trn(\pi,\{\},\{\})$ always contains the path $\pi$.
Thus any situations reachable by $\pi$ are also reachable by $\Trn(\pi,\{\},\{\})$.
Since $\Trn(\pi,\{\},\{\})$ is always in the regression of $\PKnowsZ(\pi,\phi,\mathcal{E}^{1}(s))$
and $\Reg(\phi,\{\})=\phi$ always, we can conclude that $\PKnowsZ(\pi,\phi,\mathcal{E}^{1}(s))$
always implies $\PKnowsZ(\Trn(\pi,\{\},\{\}),\phi,s)$, which implies
$\PKnowsZ(\pi,\phi,s)$ as required. 
\end{proofsketch}
\begin{thm}
For any epistemic path $\pi$, if there is some $n\in\mathbb{N}$
such that:\[
\Dt\cup\Dt_{K}^{obs}\models\,\PKnowsZ(\pi,\phi,\mathcal{E}^{n}(s))\rightarrow\PKnowsZ(\pi,\phi,\mathcal{E}^{n+1}(s))\]
 Then:\[
\Dt\cup\Dt_{K}^{obs}\models\,\PKnowsZ(\pi,\phi,\mathcal{E}^{n}(s))\,\equiv\,\bigwedge_{m\in\mathbb{N}}\PKnowsZ(\pi,\phi,\mathcal{E}^{m}(s))\]

\end{thm}
\begin{proof}
By Theorem \ref{thm:En_impl_En-1}, $\PKnowsZ(\pi,\phi,\mathcal{E}^{n+1}(s))\rightarrow\PKnowsZ(\pi,\phi,\mathcal{E}^{n}(s))$
always. So for this hypothesised value of $n$, the implication in
the current theorem is in fact an equivalence. We then have $\PKnowsZ(\pi,\phi,\mathcal{E}^{n}(s))\rightarrow\PKnowsZ(\pi,\phi,\mathcal{E}^{m}(s))$
for $m<n$, and $\PKnowsZ(\pi,\phi,\mathcal{E}^{n}(s))\equiv\PKnowsZ(\pi,\phi,\mathcal{E}^{m}(s))$
for $m\geq n$, which is enough to establish the infinite conjunction
as required. 
\end{proof}
This pair of results allows us to replace the infinite conjunction
with a fixpoint calculation, so we will use the following suggestive
notation:\[
\PKnowsZ(\pi,\phi,\mathcal{E}^{\infty}(s))\isdef\bigwedge_{n\in\mathbb{N}}\PKnowsZ(\pi,\phi,\mathcal{E}^{n}(s))\]


We can therefore use the following as our definition of $\PKnows$:\[
\PKnows(\pi,\phi,s)\,\isdef\,\PKnowsZ(\pi,\phi,\mathcal{E}^{\infty}(s))\]


This macro would once again expand into a complicated second-order
sentence of the situation calculus, encoding both the semantics of
the epistemic path and the fixpoint definition of $\mathcal{E}^{\infty}$.
As before, regression needs to avoid such expansion by treating $\PKnows$
as a primitive fluent. It can calculate $\PKnows$ from $\PKnowsZ$
by performing a fixpoint calculation very similar to that performed
for the persistence condition, using proof techniques for dynamic
logic directly to avoid expanding $\PKnowsZ$ and $\KDo$.

Again, the question must be asked: can this really be considered an
effective technique in practice? Many of our earlier comments on the
persistence condition operator also apply here. Moreover, note that
the epistemic path regressor preserves much of the structure of $\pi$,
such that there is a good deal of similarity between the epistemic
paths in $\PKnowsZ(\pi,\phi,\mathcal{E}^{n}(s))$ and $\PKnowsZ(\pi,\phi,\mathcal{E}^{n+1}(s))$.
Our preliminary implementation contains some special-case logic to
simplify the reasoning based on this similarity, and we are currently
investigating further such enhancements.

Clearly this definition of $\PKnows$ operates in a very similar way
to the regression rule for $\Knows$ from equation \eqref{eqn:R_do_c_s},
using a fixpoint calculation to account for arbitrarily long sequences
of hidden actions. In the following section we formalism the precise
relationship between $\Knows$ and $\PKnows$.


\section{The Link with Individual Knowledge\label{sub:The-Link-with-IK}}

The last remaining link is the most important of all: showing that
this new path-based account of knowledge actually captures the knowledge
of the agents, according to the semantics of individual knowledge
developed in Section \ref{sec:Obs-Knowledge}. The following series
of theorems establish this important link. First, we produce special
cases of the regression rule for $\PKnows(\pi,\phi)$ for the case
of a single agent:

\begin{thm}
\label{thm:Pknows_PbU_S0}For any $agt$ and $\phi$:\[
\Dt\cup\Dt_{K}^{obs}\,\models\,\PKnows(agt,\phi,S_{0})\,\equiv\,\PKnowsZ(agt,\Pst(\phi,PbU(agt)),S_{0})\]

\end{thm}
\begin{proofsketch}
By stepping through the regression of $\PKnowsZ(agt,\phi,do(\{\},S_{0}))$
we show that for any $n$, $\PKnowsZ(agt,\phi,\mathcal{E}^{n}(S_{0}))\equiv\PKnowsZ(agt,\Pst^{n}(\phi,\PbU(agt)),S_{0})$.
The fixpoint calculation for $\PKnowsZ(agt,\phi,\mathcal{E}^{\infty}(S_{0}))$
is then clearly the same calculation required to derive $\PKnowsZ(agt,\Pst(\phi,PbU(agt)),S_{0})$
and we can equate the two expressions. 
\end{proofsketch}
\medskip{}


\begin{thm}
\label{thm:Pknows_PbU_do}For any $agt$, $\phi$, $c$ and $s$:\begin{multline*}
\PKnows(agt,\phi,do(c,s))\,\equiv\,\exists z:\, Obs(agt,c,s)=z\\
\wedge\,\left[z=\{\}\,\rightarrow\,\PKnows(agt,\Pst(\phi,\PbU(agt)),s)\right]\\
\wedge\,\left[z\neq\{\}\,\rightarrow\,\PKnows(agt,\forall c':\,\left(Poss(c')\wedge Obs(agt,c')=z\right)\rightarrow\Reg(\Pst(\phi,\PbU(agt)),c),s)\right]\end{multline*}

\end{thm}
\begin{proofsketch}
Repeating the calculations from lemma \ref{thm:Pknows_PbU_S0} on
$\PKnowsZ(agt,\phi,\mathcal{E}^{1}(do(c,s)))$, and pushing the application
of $\mathcal{E}^{\infty}$ past the actions $c$, we obtain the following:\[
\PKnows(agt,\phi,do(c,s))\,\equiv\,\PKnowsZ(agt,\Pst(\phi,\PbU(agt)),do(c,\mathcal{E}^{\infty}(s)))\]


Regressing the RHS through the action $c$, we obtain:\begin{align*}
\PKnows(agt,\phi,do(c,s))\,\equiv\,\, & \forall c':\,\PKnowsZ(\Trn(agt,c,c'),\Reg(\Pst(\phi,\PbU(agt)),c'),\mathcal{E}^{\infty}(s))\\
\equiv\,\, & \forall c':\,\PKnows(\Trn(agt,c,c'),\Reg(\Pst(\phi,\PbU(agt)),c'),s)\end{align*}


Simplifying $\Trn(agt,c,c')$ and using it to re-arrange the above
expression gives the desired result. 
\end{proofsketch}
We will also need the following result for individual knowledge:

\begin{thm}
\label{thm:Knows_impl_KnowsPbU}For any $agt$, $\phi$ and $s$:\[
\Dt\cup\Dt_{K}^{obs}\,\models\,\Knows(agt,\phi,s)\,\equiv\,\Knows(agt,\Pst(\phi,PbU(agt)),s)\]

\end{thm}
\begin{proofsketch}
By induction on the regression rules for knowledge, and using the
following property of the persistence condition ({}``if $\phi$ persists,
then $\Pst(\phi,\alpha)$ persists''):\[
\forall s':\,\Pst(\phi,\alpha)[s]\,\wedge\, s\leq_{\alpha}s'\,\rightarrow\,\Pst(\phi,\alpha)[s']\]


We consider three cases: $s=S_{0}$, and $s=do(c,s)$ with $c$ both
observable and unobservable. Each case requires only a simple re-arrangement
of the relevant regression rule. 
\end{proofsketch}
\medskip{}


Finally, we are in a position to state the major theorem of this section
- that $\PKnows$ for a single agent is equivalent to the standard
$\Knows$ macro.

\begin{thm}
For any $agt$, $\phi$ and $s$:

\[
\Dt\cup\Dt_{K}^{obs}\,\models\,\Knows(agt,\phi,s)\,\,\equiv\,\,\PKnows(agt,\phi,s)\]

\end{thm}
\begin{proof}
By induction on the regression rules for each expression. For the
$S_{0}$ case, we require the following result which is true by the
definition of $\KDo$:\[
\KnowsZ(agt,\phi,S_{0})\equiv\PKnowsZ(agt,\phi,S_{0})\]
 The regression rule for $\Knows$ from equation \eqref{eqn:R_s0}
then precisely matches the result of theorem \ref{thm:Pknows_PbU_S0}
and we have the required equivalence. For the $do(c,s)$ case, we
can substitute the result of theorem \ref{thm:Knows_impl_KnowsPbU}
into the regression rule for $\Knows$ from equation \eqref{eqn:R_do_c_s}
to produce an expression precisely matching the result of theorem
\ref{thm:Pknows_PbU_do}. Using $\Knows(agt,\phi,s)\equiv\PKnows(agt,\phi,s)$
from the inductive hypothesis renders the two equivalent. 
\end{proof}
\medskip{}


Thus the expressions $\Knows(agt,\phi,s)$ and $\PKnows(agt,\phi,s)$
are equivalent under our formulation. This link is all that is required
to validate the additional complex modalities shown in equation \eqref{eq:pknows_identities},
which we will repeat below for convenience.

\begin{thm}
The following identities hold under the theory of action $\Dt\cup\Dt_{K}^{obs}$:\begin{align*}
\Knows(A,\phi,s) & \,\,\equiv\,\,\PKnows(A,\phi,s)\\
\Knows(A,\Knows(B,\phi),s) & \,\,\equiv\,\,\PKnows(A;B,\phi,s)\\
\Knows(A,\phi)\wedge\Knows(B,\phi,s) & \,\,\equiv\,\,\PKnows(A\cup B,\phi,s)\\
\EKnows(G,\phi,s) & \,\,\equiv\,\,\PKnows(\bigcup_{a\in G}a,\phi,s)\\
\CKnows(G,\phi,s) & \,\,\equiv\,\,\PKnows((\bigcup_{a\in G}a)^{*},\phi,s)\end{align*}

\end{thm}
\begin{proof}
Each follows from the equivalence of $\Knows(agt,\phi,s)$ and $\PKnows(agt,\phi,s)$,
using the semantics of first-order dynamic logic as defined by $\KDo$.
For example, in the $\CKnows$ case we argue as follows: By definition,
$\PKnows((\bigcup_{a\in G}a)^{*},\phi,s)$ is the transitive closure
of the union of $\PKnows(a,\phi,s)$ for $a\in G$. Also by definition,
$\CKnows(G,\phi,s)$ is the transitive closure of the union of $\Knows(a,\phi,s)$
for $a\in G$. Since $\PKnows(a,\phi,s)$ and $\Knows(a,\phi,s)$
are the same relation, their transitive closures are also the same
and the identity is entailed. 
\end{proof}

\section{Answering the Regressed Query}

We are now in a position to reduce an epistemic query $\PKnows(\pi,\phi,s)$
at some future situation to an epistemic query $\PKnows(\Trn^{*}(\pi),\Reg^{*}(\phi),S_{0})$
at the initial situation. While this is a significant gain for effective
automated reasoning, it still remains to answer the regressed query.

As with individual knowledge, we assume that regressed queries will
be handled by a special-purpose theorem prover rather than by expanding
the macros. However, we should note that validity in first-order dynamic
logic is undecidable; in fact it is $\Pi_{1}^{1}$-hard \citep{kooi07dyn_termmodal_logic}.
As with previous work in the situation calculus, we must assume that
axioms about the initial situation are in a restricted form amenable
to effective reasoning. There are several special cases that can simplify
answering the regressed query.

A common simplifying assumption is that the potential values of each
variable can be finitely enumerated. In this case it is possible to
translate our epistemic paths into propositional dynamic logic, which
is decidable. The only difficulty is the elimination of variable bindings
inside an iteration operator, which can be handled using a Kleene-style
technique similar to the $\mathcal{K}$ translator of \citet{vanBenthem06lcc}.

Alternately, it may be that the initial situation is completely known
and uncertainty is introduced only due to partial observability of
actions. In this case the initial epistemic frame contains the lone
situation $S_{0}$, and the regressed path can be reduced to a series
of tests and variable re-bindings.

We have produced a preliminary implementation for propositional domains
using the PDL prover from the Tableaux Workbench suite \citep{abate07twb_pdl},
and have used it to verify some simple examples. We are also investigating
further techniques for answering the regressed query in a more effective
manner.


\section{An Illustrative Example}

With this technical machinery in place, we are now able to reason
about the group-level epistemic modalities of a team of agents. To
demonstrate, we revisit our example domain from Subsection \ref{sub:An-Illustrative-Example}.
We add the the following initial knowledge axiom to $\Dt_{S_{0}}$:\begin{gather*}
\PKnowsZ((Ann\cup Bob)^{*},InRoom(Ann)\wedge InRoom(Bob),S_{0})\end{gather*}


The preconditions and effects of actions in this domain are relatively
simple, such that there is nothing that can be accomplished by a sequence
of hidden actions that cannot be accomplished by a single hidden action.
More formally, it is possible to show that:\[
\phi[\mathcal{E}^{\infty}(s)]\,\equiv\,\phi[\mathcal{E}^{1}(s)]\]
 In other words, the fixpoint calculation required for reasoning about
$\PKnows$ will always terminate after a single iteration. We will
therefore use the following identity to simplify presentation of the
example:\begin{align*}
\Reg(\PKnows(\pi,\phi,do(c,s))\,\Rightarrow & \,\,\,\PKnowsZ(\pi,\phi,\mathcal{E}^{\infty}(do(c,s)))\\
\equiv & \,\,\,\PKnowsZ(\pi,\phi,\mathcal{E}^{1}(do(c,s)))\\
\equiv & \,\,\,\PKnowsZ(\pi,\phi,do(\{\},do(c,s)))\\
\equiv & \,\,\,\forall c',c'':\,\PKnows(\Trn(\Trn(\pi,\{\},c'),c,c''),\Reg(\Reg(\phi,c'),c''),s)\end{align*}


To keep the presentation compact, we will abbreviate agent names to
$A$ and $B$ and the fluent $InRoom(agt)$ to $IR(agt)$.

\begin{example}
After Bob reads the invitation, it is common knowledge that he knows
where the party is:\[
\Dt\cup\Dt_{K}^{obs}\,\models\,\PKnows((A\cup B)^{*},\exists x:\Knows(B,loc=x),do(\{read(B)\},S_{0}))\]

\end{example}
This example hinges on the fact that initially, it is common knowledge
that both agents are in the room. It is thus common knowledge that
the occurrence of $read(Bob)$ will be observed by both agents. Suppressing
the inner expression for the moment, regressing using the above-mentioned
identity gives:\begin{multline*}
\Reg(\PKnows((A\cup B)^{*},\phi,do(read(B),S_{0})))\\
\Rightarrow\,\,\forall c,c':\,\PKnows(\Trn(\Trn((A\cup B)^{*},\{\},c),read(B),c'),\Reg(\Reg(\phi,c),c'),S_{0})\\
\Rightarrow\,\,\forall c,c',c'':\,\PKnowsZ(\Trn(\Trn(\Trn((A\cup B)^{*},\{\},c),read(B),c'),\{\},c''),\Reg(\Reg(\Reg(\phi,c),c'),c''),S_{0})\end{multline*}
 We begin by evaluating $\Trn(\pi,\{\},c)$ :\[
\Trn((A\cup B)^{*},\{\},c)\,\Rightarrow\,\exists x\,;\,?x=\{\}\,;\,(\TrnA(A,x)\cup\TrnA(B,x))^{*}\,;\,?x=c\]
 With the cases for $\TrnA(agt,x)$ both evaluating to:\[
\TrnA(agt,x)\,\Rightarrow\,\exists z\,;\,?Obs(agt,x)=z\,;\, agt\,;\,\exists x\,;\,?Poss(x)\vee x=\{\}\,;\,?Obs(agt,x)=z\]
 Since we know all possible observations that Ann could make, let
us expand the path $\TrnA(A,x)$ by enumerating the possible values
for $z$. The first line in the result gives the case where $Obs(Ann,x)=\{\}$,
and the last line is the case where $Obs(Ann,x)=read(Ann)\#r$. All
other lines are where Ann simply observes the action $x$. \begin{gather*}
\TrnA(A,x)\,\Rightarrow\,\,\,\,\,\,\,\,\,\,\,\,\,\,\,\,\,\,\,\,\,\,\,\,\,\,\,\,\,\,\,\,\,\,\,\,\,\,\,\,\,\,\,\,\,\,\,\,\,\,\,\,\,\,\,\,\,\,\,\,\,\,\,\,\,\,\,\,\,\,\,\,\,\,\,\,\,\,\,\,\,\,\,\,\,\,\,\,\,\,\,\,\,\,\,\,\,\,\,\,\,\,\,\,\,\,\,\,\,\,\,\,\,\,\,\,\,\,\,\,\,\,\,\,\,\,\,\,\,\,\,\,\,\,\,\,\,\,\,\,\,\,\,\,\,\,\,\,\,\,\,\,\,\,\,\,\,\,\,\,\,\,\,\\
?\, x=\{\}\vee(x=read(B)\wedge\neg IR(A)))\,;\, A\,;\,\exists x\,;\,?\, x=\{\}\vee(x=read(B)\wedge IR(B)\wedge\neg IR(A))\\
\cup\,\,?x=enter(A)\,;\, A\,;\,\exists x\,;\,?x=enter(A)\wedge IR(A)\\
\cup\,\,?x=enter(B)\,;\, A\,;\,\exists x\,;\,?x=enter(B)\wedge IR(B)\\
\cup\,\,?x=leave(A)\,;\, A\,;\,\exists x\,;\,?x=leave(A)\wedge\neg IR(A)\\
\cup\,\,?x=leave(B)\,;\, A\,;\,\exists x\,;\,?x=leave(B)\wedge\neg IR(B)\\
\cup\,\,?\, x=read(B)\wedge IR(A)\,;\, A\,;\,\exists x\,;\,?\, x=read(B)\wedge IR(A)\wedge IR(B)\\
\cup\,\,\exists r\,;\,?\, x=read(A)\wedge loc(r)\,;\, A\,;\,\exists x\,;\,?\, x=read(A)\wedge IR(A)\wedge loc(r)\end{gather*}
 Now consider starting with $x$ set to $\{\}$ and following this
path some number of times. On the first iteration, $x$ may get rebound
to either $\{\}$ or $read(B)$. Subsequent iterations may alternate
between these values, but can never bind $x$ to any other action.
An analogous argument for $\TrnA(B,x)$ shows that the path $\Trn((A\cup B)^{*},\{\},c)$
can be simplified to consider only the actions $\{\}$, $read(A)$
and $read(B)$, with any other binding of $x$ being unreachable:\begin{multline*}
\Trn((A\cup B)^{*},\{\},c)\,\Rightarrow\,\exists x\,;\,?x=\{\}\,;\,(\\
?\, x=\{\}\vee(x=read(B)\wedge\neg IR(A)))\,;\, A\,;\,\exists x\,;\,?\, x=\{\}\vee(x=read(B)\wedge IR(B)\wedge\neg IR(A))\\
\cup\,\,?\, x=\{\}\vee(x=read(A)\wedge\neg IR(B)))\,;\, B\,;\,\exists x\,;\,?\, x=\{\}\vee(x=read(A)\wedge IR(A)\wedge\neg IR(B))\\
\cup\,\,?\, x=read(B)\wedge IR(A)\,;\, A\,;\,\exists x\,;\,?\, x=read(B)\wedge IR(A)\wedge IR(B)\\
\cup\,\,\exists r\,;\,?\, x=read(A)\wedge loc(r)\,;\, A\,;\,\exists x\,;\,?\, x=read(A)\wedge IR(A)\wedge loc(r)\\
\cup\,\,?\, x=read(A)\wedge IR(B)\,;\, B\,;\,\exists x\,;\,?\, x=read(A)\wedge IR(B)\wedge IR(A)\\
\cup\,\,\exists r\,;\,?\, x=read(B)\wedge loc(r)\,;\, B\,;\,\exists x\,;\,?\, x=read(B)\wedge IR(B)\wedge loc(r)\\
)^{*}\,;\,?x=c\end{multline*}
 This path matches with our intuitions about the observability of
actions. Since the $enter$ and $leave$ actions are always public,
they can never be mistaken for the empty action and are irrelevant
in the path $\Trn(\pi,\{\},c)$. Moreover, the binding of $x$ may
only switch from $\{\}$ to $read(B)$ if there is a possible world
in which Ann is not in the room, reflecting the conditions under which
that action would be hidden.

Next we evaluate the $\Trn(\pi,read(B),c')$ portion, which will produce:\begin{multline*}
\Trn(\Trn((A\cup B)^{*},\{\},c),read(B),c')\,\Rightarrow\,\exists x'\,;\,?x'=read(B)\,;\,\exists x\,;\,?x=\{\}\,;\\
(\TrnA(\TrnA(A,x),x')\cup\TrnA(\TrnA(B,x),x'))^{*}\,;\,?x=c\,;\,?x'=c'\end{multline*}
 As before, the only possible values of $x'$ are $\{\}$, $read(A)$
and $read(B)$. None of these actions affect any of the tests present
in $\Trn((A\cup B)^{*},\{\},c)$, so we can safely leave them unchanged
in the resulting path:\begin{multline*}
\Trn(\Trn((A\cup B)^{*},\{\},c),read(B),c')\,\Rightarrow\,\exists x'\,;\,?x'=read(B)\,;\,\exists x\,;\,?x=\{\}\,;\,(\\
?x=\{\}\vee(x=read(B)\wedge\neg IR(A)));\TrnA(A,x');\exists x;?x=\{\}\vee(x=read(B)\wedge IR(B)\wedge\neg IR(A))\\
\cup\,?x=\{\}\vee(x=read(A)\wedge\neg IR(B)));\TrnA(B,x');\exists x;?x=\{\}\vee(x=read(A)\wedge IR(A)\wedge\neg IR(B))\\
\cup\,\,?\, x=read(B)\wedge IR(A)\,;\,\TrnA(A,x')\,;\,\exists x\,;\,?\, x=read(B)\wedge IR(A)\wedge IR(B)\\
\cup\,\,\exists r\,;\,?\, x=read(A)\wedge loc(r)\,;\,\TrnA(A,x')\,;\,\exists x\,;\,?\, x=read(A)\wedge IR(A)\wedge loc(r)\\
\cup\,\,?\, x=read(A)\wedge IR(B)\,;\,\TrnA(B,x')\,;\,\exists x\,;\,?\, x=read(A)\wedge IR(B)\wedge IR(A)\\
\cup\,\,\exists r\,;\,?\, x=read(B)\wedge loc(r)\,;\,\TrnA(B,x')\,;\,\exists x\,;\,?\, x=read(B)\wedge IR(B)\wedge loc(r)\\
)^{*}\,;\,?x=c\,;\, x'=c'\end{multline*}
 We can complete this expansion using the previously calculated result
for $\TrnA(A,x)$, but will proceed leaving this implicit. Finally,
we must apply $\Trn$ again for the $\Trn(\pi,\{\},c'')$ portion.
The process is identical and we will not repeat it here.

We next evaluate this path over the initial epistemic relation of
the agents, about which we know:\[
\PKnowsZ((A\cup B)^{*},IR(A)\wedge IR(B),S_{0})\]


Consider how this restricts the possible bindings of $x$, $x'$ etc.
For $x$ to switch from its initial value of $\{\}$ to $read(A)$
(resp. $read(B)$) the path must traverse a test for $\neg IR(B)$
(resp. $\neg IR(A)$). Since we know that these tests will never succeed
on the initial epistemic frame, we can conclude that $x$ will always
be bound to $\{\}$. Since the path insists that $x=c$ at termination,
this is the only interesting value for $c$ - any value of $c$ other
than $\{\}$ will result in no worlds being reachable by the regressed
path and will thus have $\PKnowsZ$ vacuously true. By a similar argument,
we find that the only interesting values of $c'$ and $c''$ are $read(B)$
and $\{\}$ respectively.

Turning now to the regression of the inner formula, we have:\begin{gather*}
\Reg(\Reg(\Reg(\exists x:\,\Knows(B,loc=x),c),c'),c'')\\
\Rightarrow\Reg(\Reg(\Reg(\exists x:\,\Knows(B,loc=x),\{\}),read(B)),\{\})\\
\Rightarrow\Reg(\Reg(\exists x:\,\Knows(B,loc=x),read(B),\{\})\\
\Rightarrow true\end{gather*}


Thus for $c=\{\}$, $c'=read(B)$, $c''=\{\}$ the known formula is
a tautology, while for any other values the regressed epistemic path
has no reachable worlds. The example is therefore entailed by the
domain.

~

\begin{example}
After Bob reads the invitation, the location is not common knowledge\[
\Dt\cup\Dt_{K}^{obs}\,\not\models\,\PKnows((Ann\cup Bob)^{*},loc=C,do(read(Bob),S_{0}))\]

\end{example}
Regression of the epistemic path proceeds as with the previous example,
but regression of the inner formula no longer produces a tautology.
If we ignore the path components dealing with $c$ and $c''$, which
from the previous example we know to be redundant, regressing this
expression produces the following:

\begin{multline*}
\PKnowsZ(\dots\\
\cup\,\,?\, x'=read(B)\wedge IR(A)\,;\, A\,;\,\exists x'\,;\,?\, x'=read(B)\wedge IR(A)\wedge IR(B)\\
\cup\,\,\exists r\,;\,?\, x'=read(B)\wedge loc(r)\,;\, B\,;\,\exists x'\,;\,?\, x'=read(B)\wedge IR(B)\wedge loc(r)\\
\cup\dots,loc=C,S_{0})\end{multline*}


This means, very roughly, that $read(B)$ could result in common knowledge
that $loc=C$ if it was commonly known that the preconditions of $read(B)$
implied $loc=C$. Since this is not given in the domain description,
the example is not entailed.

~

\begin{example}
After Ann also reads the invitation, the location becomes common knowledge:\[
\Dt\cup\Dt_{K}^{obs}\,\models\,\PKnows((Ann\cup Bob)^{*},loc=C,do(read(Ann),do(read(Bob),S_{0})))\]

\end{example}
The additional action here can be handled in the same manner as the
previous example. We will not repeat the working, but simply state
that this indeed is a consequence of our theory.


\section{Discussion\label{sub:CEM-Discussion}}

In this section we have introduced a technique for representing and
reasoning about complex epistemic modalities in the situation calculus.
In order to formulate an effective reasoning procedure, we have had
to move beyond just $\Knows$ and $\CKnows$ and introduce a powerful
epistemic path language based on dynamic logic. Mirroring the development
of knowledge for individual agents, group-level modalities are introduced
as macros of the form $\PKnows(\pi,\phi,s)$ where $\pi$ is a complex
epistemic path. To avoid having to expand these macros during reasoning,
we have modified the regression operator to treat them as primitive
fluents.

Demonstrating the utility of our approach, we have presented an example
of effective automated reasoning about common knowledge in an asynchronous,
partially observable domain. This powerful new ability is a first
for the situation calculus.

Our development has clear parallels with the development of LCC \citep{vanBenthem06lcc}.
We choose the situation calculus for its much richer ontology, e.g.
preconditions and effects are first order, while actions take arguments
and may be performed concurrently. On one hand, this forces us to
use a more powerful dynamic logic for our epistemic language and run
the risk of undecidability. On the other, it actually simplifies some
aspects of our presentation. We do not need explicit update frames,
and the definition of our path regressor does not require an auxiliary
Kleene-style operator to handle iteration.

In synchronous domains with a finite state-space the situation calculus
may not offer a gain in expressiveness, but it can certainly provide
a more succinct axiomatisation. Moving beyond such domains, our formalism
offers the potential to incorporate other rich domain features that
have been developed for the situation calculus, such as continuous
time and actions with duration \citep{reiter96sc_nat_conc}.

The expressiveness of our epistemic path language means that answering
a regressed knowledge query can be difficult in the general case.
However, starting from the expression for common knowledge, the epistemic
path regressor $\Trn$ will generate only a fragment of the full epistemic
language. For example, it generates only formulae without nested iteration
operators, a property which is known to simplify proof search in PDL
\citep{abate07twb_pdl}. Restricting the domain can weaken this generated
fragment, as shown in \citep{vanBenthem06lcc} where a relativised
common-knowledge operator is proven sufficient for domains in which
all actions are publicly observable. Identifying restrictions such
as this that can simplify reasoning in the epistemic language is a
promising avenue for future research.

As with our account of individual-level knowledge, hidden actions
are handled using a fixpoint calculation. However, this calculation
is no longer based on objective formulae and the unique names axioms,
but requires reasoning in first-order dynamic logic. The above considerations
must be taken into account not only when answering the final regressed
query, but also when generating the intermediate fixpoints.


\section{TODO: Distributed Knowledge}

\begin{itemize}
\item Sharing observation histories 
\item Approximations (e.g. \char`\"{}someone knows\char`\"{}) 
\end{itemize}
