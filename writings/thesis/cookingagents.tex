

\chapter{Axioms for the {}``Cooking Agents''}

\label{ch:cookingagents}

This appendix provides the axioms for the {}``cooking agents'' example
domain used in Chapters \ref{ch:mindigolog} and \ref{ch:jointexec}.
While the different chapters use slightly different variants of the
domain, the major details are unchanged between chapters.


\section*{Synchronous, with Time and Natural Actions}

In this domain there are three agents named $Jon$, $Jim$ and $Joe$:\[
\forall agt:\,\, agt=Jim\,\vee\, agt=Jon\,\vee\, agt=Joe\]


There are various types of ingredient and utensil, and types are represented
explicitly as terms such as $Lettuce$ and $Bowl.$ Individual objects
of these types are named e.g. $Lettuce1$, $Bowl2$: We have a rigid
predicate $ObjIsType(obj,typ)$ that relates these two sorts of object.
It is defined as the completion of the following clauses:\begin{gather*}
IsType(obj,Bowl)\,\rightarrow\, obj=Bowl1\,\vee\, obj=Bowl2\,\vee\, Bowl3\\
IsType(obj,Board)\,\rightarrow\, obj=Board1\,\vee\, obj=Board2\\
IsType(obj,Egg)\,\rightarrow\, Egg1\\
IsType(obj,Tomato)\,\rightarrow\, obj=Tomato1\,\vee\, obj=Tomato2\,\vee\, obj=Tomato3\\
IsType(obj,Lettuce)\,\rightarrow\, obj=Lettuce1\,\vee\, obj=Lettuce2\\
IsType(obj,Carrot)\,\rightarrow\, obj=Carrot1\,\vee\, obj=Carrot2\,\vee\, obj=Carrot3\\
IsType(obj,Cheese)\,\rightarrow\, obj=Cheese1\,\vee\, obj=Cheese2\end{gather*}


Different object super-types are identified using:\begin{gather*}
IsContainer(obj)\,\equiv\, IsType(obj,Bowl)\,\vee\, IsType(obj,Board)\\
IsIngredient(obj)\,\equiv\, IsType(obj,Egg)\,\vee\, IsType(obj,Lettuce)\,\vee\,\dots\end{gather*}


The available actions are $release$, $acquire$, $placeIn$, and
$transer$, along with $beginTask$ and $endTask$. There are two
tasks: $chop(cont)$ chops the contents of a container, and $mix(cont,t)$
mixes the contents of a container for the given time.

We have the following fluents and successor state axioms. An ingredient
is used if the agent places it in some container:\begin{multline*}
Used(obj,do(c\#t,s))\,\equiv\, IsIngredient(obj)\\
\wedge\,\left(\exists agt,cnt:placeIn(agt,obj,cnt)\in c\right)\vee\, Used(obj,s)\end{multline*}


An agent has an object after he acquires it, and ceases to have it
when it is released or becomes used:\begin{multline*}
HasObject(agt,obj,do(c\#t,s))\,\equiv\, acquire(agt,obj)\in c\\
\vee HasObject(agt,obj,s)\,\wedge\neg\left(release(agt,obj)\in c\right.\\
\left.\vee IsIngredient(obj)\wedge\exists cnt:\, placeIn(agt,obj,cnt)\in c\right)\end{multline*}


The contents of a container is simply the set of things that have
been placed into it. For this simple example, we do not represent
the \emph{state} of those ingredients, e.g. mixed or chopped:\begin{multline*}
Contents(obj,cnts,do(c\#t,s))\,\equiv\,\left(\exists cnts_{n},cnts_{o}:\,\mathbf{NewConts}(obj,cnts_{n},c,s)\right.\\
\left.\wedge\, Contents(obj,cnts_{o},s)\wedge cnts=cnts_{n}\cup cnts_{o}\right)\\
\vee\,\,\left(cnts=\{\}\wedge\mathbf{LostContents}(obj,c)\right)\\
\vee\,\,\left(Contents(obj,cnts,s)\right.\wedge\\
\left.\neg(\exists cnts_{n},cnts_{o}:\,\mathbf{NewConts}(obj,cnts_{n},c,s)\vee\mathbf{LostContents}(obj,c)\right)\end{multline*}
\begin{multline*}
\mathbf{NewConts}(obj,cnts,c,s)\,\equiv\,\exists agt,igr:\, placeIn(agt,igr,obj)\in c\wedge cnts=\{igr\}\\
\vee\,\,\exists agt,obj':\, transfer(agt,obj',obj)\in c\wedge Contents(obj',cbts,s)\end{multline*}
\[
\mathbf{LostConts}(obj,c)\,\equiv\,\exists agt,obj':\, transfer(agt,obj,obj')\in c\]


An agent can be doing a long-running task, with time $t_{r}$ remaining
until completion. The rigid function $duration(tsk)$ gives the running
time of a task:\begin{multline*}
DoingTask(agt,tsk,t_{r},do(c\#t,s))\,\equiv\,\\
beginTask(agt,tsk)\in c\wedge t_{r}=duration(tsk)\\
\vee\exists t_{r}':\, DoingTask(agt,tsk,t_{r}',s)\wedge t_{r}=t_{r}'-t\wedge endTask(agt,tsk)\not\in c\end{multline*}


The possibility axioms for individual actions are:\[
Poss(acquire(agt,obj)\#t,s)\,\equiv\,\neg Used(obj)\,\wedge\,\neg\exists agt':\, HasObject(agt,obj,s)\]
\[
Poss(release(agt,obj)\#t,s)\,\equiv\, HasObject(agt,obj,s)\]
\[
Poss(placeIn(agt,obj,cnt)\#t,s)\,\equiv\, HasObject(agt,obj,s)\wedge HasObject(agt,cnt,s)\]
\begin{multline*}
Poss(transfer(agt,cnt,cont')\#t,s)\,\equiv\\
\, HasObject(agt,cnt,s)\wedge HasObject(agt,cnt',s)\end{multline*}
\begin{multline*}
Poss(beginTask(agt,tsk)\#t,s)\,\equiv\,\\
\exists cnt,t:\, tsk=mix(cnt,t)\wedge HasObject(agt,cnt,s)\wedge ObjIsType(cnt,Bowl)\\
\vee\exists cnt:\, tsk=chop(cnt)\wedge HasObject(agt,cnt,s)\wedge ObjIsType(cnt,Board)\end{multline*}
\[
Poss(endTask(agt,tsk)\#t,s)\,\equiv\,\exists t_{r}:\, DoingTask(agt,tsk,t_{r},s)\wedge t=start(s)+t_{r}\]


Concurrent actions are possible if they are all individually possible
and no pair of action is in conflict:\[
Poss(c\#t,s)\,\equiv\,\forall a,a'\in c:\, Poss(a\#t,s)\wedge Poss(a'\#t,s)\wedge\neg Conflicts(a,a',s)\]


Actions conflict if they are performed by the same agent, or are attempts
to acquire the same resource:\begin{multline*}
Conflicts(a,a',s)\,\equiv\, actor(a)=actor(a')\\
\vee\,\exists agt,agt',obj:\, a=acquire(agt,obj)\wedge a'=acquire(agt',obj)\end{multline*}


Initially all containers are empty, no-one has any objects, and all
ingredients apart from possibly the egg are not used:\begin{gather*}
\forall cnt:\, IsContainer(cnt)\,\rightarrow\, Contents(cnt,\{\},S_{0})\\
\forall agt,obj:\,\neg HasObject(agt,obj,S_{0})\\
\forall igr:\, ObjIsType(igr,Egg)\,\vee\,\neg Used(igr,S_{0})\end{gather*}


These axioms suffice for the example domain used in Chapter \ref{ch:mindigolog}


\section*{Asynchronous, without Time or Natural Actions}

For Chapter \ref{ch:jointexec} we drop the temporal component, and
collapse the tasks $mix$ and $chop$ into primitive actions:\[
Poss(mix(agt,cnt),s)\,\equiv\, HasObject(agt,cnt,s)\wedge ObjIsType(cnt,Bowl)\]
\[
Poss(chop(agt,cnt),s)\,\equiv\, HasObject(agt,cnt,s)\wedge ObjIsType(cnt,Board)\]


We introduce a sensing action $checkFor(agt,typ)$ which determines
whether all objects of that type are unused:\[
Poss(checkFor(agt,typ),s)\,\equiv\,\top\]
\begin{multline*}
SR(checkFor(agt,typ),s)=r\,\equiv\,\\
\shoveright{r="T"\wedge\forall obj:\, ObjIsType(obj,typ)\rightarrow\neg Used(obj,s)}\\
\vee r="F"\wedge\exists obj:\, ObjIsType(obj,typ)\wedge Used(obj,s)\end{multline*}


We adopt the $CanObs/CanSense$ axioms for observability and make
all actions private except $acquire$ and $release$:\begin{multline*}
CanObs(agt,a,s)\,\equiv\, actor(a)=agt\,\vee\,\exists agt',obj:\, a=acquire(agt',obj)\\
\vee\,\exists agt',obj:\, a=release(agt',obj)\end{multline*}
\[
CanSense(agt,a,s)\,\equiv\, actor(a)=agt\]


Finally, we identify independent actions as those that deal with different
objects, which much be axiomatised by enumerating the each possible
case. We will not present such an enumeration here.

These axioms suffice for the example domain in Chapter \ref{ch:jointexec}.

