%% LyX 1.4.1 created this file.  For more info, see http://www.lyx.org/.
%% Do not edit unless you really know what you are doing.
\documentclass[letterpaper]{article}
\usepackage[T1]{fontenc}
\usepackage[latin1]{inputenc}
\usepackage{amsmath}

\makeatletter

%%%%%%%%%%%%%%%%%%%%%%%%%%%%%% LyX specific LaTeX commands.
\newcommand{\noun}[1]{\textsc{#1}}

%%%%%%%%%%%%%%%%%%%%%%%%%%%%%% User specified LaTeX commands.



\usepackage{ijcai07}

\usepackage{times}

\usepackage{helvet}

\usepackage{courier}

\usepackage{amsthm}




\newcommand{\isdef}{\hbox{$\stackrel{\mbox{\tiny def}}{=}$}}


\newtheorem{theorem}{Theorem}



\makeatother
\begin{document}


\title{Multi-Agent Knowledge in the Situation Calculus with\\
 Partial Awareness of Actions}


\author{Ryan Kelly and Adrian Pearce\\
 Department of Computer Science and Software Engineering\\
 NICTA Victoria Laboratory\\
 The University of Melbourne\\
 Victoria, 3010, Australia\\
 \{rfk,adrian\}@csse.unimelb.edu.au}

\maketitle
\begin{abstract}
We develop significant improvements to the existing accounts of knowledge
in the situation calculus. Situation calculus models of multi-agent
domains typically make one of two assumptions: that all agents are
aware of all actions that have been performed, or that each agent
is completely ignorant of the actions performed by others. By reifying
the observations made by each agent as the world evolves, we develop
an enhanced account of knowledge that spans these two extremes. We
show that our approach can naturally model agents who passively monitor
the value of fluents in their environment, and develop a generalized
model of communication as sensing actions. Finally, we show how to
reason effectively in this formalism by extending the regression operator
to accommodate limited forms of quantification over situations. 
\end{abstract}

\section{Introduction}

The situation calculus \cite{McCHay69sitcalc,pirri99contributions_sitcalc}
is one of the most popular formalisms for reasoning about dynamic
worlds. Since its inception, many extensions have been proposed to
enrich its ontology with concepts such as knowledge \cite{scherl03sc_knowledge}
and concurrent actions \cite{reiter96sc_nat_conc}. These extensions
can be combined to provide a rich formalism for modeling multi-agent
domains.

An almost universal assumption when working with the situation calculus
is that agents are fully aware of the actions that have been performed
in the world. Indeed, the very notion of 'situation' is formalized
as a history of all actions that have occurred. In the rare cases
that this assumption is not employed (e.g. \cite{Lesperance99sitcalc_approach}),
the opposing extreme is posited - that agents are completely unaware
of any actions performed by others.

Both of these assumptions are unrealistic in many real-world domains.
Instead, we reify the notion of \emph{observations} and axiomatize
the conditions under which an agent will observe the occurrence of
an action. By ensuring that the agents consider possible any situation
that agrees with what they have observed, we develop an enhanced account
of knowledge that spans these two extremes, allowing a wider range
of domains to be modeled.

It is frequently infeasible for agents to directly determine what
actions have occurred. Instead, they may be aware simply that the
values of certain fluents have changed. By including a special class
of actions that notify agents about fluent change, we show that such
domains can be easily accommodated by our formalism.

Multi-agent systems almost invariably require communication between
agents, typically represented by explicit communication actions. Such
actions are intricately connected with the knowledge states of the
agents involved, and existing accounts of knowledge such as include
them explicitly in the dynamics of knowledge change \cite{shapiro01casl_feat_inter}.
However, it should not be necessary to alter such fundamental axioms
to add or modify communication actions. Using our generalized account
of knowledge we propose an account of communication actions as sensing,
allowing new communication actions to be introduced with ease.

Our new semantics or knowledge utilizes quantification over situation
terms, which is traditionally problematic for automated reasoning.
Despite this, we show that the standard tool for efficient automated
reasoning in the situation calculus - the regression operator \cite{reiter91frameprob}
- can be extended to handle our new semantics.


\section{The Multi-Agent Situation Calculus}

We work in a version of the the situation calculus as described in
\cite{pirri99contributions_sitcalc}, enriched with concurrent actions
\cite{reiter96sc_nat_conc} to more realistically represent the dynamics
of a multi-agent system. We use the approach of \cite{shapiro01casl_feat_inter}
for representing multiple agents, and begin from the standard account
of knowledge given by \cite{scherl03sc_knowledge,scherl03conc_knowledge}.
A brief overview is presented below.

There are a number of additional extensions to the situation calculus
that are useful for modeling multi-agent domains, in particular explicit
time and natural actions as described in \cite{reiter96sc_nat_conc}.
These extensions are compatible with our approach, and have been excluded
from this paper largely for space reasons.

The situation calculus is a many-sorted language of first-order logic,
augmented with a second-order induction axiom. Its has the following
types: \noun{Agent} terms (variables $agt_{y}^{x}$) represent the
agents operating in the world; \noun{Action} terms (variables $a_{y}^{x}$)
represent individual instantaneous events that can cause the state
of the world to change; \noun{Concurrent} terms (variables $c_{y}^{x}$)
are sets of actions that occur simultaneously; \noun{Situation} terms
(variables $s_{y}^{x}$) are histories of the concurrent actions that
have occurred in the world, with the initial situation represented
by $S_{0}$ and successive situations built up using the function
$do\,:\, Concurrent\times Situation\rightarrow Situation$; \noun{Result}
terms (variables $r_{y}^{x}$) represent sensing results returned
by actions, and are written inside double-quotes; \noun{Object} terms
(variables $obj_{y}^{x}$) represent any other object in the domain.
We also distinguish \noun{Fluents} as predicates or functions representing
properties of the world that may change from one situation to another.
Fluents take a situation term as their final argument.

A collection of situation calculus statements that describe the behavior
of a dynamic world is referred to as a \emph{theory of action} and
is typically represented by $\Sigma$. Queries about the behavior
or evolution of the world are posed as entailment queries relative
to the theory of action.


\subsubsection{Action Precondition Axioms}

There is a distinguished fluent predicate $Poss(a,s)$ that indicates
when it is possible to perform an action in a given situation. For
example, it is only possible for an agent to drop an object if they
are holding it: %
\footnote{We follow the convention that lower-case variables are implicitly
universally quantified %
} \begin{equation}
Poss(drop(agt,obj),s)\equiv holding(agt,obj,s)\end{equation}


For concurrent actions, an arbitrary combination of actions $\{ a_{1},a_{2}\}$
is not guaranteed to be possible. While the actions $moveTo(Chair)$
and $moveTo(Table)$ may individually be possible, performing them
concurrently clearly is not. This is known as the precondition interaction
problem \cite{reiter96sc_nat_conc} and is an area of ongoing research.
A simple solution is to explicitly state which actions cannot be performed
together using a predicate $Conflicts$: \begin{multline}
Poss(c,s)\equiv\\
\forall a\left[a\in c\rightarrow Poss(a,s)\right]\wedge\neg Conflicts(c,s)\end{multline}


For this work, we make no particular commitment toward a solution
to this problem.


\subsubsection{Successor State Axioms}

The truth of the various fluents is specified by defining what is
true of the initial situation, and collecting the effects of various
actions into \emph{successor state axioms}. Such axioms provide a
general solution to the infamous frame problem \cite{reiter91frameprob}.
They have the general form: \begin{multline}
F(\overrightarrow{x},do(c,s))\equiv\\
\Gamma^{+}(\overrightarrow{x},c,s)\,\vee\, F(\overrightarrow{x},s)\wedge\neg\Gamma^{-}(\overrightarrow{x},c,s)\end{multline}


Here $F$ is a fluent with non-situation arguments $\overrightarrow{x}$,
$\Gamma^{+}$ is a formula giving the conditions under which $F$
will become true, and $\Gamma^{-}$ a formula giving the conditions
under which $f$ will become false. In words, these axioms state that
{}``$F$ is true after doing actions $c$ if $c$ causes $F$ to
become true, or $F$ is currently true and $c$ doesn't cause $F$
to become false''.


\subsubsection{Uniform Formulae}

An important class of formulae in the situation calculus are the \emph{uniform
formulae} - those that refer to a single situation term. Such formulae
represent a property of a single situation. Throughout this paper,
we will use the meta-variable $\phi$ to refer to an arbitrary uniform
formula. Where no confusion can arise, we suppress the situation terms
in uniform formulae to simplify the presentation. (TODO: reference
encoding ufs as terms)

It is often useful to apply some uniform formula to a given situation
term. The formula $\phi[s]$ represents the uniform formula $\phi$
with all occurrences of its unique situation term replaced by the
situation $s$.


\subsubsection{Communication}

Inter-agent communication is typically modeled using special communication
actions. Representative of this approach are the $informWhether(agt_{1},agt_{2},\phi)$
and $informRef(agt_{1},agt_{2},\theta)$ actions of \cite{shapiro01casl_feat_inter},
whereby $agt_{1}$ informs $agt_{2}$ about the truth of a formula
or the referent of a function respectively. We shorten these to $infWhether$
and $infRef$ in our presentation below.


\subsubsection{Knowledge and Sensing}

The semantics of knowledge in the situation calculus \cite{scherl03sc_knowledge}
are based on the popular {}``possible worlds'' model. A knowledge
fluent $K(agt,s',s)$ is used to indicate that {}``in situation $s$,
the agent $agt$ considers the alternate situation $s'$ to be possible''.
A knowledge operator $\mathbf{Knows}$ is then introduced as a macro:
\begin{equation}
\mathbf{Knows}(agt,\phi,s)\isdef\forall s'\, K(agt,s',s)\rightarrow\phi[s']\label{eqn:knows_def}\end{equation}


Thus, an agent knows something if it is true in all situations that
the agent considers possible. This can be specialized with further
macros, such as an agent knowing whether or not some formula holds:
\begin{multline}
\mathbf{KWhether}(agt,\phi,s)\isdef\\
\mathbf{Knows}(agt,\phi,s)\vee\mathbf{Knows}(agt,\neg\phi,s)\end{multline}


To allow actions to return sensing information, the sensing result
function $SR(a,s)$ is introduced which gives the result returned
by the action $a$ when executed in a situation $s$. This is defined
by sensing result axioms of the form: \begin{equation}
SR(a(\overrightarrow{x}),s)=r\equiv\phi_{a}(\overrightarrow{x},r,t,s)\end{equation}


For example, consider the action $sense_{F}$ which senses whether
fluent $F$ is true or false. Its sensing result axiom would be given
by: \begin{multline}
SR(sense_{F},s)=r\equiv\\
(r="YES"\wedge F(s))\vee(r="NO"\wedge\neg F(s))\end{multline}


For actions that don't return sensing information, the value of $SR$
is set to some arbitrary constant such as $"OK"$.

Combining the multi-agent semantics of \cite{shapiro01casl_feat_inter}
with the handling of concurrent actions in \cite{scherl03conc_knowledge},
the common form of successor state axiom for the knowledge fluent
is: \begin{multline}
K(agt,s'',do(c,s))\equiv\\
\exists s'\,.\, s''=do(c,s')\,\wedge K(s',s)\wedge Poss(c,s')\\
\wedge\,\forall a\left[a\in c\wedge agent(a)=agt\rightarrow SR(a,s)=SR(a,s')\right]\\
\wedge\,\forall agt_{2},\phi\left[infWhether(agt_{2},agt,\phi)\in c\rightarrow\phi[s']\equiv\phi[s]\right]\\
\wedge\,\forall agt_{2},\theta\left[infRef(agt_{2},agt,\theta)\in c\rightarrow\theta[s']=\theta[s]\right]\end{multline}


In words, this states that $s''$ is considered a possible alternative
to $do(c,s)$ when $s''$ is the result of doing the actions in $c$
in a situation $s'$ that is a possible alternative to $s$. The situation
$s''$ must furthermore be possible, the sensing results must agree
for all actions in $c$ that were carried out by the agent, and the
situation must be consistent with what it has been informed by others.

Alternate possible worlds to the initial situation are identified
by truth of the predicate $K_{0}(s)$. This allows the $K$ fluent
to be fully specified, by asserting that: \begin{equation}
K(s,S_{0})\equiv K_{0}(s)\end{equation}


While a very powerful formalism, this successor state axiom still
suffers from several limitations:

\begin{itemize}
\item Each agent must be aware of \emph{all} actions that have occurred 
\item The semantics of communication actions are tightly bound to the account
of knowledge 
\end{itemize}
By adopting instead the form of successor state axiom proposed in
\cite{Lesperance99sitcalc_approach}, we get the opposite extreme:
agents are only aware of actions that they themselves perform: \begin{multline}
K(agt,s'',do(c,s))\equiv\\
\exists s',s^{*},c'\,.\, s''=do(c',s^{*})\,\wedge K(s',s)\wedge Poss(c',s^{*})\\
\wedge\,\mathbf{ExoOnly}(agt,s',s^{*})\\
\wedge\,\forall a\left[agent(a)=agt\rightarrow a\in c'\equiv a\in c\right]\\
\wedge\,\forall a\left[a\in c\wedge agent(a)=agt\rightarrow SR(a,s)=SR(a,s')\right]\\
\wedge\,\forall agt_{2},\phi\left[infWhether(agt_{2},agt,\phi)\in c\rightarrow\phi[s']\equiv\phi[s]\right]\\
\wedge\,\forall agt_{2},\theta\left[infRef(agt_{2},agt,\theta)\in c\rightarrow\theta[s']=\theta[s]\right]\end{multline}


Where the macro $\mathbf{ExoOnly}$ indicates that two situations
are connected only by actions performed by other agents: \begin{multline}
\mathbf{ExoOnly}(agt,s,s'')\,\isdef\, s\leq s''\wedge\\
\forall s',c,a\left[s<do(c,s^{*})\leq s''\wedge a\in c\rightarrow agent(a)\neq agt\right]\end{multline}


With this modified approach, agents consider possible any situation
which agrees on the actions that they have performed. Importantly,
there may have been an arbitrary sequence of situations between $s'$
and $s''$ which the agent knew nothing about, because they consisted
entirely of exogenous (from $agt$'s point of view) actions. Again,
there are limitations to this approach:

\begin{itemize}
\item Agents can \emph{never} be aware of the actions performed by others 
\item After performing an action, the agent does not allow for the possibility
that other agents have performed subsequent actions 
\end{itemize}
The last point involves some subtlety. Suppose that an agent $agt$
has just performed the action $a_{1}$, so the world is in some situation
$do(a_{1},s)$. Another agent then performs the action $a_{2}$, leaving
the world in the situation $do(a_{2},do(a_{1},s))$. However, $agt$
would not be aware that the state of the world has changed, so its
set of possible situations should remain the same. Unfortunately,
$K(s',do(a_{2},do(a_{1},s)))\not\equiv K(s',do(a_{1},s))$ under existing
formulations. To faithfully respect this aspect of the knowledge of
agents in the real world, the successor state axiom for $K$ must
permit any arbitrary future of $do(a_{1},s)$ that can be brought
about entirely through exogenous actions.

We overcome these limitations with the development of an improved
successor state axiom for $K$.


\section{New Semantics for Knowledge}


\subsubsection{Awareness of Actions}

The existing accounts of knowledge presented above all employ an assumption
about when an agent is aware of the occurrence of an action - either
{}``agents are always aware of actions'' or {}``agents are only
aware of actions that they themselves perform''. We replace such
assumptions with a question: {}``when will an agent be aware of the
occurrence of an action?''. By providing a way to answer this question
within the theory of action, a more general account of knowledge is
obtained.

Another distinguished fluent is introduced, akin to $Poss$ but describing
when actions will be observed by agents: $CanObs(agt,a,s)$ indicates
that agent $agt$ would be able to observe action $a$ being performed
in situation $s$. We assume that in an implemented system this {}``awareness''
is generated by some other component of the agent's software. The
important point is that if $CanObs(agt,a,s)$ is true and action $a$
occurs, then $agt$ will be made aware of it. Likewise, if $agt$
is not aware that action $a$ occurred, then it did not occur. The
axiomatisation of this predicate, like that of $Poss$, is the responsibility
of the domain modeler.

There is a related assumption implicit in the handling of sensing
actions: that only the agent performing a sensing action is aware
of its result. Such a restriction is common in the real world, but
certainly not universal. Consider an agent waiting for a train who
activates a speaker to acquire information about when it will arrive.
The results of this sensing action would be available to any other
agents within earshot. We add an analogous predicate $CanSense(agt,a,s)$
to indicate when the sensing information resulting from an action
is available to an agent.

Since we will often want to compare what is observed by an agent in
different situations, we introduce a new sort for \emph{sets of action-result
pairs}. We then define the function $Observations(agt,c,s)$ which
returns precisely the set of $<action,result>$ pairs that $agt$
would be aware of if the concurrent actions $c$ were performed in
situation $s$ . The special result term {}``?'' indicates that
the result of an action was not available. \begin{multline}
Observations(agt,c,s)=o\equiv\\
\forall a,r\,\left[<a,r>\in o\equiv a\in c\wedge CanObs(agt,a,s)\right.\\
\wedge\,\left(CanSense(agt,a,s)\wedge SR(a,s)=r\right.\\
\left.\vee\,\left.\neg CanSense(agt,a,s)\wedge r="?"\right)\right]\label{eqn:observations_orig}\end{multline}


Like their counterpart $Poss$, $CanObs$ and $CanSense$ may suffer
from interaction problems when concurrent actions are considered -
for example, one agent moving in front of another may mean the second
agent's actions are no longer observable. While we are confident an
approach similar to that used for precondition interaction can be
applied, this remains an open problem.


\subsubsection{Successor State Axiom for $K$}

What, then, are the considerations required for the successor state
axiom of $K$ for any particular agent? Generalizing the considerations
of \cite{scherl03sc_knowledge}, \cite{scherl03conc_knowledge}, \cite{Lesperance99sitcalc_approach}
and \cite{shapiro01casl_feat_inter}, situations considered possible
must:

\begin{itemize}
\item be successors to situations previously considered possible by the
agent 
\item be possible according to $Poss$ 
\item agree on all actions that the agent observed to occur 
\item agree with all sensing results that the agent observed 
\item admit the possibility of a sequence of actions that was totally unobservable
by the agent 
\end{itemize}
Having the observations made by each agent explicitly reified in the
language makes formalizing these considerations significantly easier.

As the resulting successor state axiom will be quite complex, we first
introduce the abbreviation $\mathbf{Unobs}(agt,s,s')$, which is true
when $agt$ would not be aware of any of the actions leading from
$s$ to $s'$: \begin{multline}
\mathbf{Unobs}(agt,s,s''))\,\isdef\,\neg\exists c',s'\,.\, s<do(c',s')\leq s''\\
\wedge\, Observations(agt,c',s')\neq\{\}\label{eqn:unobs_defn}\end{multline}


If the agent considers possible a situation $s$ then it must also
consider possible any situation $s'$ such that $\mathbf{Unobs}(s,s')$,
as it would be completely unaware of the world changing from $s$
to $s'$. Also note that $\mathbf{Unobs}(s,s)$ is always true.

Let us also introduce an additional predicate $Legal(s)$, which states
that a situation can legally be brought about in the world. This facilitates
the incorporation of additional extensions to the situation calculus,
such as natural actions. For the moment, it suffices that it constrains
actions to be possible: \begin{multline}
Legal(S_{0})\equiv True\\
\shoveleft{Legal(do(c,s))\equiv Poss(c,s)\wedge Legal(s)}\end{multline}


We then posit the following successor state axiom, capturing the considerations
outlined above. \begin{multline}
K(agt,s'',do(c,s))\equiv\\
Observations(agt,c,s)=\{\}\wedge K(agt,s'',s)\\
\shoveleft{\vee\,\,\exists o\, Observations(agt,c,s)=o\,\wedge}\\
o\neq\{\}\wedge Legal(s'')\,\wedge\\
\exists c',s'\, Observations(agt,c',s')=o\,\wedge\\
K(s',s)\wedge\mathbf{Unobs}(agt,do(c',s'),s'')\label{eqn:new_k_ssa}\end{multline}


Thus the agents consider possible any legal situation that agrees
with what they have observed. It remains to specify the truth of the
$K$ fluent in the initial situation. Assuming that alternate initial
situations are identified by the truth of $K_{0}(s)$ as in previous
works, then the following captures the intended semantics: \begin{equation}
K(s,S_{0})\equiv Legal(s)\wedge\exists s'\,\left[K_{0}(s')\wedge\mathbf{Unobs}(s',s)\right]\label{eqn:new_k_s0}\end{equation}


TODO: yack about how much better it is.

It is straightforward to verify the following properties of the $K$
relation:

{*} knowledge producing effects

{*} memory

{*} default persistence of ignorance

{*} ignorance of actions affected by unobservable actions

{*} persistence of transitivity, symmetry, reflexivity, euclidianness

TODO: is it? then sit down with pen and paper and do so!


\subsubsection{Awareness of Fluent Change}

In some domains, it may not be feasible for an agent to infer that
a particular action has taken place. For example, suppose that an
agent passively monitors the state of a light bulb in its environment,
such that it notices it changing from dark to illuminated. While it
knows that \emph{some} action must have occurred, it need not be aware
of precisely which action took place. Indeed, some may argue that
this is the most common case for agents operating in the physical
world - they are not directly aware of actions, but of changes to
fluents.

To model this, we introduce a special class of actions called \emph{change
actions}, whose sole purpose is to inform agents that some property
of the world has changed. They can never be executed and so never
appear in situation terms directly. Instead, they are included in
the list of observations made by an agent by modifying the $Observations$
predicate: \begin{multline}
Observations(agt,c,s)=o\equiv\\
\forall a,r\,\left[<a,r>\in o\equiv\right.\\
\left[a\in c\,\vee\, changeAction(a)\wedge SR(a,s)\neq SR(a,do(c,s))\right]\\
\wedge CanObs(agt,a,s)\\
\wedge\,\left(CanSense(agt,a,s)\wedge SR(a,s)=r\right.\\
\left.\vee\,\left.\neg CanSense(agt,a,s)\wedge r="?"\right)\right]\end{multline}


According to this definition, change actions appear in the list of
observations only when the corresponding sensing result changes in
the resulting situation. By making agents aware of these actions using
$CanObs$ and $CanSense$, our formalism neatly handles this form
of environmental awareness.


\subsubsection{Communication as Sensing}

The traditional incorporation of communication actions directly into
the successor state axiom for $K$ unnecessarily complicates the semantics
of knowledge and makes it more difficult to introduce new communication
actions. Our formulation provides a ready alternative - encode communication
actions are sensing actions where both sender and recipient are aware
of the outcome. Consider formulating the action $infWhether$ in this
framework: \begin{multline}
Poss(infWhether(agt_{1},agt_{2},\phi),s)\equiv\\
\shoveright{\mathbf{KWhether}(agt_{1},\phi,s})\\
\shoveleft{SR(infWhether(agt_{2},agt_{3},\phi),s)=r\equiv}\\
\shoveright{(r="YES"\wedge\phi[s])\,\vee\,(r="NO"\wedge\neg\phi[s])}\\
\shoveleft{CanObs(agt_{1},infWhether(agt_{2},agt_{3},\phi),s)\equiv}\\
\shoveright{agt_{1}=agt_{2}\,\vee\, agt_{1}=agt_{3}}\\
\shoveleft{CanSense(agt_{1},infWhether(agt_{2},agt_{3},\phi),s)\equiv}\\
agt_{1}=agt_{2}\,\vee\, agt_{1}=agt_{3}\end{multline}


It is straightforward to show that the intended semantics of this
communication are now enforced by the general successor state axiom
for $K$. Only sender and recipient will be aware that the communication
has occurred, and their state of knowledge will be updated to take
into account the communicated truth/falsity of $\phi$ in accordance
with $SR$.


\section{Reasoning}

The standard approach to reasoning in the situation calculus \cite{reiter91frameprob}
involves the regression meta-operator $\mathcal{R}$. This operator
takes a situation calculus formula referring to a situation term $do(c,s)$
and transforms it into an equivalent formula that refers only to situation
$s$. By repeatedly applying this operator, a formula that refers
only to $S_{0}$ can be obtained. As a result, many of the axioms
from the theory of action $\Sigma$ are not needed when reasoning
about this formula. The regression approach is employed for reasoning
about knowledge in \cite{scherl03sc_knowledge}, with the resulting
formula requiring modal reasoning about the initial situation.

Unfortunately, regression cannot be applied to formulae that quantify
over situations. A second-order induction axiom is required \cite{Reiter93proving},
making such formulae problematic for automated reasoning. The appearance
of the macro $\mathbf{Unobs}$ in equation \ref{eqn:new_k_ssa} means
that our new successor state axiom is not regressable using standard
techniques. To permit an effective reasoning procedure, we appeal
to the idea of {}``persistence conditions''.

In a companion paper (TODO:cite ourselves) we show how to calculate
the \emph{persistence condition} $\mathcal{P}(\phi,\alpha)$ of formula
$\phi$ with respect to actions satisfying conditions $\alpha$, such
that: \begin{multline}
\mathcal{P}(\phi,\alpha)[s]\equiv\forall s''\, s\leq s''\wedge\\
\neg\exists c',s'\, s<do(c',s')\leq s''\wedge\neg\alpha(c',s')\rightarrow\phi[s'']\label{eqn:P_defn}\end{multline}


The truth of $\mathcal{P}(\phi,\alpha)$ in a situation $s$ guarantees
that, as long as only actions that satisfy $\alpha$ occur, the formula
$\phi$ will remain true in all situations in the future of $s$.
Like the regression operator $\mathcal{R}$, $\mathcal{P}$ is a meta-operator
that transforms situation calculus formulae rather than a predicate
in the language. The procedure for determining $\mathcal{P}$ replaces
second-order induction with iterated first-order reasoning.

By having $\alpha$ single out unobservable actions, the persistence
condition can be combined with the techniques for regressing knowledge
of \cite{scherl03sc_knowledge}. Assume that the $K$ fluent appears
only in the context of a $\mathbf{Knows}$ macro. Then, from definition
\ref{eqn:knows_def}, we have: \begin{multline}
\mathbf{Knows}(agt,\phi,do(c,s))\equiv\\
\forall s''\, K(agt,s'',do(c,s))\rightarrow\phi[s'']\end{multline}


Employing the successor state axiom from equation \ref{eqn:new_k_ssa},
this can be re-written in the equivalent form: \begin{multline}
\mathbf{Knows}(agt,\phi,do(c,s))\equiv\\
\forall s''\left(Observations(agt,c,s)=\{\}\wedge K(agt,s'',s)\right.\\
\vee\,\,\exists o\, Observations(agt,c,s)=o\wedge\\
o\neq\{\}\wedge Legal(s'')\wedge\\
\exists c',s'\, Observations(agt,c',s')=o\\
\left.\wedge K(s',s)\wedge\mathbf{Unobs}(agt,do(c',s'),s'')\right)\,\rightarrow\,\phi[s'']\end{multline}


Through a series of equivalence-preserving transformations, this can
be re-written as: \begin{multline}
\mathbf{Knows}(agt,\phi,do(c,s))\equiv\\
\left(Observations(agt,c,s)=\{\}\rightarrow\forall s''\, K(agt,s'',s)\rightarrow\phi[s'']\right)\\
\wedge\left(\exists o\,.\, Observations(agt,c,s)=o\wedge o\neq\{\}\rightarrow\right.\\
\forall s'\, K(s',s)\rightarrow\forall c'\, Observations(agt,c',s')=o\rightarrow\\
\left.\forall s''\, Legal(s'')\wedge\mathbf{Unobs}(agt,do(c',s'),s'')\,\rightarrow\,\phi[s'']\right)\end{multline}


Note that both of these conjuncts now contain sub-formulae matching
the form of the $\mathbf{Knows}$ macro. It can thus be substituted
back in to give: \begin{multline}
\mathbf{Knows}(agt,\phi,do(c,s))\equiv\\
\left(Observations(agt,c,s)=\{\}\rightarrow\mathbf{Knows}(agt,\phi,s)\right)\\
\wedge\left(\exists o\,.\, Observations(agt,c,s)=o\wedge o\neq\{\}\rightarrow\right.\\
\left.\mathbf{Knows}(agt,\mathbf{PK}(\phi,o,s'),s)\right)\end{multline}


Where we have defined the abbreviation $\mathbf{PK}(\phi,o,s')$ as:
\begin{multline}
\mathbf{PK}(\phi,o,s')\isdef\forall c'\, Observations(agt,c',s')=o\rightarrow\\
\forall s''\, Legal(s'')\wedge\mathbf{Unobs}(agt,do(c',s'),s'')\,\rightarrow\,\phi[s'']\end{multline}


However, for $\mathbf{PK}(\phi,o)$ to appear in the $\mathbf{Knows}$
macro it must be uniform in the situation variable $s'$. This can
be achieved using the regression and persistence operators. Noting
that equation \ref{eqn:unobs_defn} matches the form of equation \ref{eqn:P_defn},
and regressing to make the expression uniform in $s'$, we develop
the following equivalence: \begin{multline}
\mathbf{PK}(\phi,o,s')\equiv\forall c'\, Observations(agt,c',s')=o\rightarrow\\
\mathcal{R}(\mathcal{P}(\phi,Legal()\wedge Observations(agt)=\{\})[do(c',s')])\end{multline}


We have thus reduced reasoning about knowledge in $do(c,s)$ to reasoning
about knowledge in $s$. This is precisely the behavior required by
the regression operator. TODO: formalize, refer back to other clauses

Additional care must be taken when dealing with knowledge in the initial
situation. By a straightforward transformation on of equations \ref{eqn:knows_def}
and \ref{eqn:new_k_s0}: \begin{multline}
\mathbf{Knows}(\phi,S_{0})\equiv\\
\forall s\, K_{0}(s)\rightarrow\{\forall s'\, Legal(s')\wedge\mathbf{Unobs}(s,s')\rightarrow\phi[s']\}\end{multline}
 By applying the persistence condition operator, this can be reduced
to reasoning about a uniform formula: 

\begin{multline}
\mathbf{Knows}(\phi,S_{0})\equiv\\
\forall s\, K_{0}(s)\rightarrow\mathcal{P}(\phi,Legal()\wedge Observations(agt)=\{\})[s]\end{multline}


This is standard modal reasoning over the $K_{0}$ relation, as employed
by \cite{scherl03sc_knowledge} for determining knowledge in the initial
situation. TODO: wind up this section better


\section{Conclusions}

In this paper we have significantly increased the scope of the situation
calculus for modeling knowledge in multi-agent domains. By explicitly
reifying the observations made by each agent as the world evolves,
we have generalized the dynamics of knowledge update to allow partial
awareness of actions. We have shown that this can accommodate the
important case where agents are aware that a property of their environment
has changed, but do not know the precise actions responsible. A pleasing
offshoot of this generalization is that communication actions, previously
tied intimately to the dynamics of knowledge, can now be added or
modified with the same ease as regular actions. Despite requiring
universal quantification over future situations, we have shown that
the regression operator - the standard tool for efficient reasoning
in the situation calculus - can be adapted to our new formalism.

With our improved semantics of knowledge, the situation calculus is
well positioned for representing and reasoning about a much broader
range of more realistic multi-agent domains.

\bibliographystyle{named} \bibliographystyle{named}
\bibliography{/storage/uni/pgrad/library/references}


\end{document}
