\documentclass{ifaamas-submission}

\makeatletter

\newcommand{\noun}[1]{\textsc{#1}}
\newcommand{\isdef}{\hbox{$\stackrel{\mbox{\tiny def}}{=}$}}
\newtheorem{theorem}{Theorem}
\newdef{definition}{Definition}

\makeatother
\begin{document}

\conferenceinfo{AAMAS'07} {May 14--18 2007, Honolulu, Hawai'i, USA.}
\CopyrightYear{2007}

\title{Knowledge and Observations in the Situation Calculus}

\numberofauthors{1}

\author{
\alignauthor
Ryan F. Kelly and Adrian R. Pearce\\
\affaddr{NICTA Victoria Research Laboratory}\\
\affaddr{Department of Computer Science and Software Engineering}\\
\affaddr{The University of Melbourne}\\
\affaddr{Victoria, 3010, Australia}\\
\email{\{rfk,adrian\}@csse.unimelb.edu.au}
}

\maketitle

\begin{abstract}
We present a powerful new account of multi-agent knowledge in the
situation calculus and an effective reasoning procedure for handling
knowledge queries. Our approach generalizes existing work by reifying
the observations made by each agent as the world evolves, allowing
for agents that are partially or completely unaware of some of the
actions that have occurred. This also enables agents to reason effectively
about knowledge using only their internal history of observations,
rather than requiring a full history of the world. The result is a
more robust and flexible account of knowledge suitable for use in
partially-observable multi-agent domains. 
\end{abstract}

\category{I.2.4}{Artificial Intelligence}{Knowledge Representation Formalisms and Methods}

\terms{Theory}

\keywords{Situation Calculus, Knowledge, Action, Observability}

\section{Introduction}

The situation calculus \cite{pirri99contributions_sitcalc},
 extended with knowledge \cite{scherl03sc_knowledge} and concurrent actions
\cite{scherl03conc_knowledge}, provides a rich formalism for modeling complex
domains such as multi-agent systems.  However, existing techniques for
effective reasoning require an omniscient viewpoint, with queries posed
relative to the current situation.
This makes it difficult for situated agents to reason about their environment
unless they have a complete history of the world, which is unrealistic in many domains.

We overcome this limitation by explicitly reifying the observations made by
each agent as the world evolves. Knowledge is axiomatised such that each agent
considers possible any situation compatible with what it has observed.
We then extend the traditional definition of regression
over knowledge queries to operate using the agent's history of
observations rather than a full situation term.  Our approach thus
allows a situated agent to reason effectively about its own knowledge of
the world even when
it is only partially aware of the actions that have occurred.

\section{Background}

Our work utilizes the situation calculus \cite{pirri99contributions_sitcalc} 
enriched with concurrent actions \cite{scherl03conc_knowledge} and multiple
agents. A \emph{basic action theory} is represented by $\mathcal{D}$.
We utilize the notion of \emph{action description predicates} as in
\cite{kelly07sc_persistence}, which include the familiar action possibility
predicate $Poss$.
The ordering relation $s <_{\alpha} s'$ should be read as {}``$s'$ is in
the future of $s$, and all intermediate actions satisfy $\alpha$''.
Situations satisfying $S_0 \leq_{Poss} s$ are termed {}``legal situations''
and are identified using $Legal(s)$.
The \emph{uniform formulae} \cite{pirri99contributions_sitcalc}
 represent properties of the state of the world.
We write $\phi$ for an arbitrary uniform
formula and $\phi[s]$ for a uniform formula with its situation
term replaced by $s$.

The standard semantics of knowledge 
\cite{scherl03sc_knowledge,scherl03conc_knowledge} are
based on a {}``possible worlds'' formulation. A knowledge fluent
$K(agt,s',s)$ indicates that {}``in situation $s$, the
agent $agt$ considers the alternate situation $s'$ to be possible''.
The macro $\mathbf{Knows}$ then acts as a shorthand for
the standard possible-worlds definition of knowledge:
\begin{equation*}
\mathbf{Knows}(agt,\phi,s)\,\isdef\,\forall s'\,.\, K(agt,s',s)\rightarrow\phi[s']\label{eqn:knows_def}
\end{equation*}

The successor state axiom for the knowledge fluent (with
multiple agents and concurrent actions) is:
\begin{multline*}
K(agt,s'',do(c,s))\equiv\\
\exists s'\,.\, s''=do(c,s')\,\wedge K(agt,s',s)\wedge Poss(c,s')\\
\wedge\,\forall a\in c.\left[agent(a)=agt\rightarrow SR(a,s)=SR(a,s')\right]\label{eqn:k_ssa_standard}
\end{multline*}
 
It is also necessary to permit alternate initial situations, which are
identified by $Init(s)$. Every situation is \emph{rooted at}
some initial situation, identified using $Root(s)$.

While powerful, this formulation has an important limitation: each
agent is assumed to be aware of all actions that have occurred.
While this is clearly infeasible for many multi-agent domains, it permits
the regression operator \cite{pirri99contributions_sitcalc} to be used
for effective reasoning about knowledge queries.
For formulations that do not make this assumption (see e.g. 
 \cite{Lesperance99sitcalc_approach}) the successor state axiom for $K$
involves universal quantification over situation terms and standard
effective reasoning techniques cannot be applied.

To construct a more general formalism while retaining regression as
a reasoning tool, we utilise the recently developed notion of 
\emph{persistence condition} \cite{kelly07sc_persistence}. The operator
$\mathcal{P}_{\mathcal{D}}(\phi,\alpha)$ produces a uniform formula
that will hold in $s$ if and only if $\phi$ will remain true in all future 
situations brought about by actions 
satisfying $\alpha$ (we say $\phi$ \emph{persists} under $\alpha$):
\begin{equation*}
\mathcal{D} \models \left(\forall s'\,.\,s \leq_{\alpha} s' \rightarrow \phi[s']\right)\,\,\equiv\,\, \mathcal{P}_{\mathcal{D}}(\phi,\alpha)[s]
\end{equation*}

\section{New Semantics of Knowledge}

\label{sec:New-Semantics}

To generalize the existing account of knowledge in a robust way, we introduce 
a distinction between \emph{actions}, which cause changes to the state
of the world, and \emph{observations}, which cause an agent to become
\emph{aware} of some change in the state of the world.
\begin{definition}
An \emph{observation} is a notification received by an agent that
makes it aware of some change in the state of the world. When an agent
receives such a notification, we say that it {}``observed'' or {}``perceived''
that observation.
\end{definition}
For simplicity we assume that agents perceive observations 
in the same instant as the actions causing them. We make
no commitment as to how these notifications are generated, preferring
a clean delineation between the task of observing change and the dynamics
of knowledge update.

Let us introduce an additional sort \noun{Observation}
to the situation calculus, for the moment without
commitment towards what this sort will contain. We
then introduce the function $Obs(agt,c,s)=o$, returning
a set of observations, to mean {}``when the actions $c$ are performed
in situation $s$, agent $agt$ will perceive the observations $o$''.

The concept of an \emph{observation history} follows naturally - it
is the sequence of all observations made by an agent as the world
has evolved. Using $\epsilon$ to represent the empty history,
the function $ObsHist$ can be defined  to give the observation history
associated with a particular situation:
\begin{multline}
Init(s)\rightarrow ObsHist(agt,s)=\epsilon\\
\shoveleft{ObsHist(agt,do(c,s))=h\,\,\equiv\,\,}
\exists o\,.\, Obs(agt,c,s)=o\\
\shoveright{\wedge\,\left(o=\{\}\rightarrow h=ObsHist(agt,s)\right)}\\
\wedge\,\left(o\neq\{\}\rightarrow h=o\cdot ObsHist(agt,s)\right)\label{eqn:obshist_defn}\end{multline}

There is a strong analogue between situations and observation histories.
A situation represents a complete, global history of all the actions
that have occurred in the world, while an observation history is an
agent's local history of all the observations it has made. The situation
is an omniscient view of the world, the observation history a local
view.

It is a basic tenet of epistemic reasoning that an agent's knowledge must
depend solely on its local history: the knowledge that it started out with
combined with the observations it has made since then.
Clearly, $s'$ should be related to $s$ if their root situations are
$K$-related, they result in the same sequence of observations, and 
$s'$ is legal:
\begin{multline}
\label{eq:k-desired}
\mathcal{D}\models K(agt,s',s)\equiv K(Root(s'),Root(s))\,\wedge\\
Legal(s')\,\wedge\, ObsHist(agt,s')=ObsHist(agt,s)
\end{multline}

Unfortunately this formulation cannot be used directly in a basic action
theory, as these require that fluent change be specified
using successor state axioms.  We must formulate an appropriate axiom for
$K$ which enforces equation (\ref{eq:k-desired}).

We first introduce $PbU(agt,c,s)$ as a notational convenience for {}``possible
but unobservable'',
indicating that the actions $c$ are possible in $s$, but no observations
would be perceived by the agent $agt$ if they are performed:
\begin{equation}
PbU(agt,c,s)\equiv
Poss(c,s)\wedge Obs(agt,c,s)=\{\}\label{eq:PbU_defn}
\end{equation}

By stating that $s\leq_{PbU(agt)}s'$ we assert that an agent would
perceive no observations were the world to change from situation $s$ to
$s'$. If it considers $s$ possible then it must also consider $s'$ possible.
 Following this intuition, the successor state axiom below
captures the desired dynamics of $K$:
\begin{multline}
K(agt,s'',do(c,s))\equiv\\
\shoveleft{\,\,\,\,\,\,\,\,\,\,\left[\, Obs(agt,c,s)=\{\}\rightarrow K(agt,s'',s)\,\right]}\\
\shoveleft{\,\,\wedge\,\,\,\left[\, Obs(agt,c,s)\neq\{\}\rightarrow\right.}\\
\exists c',s'\,.\, Obs(agt,c',s')=Obs(agt,c,s)\\
\left.\wedge\, Poss(c',s')\wedge K(agt,s',s)\wedge do(c',s')\leq_{PbU(agt)}s''\,\right]\label{eqn:new_k_ssa}
\end{multline}

If $c$ was totally unobservable, the agent's state of knowledge does
not change. Otherwise, it considers possible any legal successor to
a possible alternate situation $s'$ that can be brought about by
actions $c'$ that result in identical observations. It also considers
possible any future of such a situation in which is would perceive no
more observations.

Since situations satisfying
 $S_{0}\leq_{PbU(agt)}s$ holds must be $K$-related to $S_{0}$ for
equation (\ref{eq:k-desired}) to be satisfied, we
cannot use $K$ to encode the agent's initial knowledge.
We introduce $K_{0}$ for this purpose as follows:
\begin{multline}
Init(s)\rightarrow\\
K(agt,s'',s)\equiv \exists s'\,.\, K_{0}(agt,s',s)\wedge s'\leq_{PbU(agt)}s''
\label{eqn:new_k_s0}
\end{multline}

This axiomatisation is enough to ensure that knowledge meets to requirements
set out in equation (\ref{eq:k-desired}).

\begin{theorem}
\label{thm:k_obs_equiv} For any basic action theory $\mathcal{D}$ axiomatising
knowledge according to equations (\ref{eqn:new_k_ssa},\ref{eqn:new_k_s0}):
\begin{multline*}
\mathcal{D}\models K(agt,s',s)\equiv K(Root(s'),Root(s))\,\wedge\\
Legal(s')\,\wedge\, ObsHist(agt,s')=ObsHist(agt,s)
\end{multline*}
\end{theorem}
\begin{proof}
Straightforward, using equations (\ref{eqn:obshist_defn},\ref{eq:PbU_defn},\ref{eqn:new_k_ssa},\ref{eqn:new_k_s0}). 
\end{proof}

Using this new formulation, an agent's knowledge is completely decoupled
from the global notion of actions, instead depending only on the local
information that it has observed. By allowing different kinds of term in
the \noun{Observation} sort, and axiomatising the $Obs$ function in
different ways, a wide variety of domains can be modelled.

Let us begin by considering the standard account of knowledge from
\cite{scherl03sc_knowledge}. Its basic assumption that {}``all agents
are aware of all actions'' can be captured by allowing the \noun{Observation}
sort to contain \noun{Action} terms and including the following as an axiom:
\begin{equation}
a\in Obs(agt,c,s)\equiv a\in c\label{eq:ax_obs_std1}
\end{equation}

What about sensing information? We can extend the \noun{Observations}
sort to contain terms of the form \emph{$(Action=Result)$} and axiomatize
like so:
\begin{multline}
(a=r)\in Obs(agt,c,s)\equiv\\
a\in c\wedge SR(a,s)=r\wedge agent(a)=agt\label{eq:ax_obs_std2}
\end{multline}

Using equations (\ref{eq:ax_obs_std1},\ref{eq:ax_obs_std2}) as axioms, 
our new account of knowledge behaves
identically to the standard account.
To generalize this for partial observability of actions we introduce
a new action description predicate specifying
when actions will be observed by agents: $CanObs(agt,a,s)$ indicates
that agent $agt$ would observe action $a$ being performed in situation
$s$. We can then formulate $Obs()$ according to:
\begin{equation*}
a\in Obs(agt,c,s)\equiv a\in c\wedge CanObs(agt,a,s)
\end{equation*}

There is an additional assumption in the standard handling of sensing
actions: only the agent performing a sensing action is aware of its
result. We can lift this assumption by introducing an analogous
predicate $CanSense(agt,a,s)$ to indicate when sensing information
is available to an agent. We then include bare action terms in an
agent's observations when it observes the action but not its result,
and \emph{(Action=Result)} terms when it also senses the result:
\begin{multline*}
a\in Obs(agt,c,s)\equiv a\in c\\
\wedge CanObs(agt,a,s)\wedge\neg CanSense(agt,a,s)\end{multline*}
\begin{multline*}
(a=r)\in Obs(agt,c,s)\equiv a\in c\wedge SR(a,s)=r\\
\wedge CanObs(agt,a,s)\wedge CanSense(agt,a,s)
\end{multline*}

This allows one to explicitly axiomatize the conditions under which an agent
will be aware of the occurrence of an action. Even more complex domains can
be modelled by further refining the $Obs$ function, without modifying the
dynamics of knowledge update.

\section{Reasoning}

\label{sec:Reasoning}

Standard regression techniques cannot be applied to our formalism, 
since equation (\ref{eqn:new_k_ssa}) uses $\leq_{PbU}$ and so universally 
quantifies over situations.
We have developed an effective reasoning procedure
by using the persistence condition meta-operator \cite{kelly07sc_persistence}
to augment the regression techniques in \cite{scherl03sc_knowledge}.

Assuming that the knowledge fluent $K$ appears only in the context
of a $\mathbf{Knows}$ macro, we propose the following to replace
the regression clause in \cite{scherl03sc_knowledge} for $\mathbf{Knows}$:
\begin{multline}
\mathcal{R}_{\mathcal{D}}(\mathbf{Knows}(agt,\phi,do(c,s)))=\\
\exists o\,.\, Obs(agt,c,s)=o
\wedge\left[o=\{\}\rightarrow\mathbf{Knows}(agt,\phi,s)\right]\\
\wedge\,\left[o\neq\{\}\rightarrow\mathbf{Knows}(agt,\forall c'.\, Obs(agt,c',s)=o\right.\\
\left.\wedge Poss(c',s)\rightarrow\mathcal{R}_{\mathcal{D}}(\mathcal{P}_{\mathcal{D}}(\phi,PbU(agt))[do(c',s)]),s)\right]\label{eqn:R_do_c_s}
\end{multline}
As required, this reduces a knowledge query at $do(c,s)$ to a knowledge
query at $s$. It is also intuitively appealing: to know that $\phi$
holds, the agent must know that in all situations that agree with
its observations, $\phi$ cannot become false without it perceiving an
observation.

We must also specify the regression of $\mathbf{Knows}$ in the initial
situation, as equation (\ref{eqn:new_k_s0}) also uses $\leq_{PbU}$.
This clause results in standard first-order modal reasoning
over the $K_{0}$ relation, similarly to \cite{scherl03sc_knowledge}:
\begin{multline}
\mathcal{R}_{\mathcal{D}}(\mathbf{Knows}(agt,\phi,S_{0}))=\\
\forall s\, K_{0}(agt,s,S_{0})\rightarrow\mathcal{P}_{\mathcal{D}}(\phi,PbU(agt))[s]\label{eqn:R_s0}
\end{multline}

The proof that our modified regression
operator in equations (\ref{eqn:R_do_c_s},\ref{eqn:R_s0}) preserves
equivalence proceeds by expanding the definition for $\mathbf{Knows}$
using our new successor state axiom for $K$, collecting sub-formulae
that match the form of the $\mathbf{Knows}$ macro, and using regression
and persistence to render the resulting knowledge expression uniform
in $s$.  Space restrictions prohibit a detailed exposition.

We can thus handle knowledge queries using regression, the standard
technique for effective reasoning in the situation calculus.
However, it would be unreasonable for a situated agent to ask {}``do
I know $\phi$ in the current situation?'' using the situation calculus
query $\mathcal{D}\models\mathbf{Knows}(agt,\phi,s)$, as it cannot
be expected to have the full current situation $s$. However, it will
have its current observation history $h$.
The regression rules in equations (\ref{eqn:R_do_c_s},\ref{eqn:R_s0})
can be modified to operate using an observation history rather than a
situation term, with the result being significantly simpler due to the
absence of empty observation sets:
\begin{multline}
\mathcal{R}_{\mathcal{D}}(\mathbf{Knows}(agt,\phi,o\cdot h))=\\
\mathbf{Knows}(agt,\forall c\,.\, Obs(agt,c,s)=o\\
\wedge Poss(c,s)\rightarrow\mathcal{R}_{\mathcal{D}}(\mathcal{P}_{\mathcal{D}}(\phi,PbU(agt))[do(c,s)]),h)
\end{multline}
\begin{multline}
\mathcal{R}_{\mathcal{D}}(\mathbf{Knows}(agt,\phi,\epsilon))=\\
\forall s\,.\, K_{0}(agt,s,S_{0})\rightarrow\mathcal{P}_{\mathcal{D}}(\phi,PbU(agt))[s]
\end{multline}

It is a straightforward consequence of Theorem \ref{thm:k_obs_equiv}
that $\mathcal{R}_{\mathcal{D}}(\mathbf{Knows}(agt,\phi,s))$  and
$\mathcal{R}_{\mathcal{D}}(\mathbf{Knows}(agt,\phi,ObsHist(s)))$
are equivalent.
Agents can thus reason about their own knowledge
using only their local information.

\section{Conclusions}

\label{sec:Conclusions}In this paper we have significantly increased
the scope of the situation calculus for modeling knowledge in complex
domains, where there may be multiple agents and partial observability
of actions. By explicitly reifying the observations made by each agent
as the world evolves, we have generalized the dynamics of knowledge
update. Despite requiring
universal quantification over future situations, we have shown that
the regression operator can be adapted to allow effective reasoning
within our new formalism. It can also be used to reason from the internal
perspective of a single agent, allowing agents to reason about their
own world based solely on their local information.
With our new semantics of knowledge, the situation calculus is well
positioned for representing, reasoning about, and implementing more
complex, realistic multi-agent systems.

\bibliographystyle{abbrv}
\bibliography{/storage/uni/pgrad/library/references}


\end{document}
