%\documentclass[handout,a4paper]{beamer}
\documentclass{beamer}

\usepackage{amsmath}
\usepackage{graphicx}

\newcommand{\isdef}{\hbox{$\stackrel{\mbox{\tiny def}}{=}$}}

\mode<presentation>{\usetheme{Dresden}}

\title{Knowledge, Observation,\\ and the Frame Problem}
\author[Ryan Kelly (rfk@csse.unimelb.edu.au)]{Ryan Kelly \and Adrian Pearce}
\begin{document}

\begin{frame}
  \titlepage
\end{frame}

\begin{frame}
  \frametitle{Outline}
  \tableofcontents
\end{frame}

\section{Motivation}

\begin{frame}
\frametitle{Motivation}
What do we want?
\begin{itemize}
\item A powerful account of multi-agent knowledge
  \begin{itemize}
  \item Concurrent Actions
  \item Partial Observability
  \end{itemize}
\item that supports complex symbolic reasoning
\item that allows agents to reason about their own world
\item but is still computationally feasible
\end{itemize}
\end{frame}


\section{Knowledge}

\begin{frame}
\frametitle{Knowledge}

Knowledge is a modal operator
\begin{itemize}
\item Typically represented as: $\mathbf{Knows}(\phi)$ or $\mathbf{Knows}(Thomas,\phi)$
\item $\neg\mathbf{Knows}(\phi)\ \ \not\rightarrow\ \ \mathbf{Knows}(\neg\phi)$
\end{itemize}
\ \\
\ \\
Knowledge is "true belief":
\[ \mathbf{Knows}(\phi)\ \rightarrow\ \phi \]
\ \\
\ \\
Thus, it makes sense to approach knowledge using modal logic.
Knowledge is given semantics in terms of "possible worlds".

\end{frame}

\begin{frame}
\frametitle{Possible Worlds Semantics}

If an agent is unsure about the state of the world, there must be several
different states of the world that it considers possible.

\begin{center}
  \includegraphics[scale=0.4]{poss_worlds.png}
\end{center}

The agent \emph{knows} $\phi$ iff $\phi$ is true in all possible worlds.

\end{frame}

\begin{frame}
\frametitle{Knowledge and Action}
"Possible worlds" works well for reasoning about a static knowledge base,
but what if the world itself is changing?
\begin{itemize}
\item  Does the agent \emph{know} that the world has changed?
\item  How should the worlds considered possible change in response?
\item  What of all the associated reasoning-about-actions problems? (frame, quantification, ramification, ...)
\end{itemize}
\ \\
\ \\
Need to integrate with a powerful \emph{theory of action}.
\pause
\ \\
\ \\
I happen to know just the thing...
\end{frame}

\section{Knowledge \& Action}

\begin{frame}
\frametitle{The Situation Calculus}
\emph{Actions} are instantaneous events causing the world to change
\begin{itemize}
  \item $pickup(Thomas,Bowl)$, $beginTask(Richard,mix(Bowl,1))$
\end{itemize}
\emph{Situations} are histories of actions that have been performed
\begin{itemize}
  \item $S_0$, $do(pickup(Harriet,Knife),S_0)$
\end{itemize}
\emph{Fluents} are situation-dependent properties of the world
\begin{itemize}
  \item $Poss(a,s)$, $Holding(Harriet,Knife,s)$
\end{itemize}
\emph{Successor State Axioms} define what holds after an action in terms of what held before the action
\begin{itemize}
  \item $Holding(agt,obj,do(a,s)) \equiv a = pickup(agt,obj)$ \\
        $\,\,\,\,\,\,\,\,\,\,\,\,\,\vee Holding(agt,obj,s) \wedge \neg\left(a = drop(agt,obj)\right)$
\end{itemize}
\end{frame}

\begin{frame}
\frametitle{Knowledge in the Situation Calculus}

Why the situation calculus?
\begin{itemize}
\item Elegant monotonic solution to the frame problem
\item Effective reasoning procedure
\item "situations" provide a direct analog to "possible worlds"
\end{itemize}
\ \\
\ \\
Approach due (mainly) to Scherl and Levesque:\\
"Knowledge, Action and the Frame Problem", AI, 2003

\end{frame}

\begin{frame}
\frametitle{Possible Situations}
Recall that all statements about the world are relative to a situation.
So, we must talk about the knowledge of an agent in a particular situation.
\begin{equation*}
\mathbf{Knows}(agt,\phi,s)
\end{equation*}
\pause
Replace "possible worlds" with "possible situations".  Introduce a fluent
$K(agt,s',s)$ meaning that "in situation $s$, the agent $agt$ considers it
possible that the world may be in situation $s'$".
\end{frame}

\begin{frame}
\frametitle{Possible Situations}
\begin{center}
  \includegraphics[scale=0.5]{k_relation.png}
\end{center}

We can then define knowledge as a simple macro:
\[ \mathbf{Knows}(agt,\phi,s)\ \isdef\ \forall s'\left[K(agt,s',s)\rightarrow \phi(s')\right] \]
\end{frame}

\begin{frame}
\frametitle{The Frame Problem}
To specify what the agent knows initially, restrict the properties of situations
K-related to $S_0$:
\begin{gather*}
  \mathbf{Knows}(Richard,\neg Holding(Thomas,Knife),S_0) \\
  \Rightarrow \ \ \forall s \left[K(Richard,s,S_0) \rightarrow \neg Holding(Thomas,Knife,s) \right]
\end{gather*}

\pause
Since $K$ is just a fluent, we specify how it changes  between situations
with a successor state axiom:
\begin{equation*}
 K(s'',do(a,s)) \equiv \exists s' . \ s''=do(a,s')
 \wedge K(s',s) \wedge Poss(a,s')
\end{equation*}

Automatically inherits the solution to the frame problem!
\end{frame}

\begin{frame}
\frametitle{The Frame Solution}
\begin{center}
  \includegraphics[scale=0.5]{frame_soln.png}
\end{center}
\end{frame}

\begin{frame}
\frametitle{Sensing}
Agents can acquire additional information about their environment by performing
"sensing actions" which return a result.  The relationships between actions
and their sensing results are captured by a new predicate $SR$:
\begin{equation*}
SR(isNiceDay,s) = "YES" \equiv \neg Raining(s) \wedge \neg Windy(s)
\end{equation*}
These are easily incorporated into the successor state axiom:
\begin{multline*}
 K(s'',do(a,s)) \equiv \exists s' . \ s''=do(a,s') \\
 \wedge K(s',s) \wedge Poss(a,s') \wedge SR(a,s) = SR(a,s')
\end{multline*}
\end{frame}

\begin{frame}
\frametitle{Effective Reasoning}
The standard tool for effective reasoning in the S.C. is the
regression operator $\mathcal{R}$. If fluent $F$ has the successor state axiom
$F(do(a,s)) \equiv \Phi(a,s)$, then:

\[ \mathcal{R}[F(do(a,s))] = \Phi(a,s) \]

\pause
In general:

\[  \mathcal{D} \models \phi(do(a,s)) \equiv \mathcal{R}[\phi(do(a,s))] \]

But the regressed term refers to $s$ rather than $do(a,s)$.

\end{frame}

\begin{frame}
\frametitle{Regression}
By repeatedly applying $\mathcal{R}$, we reach a fluent expression referring
only to $S_0$.  This can be decided without many of the axioms in $\mathcal{D}$:
\begin{equation*}
  \Sigma \cup \mathcal{D}_{S_0} \cup \mathcal{D}_{Poss} \cup \mathcal{D}_{ssa} \cup \mathcal{D}_{una} \models \phi(s)
\end{equation*}
\begin{center}
iff
\end{center}
\begin{equation*}
  \mathcal{D}_{S_0} \cup \mathcal{D}_{una} \models \mathcal{R}^{*}[\phi(s)](S_0)
\end{equation*}
This is usually easier to answer, since regression has already done some
of the reasoning for us.
\end{frame}

\begin{frame}
\frametitle{Regressing Knowledge}
Unfortunately, $\mathcal{R}$ cannot be applied to formulae that quantify
over situations.  We must therefore special-case
$\mathcal{R}[\mathbf{Knows}(\phi,s)]$.
\ \\
\ \\
Based on the successor state axioms for $K$, Scherl and Levesque develop
the following regression rule:
\begin{multline*}
  \mathcal{R}[\mathbf{Knows}(\phi,do(a,s))]\ =\ 
    \exists y . SR(a,s) = y \wedge \\
    \mathbf{Knows}(Poss(a) \wedge SR(a)=y\ \rightarrow\ \mathcal{R}[\phi(do(a,s))],s)
\end{multline*}

We can thus transform a query about knowledge in an arbitrary situation to
a query about knowledge in the initial situation, which can (as before) be
tackled more effectively.

\end{frame}

\begin{frame}
\frametitle{Shortcomings}
This is a very powerful framework...but:
\begin{itemize}
 \item Does not handle concurrent actions
 \item Agents are assumed to be aware of \emph{all} actions that occur
 \item Requires omniscient viewpoint
\end{itemize}
\ \\
\ \\
\pause
In short, it's not suitable for complex multi-agent domains.
\ \\
\ \\
Can we do better, while maintaining the theoretical and practical elegance
of this approach?
\end{frame}

\section{Observations}

\begin{frame}
\frametitle{Concurrent Actions}
Basically, replace actions with sets of actions that all occur at the same
instant:
\begin{equation*}
  do(\left\{pickup(Richard,Knife),drop(Thomas,Bowl)\right\},s)
\end{equation*}
Agents may not be aware of all actions that have ocurred.  In fact, they
may not even be aware that \emph{any} actions have occurred.
\ \\
\ \\
When a set of actions occurs, what will each agent observe?
\end{frame}

\begin{frame}
\frametitle{Observations}
Well, let's talk about it explicitly:
\begin{equation*}
  Obs(agt,c,s) = \{o1,o2,...\}
\end{equation*}
Note that it's possible for $Obs(agt,c,s) = \{\}$. In any situation, each
agent will have made some sequence of observations:
\begin{multline*}
  ObsHist(agt,S_0) = \epsilon \\
  \shoveleft{ObsHist(agt,do(c,s)) = h \,\equiv\, \exists o . Obs(agt,c,s) = o}\\
  \shoveright{\wedge\, \left(o = \{\} \rightarrow h = ObsHist(agt,s)\right)}\\
  \wedge\, \left(o \neq \{\} \rightarrow h = o \cdot ObsHist(agt,s)\right)
\end{multline*}
\end{frame}

\begin{frame}
\frametitle{Global vs Local Perspective}
\begin{itemize}
\item \emph{Action}: global event, changes state of world
\item \emph{Observation}: local event, changes state of agent's knowledge\\
\ \\
\item \emph{Situation}: history of actions, omniscient viewpoint
\item \emph{Obs Hist}: history of observations, local viewpoint
\end{itemize}
\end{frame}

\begin{frame}
\frametitle{Knowledge follows  Observation}
Halpern \& Moses, 1990:\\
"an agent's knowledge at a given time must depend only on its local history:
the information that it started out with combined with the events it has
observed since then"
\pause
\ \\
\ \\
Clearly, we require:
\begin{multline*}
K(agt,s',s)\ \equiv\ ObsHist(agt,s') = ObsHist(agt,s)
\end{multline*}
We must enforce this in the successor state axiom for $K$.
\end{frame}

\begin{frame}
\frametitle{K Graphically}
\begin{center}
  \includegraphics[scale=0.5]{k_unobs.png}
\end{center}
\end{frame}

\begin{frame}
\frametitle{SSA for Knowledge}
First, some notation:
\begin{equation*}
  s <_{\alpha} do(c,s')\ \equiv\ s \leq_{\alpha} s' \wedge \alpha(c,s')
\end{equation*}
\begin{equation*}
  PbU(agt,c,s)\ \isdef\ Poss(c,s) \wedge Obs(agt,c,s) = \{\}
\end{equation*}
\pause
Then the following successor state axiom captures the intended dynamics of
knowledge update:
\begin{multline*}
  K(agt,s'',do(c,s))\ \equiv\ \exists o . Obs(agt,c,s) = o  \\
  \wedge \left(o = \{\} \rightarrow K(agt,s'',s)\right) \\
  \wedge \left(o \neq \{\} \rightarrow \exists c',s' . K(agt,s',s)\right. \\
  \left.\wedge Obs(agt,c',s') = o \wedge Poss(c',s') \wedge do(c',s') \leq_{PbU(agt)} s''\right)
\end{multline*}
\end{frame}

\begin{frame}
\frametitle{Axiomatizing Observations}
But what are these "observations" really?  In the simplest case, they
can just be actions that the agent is aware of, e.g:
\begin{equation*}
a \in Obs(agt,c,s)\ equiv\ a \in c \wedge actor(a) = agt
\end{equation*}
\begin{equation*}
a \in Obs(agt,c,s)\ equiv\ a \in c \wedge InSameRoomAs(agt,actor(a))
\end{equation*}
\ \\
\ \\
\pause
To handle sensing information, we can use $(action = result)$ pairs:
\begin{equation*}
(a=r) \in Obs(agt,c,s)\ equiv\ a \in c \wedge actor(a) = agt \wedge SR(a,s) = r
\end{equation*}
\ \\
\ \\
More complex scenarios are also possible.
\end{frame}

\begin{frame}
\frametitle{SSA Comparison}
We've gone from this:
\begin{multline*}
 K(s'',do(a,s)) \equiv \exists s' . \ s''=do(a,s') \\
 \wedge K(s',s) \wedge Poss(a,s') \wedge SR(a,s) = SR(a,s')
\end{multline*}
\pause
To this:
\begin{multline*}
  K(agt,s'',do(c,s))\ \equiv\ \exists o . Obs(agt,c,s) = o  \\
  \wedge \left(o = \{\} \rightarrow K(agt,s'',s)\right) \\
  \wedge \left(o \neq \{\} \rightarrow \exists c',s' . K(agt,s',s)\right. \\
  \left.\wedge Obs(agt,c',s') = o \wedge Poss(c',s') \wedge do(c',s') \leq_{PbU(agt)} s''\right)
\end{multline*}
\pause
It's messier, but it's also hiding a much bigger problem...
\end{frame}

\section{Reasoning}

\begin{frame}
\frametitle{Regression}
Recall that $\leq_{PbU(agt)}$ is defined by (implicitly) quantifying over
situations. So regression cannot be applied to the RHS of our successor state
axiom.
\ \\
\ \\
Observation-based knowledge \alert{cannot} be approached using
the standard regression operator.
\ \\
\ \\
\pause
In fact, universal quantification over situations requires a
second-order induction axiom.
Must we abandon hope of an effective reasoning procedure?
\end{frame}

\begin{frame}
\frametitle{Property Persistence}
Suppose we could "factor out" the quantification. Then we could get
on with the business of doing regression. 
\ \\
\ \\
Define the \emph{persistence condition} $\mathcal{P}[\phi,\alpha]$ of a formula
 $\phi$ and action conditions $\alpha$ to mean: assuming all future actions
satisfy $\alpha$, $\phi$ will remain true.
\begin{equation*}
  \mathcal{P}[\alpha,\phi](s)\ \equiv\ \forall s' . s \leq_{\alpha} s' \rightarrow \phi(s')
\end{equation*}
Like $\mathcal{R}$, the idea is to transform a query into a form that is easier
to deal with.
\end{frame}

\begin{frame}
\frametitle{$\mathcal{R}$ Graphically}
\begin{center}
  \includegraphics[scale=0.5]{k_unobs_highlighted.png}
\end{center}
\end{frame}

\begin{frame}
\frametitle{Regression using Persistence}
It becomes possible to define the regression of our $\mathbf{Knows}$ macro:
\begin{multline*}
  \mathcal{R}[\mathbf{Knows}(agt,\phi,do(c,s))] = \\
     \left[ Obs(agt,c,s) = \{\} \rightarrow \mathbf{Knows}(agt,\phi,s) \right] \\
     \wedge \left[ \exists o . Obs(agt,c,s) = o \wedge o\neq \{\} \rightarrow \right. \\
     \mathbf{Knows}(agt,\forall c' . Obs(agt,c',now)=o \rightarrow \\
     \left.\mathcal{R}[\mathcal{P}[\phi,PbU(agt)](do(c',s'))],s)\right]
\end{multline*}
\end{frame}

\begin{frame}
\frametitle{Reasoning from Observations}
We, the omniscient designer, can determine whether an agent knows something
 in any given situation:
\begin{equation*}
  \mathcal{D} \models \mathbf{Knows}(agt,\phi,s)
\end{equation*}
\begin{equation*}
  \mathcal{D}_{S_0} \cup \mathcal{D}_{una} \models \mathcal{R}^{*}[\mathbf{Knows}(agt,\phi,s)]
\end{equation*}
But the agent cannot use this to reason about its own knowledge, as it doesn't
know the current situation.  It needs to be able to reason from a local
viewpoint:
\begin{equation*}
  \mathcal{D} \models \mathbf{Knows}(agt,\phi,h)
\end{equation*}
\end{frame}

\begin{frame}
\frametitle{Regressing Over Observations}
The regression operator can be modified to act over observation histories,
instead of over situations:
\begin{multline*}
  \mathcal{R}[\mathbf{Knows}(agt,\phi,o \cdot h)] = \\
  \mathbf{Knows}(agt,\forall c' . Obs(agt,c',s')=o \rightarrow \\
     \mathcal{R}[\mathcal{P}[\phi,PbU(agt)](do(c',s'))],h)
\end{multline*}
We can equip agents with a situation calculus model of their own environment.
\end{frame}

\begin{frame}
\frametitle{Demonstration}
Simple test domain: three agents and several objects which they can pickup,
putdown and drop.
\ \\
\ \\
Agents can only observe their own actions.
\end{frame}

\section{Conclusions}

\begin{frame}
\frametitle{Conclusions}
What did we want?
\begin{itemize}
\item A powerful account of multi-agent knowledge
  \begin{itemize}
  \item Concurrent Actions
  \item Partial Observability
  \end{itemize}
\item that supports complex symbolic reasoning
\item that allows agents to reason about their own world
\item but is still computationally feasible
\end{itemize}
\ \\
\ \\
\pause
Future work:
\begin{itemize}
  \item Data Structures for efficient computation
  \item Interactions between concurrent actions
\end{itemize}

\end{frame}

\begin{frame}
\centering \large Thank You\\
\end{frame}
\end{document}

