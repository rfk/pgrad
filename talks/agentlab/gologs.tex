\documentclass{beamer}

\mode<presentation>{\usetheme{Dresden}}

\title{GOLOG and Friends}
\subtitle{An Introduction}
\author{Ryan Kelly}

\begin{document}

\begin{frame}
  \titlepage
\end{frame}

\begin{frame}
  \frametitle{Outline}
  \tableofcontents
\end{frame}

\section{Motivation}

\begin{frame}
\frametitle{Motivation: Why do we need Golog?}

\begin{itemize}
\item Developed for Cognitive Robotics applications
\item Robots should be able to explicitly represent and reason about their world
\item Programs should reflect the domain, not the programming environment
\end{itemize}

\end{frame}


\section{The Situation Calculus}

\begin{frame}
\frametitle{The Situation Calculus}

A FOL formalism for reasoning about dynamic worlds:

\begin{itemize}
\item Actions: $wait$, $pickup(obj)$, $goto(x,y)$
\pause
\item Situations: $S_{0}$, $do(a_{1},S_{0})$, $do(a_{2},do(a_{1},S_{0}))$
\pause
\item Fluents: $holding(obj,s)$, $(x,y)=curPosition(s)$
\pause
\item Successor State Axioms:\[
\begin{array}{cc}
holding(obj,do(a,s))\iff & a=pickup(obj)\,\vee\\
 & \left[holding(obj,s)\wedge a\neq drop(obj)\right]\end{array}\]
\end{itemize}

\end{frame}

\begin{frame}
\frametitle{The Situation Calculus}
Using a S.C. axiomatisation of their domain, agents can:
\begin{itemize}
  \item Reason about when actions are possible
  \item Reason about the effects of actions
  \item Represent sequences of actions as situations:\[
\left[a_1, a_2, a_3, a_4\right] \rightarrow do(a_4,do(a_3,do(a_2,do(a_1,S_0))))
\]
\end{itemize}
But, what about more complex actions and procedures?
\begin{itemize}
  \item $\mathbf{if}\ available(Item)\ \mathbf{then}\ pickup(Item)\ \mathbf{else}\ beSad$
  \item $\mathbf{while}\ (\exists block)\ ontable(block)\ \mathbf{do}\ remove(block)\ \mathbf{endWhile}$
  \item $\mathbf{proc}\ remove(block)\ [pickup(block);putaway(block)]\ \mathbf{endProc}$
\end{itemize}
\end{frame}

\section{GOLOG}
\begin{frame}
\frametitle{GOLOG: "alGOL in LOGic"}
Introduces programming to the S.C. by means of "complex actions":
\begin{itemize}
  \item $\delta_1;\delta_2$: Perform actions in sequence
  \item $\phi?$: Assert that a condition holds
  \item $\delta_1|\delta_2$: Choose between actions to perform
  \item $(\pi x)\delta(x)$: Choose suitable bindings for variables
  \item $\delta^*$: Perform an action zero or more times
\end{itemize}
Composing actions by means of these operators allows complex programs
to be defined.\\
\ \\
Key Point:  programs can include arbitrarily much \alert{nondeterminism}
\end{frame}

\begin{frame}
\frametitle{GOLOG: "alGOL in LOGic"}
Semantics of operators given recursively in terms of the macro $Do$.
$Do(\delta,s,s')$ means "program $\delta$ may begin in situation $s$ and end
in situation $s'$"
\begin{itemize}
  \item Primitive actions: $Do(a,s,s') \equiv Poss(a,s)\wedge s'=do(a,s)$
  \item Test actions: $Do(\phi?,s,s') \equiv \phi \wedge s=s'$
  \item Sequence: $Do(\delta_1;\delta_2) \equiv \exists s^* (Do(\delta_1,s,s^*)\wedge Do(\delta_2,s^*,s'))$
  \item Nondet. Choice: $Do(\delta_1|\delta_2,s,s') \equiv Do(\delta_1,s,s')\vee Do(\delta_2,s,s')$
  \item Nondet. Argument: $Do((\pi x)\delta(x),s,s') \equiv \exists x Do(\delta(x),s,s')$
\end{itemize}
\end{frame}

\begin{frame}
\frametitle{Executing a Golog Program}
Intuitively, an agent must find a \emph{legal execution} of its nondeterministic
program - some set of choices for the nondeterministic components that allow
the program to execute successfully.\\
\ \\
Formally, the agent must find a situation $s$ such that:\[
\exists s . Do(\delta,S_0,s)\]


Remember, a situation is a sequence of actions - in this case, the actions which must be carried out to execute the program.
\end{frame}

\begin{frame}
\frametitle{A Simple Example}
\[\delta = [(\pi\ block)(moveNextTo(block);pickup(block))];holding(Block1)?\]

Assuming that $\neg holding(Block1,S_0)$, this program has a unique
legal execution:\[
s = do(pickup(Block1),do(moveNextTo(Block1),S_0)\]

Corresponding to the action sequence:\[
[moveNextTo(Block1), pickup(Block1)]\]

The agent reasons about the world to inform its nondeterministic choices.
\end{frame}

\begin{frame}
\frametitle{A More Complicated Example}
As a slightly more complex example, consider a Golog program for
getting to uni of a morning:\[
\begin{array}{c}
ringAlarm;(hitSnooze; ringAlarm)^*;turnOffAlarm;\\
haveShower;(\pi\ food)(edible(food)?;eat(food));\\
(driveToUni\ |\ tramToUni\ |\ walkToUni);\\
(time<9:00)?
\end{array}\]

There are potentially many legal executions, depending on which actions
are possible in the world.
\end{frame}

\section{ConGolog}
\begin{frame}
\frametitle{ConGolog: Concurrency and Interrupts}
\end{frame}

\section{IndiGolog}
\begin{frame}
\frametitle{IndiGolog: Online Execution and Sensing}
\end{frame}

\section{Where?}
\begin{frame}
\frametitle{What is Golog used for?}
Many applications focused on robotics:
\begin{itemize}
  \item Mail delivery, coffee delivery robots
  \item Museum tour guide
  \item Robocup soccer team
  \item Lego Mindstorms: "Legolog"
\end{itemize}
\end{frame}
\begin{frame}
\frametitle{What is Golog used for?}
But, also being picked up in other areas
\begin{itemize}
  \item Rapid prototyping for evolutionary biology applications
  \item Service composition for the semantic web
  \item Homeland Security response plans
  \item AI for Unreal Tournament bots
  \item ...
\end{itemize}
\end{frame}

\section{Why?}
\begin{frame}
\frametitle{Why is Golog popular?}
\end{frame}


\section{The Future}
\begin{frame}
\frametitle{The Future...}
\end{frame}



\end{document}
