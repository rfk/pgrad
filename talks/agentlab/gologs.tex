\documentclass{beamer}

\newcommand{\isdef}{\hbox{$\stackrel{\mbox{\tiny def}}{=}$}}

\mode<presentation>{\usetheme{Dresden}}

\title{GOLOG and Friends}
\subtitle{An Introduction}
\author{Ryan Kelly}

\begin{document}

\begin{frame}
  \titlepage
\end{frame}

\begin{frame}
  \frametitle{Outline}
  \tableofcontents
\end{frame}

\section{Motivation}

\begin{frame}
\frametitle{Motivation: Why do we need Golog?}
Golog has been developed primarily for Cognitive Robotics applications - agents
which must do sophisticated reasoning about themselves, other agents, and their
changing environment.
\begin{itemize}
\item Agents should explicitly represent and reason about their world
\item Programs should reflect the domain, not the programming environment
\item Full planning shouldn't be needed for simple tasks
\end{itemize}
And moreover, these properties should be natural and easy to obtain in the
control program for your agent.
\end{frame}


\section{The Situation Calculus}

\begin{frame}
\frametitle{The Situation Calculus}

A F.O.L. formalism for reasoning about dynamic worlds:

\begin{itemize}
\pause
\item Actions: $wait$, $pickup(obj)$, $goto(x,y)$
\pause
\item Situations: $S_{0}$, $do(a_{1},S_{0})$, $do(a_{2},do(a_{1},S_{0}))$
\pause
\item Fluents: $holding(obj,s)$, $(x,y)=curPosition(s)$
\pause
\item Precondition Axioms: $Poss(pickup(obj),s) \leftrightarrow holding(obj,s)$
\pause
\item Successor State Axioms:\[
\begin{array}{cc}
holding(obj,do(a,s))\iff & a=pickup(obj)\,\vee\\
 & \left[holding(obj,s)\wedge a\neq drop(obj)\right]\end{array}\]
\end{itemize}
\pause
Successor State Axioms can provide an elegant solution to the infamous
frame problem.
\end{frame}

\begin{frame}
\frametitle{The Situation Calculus}
These axioms combine to give a first-order \emph{Theory of Action} $\mathbf{D}$,
 with which agents can:
\begin{itemize}
  \item Reason about when actions are possible
  \item Reason about the effects of actions
  \item Represent sequences of actions as situations:\[
\left[a_1, a_2, a_3, a_4\right] \rightarrow do(a_4,do(a_3,do(a_2,do(a_1,S_0))))
\]
\end{itemize}
\pause
But, what about more complex actions and procedures?
\begin{itemize}
  \item $\mathbf{if}\ available(Item)\ \mathbf{then}\ pickup(Item)\ \mathbf{else}\ beSad$
  \item $\mathbf{while}\ (\exists block)\ ontable(block)\ \mathbf{do}\ remove(block)\ \mathbf{endWhile}$
  \item $\mathbf{proc}\ remove(block)\ [pickup(block);putaway(block)]\ \mathbf{endProc}$
\end{itemize}
\end{frame}

\section{GOLOG}
\begin{frame}
\frametitle{GOLOG: "alGOL in LOGic"}
Introduces programming to the S.C. by means of "complex actions":
\begin{itemize}
  \pause
  \item $\delta_1;\delta_2$: Perform actions in sequence
  \pause
  \item $\phi?$: Assert that a condition holds
  \pause
  \item $\delta_1|\delta_2$: Choose between actions to perform
  \pause
  \item $(\pi x)\delta(x)$: Choose suitable bindings for variables
  \pause
  \item $\delta^*$: Perform an action zero or more times
  \pause
\end{itemize}
Composing actions by means of these operators allows complex programs
to be defined.\\
\ \\
\pause
Key Point:  programs can include \alert{nondeterminism}
\end{frame}

\begin{frame}
\frametitle{GOLOG: "alGOL in LOGic"}
Semantics of operators given recursively in terms of the macro $Do$.
$Do(\delta,s,s')$ means "program $\delta$ may begin in situation $s$ and end
in situation $s'$"
\begin{itemize}
  \pause
  \item Primitive actions: $Do(a,s,s')\ \isdef\ Poss(a,s)\wedge s'=do(a,s)$
  \pause
  \item Test actions: $Do(\phi?,s,s')\ \isdef\ \phi [s] \wedge s=s'$
  \pause
  \item Sequence: $Do(\delta_1;\delta_2)\ \isdef\ \exists s'' (Do(\delta_1,s,s'')\wedge Do(\delta_2,sxxi''^*,s'))$
  \pause
  \item Nondet. Choice: $Do(\delta_1|\delta_2,s,s')\ \isdef\ Do(\delta_1,s,s')\vee Do(\delta_2,s,s')$
  \pause
  \item Nondet. Argument: $Do((\pi x)\delta(x),s,s')\ \isdef\ \exists x Do(\delta(x),s,s')$
  \pause
  \item $\mathbf{if}\ \phi\ \mathbf{then}\ \delta_1\ \mathbf{else}\ \delta_2\ \mathbf{endIf}\ \isdef\ [\phi?;\delta_1]\ |\ [\neg\phi?;\delta_2]$
  \pause
  \item $\mathbf{while}\ \phi\ \mathbf{do}\ \delta\ \mathbf{endWhile}\ \isdef\ [[\phi?;\delta]*;\neg\phi?]$
\end{itemize}
\end{frame}

\begin{frame}
\frametitle{Executing a Golog Program}
Intuitively, an agent must find a \emph{legal execution} of its nondeterministic
program - some set of choices for the nondeterministic components that allow
the program to execute successfully.\\
\ \\
Formally, the agent must find a situation $s$ such that:\[
\mathbf{D} \models \exists s . Do(\delta,S_0,s)\]


Remember, a situation is a sequence of actions - in this case, the actions which must be carried out to execute the program.
\end{frame}

\begin{frame}
\frametitle{A Simple Example}
$[(\pi\ block)(goNextTo(block);pickup(block))];holding(Block1)?$\\
\ \\
Assuming that $\neg holding(Block1,S_0)$, this program has a unique
legal execution:\[
s = do(pickup(Block1),do(goNextTo(Block1),S_0)\]

Corresponding to the action sequence:\[
[goNextTo(Block1), pickup(Block1)]\]

The agent reasons about the world to inform its nondeterministic choices.
\end{frame}

\begin{frame}
\frametitle{A More Complicated Example}
As a slightly more complex example, consider a Golog program for
getting to uni of a morning:\[
\begin{array}{c}
ringAlarm;(hitSnooze; ringAlarm)^*;turnOffAlarm;\\
haveShower;(\pi\ food)(edible(food)?;eat(food));\\
(driveToUni\ |\ tramToUni\ |\ walkToUni);\\
(time<9:00)?
\end{array}\]

There are potentially many legal executions, depending on which actions
are possible in the world.
\end{frame}

\begin{frame}
\frametitle{Programming or Planning?}
\begin{itemize}
  \item May require reasoning about effects of actions to resolve nondeterminism
$\rightarrow$ Planning
  \item Or, procedure may be completely specified $\rightarrow$ Programming
  \item Golog occupies a middle ground $\rightarrow$ "High-Level Programming"
\end{itemize}

Full planning is possible in Golog:\[
\mathbf{while}\ \neg Goal\ \mathbf{do}\ (\pi\ a)(a)\ \mathbf{endWhile}\]

Or you can specify more of the procedure yourself, to make the agent's job
easier.\\
\ \\
Key Point: \alert{Controlled} Nondeterminism
\end{frame}

\section{ConGolog}
\begin{frame}
\frametitle{ConGolog: Concurrency and Interrupts}
"Concurrent Golog" was designed to support the concurrent execution of
several threads of control.  It also allows threads to be spawned in
response to conditions in the environment:
\begin{itemize}
  \pause
  \item $\delta_1\ ||\ \delta_2$: Execute two programs concurrently
  \pause
  \item $\delta_1\ \gg\ \delta_2$: Execute two programs concurrently, giving one priority over the other
  \pause
  \item $<\phi \rightarrow \delta>$: Execute $\delta$ when $\phi$ becomes true
\end{itemize}
\pause
This requires a modification of the $Do$ semantics.  Instead, it uses
$Trans(\delta,s,\delta',s')$ to indicate that program $\delta$ may execute
a single step, taking the world from situation $s$ to $s'$, and leaving the
program $\delta'$ still to be executed.
\end{frame}

\begin{frame}
\frametitle{ConGolog}
Concurrency is achieved by interleaving steps from both programs:
\[
\begin{array}{cc}
Trans(\delta_1\ ||\ \delta_2,s,\delta',s') \equiv & \exists \gamma . \delta'=(\gamma\ ||\ \delta_2)\wedge Trans(\delta_1,s,\gamma,s')\\
& \vee\ \exists \gamma . \delta'=(\delta_1\ ||\ \gamma)\wedge Trans(\delta_2,s,\gamma,s')
\end{array}\]
\pause
\ \\
My "getting to uni" for today:
\[
\begin{array}{c}
<runningLate\rightarrow panic>\\
\gg\ stressAboutPresentation\\
||\ (getReady;goToUni)
\end{array}\]

\end{frame}

\begin{frame}
\frametitle{ConGolog}
What might a legal execution look like?  Suppose:\[
stressAboutPresentation \equiv fidgetNervously^*\]
Then we could have:
\[ringAlarm\]
\[turnOffAlarm\]
\[fidgetNervously\]
\[haveShower\]
\[fidgetNervously\]
\[eat(toast)\]
\[tramToUni\]
\[fidgetNervously\]


\end{frame}

\section{IndiGolog}
\begin{frame}
\frametitle{IndiGolog: Online Execution and Sensing}
In order to be sure an execution is legal, Golog must plan all the way
to the end.  "Offline Execution" - for highly nondeterministic programs
this may be intractable.\\
\ \\
Furthermore, the agent might not have all the information it needs when
execution begins - may require \emph{sensing}.

\begin{itemize}
  \pause
  \item Execute online, making nondet choices arbitrarily
  \pause
  \item Perform sensing actions when extra information is required
  \pause
  \item Allow planning over individual parts of the program
\end{itemize}
\end{frame}

\begin{frame}
\frametitle{IndiGolog}
"Getting to Uni" could fail at the final test because we snoozed too long:\[
\begin{array}{c}
ringAlarm;(hitSnooze; ringAlarm)^*;turnOffAlarm;\\
haveShower;(\pi\ food)(edible(food)?;eat(food));\\
(driveToUni\ |\ tramToUni\ |\ walkToUni);\\
(time<9:00)?
\end{array}\]\\
\ \\
Instead, plan over just the nondeterministic parts:\[
\begin{array}{c}
ringAlarm;\Sigma[(hitSnooze; ringAlarm)^*;(time<8:15)?];turnOffAlarm;\\
haveShower;\Sigma[(\pi\ food)(edible(food)?;eat(food))];\\
\Sigma[(driveToUni\ |\ tramToUni\ |\ walkToUni);(time<9:00)?]

\end{array}\]\\

\end{frame}

\section{Where?}
\begin{frame}
\frametitle{What is Golog used for?}
Many applications focused on robotics:
\begin{itemize}
  \item Mail delivery, coffee delivery robots
  \item Museum tour guide
  \item Robocup soccer team
  \item Lego Mindstorms: "Legolog"
\end{itemize}
\pause
But, also being picked up in other areas
\begin{itemize}
  \item Rapid prototyping for evolutionary biology applications
  \item Service composition for the semantic web
  \item Homeland Security response plans
  \item AI for Unreal Tournament bots
  \item ...
\end{itemize}
\end{frame}

\section{Why?}
\begin{frame}
\frametitle{Why is Golog popular?}
\begin{itemize}
  \item Good level of abstraction:
  \begin{itemize}
    \item Programs based directly on actions from the domain
    \item Nondeterminism makes programs simpler and more powerful
    \item Symbolic reasoning effortlessly available
  \end{itemize}
  \item Tradeoff between programming and planning:
  \begin{itemize}
    \item Amount of nondeterminism controlled by the programmer
    \item Procedural knowledge easy to encode
    \item Full planning still available
  \end{itemize}
  \item Logic-based:
  \begin{itemize}
    \item Compact implementation in Prolog
    \item Possible to prove safety/liveness properties
  \end{itemize}
\end{itemize}
\end{frame}


\section{The Future}
\begin{frame}
\frametitle{The Future...}
Multiagent systems:
\begin{itemize}
  \item Modelling and Simulation of multi-agent systems (CASL, Levesque et. al.)
  \item Golog for team coordination and control (ReadyLog, Lakemeyer et. al.)
  \item Cooperative execution of Golog programs (Us)
\end{itemize}

Knowledge and belief
\begin{itemize}
  \item Particularly with regard to sensing and communication
\end{itemize}

Modifying the situation calculus:
\begin{itemize}
  \item Concurrency, time, ...
  \item Fluent Calculus
  \item Modal Logic
\end{itemize}
\end{frame}


\end{document}
