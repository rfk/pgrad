\documentclass{beamer}

\mode<presentation>{\usetheme{Dresden}}

\title{High-Level Program Execution for Multi-Agent Teams}
\author{Ryan Kelly}

\begin{document}

\begin{frame}
  \titlepage
\end{frame}

\begin{frame}
  \frametitle{Outline}
  \tableofcontents
\end{frame}

\section{Motivation}

\begin{frame}
\frametitle{Multi-Agent Teams}
Two categories of multi-agent system:
\begin{itemize}
  \item Open multi-agent systems
  \item Multi-agent teams
\end{itemize}
Teams may be conceptualised as a single agent, with distributed sensing,
reasoning and acting capabilities.\\
\ \\
\pause
What ideas can we borrow from single-agent programming?
\end{frame}

\begin{frame}
\frametitle{High-Level Program Execution}
A practical alternative to planning for single-agent systems. Specify
a program made up of:
\begin{itemize}
  \item Actions from the world
  \item Connected by high-level programming constructs
  \item Including nondeterminism where reasoning is required
\end{itemize}
\ \\
\ \\
Can range from deterministic programming to full planning:
\alert{Controlled Nondeterminism}
\end{frame}

\begin{frame}
\frametitle{Motivating Example: The Cooking Agents}
Several robotic chefs inhabit a kitchen, along with various ingredients,
appliances and utensils.  They must cooperate to produce a meal consisting
of several dishes.\\
\ \\
Challenges:
\begin{itemize}
  \item Concurrent execution of tasks
  \item Conflict over shared resources
  \item Coordination of shared actions
\end{itemize}

Assumptions:
\begin{itemize}
  \item Deterministic, fully observable world
  \item Complete and reliable communication
\end{itemize}
\end{frame}


\section{Golog Review}

\begin{frame}
\frametitle{The Situation Calculus}

A first-order logic formalism for reasoning about dynamic worlds:

\begin{itemize}
\pause
\item Actions: $wait$, $pickup(obj)$, $goto(x,y)$
\pause
\item Situations: $S_{0}$, $do(a_{1},S_{0})$, $do(a_{2},do(a_{1},S_{0}))$
\pause
\item Fluents: $holding(obj,s)$, $(x,y)=curPosition(s)$
\pause
\item Successor State Axioms:\[
\begin{array}{cc}
holding(obj,do(a,s))\iff & a=pickup(obj)\,\vee\\
 & \left[holding(obj,s)\wedge a\neq drop(obj)\right]\end{array}\]
\end{itemize}
Successor state axioms can provide an elegant solution to the infamous
frame problem.

\end{frame}

\begin{frame}
\frametitle{Golog}
Introduces programming to the S.C. by means of "complex actions":
\begin{itemize}
  \pause
  \item $\delta_1;\delta_2$: Perform actions in sequence
  \pause
  \item $\phi?$: Assert that a condition holds
  \pause
  \item $\delta_1|\delta_2$: Choose between actions to perform
  \pause
  \item $(\pi x)\delta(x)$: Choose suitable bindings for variables
  \pause
  \item $\delta^*$: Perform an action zero or more times
  \pause
\end{itemize}
Composing actions by means of these operators allows complex programs
to be defined.
\end{frame}

\begin{frame}
\frametitle{Example: A Salad in Golog}
Consider a Golog program for making a simple salad:\[
\begin{array}{c}
\mathbf{proc}\ MakeSalad(dest)\\
ChopTypeInto(Lettuce,dest)\ ;\\
ChopTypeInto(Carrot,dest)\ ;\\
ChopTypeInto(Tomato,dest)\ ;\\
mix(dest,1)\\
\mathbf{end}\end{array}\]
\end{frame}

\begin{frame}
\frametitle{Executing a Golog Program}
Intuitively, an agent must find a \emph{legal execution} of its nondeterministic
program - some set of choices for the nondeterministic components that allow
the program to execute successfully.\\
\ \\
Formally, the agent must find a situation $s$ such that:\[
\mathbf{D} \models \exists s . Do(\delta,S_0,s)\]

Here, $\mathbf{D}$ is the Situation Calculus theory of action, and $Do$
encodes the semantics of the Golog operators.\\
\ \\
Remember, a situation is a sequence of actions - in this case, the actions which must be carried out to execute the program.
\end{frame}

\begin{frame}
\frametitle{ConGolog: Concurrency and Interrupts}
"Concurrent Golog" was designed to support the concurrent execution of
several threads of control.  It also allows threads to be spawned in
response to conditions in the environment: \\
\ \\
\begin{itemize}
  \pause
  \item $\delta_1\ ||\ \delta_2$: Execute two programs concurrently
  \pause
  \item $\delta_1\ \gg\ \delta_2$: Execute two programs concurrently, giving one priority over the other
  \pause
  \item $<\phi \rightarrow \delta>$: Execute $\delta$ when $\phi$ becomes true
\end{itemize}
\end{frame}

\begin{frame}
\frametitle{ConGolog}
Semantics are defined in terms of "single steps" of computation.
$Trans(\delta,s,\delta',s')$ indicates that program $\delta$ may execute
a single step, taking the world from situation $s$ to $s'$, and leaving the
program $\delta'$ still to be executed.\\
\ \\
\pause
Concurrency is achieved by interleaving steps from both programs:
\[
\begin{array}{cc}
Trans(\delta_1\ ||\ \delta_2,s,\delta',s') \equiv & \exists \gamma . \delta'=(\gamma\ ||\ \delta_2)\wedge Trans(\delta_1,s,\gamma,s')\\
& \vee\ \exists \gamma . \delta'=(\delta_1\ ||\ \gamma)\wedge Trans(\delta_2,s,\gamma,s')
\end{array}\]
\end{frame}

\begin{frame}
\frametitle{A Salad in ConGolog}
When making a salad, the order of ingredients doesnt matter.  This can
be expressed using concurrency:\[
\begin{array}{c}
\mathbf{proc}\ MakeSalad(dest)\\
\left[\ ChopTypeInto(Lettuce,dest)\ ||\right.\\
ChopTypeInto(Carrot,dest)\ ||\\
\left.ChopTypeInto(Tomato,dest)\ \right]\ ;\\
mix(dest,1)\\
\mathbf{end}\end{array}\]
\end{frame}


\begin{frame}
\frametitle{IndiGolog: Online Execution and Sensing}
In order to be sure an execution is legal, Golog must plan all the way
to the end.  "Offline Execution" - for highly nondeterministic programs
this may be intractable.\\
\ \\
Furthermore, the agent might not have all the information it needs when
execution begins - may require \emph{sensing}.

\begin{itemize}
  \pause
  \item Execute online, making nondet choices arbitrarily
  \pause
  \item Perform sensing actions when extra information is required
  \pause
  \item Allow planning over individual parts of the program: $\Sigma(\delta)$
\end{itemize}
\end{frame}

\begin{frame}
\frametitle{A Salad in IndiGolog}
Regardles of what ends up in the bowl, it can always be mixed.  Only need
to plan how to add the ingredients:\[
\begin{array}{c}
\mathbf{proc}\ MakeSalad(dest)\\
\Sigma\left[\ ChopTypeInto(Lettuce,dest)\ ||\right.\\
ChopTypeInto(Carrot,dest)\ ||\\
\left.ChopTypeInto(Tomato,dest)\ \right]\ ;\\
mix(dest,1)\\
\mathbf{end}\end{array}\]
\end{frame}


\section{Golog for MA Teams?}

\begin{frame}
\frametitle{Golog for MA Teams?}
Golog has proven very powerful in single-agent domains.  But, there are
some challenges when moving to a multi-agent setting:
\begin{itemize}
  \pause
  \item Several actions can be performed at the same instant
  \pause
  \item Need to coordinate concurrent actions
  \pause
  \item Need to predict actions of others, and the environment
\end{itemize}
\ \\
\ \\
\pause
Some of these have been individually addressed in the Situation Calculus.
We integrate them into a Golog suitable for multi-agent teams: "MIndiGolog".

\end{frame}

\section{Concurrency}

\begin{frame}
\frametitle{True vs Interleaved Concurrency}
ConGolog provides \emph{interleaved concurrency} allowing multiple
high-level programs to be executed together.\\
\ \\
With mulitple agents, the world exhibits \emph{true concurrency}. Several
actions can occur at the same instant.\\
\ \\
\pause
Need to:
\begin{itemize}
  \item allow true concurrency to be represented
  \item modify the $||$ operator to take advantages of true concurrency
  \item ensure that this modification is well-behaved
\end{itemize}
\end{frame}

\begin{frame}
\frametitle{The Concurrent Situation Calculus (Reiter, TODO)}
Have the first argument of an action name the agent performing it:\[
pickup(Thomas,Lettuce1),\ mix(Richard,Bowl1,1)\]
\pause
Replace action terms with sets of actions, all of which are performed
at the same time:\[
do(\{pickup(Thomas,Lettuce1),pickup(Richard,Tomato1)\},S_0)\]
\pause
Precondition interaction:\[
Poss({pickup(Richard,Obj),pickup(Harriet,Obj)},s)\ ?\]
\pause
Basic solution:\[
Poss(C,s) \leftrightarrow \forall a \in C.Poss(a,s)\ \wedge\ \neg Conflicts(C,s)\]
\end{frame}

\begin{frame}
\frametitle{True + Interleaved Concurrency}
When executing two programs in parallel, it may be possible for the team to
execute a step from each simultaneously:\[
\begin{array}{cc}
Trans(\delta_1\ ||\ \delta_2,s,\delta',s') \equiv & \exists \gamma . \delta'=(\gamma\ ||\ \delta_2)\wedge Trans(\delta_1,s,\gamma,s')\\
& \\
& \vee\ \exists \gamma . \delta'=(\delta_1\ ||\ \gamma)\wedge Trans(\delta_2,s,\gamma,s')\\
& \\
& \vee\ \exists c_1,c_2,\gamma_1,\gamma_2.\ Trans(\delta_1,s,\gamma_1,do(c_1,s)\\
& \wedge\ Trans(\delta_2,s,\gamma_2,do(c_2,s))\\
& \wedge\ \delta'=(\gamma_1||\gamma_2) \wedge s'=do(c_1 \cup c_2,s)\\
\end{array}\]
But, this simple encoding is not well-behaved (later...)
\end{frame}

\section{Time}

\begin{frame}
\frametitle{Time}
Which is easier for a team to manage?
\begin{itemize}
  \pause
  \item Perform a list of actions concurrently
  \pause
  \item Perform a list of actions concurrently at time $t$
\end{itemize}
\ \\
\ \\
\pause
Assuming a good wall-clock is available, explicit time makes coordination easier.\\
\ \\
It is also generally a nice feature to have when representing dynamic worlds.
Example: bake a cake for \emph{10 minutes}
\end{frame}

\begin{frame}
\frametitle{The Temporal Situation Calculus}
Adding time to the S.C. has a long history (TODO: refs).\\
\ \\
We opt for a simple solution: attach an explcit time to each situation,
indicating when it was brought about:\[
\begin{array}{c}
do(a_1,t_1,S_0),\ do(a_2,t_2,do(a_1,t_1,S_0))\\
Poss(a,t,s) \leftrightarrow\ \dots
start(do(a,t,s)) = t
\end{array}\]
\pause
Temporal Golog finds situations with \emph{constraints} on the
time at which actions are performed:\[
\begin{array}{c}
do(a_2,t_2,do(a_1,t_1,S_0)\\
t_1>0,\ t_2>t_1+5
\end{array}\]
\end{frame}

\begin{frame}
\frametitle{Continuous Actions}
Some actions have a finite duration: $mix(Thomas,Bowl1,5)$
Others are instantaneous: $pickup(Robert,Lettuce1)$.
So this concurrent action is problematic:\[
\{mix(Thomas,Bowl1,5),pickup(Robert,Lettuce1)\}\]
Standard solution: decompose continuous actions into instantaneous $begin$ 
and $end$ actions:\[
begin\_mix(Thomas,Bowl1,5)$, $end\_mix(Thomas,Bowl1,5)\]
But $begin\_mix$ is not a standard action.  It cannot be performed at
any time.  Rather, it \emph{must} occur whenever it is possible.
\end{frame}

\section{Natural Actions}

\begin{frame}
\frametitle{Natural Actions (Reiter, TODO)}
\begin{itemize}
  \item Natural Action:  an action that must occur whenever it is possible,
        unless some earlier action prevents it from occuring.
  \item $Poss(na,t,s)$ predicts the times at which a natural action will occur.
  \item Least Natural Time Point: the earliest time at which a natural
        action will occur for a given situation:\[
\begin{array}{c}
lntp(s,t) \leftrightarrow \exists a [natural(a) \wedge Poss(a,t,s)] \wedge\\
\forall a,t_a [natural(a) \wedge Poss(a,t_a,s) \rightarrow t \leq t_a]
\end{array}\]
  \item Legal Situation: one in which any natural actions that could have occured, did occur:\[
\begin{array}{c}
legal(do(c,t,s)) \leftrightarrow legal(s) \wedge\\
Poss(c,t,s) \wedge start(s) \leq t \wedge\\
\forall a,t_a [natural(a) \wedge Poss(a,t_a,s) \rightarrow (a \in c \vee t < t_a)]
\end{array}\]
\end{itemize}
\end{frame}

\begin{frame}
\frametitle{Making Salad, again}
So, how does a team cooperate to make a salad?
\[
\begin{array}{c}
\mathbf{proc}\, MakeSalad(dest)\\
\left[\pi(agt)(ChopTypeInto(agt,Lettuce,dest))\,||\right.\\
\pi(agt)(ChopTypeInto(agt,Carrot,dest))\,||\\
\left.\pi(agt)(ChopTypeInto(agt,Tomato,dest))\right]\,;\\
\pi(agt)\left[acquire(agt,dest)\,;\right.\\
\,\,\,\,\,\,\,\,\,\,\,\,\,\,\,\,\,\,\,\, begin\_ task(agt,mix(dest,1))\,;\\
\,\,\,\,\,\,\,\,\,\,\,\,\,\,\,\,\,\, end\_ task(agt,mix(dest,1))\,;\\
\left.\,\,\,\, release(agt,dest)\right]\,\,\mathbf{end}\end{array}\]
\end{frame}

\section{Implementation}
Our first implementation was in Prolog:
\begin{itemize}
  \item $Trans$ rules from the semantics of ConGolog/IndiGolog
  \pause
  \item Modified for lists of concurrent actions, explicit time
  \pause
  \item Including our new rules for $||$ and primitive actions
  \pause
  \item And an axiomatisation of the "cooking agents" domain
\end{itemize}

\section{Related/Future Work}


\end{document}
