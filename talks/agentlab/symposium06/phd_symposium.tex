
%\documentclass[handout, compress]{beamer}
\documentclass[compress]{beamer}

\mode<presentation>
{
  %\usetheme{Warsaw}
  \usetheme{Dresden}
  %\usetheme{Darmstadt}
  % or ...
  %\usecolortheme{dolphin}

  \setbeamercovered{transparent}
  % or whatever (possibly just delete it)
}

\usepackage[english]{babel}

\usepackage[latin1]{inputenc}

\usepackage{amsmath}
\usepackage[all]{xy}

\usepackage{times}
\usepackage[T1]{fontenc}

\title
{A Programming Language for\\Multi-Agent Teams}

\author
{Ryan Kelly}

\institute[The University of Melbourne]
{
  Department of Computer Science and Software Engineering\\
  The University of Melbourne%\\
%(see http://www.cs.mu.oz.au/agentlab)
}

\date[The University of Melbourne]
{5th October 2005 / The University of Melbourne}

% If you have a file called "university-logo-filename.xxx", where xxx
% is a graphic format that can be processed by latex or pdflatex,
% resp., then you can add a logo as follows:

%\pgfdeclareimage[height=0.8cm]{university-logo}{MINSilvr}
%\logo{\pgfuseimage{university-logo}}

% If you wish to uncover everything in a step-wise fashion, uncomment
% the following command: 

%\beamerdefaultoverlayspecification{<+->}

\begin{document}

\begin{frame}
  \titlepage
\end{frame}

\section{Thesis}

\subsection{Title}

\begin{frame}
\frametitle{A Programming Language for Multi-Agent Teams}
Aim: to develop techniques for the cooperative execution of
a shared high-level program by a team of autonomous agents.
\\\ 
\\
\pause
So that a designer may
specify tasks as a shared, high-level, nondeterministic program and
have the team of agents cooperate to find and perform a legal execution of that
program.
\\\ 
\\
\pause
By extending the
situation calculus and the Golog programming language to handle complex
multi-agent domains.
\end{frame}

\subsection{Research Problems}
\begin{frame}
When a team of agents must cooperate closely to achieve a shared goal,
they can be usefully conceptualised as a single agent with distributed
reasoning, acting and sensing abilities.
\begin{itemize}
\item<2-> There is a \emph{common, shared} notion of the the tasks to be
performed.
\item<3-> Interested more in \emph{what} gets done rather than
\emph{who} does it.
\item<4-> How can we represent and reason about tasks for such teams in
a powerful, flexible way?  How can we handle concurrency, coordination,
contention..?
\item<5-> How can the agents robustly perform these tasks in the face of
partial information about each other and the environment?
\end{itemize}
\end{frame}

\subsection{Timeline}

\begin{frame}
\frametitle{Timeline}
\begin{itemize}
\item \textbf{Apr 05:} Fresh-faced, naive Masters candidate
\item \textbf{1st Year:} Selected technologies, began M.A. extensions
\item<2-> \textbf{Apr 06:} Converted to PhD candidature
\item<2-> \textbf{2nd Year:} Preliminary implementations, coordination strategies
\item<3-> \textbf{Apr 07:} Progress Report
\item<3-> \textbf{Jun 07:} First draft of Thesis
\item<3-> \textbf{Nov 07:} Demonstrate prototype on networked machines
\item<3-> \textbf{Nov 07:} Full draft of Thesis
\item<4-> \textbf{Apr 08:} Final Submission, Party, etc
\end{itemize}
\end{frame}

\subsection{Progress}

\begin{frame}
\frametitle{Progress}
\begin{itemize}
\item Extended semantics of IndiGolog to be more suitable for multi-agent teams
  \begin{itemize}
  \item Concurrency, Time, Predictable Exogenous Actions
  \item Implementation of "MIndiGolog" in Prolog
  \item Preliminary distributed execution planning using Oz
  \item Paper: "High-Level Program Execution for Distributed Problem Solving", accepted to IAT'06
  \end{itemize}
\end{itemize}
\end{frame}

\begin{frame}
\frametitle{Progress}
\begin{itemize}
\item Developed a robust, powerful, yet computationally "feasible" account
of multi-agent knowledge
  \begin{itemize}
  \item Developed effective reasoning procedure for universally-quantified situation calculus queries ("persistence condition" method)
  \item Preliminary implementation showing promise, but grappling with the
tractability of first-order reasoning
  \item Paper: "Property Persistence in the Situation Calculus", accepted to IJCAI'07
  \item Paper: "Knowledge and Observation in the Situation Calculus", in preparation for AAMAS'07
  \end{itemize}
\end{itemize}
\end{frame}

\begin{frame}
\frametitle{Progress}
\begin{itemize}
\item Developing strategies for coordination and cooperative execution
  \begin{itemize}
  \item Based on "knowing it is safe" to perform an action
  \item Very much a work in progress...
  \end{itemize}
\end{itemize}
\end{frame}

\subsection{Goal}

\begin{frame}
\frametitle{Ultimate Goal}
To demonstrate a team of simulated chefs, running autonomously on
several networked machines, preparing a banquet by executing a shared
high-level program.
\pause
\\\ 
\\
Or at least something approximating this...
\end{frame}
\end{document}

