
%\documentclass[handout, compress]{beamer}
\documentclass[compress]{beamer}

\mode<presentation>
{
  %\usetheme{Warsaw}
  \usetheme{Dresden}
  %\usetheme{Darmstadt}
  % or ...
  %\usecolortheme{dolphin}

  \setbeamercovered{transparent}
  % or whatever (possibly just delete it)
}

\usepackage[english]{babel}

\usepackage[latin1]{inputenc}

\usepackage{amsmath}
\usepackage[all]{xy}

\usepackage{times}
\usepackage[T1]{fontenc}

\title
{A Programming Language for Multi-Agent Teams}

\author
{Ryan kelly}

\institute[University of Melbourne]
{
  Department of Computer Science and Software Engineering\\
  University of Melbourne%\\
%(see http://www.cs.mu.oz.au/agentlab)
}

\date[The University of Melbourne]
{5th October 2005 / The University of Melbourne}

\subject{Talks}
% This is only inserted into the PDF information catalog. Can be left
% out. 

% If you have a file called "university-logo-filename.xxx", where xxx
% is a graphic format that can be processed by latex or pdflatex,
% resp., then you can add a logo as follows:

\pgfdeclareimage[height=0.8cm]{university-logo}{MINSilvr}
\logo{\pgfuseimage{university-logo}}

% If you wish to uncover everything in a step-wise fashion, uncomment
% the following command: 

%\beamerdefaultoverlayspecification{<+->}

\begin{document}

\begin{frame}
  \titlepage
\end{frame}

\begin{frame}
  \frametitle{Talk outline}
  \tableofcontents
  % You might wish to add the option [pausesections]
\end{frame}

\section{Thesis}

\subsection{Title}

\begin{frame}
\frametitle{A Programming Language for Multi-Agent Teams}
Abstract: In this thesis I...
\end{frame}

\subsection{Research problems}

\begin{frame}
When computational devices have overlapping or common objectives or when agents rely on one-another to complete joints tasks,
\begin{itemize}
\item<2> There is a lack of coordination algorithms available in the literature,
\item<3> The problem is especially challenging where agents have limited knowledge about, and control over, other agents.  
\item<4> There is a clear need for capturing and utilising richer dependency information in distributed computation tasks in ways that guarantee more effective and efficient solutions.
\end{itemize}
\end{frame}

\subsection{The plan}

\begin{frame}
\frametitle{The PhD Calendar}
\begin{itemize}
\item<1> Outline of the planned stages of your thesis
\item<2> Significant dates, including start, confirmation etc.
\end{itemize}
\end{frame}

\subsection{Progress}

\begin{frame}
\frametitle{Progress}
I have made the following progress:
\begin{itemize}
\item<2> Algorithms, protocols, methodologies developed
\item<3> Papers accepted into conferences or journals
\item<4> I don't have to ask what coffee to order at Baretto's (they know).
\end{itemize}
\end{frame}

\end{document}

