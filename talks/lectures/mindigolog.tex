
%\documentclass[handout, compress]{beamer}
\documentclass[compress]{beamer}

\mode<presentation>
{
  %\usetheme{Warsaw}
  \usetheme{Copenhagen}
  %\usetheme{Darmstadt}
  % or ...
  %\usecolortheme{dolphin}

  \setbeamercovered{transparent}
  % or whatever (possibly just delete it)
}

\usepackage[english]{babel}

\usepackage[latin1]{inputenc}

\usepackage{amsmath}
\usepackage[all]{xy}

\usepackage{times}
\usepackage[T1]{fontenc}

\usepackage{verbatim}
\usepackage{subfigure}


\title{MIndiGolog: Multi-Agent Golog}


\author{Ryan Kelly}

\institute[The University of Melbourne]
{
  Department of Computer Science and Software Engineering\\
  The University of Melbourne\\
  Victoria, 3010, Australia\\
  rfk@csse.unimelb.edu.au
}

\date[The University of Melbourne]
{27th March 2007}

% If you have a file called "university-logo-filename.xxx", where xxx
% is a graphic format that can be processed by latex or pdflatex,
% resp., then you can add a logo as follows:

\pgfdeclareimage[height=1.2cm]{university-logo}{MINSilvr}
\logo{\pgfuseimage{university-logo}}

\begin{document}

\begin{frame}
  \titlepage
\end{frame}

\begin{frame}
\centering "MIndiGolog"\\
\begin{itemize}
\item Operators of ConGolog (Concurrency, Interrupts)
\item Controlled Search of IndiGolog (in a later lecture)
\item Richer Theory of Action:
\begin{itemize}
\item Robust integration of \emph{true concurrency}
\item Explicit temporal component
\item Seamless integration of \emph{natural actions}
\end{itemize}
\end{itemize}
\ \\
\ \\
Vision: the cooperative execution of a shared high-level program by a team of autonomous agents.
\end{frame}

\begin{frame}
\frametitle{Example Task Specification}
\begin{multline*}
\mathbf{proc}\, MakeSalad(dest)\\
\left[\pi(agt)(ChopTypeInto(agt,Lettuce,dest))\,||\right.\\
\pi(agt)(ChopTypeInto(agt,Carrot,dest))\,||\\
\left.\pi(agt)(ChopTypeInto(agt,Tomato,dest))\right]\,;\\
\pi(agt)\left[acquire(agt,dest)\,;\right.\\
\,beginTask(agt,mix(dest,1))\,;\\
\left.\, release(agt,dest)\right]\,\,\mathbf{end}
\end{multline*}
\end{frame}

\begin{frame}
\frametitle{Concurrent Situation Calculus}
Replace action terms with sets of actions, all of which are performed
at the same time:\[
do(\{pickup(Thomas,Lettuce1),pickup(Richard,Tomato1)\},S_0)\]

Precondition interaction:\[
Poss(\{pickup(Richard,Obj),pickup(Harriet,Obj)\},s)\ ?\]

Basic solution:\[
Poss(C,s) \leftrightarrow \forall a \in C.Poss(a,s)\ \wedge\ \neg Conflicts(C,s)\]
\end{frame}

\begin{frame}
\frametitle{Integrating True Concurrency}
\begin{multline*}
\left[\pi(agt)(ChopTypeInto(agt,Lettuce,dest))\,||\right.\\
\pi(agt)(ChopTypeInto(agt,Carrot,dest))\,||\\
\left.\pi(agt)(ChopTypeInto(agt,Tomato,dest))\right]
\end{multline*}

Should take advantage of true concurrency.  Basic idea:
\begin{multline*}
Trans(\delta_{1}||\delta_{2},s,\delta',s')\equiv\exists\gamma\,.\, Trans(\delta_{1},s,\gamma,s')\wedge\delta'=(\gamma||\delta_{2})\\
\shoveright{\vee\exists\gamma\,.\, Trans(\delta_{2},s,\gamma,s')\wedge\delta'=(\delta_{1}||\gamma)}\\
\shoveright{\vee\exists c_{1},c_{2},\gamma_{1},\gamma_{2},t\,.\, Trans(\delta_{1},s,\gamma_{1},do(c_{1},t,s))\wedge}\\ 
Trans(\delta_{2},s,\gamma_{2},do(c_{2},t,s))\wedge\delta'=(\gamma_{1}||\gamma_{2})\wedge s'=do(c_{1}\cup c_{2},t,s)
\end{multline*}
\end{frame}

\begin{frame}
\frametitle{Robust Concurrency}
The combination of actions $(c_1\cup c_2)$ may not be possible.
\begin{itemize}
  \item Must check this explicitly
\end{itemize}
\ \\
\ \\
The same \emph{agent-initiated} action mustn't $Trans$ both programs.
\begin{itemize}
  \item otherwise dangerous 'skipping' of actions can occur
  \item if two concurrent programs both call for $pay(Ryan,\$100)$ to be performed, it had better be performed twice!
  \item Natural actions can transition both programs
\end{itemize}
\end{frame}

\begin{frame}
\frametitle{Robust Concurrency}
\begin{multline*}
Trans(\delta_{1}||\delta_{2},s,\delta',s')\equiv\exists\gamma\,.\, Trans(\delta_{1},s,\gamma,s')\wedge\delta'=(\gamma||\delta_{2})\\
\shoveright{\vee\exists\gamma\,.\, Trans(\delta_{2},s,\gamma,s')\wedge\delta'=(\delta_{1}||\gamma)}\\
\shoveright{\vee\exists c_{1},c_{2},\gamma_{1},\gamma_{2},t\,.\, Trans(\delta_{1},s,\gamma_{1},do(c_{1},t,s))}\\
\shoveright{\wedge Trans(\delta_{2},s,\gamma_{2},do(c_{2},t,s))\wedge\delta'=(\gamma_1||\gamma_2)\wedge s'=do(c_1\cup c_2,t,s)}\\
\wedge Poss(c_1\cup c_2,t,s)\wedge\forall a.\left[a\in c_{1}\wedge a\in c_{2}\rightarrow Natural(a)\right]
\end{multline*}
\end{frame}

\begin{frame}
\frametitle{Time}
Which is easier for a team to manage?
\begin{itemize}
  \pause
  \item Perform a list of actions concurrently
  \pause
  \item Perform a list of actions concurrently at time $t$
\end{itemize}
\ \\
\ \\
\pause
Assuming a good wall-clock is available, explicit time makes coordination easier.\\
\ \\
It is also generally a nice feature to have when representing dynamic worlds.
Example: bake a cake for \emph{10 minutes}
\end{frame}

\begin{frame}
\frametitle{The Temporal Situation Calculus}
We opt for a simple solution: attach an explicit time to each situation,
indicating when it was brought about:\[
\begin{array}{c}
do(a_1,t_1,S_0),\ do(a_2,t_2,do(a_1,t_1,S_0))\\
Poss(a,t,s) \leftrightarrow\ \dots\\
start(do(a,t,s)) = t
\end{array}\]

Temporal Golog finds situations with \emph{constraints} on the
time at which actions are performed:\[
\begin{array}{c}
do(a_2,t_2,do(a_1,t_1,S_0)\\
t_1>0,\ t_2>t_1+5
\end{array}\]
\end{frame}


\begin{frame}
\frametitle{Continuous Actions}
Some actions have a finite duration: $mix(Thomas,Bowl1,5)$
Others are instantaneous: $pickup(Robert,Lettuce1)$.
So this concurrent action is problematic:\[
\{mix(Thomas,Bowl1,5),pickup(Robert,Lettuce1)\}\]
Standard solution: decompose continuous actions into instantaneous $begin$
and $end$ actions:\[
begin\_mix(Thomas,Bowl1,5) \dots end\_mix(Thomas,Bowl1,5)\]
But $end\_mix()$ is not a standard action.  It cannot be performed at
any time.  Rather, it \emph{must} occur whenever it is possible.
\end{frame}

\begin{frame}
\frametitle{Natural Actions}
\begin{itemize}
  \item Natural Action:  an action that must occur whenever it is possible,
        unless some earlier action prevents it.
  \item $Poss(na,t,s)$ predicts when natural actions will occur.
  \item Least Natural Time Point: the earliest time at which a natural
        action will occur for a given situation:\[
\begin{array}{c}
lntp(s,t) \leftrightarrow \exists a [natural(a) \wedge Poss(a,t,s)] \wedge\\
\forall a,t_a [natural(a) \wedge Poss(a,t_a,s) \rightarrow t \leq t_a]
\end{array}\]
  \item Legal Situation: one in which any natural actions that could have occured, did occur:\[
\begin{array}{c}
legal(do(c,t,s)) \leftrightarrow legal(s) \wedge\\
Poss(c,t,s) \wedge start(s) \leq t \wedge\\
\forall a,t_a [natural(a) \wedge Poss(a,t_a,s) \rightarrow (a \in c \vee t < t_a)]
\end{array}\]
\end{itemize}
\end{frame}

\begin{frame}
\frametitle{Integrating Natural Actions}
$Trans$ for a single action can proceed in one of four ways:
\begin{itemize}
\item If no natural actions are possible, just do the action
\item If there are natural actions possible, either:
\begin{itemize}
\item Do the action \emph{before} the LNTP
\item Do the action \emph{at} the LNTP, with the natural actions
\item Wait for the natural actions to occur
\end{itemize}
\end{itemize}
\ \\
\ \\
The equation for this wont fit on these slides in a readable manner...
\end{frame}

\begin{frame}
\frametitle{In Summary}
\begin{itemize}
\item The same powerful operators as ConGolog
\item A Robustly Multi-Agent Semantics:
\begin{itemize}
\item True Concurrency
\item Explicit Time
\item Natural Actions
\end{itemize}
\end{itemize}
\end{frame}

\end{document}
