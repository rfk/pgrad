
%\documentclass[handout, compress]{beamer}
\documentclass[compress]{beamer}

\mode<presentation>
{
  %\usetheme{Warsaw}
  \usetheme{Dresden}
  %\usetheme{Darmstadt}
  % or ...
  %\usecolortheme{dolphin}

  \setbeamercovered{transparent}
  % or whatever (possibly just delete it)
}

\usepackage[english]{babel}

\usepackage[latin1]{inputenc}

\usepackage{amsmath}
\usepackage[all]{xy}

\usepackage{times}
\usepackage[T1]{fontenc}

\usepackage{algorithm}
\usepackage{algorithmic}

\newcommand{\isdef}{\hbox{$\stackrel{\mbox{\tiny def}}{=}$}}


\title
{Property Persistence in the Situation Calculus}

\author
{Ryan F. Kelly\\
and Adrian R. Pearce}

\institute[The University of Melbourne]
{
  NICTA Victoria Research Laboratory\\
  Department of Computer Science and Software Engineering\\
  The University of Melbourne\\
  Victoria, 3010, Australia\\
  \{rfk,adrian\}@csse.unimelb.edu.au
}

\date[The University of Melbourne]
{10th January 2007}

% If you have a file called "university-logo-filename.xxx", where xxx
% is a graphic format that can be processed by latex or pdflatex,
% resp., then you can add a logo as follows:

\pgfdeclareimage[height=1.2cm]{university-logo}{MINSilvr}
\logo{\pgfuseimage{university-logo}}

% If you wish to uncover everything in a step-wise fashion, uncomment
% the following command: 

%\beamerdefaultoverlayspecification{<+->}

\begin{document}

\begin{frame}
  \titlepage
\end{frame}

\section{Introduction}

\begin{frame}
\centering "Property Persistence in the Situation Calculus"
\ \\
\ \\
\begin{itemize}
\item Motivation
\item Situation Calculus Review
\item The Persistence Problem
\item The Persistence Condition
\item Applications
\end{itemize}
\end{frame}

\section{Motivation}

\begin{frame}
\frametitle{Motivation}
A situation calculus theory $\mathcal{D}$ contains a second-order induction axiom, so automated reasoning can be tricky:
\begin{equation*}
\mathcal{D} \models_{SOL} \psi
\end{equation*}
For some queries we can discard the second-order axiom:
\begin{equation*}
\mathcal{D} \models_{SOL} \exists s.\psi\,\,\,\,\,\mathit{iff}\,\,\,\,\,\mathcal{D}-\{I\}\models_{FOL}\exists s.\psi
\end{equation*}
Sometimes we can even "compile away" other axioms:
\begin{equation*}
\mathcal{D} \models_{SOL} \phi\,\,\,\,\,\mathit{iff}\,\,\,\,\,\mathcal{D}_{una}\cup\mathcal{D}_{S_0}\models_{FOL} \mathcal{R}^*_{\mathcal{D}}[\phi]
\end{equation*}
\end{frame}

\begin{frame}
\frametitle{Motivation}
Many interesting problems require \emph{universal} quantification over
situations, meaning they require the induction axiom.
\begin{itemize}
  \item Goal impossibility: $\,\,\,\,\mathcal{D} \models \forall s\,.\,S_0 \leq s \rightarrow \neg G(s)$
  \item Verifying state constraints
  \item Need for cooperation: $\,\,\,\,\mathcal{D} \models \forall s'\,.\,s \leq_{OwnAction} s' \rightarrow \neg G(s')$
  \item Knowledge with partially observability of actions
\end{itemize}
\ \\
\ \\
Aim: effective automated reasoning for such queries
\end{frame}

\section{Background}

\begin{frame}
\frametitle{Review: The Situation Calculus}
\emph{Actions} are instantaneous events causing the world to change
\begin{itemize}
  \item $pickup(Thomas,Bowl)$, $beginTask(Richard,mix(Bowl,1))$
\end{itemize}
\emph{Situations} are histories of actions that have been performed
\begin{itemize}
  \item $S_0$, $do(pickup(Harriet,Knife),S_0)$
\end{itemize}
\emph{Fluents} are situation-dependent properties of the world
\begin{itemize}
  \item $Poss(a,s)$, $Holding(Harriet,Knife1,s)$
\end{itemize}
\emph{Successor State Axioms} solve the frame problem
\begin{itemize}
  \item $Holding(i,r,do(a,s)) \equiv$ \\ $\,\,\,\,\,a=pickup(i,r) \vee Holding(i,r,s) \wedge a\neq drop(i,r)$
\end{itemize}
\ \\
\end{frame}

\begin{frame}
\frametitle{Review: The Situation Calculus}
Theories limited to a particular form for effective reasoning:
\begin{equation*}
\mathcal{D}\,\,=\,\,\Sigma\cup\mathcal{D}_{ss}\cup\mathcal{D}_{ap}\cup\mathcal{D}_{una}\cup\mathcal{D}_{S_0}
\end{equation*}
\ \\
\ \\
"Uniform Formulae" (or \emph{properties})
\begin{itemize}
 \item combinations of fluents referring to common situation
 \item $\phi[s]$ indicates replacement of situation term with $s$
\end{itemize}

\end{frame}

\begin{frame}
\frametitle{Review: The Situation Calculus}
Ordering over situations:
\begin{gather*}
\neg\left(s \sqsubset S_0\right)\\
s\sqsubset do(a,s')\equiv s\sqsubseteq s'
\end{gather*}
\ \\
\ \\
"Action Description Predicates" (e.g. $Poss(a,s)$):
\begin{equation*}
\alpha(A(\overrightarrow{x}),s) \equiv \Pi_A(\overrightarrow{x},s)\,\,\,\,\,\in\,\,\,\,\,\mathcal{D}_{ap}
\end{equation*}
\begin{gather*}
\neg\left(s <_{\alpha} S_0\right)\\
s <_{\alpha} do(a,s')\equiv s \leq_{\alpha} s' \wedge \alpha(a,s')
\end{gather*}

\end{frame}

\begin{frame}
\frametitle{Review: The Situation Calculus}
Regression is a syntactic transformation that "winds back" a formula
uniform in $do(a,s)$ to a formula uniform in $s$:
\begin{equation*}
\mathcal{D} \models \phi \equiv \mathcal{R}_{\mathcal{D}}[\phi]
\end{equation*}
If $\phi$ is uniform in a ground situation term, we can regress it all the
way back to one uniform in $S_0$:
\begin{equation*}
\mathcal{D} \models \phi\,\,\,\,\,\mathit{iff}\,\,\,\,\,\mathcal{D}_{una}\cup\mathcal{D}_{S_0} \models \mathcal{R}^*_{\mathcal{D}}[\phi]
\end{equation*}
This idea is the main technique for effective reasoning in the situation calculus. However, not all formulae can be regressed.
\end{frame}


\section{Property Persistence}

\begin{frame}
\frametitle{Property Persistence}
We are often interested in queries of the form:
\begin{equation*}
\mathcal{D} \models \forall s'\,.\,s \leq_{\alpha} s' \rightarrow \phi[s']
\end{equation*}
\ \\
"$\phi$ holds in situation s, and if all further actions satisfy $\alpha$ then $\phi$ will continue to hold"
\ \\
\ \\
Or just: "$\phi$ persists under $\alpha$"
\ \\
\ \\
Universal quantification over situations means that we cannot use regression
for such queries.
\end{frame}

\section{Persistence Condition}

\begin{frame}
\frametitle{The Persistence Condition}
Idea: for a given $\phi$ and $\alpha$, calculate a uniform formula that implies the persistence of $\phi$ under $\alpha$:
\begin{equation*}
  \mathcal{D}-\mathcal{D}_{S_0} \models \left(\forall s'\,.\,s \leq_{\alpha} s' \rightarrow \phi[s']\right) \equiv \mathcal{P}_{\mathcal{D}}[\phi,\alpha][s]
\end{equation*}

Since this "persistence condition" is uniform in $s$, we can handle it using
standard effective reasoning techniques.
\ \\
\ \\
But, does it exist? Can we calculate it effectively?
\end{frame}

\begin{frame}
\frametitle{A Tantalizing Possibility}
Generalizing a result from Lin and Reiter's work on state constraints:
\begin{gather*}
\mathcal{D}-\mathcal{D}_{S_{0}}\models\forall s\,.\,\phi[s]\rightarrow\left(\forall s'\,.\, s\leq_{\alpha}s'\rightarrow\phi[s']\right)\\
\mathit{iff}\\
\mathcal{D}_{una}\models\forall s,a\,.\,\phi[s]\wedge\mathcal{R}_{\mathcal{D}}[\alpha(a,s)]\rightarrow\mathcal{R}_{\mathcal{D}}[\phi[do(a,s)]]
\end{gather*}
\ \\
\ \\
So, it's relatively straightforward to check whether $\phi \equiv \mathcal{P}_{\mathcal{D}}[\phi,\alpha]$
\end{frame}

\begin{frame}
\frametitle{Persistence to Finite Depth}
Define "persistence to depth 1" as:
\begin{equation*}
  \mathcal{P}^1_{\mathcal{D}}[\phi,\alpha]\,\,\isdef\,\,\phi[s] \wedge \forall a\,.\,\mathcal{R}_{\mathcal{D}}[\alpha(a,s)\rightarrow \phi[do(a,s)]]
\end{equation*}

Then "persistence to depth $n$" can be defined recursively:
\begin{equation*}
  \mathcal{P}^n_{\mathcal{D}}[\phi,\alpha]\,\,\isdef\,\,\mathcal{P}^1_{\mathcal{D}}[\mathcal{P}^{n-1}_{\mathcal{D}}[\phi,\alpha],\alpha]
\end{equation*}

Note that $\mathcal{P}^{n}_{\mathcal{D}}$ is always uniform in $s$.
\end{frame}

\begin{frame}
\frametitle{Persistence as a Fixed-Point}
It is possible to show that:
\begin{gather*}
\mathcal{D}_{una}\models\forall s\,.\,\mathcal{P}_{\mathcal{D}}^{n}[\phi,\alpha][s]\rightarrow\mathcal{P}_{\mathcal{D}}^{n+1}[\phi,\alpha][s]\label{eqn:pn_persists}\\
\mathit{iff}\nonumber \\
\mathcal{D}-\mathcal{D}_{s_{0}}\models\forall s\,.\,\mathcal{P}_{\mathcal{D}}^{n}[\phi,\alpha][s]\equiv\mathcal{P_{D}}[\phi,\alpha][s]\label{eqn:pn_equiv_persists}
\end{gather*}
\ \\
\ \\
So $\mathcal{P}_{\mathcal{D}}$ is a fixed-point of $\mathcal{P}^1_{\mathcal{D}}$

We can seek it using a straightforward iterative algorithm.
\end{frame}

\begin{frame}
\frametitle{Properties of the Fixed-Point}
Since $\mathcal{P}^1_{\mathcal{D}}$ is monotone, $\mathcal{P}_{\mathcal{D}}$:
\begin{itemize}
  \item is guaranteed to exist
  \item can be constructed by trans-finite iteration
\end{itemize}

Termination with \emph{finite} iteration requires that applications of $\mathcal{P}^1_{\mathcal{D}}$ do not create infinite chains.  We are developing syntactic restrictions on successor state axioms that can guarantee this.
\end{frame}

\begin{frame}
\frametitle{Effectiveness}
Is this an effective technique in practice?
\begin{itemize}
 \item purely first-order reasoning
 \item uses only the unique names axioms
 \item result is uniform, so can be used with regression
 \item $\mathcal{P}_{\mathcal{D}}$ can be cached for a given $\phi$ and $\alpha$
 \item many properties fail to persist for small $n$
\end{itemize}
\ \\
\ \\
We also inherit potential disadvantages of regression operator:
\begin{itemize}
  \item Length of $\mathcal{P}_{\mathcal{D}}[\phi,\alpha]$ can be exponential in length of $\phi$
\end{itemize}
\end{frame}

\section{Applications}

\begin{frame}
\frametitle{Applications}
Goal Impossibility:
\begin{equation*}
\mathcal{D}_{una}\cup\mathcal{D}_{S_0} \models \mathcal{P}_{\mathcal{D}}[\neg G(s),Poss][S_0]
\end{equation*}

Goal Futility:
\begin{equation*}
\mathcal{D}_{una}\cup\mathcal{D}_{S_0} \models \mathcal{R}^*_{\mathcal{D}}[\mathcal{P}_{\mathcal{D}}[\neg G(s),Poss][s]]
\end{equation*}

Knowledge with Partial Observability of Actions:
\begin{multline*}
\mathcal{R}_{\mathcal{D}}[\mathbf{Knows}(\phi,do(a,s))]\,\,=\\
  \mathbf{Knows}(\mathcal{R}_{\mathcal{D}}[\mathcal{P}_{\mathcal{D}}[\phi,CantObs][do(a,now)]],s)
\end{multline*}
\end{frame}

\section{Conclusions}

\begin{frame}
\frametitle{Conclusions}
We have significantly increased the scope of queries that can be effectively
posed in the situation calculus:
\begin{itemize}
 \item Identified useful subset of universally quantified queries
 \item Developed algorithm to transform them to managable form
 \item Integrates well with existing tools for effective reasoning
\end{itemize}
Still to come:
\begin{itemize}
 \item Stronger termination guarantees
 \item SitCalc extensions: concurrent actions, continuous time, ...
\end{itemize}
\end{frame}

\begin{frame}
\centering \large Thank You\\
\end{frame}

\end{document}
