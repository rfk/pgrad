
%\documentclass[handout, compress]{beamer}
\documentclass[compress]{beamer}

\mode<presentation>
{
  %\usetheme{Warsaw}
  \usetheme{Dresden}
  %\usetheme{Darmstadt}
  % or ...
  %\usecolortheme{dolphin}

  \setbeamercovered{transparent}
  % or whatever (possibly just delete it)
}

\usepackage[english]{babel}

\usepackage[latin1]{inputenc}

\usepackage{amsmath}
\usepackage[all]{xy}

\usepackage{times}
\usepackage[T1]{fontenc}

\title
{Towards High-Level Programming\\ for Distributed Problem Solving}

\author
{Ryan F. Kelly\\
and Adrian R. Pearce}

\institute[The University of Melbourne]
{
  Department of Computer Science and Software Engineering\\
  The University of Melbourne
}

\date[The University of Melbourne]
{20th December 2006}

% If you have a file called "university-logo-filename.xxx", where xxx
% is a graphic format that can be processed by latex or pdflatex,
% resp., then you can add a logo as follows:

%\pgfdeclareimage[height=0.8cm]{university-logo}{MINSilvr}
%\logo{\pgfuseimage{university-logo}}

% If you wish to uncover everything in a step-wise fashion, uncomment
% the following command: 

%\beamerdefaultoverlayspecification{<+->}

\begin{document}

\begin{frame}
  \titlepage
\end{frame}

\section{Introduction}

\begin{frame}
\frametitle{Title Keywords}
\centering "Towards High-Level Programming\\
 for Distributed Problem Solving"
\ \\
\ \\
\begin{itemize}
\pause
\item Towards
\pause
\item High-Level Programming
\pause
\item Distributed
\end{itemize}
\end{frame}

\begin{frame}
\frametitle{Outline}
\tableofcontents
\end{frame}

\section{Motivation}

\begin{frame}
\frametitle{Distributed Problem Solving}
A team of agents cooperating closely to perform a shared task.
\begin{itemize}
\item Task Specification
\item Cooperative Execution Strategy
\end{itemize}
\ \\
\ \\
\pause
Many interesting problems outside our research scope:
\begin{itemize}
\item Team formation and maintenance
\item Formulating joint plans
\item Detecting and recovering from failure
\item ...
\end{itemize}
\end{frame}

\begin{frame}
\frametitle{Task Specification}
HTN is a popular and powerful choice.
\ \\
\ \\
High-Level Programming is a promising alternative from single-agent systems:
\begin{itemize}
\item Primitive actions from the agent's world
\item Connected by standard programming constructs
\item Containing controlled amounts of nondeterminism
\end{itemize}
\ \\
\ \\
GOLOG, Dynamic Logic
\end{frame}

\begin{frame}
\frametitle{Why High-Level Programming?}
\begin{itemize}
\item Natural, flexible task specification
\item Powerful nondeterminism control
  \begin{itemize}
  \item order of actions, who does what, ...
  \end{itemize}
\item Sophisticated logic of action
  \begin{itemize}
  \item Concurrent actions, continuous actions, explicit time, ...
  \end{itemize}
\end{itemize}
\ \\
\ \\
Vision: the cooperative, distributed execution of a shared high-level program by a team of autonomous agents.
TODO: note ReadyLog work as motivator
\end{frame}


\section{Background}

\begin{frame}
\frametitle{Review: Situation Calculus}
\emph{Actions} are instantaneous events causing the world to change
\begin{itemize}
  \item $pickup(Thomas,Bowl)$, $beginTask(Richard,mix(Bowl,1))$
\end{itemize}
\emph{Situations} are histories of actions that have been performed
\begin{itemize}
  \item $S_0$, $do(pickup(Harriet,Knife),S_0)$
\end{itemize}
\emph{Fluents} are situation-dependent properties of the world
\begin{itemize}
  \item $Holding(Harriet,Knife1,s)$
\end{itemize}
Successor State Axioms - monotonic solution to the frame problem
\end{frame}

\begin{frame}
\frametitle{Situation Calculus Extensions}
Concurrency:
\begin{itemize}
  \item Actions replaced with \emph{sets} of actions: $do({a_1,a_2},s)$
  \item Need to be careful about interactions
\end{itemize}
\pause
Explicit Time:
\begin{itemize}
  \item Add temporal component: $do(c,t,s)$
  \item Need to be careful about interactions
\end{itemize}
\pause
Predictable Exogenousl Actions ("Natural Actions"):
\begin{itemize}
  \item \epmh{Must} occur whenever $Poss(a,t,s)$ permits
  \item "Legal" situations respect this requirement: $Legal(s)$
\end{itemize}
\end{frame}

\begin{frame}
\frametitle{Review: Golog}
The "Golog Family":
\begin{itemize}
  \item Original GOLOG
  \item ConColog: interleaved concurrency
  \item IndiGolog: online execution
  \item ...
  \item MIndiGolog
\end{itemize}
Standard Operators:
\begin{itemize}
  \item Sequence:  $\delta_1\,;\,\delta_2$
  \item Choice: $\delta_1\,|\,\delta_2$
  \item Nondet Args: $\pi(x)\delta(x)$
  \item Concurrent Exec: $\delta_1\,||\,\delta_2$
\end{itemize}
\end{frame}

\begin{frame}
\frametitle{Golog Example}
\end{frame}

\section{MIndiGolog}

\begin{frame}
\frametitle{Concurrent Execution}
\end{frame}

\begin{frame}
\frametitle{Natural Actions}
\end{frame}

\begin{frame}
\frametitle{Example Execution}
\end{frame}

\section{Distributed Planning}

\begin{frame}
\frametitle{Distributed Logic Programing with Oz}
\end{frame}

\section{Conclusions}

\begin{frame}
\frametitle{Achievements}
\end{frame}

\begin{frame}
\frametitle{Future Work}
\end{frame}

\begin{frame}
\end{frame}

\end{document}
