
%\documentclass[handout, compress]{beamer}
\documentclass[compress]{beamer}

\mode<presentation>
{
  %\usetheme{Warsaw}
  \usetheme{Dresden}
  %\usetheme{Darmstadt}
  % or ...
  %\usecolortheme{dolphin}

  \setbeamercovered{transparent}
  % or whatever (possibly just delete it)
}

\usepackage[english]{babel}

\usepackage[latin1]{inputenc}

\usepackage{amsmath}
\usepackage[all]{xy}

\usepackage{times}
\usepackage[T1]{fontenc}

\usepackage{verbatim}
\usepackage{subfigure}


\title
{Towards High-Level Programming\\ for Distributed Problem Solving}

\author
{Ryan F. Kelly\\
and Adrian R. Pearce}

\institute[The University of Melbourne]
{
  Department of Computer Science and Software Engineering\\
  The University of Melbourne\\
  Victoria, 3010, Australia\\
  \{rfk,adrian\}@csse.unimelb.edu.au
}

\date[The University of Melbourne]
{20th December 2006}

% If you have a file called "university-logo-filename.xxx", where xxx
% is a graphic format that can be processed by latex or pdflatex,
% resp., then you can add a logo as follows:

\pgfdeclareimage[height=1.2cm]{university-logo}{MINSilvr}
\logo{\pgfuseimage{university-logo}}

% If you wish to uncover everything in a step-wise fashion, uncomment
% the following command: 

%\beamerdefaultoverlayspecification{<+->}

\begin{document}

\begin{frame}
  \titlepage
\end{frame}

\section{Introduction}

\begin{frame}
\centering "Towards High-Level Programming\\
 for Distributed Problem Solving"
\ \\
\ \\
\begin{itemize}
\pause
\item Towards
\pause
\item High-Level Programming
  \begin{itemize}
  \item Task Specification: "MIndiGolog"
  \end{itemize}
\pause
\item Distributed
  \begin{itemize}
  \item Cooperative Planning: Oz
  \end{itemize}
\end{itemize}
\end{frame}

\section{Motivation}

\begin{frame}
\frametitle{Task Specification}
High-Level Programming is a promising approach from single-agent systems:
\begin{itemize}
\item Primitive actions from the agent's world
\item Connected by standard programming constructs
\item Containing controlled amounts of nondeterminism
\item e.g. GOLOG, Dynamic Logic
\end{itemize}
\ \\
\ \\
Vision: the cooperative execution of a shared high-level program by a team of autonomous agents.
\end{frame}

\begin{frame}
\frametitle{Example Task Specification}
\begin{multline*}
\mathbf{proc}\, MakeSalad(dest)\\
\left[\pi(agt)(ChopTypeInto(agt,Lettuce,dest))\,||\right.\\
\pi(agt)(ChopTypeInto(agt,Carrot,dest))\,||\\
\left.\pi(agt)(ChopTypeInto(agt,Tomato,dest))\right]\,;\\
\pi(agt)\left[acquire(agt,dest)\,;\right.\\
\,beginTask(agt,mix(dest,1))\,;\\
\left.\, release(agt,dest)\right]\,\,\mathbf{end}
\end{multline*}
\end{frame}

\begin{frame}
\frametitle{Why High-Level Programming?}
\begin{itemize}
\item Natural, flexible task specification
\item Powerful nondeterminism control
  \begin{itemize}
  \item order of actions, who does what, ...
  \end{itemize}
\item Sophisticated logic of action
  \begin{itemize}
  \item Concurrent actions, continuous actions, explicit time, ...
  \end{itemize}
\end{itemize}
\ \\
\ \\
Ferrein, Lakemeyer et.al. have successfully controlled a RoboCup team using
a Golog variant called "ReadyLog".
\end{frame}

\begin{frame}
\frametitle{Golog for Multiple Agents}
The "Golog Family" includes:
\begin{itemize}
  \item Original GOLOG
  \item ConColog: interleaved concurrency
  \item IndiGolog: online execution
\end{itemize}
\ \\
\ \\
MIndiGolog facilitates this approach in M.A. domains:
\begin{itemize}
\item Robust integration of \emph{true concurrency}
\item Explicit temporal component
\item Seamless integration of \emph{natural actions}
\end{itemize}
\end{frame}


\section{Background}

\begin{frame}
\frametitle{Review: Situation Calculus}
\emph{Actions} are instantaneous events causing the world to change
\begin{itemize}
  \item $pickup(Thomas,Bowl)$, $beginTask(Richard,mix(Bowl,1))$
\end{itemize}
\ \\
\ \\
\emph{Situations} are histories of actions that have been performed
\begin{itemize}
  \item $S_0$, $do(pickup(Harriet,Knife),S_0)$
\end{itemize}
\ \\
\ \\
\emph{Fluents} are situation-dependent properties of the world
\begin{itemize}
  \item $Poss(a,s)$, $Holding(Harriet,Knife1,s)$
\end{itemize}
\end{frame}

\begin{frame}
\frametitle{Situation Calculus Extensions}
Concurrency:
\begin{itemize}
  \item Actions replaced with \emph{sets} of actions: $do(\{a_1,a_2\},s)$
  \item Need to be careful about interactions
\end{itemize}
Explicit Time:
\begin{itemize}
  \item Add temporal component: $do(c,t,s)$
\end{itemize}
Predictable Exogenous Actions ("Natural Actions"):
\begin{itemize}
  \item \emph{Must} occur whenever $Poss(a,t,s)$ permits
  \item "Legal" situations respect this requirement: $Legal(s)$
\end{itemize}
\end{frame}

\begin{frame}
\frametitle{Review: Golog}
Operators defined using transition semantics:
\begin{multline*}
Trans(\delta_{1}||\delta_{2},s,\delta',s')\equiv\exists\gamma.Trans(\delta_{1},s,\gamma,s')\wedge\delta'=(\gamma||\delta_{2})\\
\vee\,\exists\gamma.Trans(\delta_{2},s,\gamma,s')\wedge\delta'=(\delta_{1}||\gamma)
\end{multline*}
\ \\
\ \\
Agents must reason about their actions to plan a \emph{legal execution} of
a high-level program $\delta$:
\begin{equation*}
\mathcal{D}\models\exists s\,.\,Trans*(\delta,S_0,\delta',s)\wedge Final(\delta',s)
\end{equation*}
\end{frame}

\section{MIndiGolog}

\begin{frame}
\frametitle{Integrating True Concurrency}
\begin{multline*}
\left[\pi(agt)(ChopTypeInto(agt,Lettuce,dest))\,||\right.\\
\pi(agt)(ChopTypeInto(agt,Carrot,dest))\,||\\
\left.\pi(agt)(ChopTypeInto(agt,Tomato,dest))\right]
\end{multline*}
\pause
Should take advantage of true concurrency.  Basic idea:
\begin{multline*}
Trans(\delta_{1}||\delta_{2},s,\delta',s')\equiv\exists\gamma\,.\, Trans(\delta_{1},s,\gamma,s')\wedge\delta'=(\gamma||\delta_{2})\\
\shoveright{\vee\exists\gamma\,.\, Trans(\delta_{2},s,\gamma,s')\wedge\delta'=(\delta_{1}||\gamma)}\\
\shoveright{\vee\exists c_{1},c_{2},\gamma_{1},\gamma_{2},t\,.\, Trans(\delta_{1},s,\gamma_{1},do(c_{1},t,s))\wedge}\\ 
Trans(\delta_{2},s,\gamma_{2},do(c_{2},t,s))\wedge\delta'=(\gamma_{1}||\gamma_{2})\wedge s'=do(c_{1}\cup c_{2},t,s)
\end{multline*}
\end{frame}

\begin{frame}
\frametitle{Robust Concurrency}
The combination of actions $(c_1\cup c_2)$ may not be possible.
\begin{itemize}
  \item Must check this explicitly
\end{itemize}
\ \\
\ \\
The same \emph{agent-initiated} action mustn't $Trans$ both programs.
\begin{itemize}
  \item otherwise dangerous 'skipping' of actions can occur
  \item if two concurrent programs both call for $pay(Ryan,\$100)$ to be performed, it had better be performed twice!
  \item Natural actions can transition both programs
\end{itemize}
\end{frame}

\begin{frame}
\frametitle{Robust Concurrency}
\begin{multline*}
Trans(\delta_{1}||\delta_{2},s,\delta',s')\equiv\exists\gamma\,.\, Trans(\delta_{1},s,\gamma,s')\wedge\delta'=(\gamma||\delta_{2})\\
\shoveright{\vee\exists\gamma\,.\, Trans(\delta_{2},s,\gamma,s')\wedge\delta'=(\delta_{1}||\gamma)}\\
\shoveright{\vee\exists c_{1},c_{2},\gamma_{1},\gamma_{2},t\,.\, Trans(\delta_{1},s,\gamma_{1},do(c_{1},t,s))}\\
\shoveright{\wedge Trans(\delta_{2},s,\gamma_{2},do(c_{2},t,s))\wedge\delta'=(\gamma_1||\gamma_2)\wedge s'=do(c_1\cup c_2,t,s)}\\
\wedge Poss(c_1\cup c_2,t,s)\wedge\forall a.\left[a\in c_{1}\wedge a\in c_{2}\rightarrow Natural(a)\right]
\end{multline*}
\end{frame}

\begin{frame}
\frametitle{Natural Actions}
Previous work required the programmer to explicitly predict natural actions
and include them in the program.
\ \\
\ \\
We modify $Trans$ to predict the occurrence of natural actions automatically.
\begin{itemize}
  \item The programmer can completely ignore natural actions
  \item Relevant ones can be included directly in the program, to mean "wait for this natural action to occur"
\end{itemize}
\end{frame}

\section{Distributed Planning}

\begin{frame}
\frametitle{Distributed Execution}
\begin{itemize}
 \item Agents can each plan a legal execution individually
 \item Identical search strategy produces identical results
 \item Coordination without communication!
 \item Requires a fully observable, completely known world
\end{itemize}
\ \\
\ \\
But, we can also take advantage of communication to share the planning
workload between agents.

\end{frame}

\begin{frame}
\frametitle{Execution Planning is a Logic Program}
The semantics of Golog can be neatly encoded as a logic program,
traditionally in Prolog.  We use Oz for its strong distributed programming
support.

\small \verbatiminput{listings/goloz_trans.oz}

\end{frame}

\begin{frame}
\frametitle{Distributed Logic Programing with Oz}
By utilizing techniques for distributed logic programming, the agents
can share the computational workload:

\small \verbatiminput{listings/goloz_do_parallel.oz}

But, what about partially-observable domains?

\end{frame}

\section{Conclusions}

\begin{frame}
\frametitle{Conclusions}
We have:
\begin{itemize}
  \item Developed MIndigGolog: extension to IndiGolog suitable for specifying the behavior of multi-agent teams
  \item Demonstrated that the execution-planning workload can be shared using off-the-shelf techniques
\end{itemize}

In progress:
\begin{itemize}
  \item An account of multi-agent knowledge in partially-observable domains
  \item A knowledge-based distributed execution strategy
  \item Workload sharing using distributed constraint satisfaction
\end{itemize}
\end{frame}

\begin{frame}
\centering \large Thank You\\
\end{frame}

\end{document}
