\documentclass[compress]{beamer}

\mode<presentation>
{
  \usetheme{Dresden}
%  \setbeamercovered{transparent}
}

\usepackage[english]{babel}

\usepackage[latin1]{inputenc}

\usepackage{amsmath}
\usepackage[all]{xy}

\usepackage{times}
\usepackage[T1]{fontenc}

\usepackage{algorithm}
\usepackage{algorithmic}

\newcommand{\isdef}{\hbox{$\stackrel{\mbox{\tiny def}}{=}$}}
\newcommand{\Dt}{\mathcal{D}}
\newcommand{\Reg}{\mathcal{R}}
\newcommand{\Pst}{\mathcal{P}}
\newcommand{\Trn}{\mathcal{T}}
\newcommand{\Kln}{\mathcal{K}}
\newcommand{\TrnA}{\Trn_{a}}
\newcommand{\EKnows}{\mathbf{EKnows}}
\newcommand{\Knows}{\mathbf{Knows}}
\newcommand{\CKnows}{\mathbf{CKnows}}
\newcommand{\KnowsZ}{\mathbf{Knows_{0}}}
\newcommand{\PKnowsZ}{\mathbf{PKnows_{0}}}
\newcommand{\PKnows}{\mathbf{PKnows}}
\newcommand{\KTrans}{\mathbf{KDo}}
\newcommand{\KDo}{\mathbf{KDo}}
\newcommand{\EDo}{\mathbf{EDo}}
\newcommand{\vars}[1]{\bar{#1}}
\newcommand{\PbU}{PbU}


\title
{Complex Epistemic Modalities \\in the Situation Calculus}

\author
{Ryan F. Kelly\\
and Adrian R. Pearce}

\institute[The University of Melbourne]
{
  Department of Computer Science and Software Engineering\\
  The University of Melbourne\\
  Victoria, 3010, Australia\\
  \{rfk,adrian\}@csse.unimelb.edu.au
}

\date[The University of Melbourne]
{19th September 2007}

\pgfdeclareimage[height=1.2cm]{university-logo}{MINSilvr}
\logo{\pgfuseimage{university-logo}}

\begin{document}

\begin{frame}
  \titlepage
\end{frame}

\section{Motivation}

\begin{frame}
\frametitle{Motivation}
We want:
\begin{itemize}
\item A rich theory of knowledge and action; with
\item Group-level knowledge operators; and
\item An effective reasoning procedure
\end{itemize}
\ \\
\ \\
\pause
Or, more concretely:
\begin{itemize}
\item The Situation Calculus; with
\item Common Knowledge; and
\item A Regression Rule
\end{itemize}
\end{frame}

\section{Background}

\begin{frame}
\frametitle{The Situation Calculus}
\emph{Actions} are instantaneous events causing the world to change
\begin{itemize}
  \item $read(Ann)$
\end{itemize}
\ \\
\ \\
\emph{Situations} are histories of actions that have been performed
\begin{itemize}
  \item $S_0$,\ \ \ $do(\{read(Bob)\},S_0)$
\end{itemize}
\ \\
\ \\
\emph{Fluents} are situation-dependent properties of the world
\begin{itemize}
  \item $Poss(c,s)$, $loc(s) = x$
\end{itemize}
\end{frame}

\begin{frame}
\frametitle{Knowledge in the Situation Calculus}
We can reify possible worlds using "possible situations":
\begin{equation*}
 K(agt,s',s)
\end{equation*}
\ \\
\ \\
And define knowledge using a macro:
\begin{equation*}
\Knows(agt,\phi,s)\,\,\isdef\,\,\forall s': \left[ K(agt,s',s) \rightarrow \phi[s']\right]
\end{equation*}
\end{frame}

\begin{frame}
\frametitle{Knowledge in the Situation Calculus}
Sensing etc. is axiomatised by \emph{observations}:
\begin{gather*}
Obs(Ann,\{read(Ann\},s) = \{read(Ann)\#loc(s)\}\\
Obs(Bob,\{read(Ann)\},s) = \{read(Ann)\}
\end{gather*}
\pause
Successor state axiom for $K$:
\begin{multline*}
K(agt,s'',do(c,s)) \equiv \exists c',s': s'' = do(c',s') \wedge K(agt,s',s)\\
  Poss(c',s') \wedge Obs(agt,c',s') = Obs(agt,c,s)
\end{multline*}
\end{frame}

\begin{frame}
\begin{center}
  \includegraphics[scale=0.35]{k_regression1}
\end{center}
\end{frame}

\begin{frame}
\begin{center}
  \includegraphics[scale=0.35]{k_regression2}
\end{center}
\end{frame}

\begin{frame}
\begin{center}
  \includegraphics[scale=0.35]{k_regression3}
\end{center}
\end{frame}

\begin{frame}
\begin{center}
  \includegraphics[scale=0.35]{k_regression4}
\end{center}
\end{frame}

\begin{frame}
\frametitle{Reasoning about Knowledge}
This permits a regression rule for knowledge:
\begin{multline*}
\Reg(\Knows(agt,\phi,do(c,s)))\,\,\isdef \\
  \exists o: o = Obs(agt,c,s)\,\wedge \\
  \Knows(agt,\forall c': Poss(c')\wedge Obs(agt,c')=o \rightarrow \Reg(\phi,c'),s)
\end{multline*}
\ \\
\ \\
We can reason effectively by reducing a query about future knowledge to a query about initial knowledge.
\end{frame}

\section{Group Knowledge}

\begin{frame}
\frametitle{Group-Level Knowledge}
Simple group-level modalities like "everyone knows" can be expanded down to individual-level knowledge:
\begin{gather*}
\EKnows(G,\phi,s)\,\,\isdef\,\,\bigwedge_{agt \in G} \Knows(agt,\phi,s)\\
\EKnows^{n}(G,\phi,s)\,\,\isdef\,\,\EKnows(G,\EKnows^{n-1}(G,\phi),s)
\end{gather*}
\ \\
\ \\
\pause
But common knowledge is infinitary:
\begin{equation*}
\CKnows(G,\phi,s)\,\,\,\isdef\,\,\,\EKnows^{\infty}(G,\phi,s)\,\,\,\pause \isdef\,\,\,???
\end{equation*}
\end{frame}

\begin{frame}
\frametitle{Common Knowledge}
$\CKnows$ is usually specified with a second-order axiom.

This precludes using regression for common knowledge.
\ \\
\ \\
\pause
But maybe we can formulate a special-case rule?
\begin{multline*}
\Reg(\CKnows(G,\phi,do(c,s)))\,\,\isdef\,\,\exists o: o = \mathbf{CObs}(G,c,s)\,\wedge \\
  \CKnows(G,\forall c': Poss(c')\wedge \mathbf{CObs}(agt,c')=o \rightarrow \Reg(\phi,c'),s)
\end{multline*}
\pause
This is \alert{impossible}, even in very restricted domains.
\end{frame}

\begin{frame}
\begin{center}
  \includegraphics[scale=0.35]{pk_regression}
\end{center}
\end{frame}

\begin{frame}
\centering To express the regression of common knowledge,\\
we need a more powerful epistemic language \\
\end{frame}

\section{Epistemic Paths}

\begin{frame}
\frametitle{Dynamic Logic}
Dynamic logic is typically conceived as a logic of action:
\begin{align*}
Sequence: & \,\,\,\,\,\,\, A_1\,;\,A_2 \\
Choice: & \,\,\,\,\,\,\, A_1\,\cup\,A_2 \\
Test: & \,\,\,\,\,\,\, ?\phi\,;\,A_1 \\
Iteration: &  \,\,\,\,\,\,\, (A_1\,;\,?\phi\,;\,A_2)^*
\end{align*}
\ \\
\ \\
\pause
More generally, it is a language for exploring Kripke structures.
\end{frame}

\begin{frame}
\frametitle{Epistemic Path Language}
van Bentham, van Eijck \& Kooi, 2006,\\
"Logics of Communication and Change":
\ \\
\ \\
What if we interpret dynamic logic epistemically?
\ \\
\ \\
\pause
Great idea!  Let's capture it in the situation calculus:
\begin{itemize}
\item first-order preconditions and effects
\item quantifying-in, de-dicto/de-re
\item sets of concurrent actions
\end{itemize}
\end{frame}

\begin{frame}
\frametitle{Knowledge using Epistemic Paths}
We introduce a new macro $\PKnows(\pi,\phi,s)$ subsuming all group-level knowledge modalities:
\begin{align*}
\Knows(Ann,\phi) &  \,\,\Rightarrow\,\,\PKnows(Ann,\phi) \\
\Knows(Ann,\Knows(Bob,\phi)) & \,\,\Rightarrow\,\,\PKnows(Ann ; Bob,\phi) \\
\EKnows(\{Ann,Bob\},\phi) & \,\,\Rightarrow\,\,\PKnows(Ann \cup Bob,\phi) \\
\CKnows(\{Ann,Bob\},\phi) & \,\,\Rightarrow\,\,\PKnows((Ann \cup Bob)^*,\phi)
\end{align*}
\end{frame}

\begin{frame}
\frametitle{Semantics of Epistemic Paths}
We need to use \emph{first-order} dynamic logic:
\begin{align*}
\KDo(\pi,s,s') & \,\,\isdef\,\,\exists \mu,\mu': \KDo(\pi,\mu,s,\mu',s') \\
\KDo(agt,\mu,s,\mu',s') & \,\,\isdef\,\,\mu = \mu' \wedge K(agt,s',s) \\
\KDo(?\phi,\mu,s,\mu',s') & \,\,\isdef\,\,s = s' \wedge \mu = \mu' \wedge \mu(\phi)[s] \\
\KDo(\pi_1 ; \pi_2,\mu,s,\mu',s') & \,\,\isdef\,\,\exists \mu'',s'': \KDo(\pi_1,\mu,s,\mu'',s'') \\
\,\,  & \,\,\,\,\,\,\,\,\,\,\,\,\,\,\,\,\,\,\,\,\,\,\,\, \wedge \KDo(\pi_2,\mu'',s'',\mu',s')\\
\KDo(\exists x,\mu,s,\mu',s') & \,\,\isdef\,\,s = s' \wedge \exists z: \mu' = \mu[x/z] \\
\dots & \dots\dots
\end{align*}
\end{frame}

\begin{frame}
\frametitle{Knowledge using Epistemic Paths}
The definition of $\PKnows$ mirrors $\Knows$:
\begin{equation*}
\PKnows(\pi,\phi,s)\,\,\isdef\,\,\forall s': \left[ \KDo(\pi,s',s) \rightarrow \phi[s']\right]
\end{equation*}
\ \\
\ \\
\pause
The regression rule for this modality is:
\begin{multline*}
\Reg(\PKnows(\pi,\phi,do(c,s)))\,\,\isdef \\
  \forall c': \PKnows(\Trn(\pi,c,c'),\Reg(\phi,c'),s)
\end{multline*}
\end{frame}

\section{Path Regressor}

\begin{frame}
\frametitle{The Epistemic Path Regressor}
The epistemic path regressor $\Trn$ uses a fresh variable $x$ to track the action that was performed in the current situation:
\begin{equation*}
\Trn(\pi,c,c')\,\,\isdef\,\,\exists x ; ?x=c ; \TrnA(\pi,x) ; ?x=c'
\end{equation*}

And the auxiliary regressor $\TrnA$ encodes the local semantics of each operator:
\begin{multline*}
\TrnA(agt,x) \,\,\isdef \,\,\,\exists o\,;\,?Obs(agt,x)=o\,;\,agt\,;\\
   \exists x\,;\,?\left(Poss(x)\wedge Obs(agt,x)=o\right)
\end{multline*}
\end{frame}

\begin{frame}
\begin{center}
  \includegraphics[scale=0.35]{pk_regression}
\end{center}
\end{frame}

\begin{frame}
\frametitle{The Epistemic Path Regressor}
The other cases for $\TrnA$ are straightforward:
\begin{align*}
\TrnA(?\phi,x) & \,\,\,\isdef\,\,\,?\Reg(\phi,x) \\
\TrnA(\exists y,x) & \,\,\,\isdef\,\,\,\exists y \\
\TrnA(\pi_1 ; \pi_2,x) & \,\,\,\isdef\,\,\,\TrnA(\pi_1,x) ; \TrnA(\pi_2,x) \\
\TrnA(\pi_1 \cup \pi_2,x) & \,\,\,\isdef\,\,\,\TrnA(\pi_1,x) \cup \TrnA(\pi_2,x) \\
\TrnA(\pi^*,x) & \,\,\,\isdef\,\,\,\TrnA(\pi,x)^*
\end{align*}
\end{frame}

\begin{frame}
\frametitle{We can now prove...}
\begin{equation*}
\PKnows(Bob,loc = C,do(read(Bob),S_0))
\end{equation*}
\pause
\begin{equation*}
\neg\PKnows((Ann \cup Bob)^*,loc = C,do(read(Bob),S_0))
\end{equation*}
\pause
\begin{equation*}
\PKnows((Ann \cup Bob)^*,\mathbf{KRef}(Bob,loc),do(read(Bob),S_0))
\end{equation*}
\pause
\begin{equation*}
\PKnows((Ann \cup Bob)^*,loc = C,do(read(A), do(read(Bob),S_0)))
\end{equation*}
\end{frame}

\section{Conclusion}

\begin{frame}
\frametitle{Achievements}
We now have:
\begin{itemize}
\item The Situation Calculus; with
\item Common Knowledge; and
\item A Regression Rule
\end{itemize}
\ \\
\ \\
\pause
Which gives us:
\begin{itemize}
\item A rich theory of knowledge and action; with
\item Group-level knowledge operators; and
\item An effective reasoning procedure
\end{itemize}
\end{frame}


\begin{frame}
\frametitle{Further work}
Answering the regressed query:
\begin{itemize}
\item FODL is highly undecidable in general!
\item But for finite domains, our paths are reducible to PDL.
\item Is there a fragment of FODL with nicer properties?
\end{itemize}
\ \\
\ \\
Asynchronous domains:
\begin{itemize}
\item What if $Obs(agt,c,s) = \{\}$ ?
\item Must reason about arbitrarily-many hidden actions
\pause
\item Journal version and thesis coming soon...
\end{itemize}
\end{frame}

\begin{frame}
\centering \large Thank You\\
\end{frame}

\end{document}
