\documentclass[compress]{beamer}

\mode<presentation>
{
  \usetheme{Dresden}
  \setbeamercovered{transparent}
}

\usepackage[english]{babel}

\usepackage[latin1]{inputenc}

\usepackage{amsmath}
\usepackage[all]{xy}

\usepackage{times}
\usepackage[T1]{fontenc}

\usepackage{algorithm}
\usepackage{algorithmic}

\newcommand{\isdef}{\hbox{$\stackrel{\mbox{\tiny def}}{=}$}}
\newcommand{\Dt}{\mathcal{D}}
\newcommand{\Reg}{\mathcal{R}}
\newcommand{\Pst}{\mathcal{P}}
\newcommand{\Trn}{\mathcal{T}}
\newcommand{\Kln}{\mathcal{K}}
\newcommand{\TrnA}{\Trn_{a}}
\newcommand{\EKnows}{\mathbf{EKnows}}
\newcommand{\Knows}{\mathbf{Knows}}
\newcommand{\CKnows}{\mathbf{CKnows}}
\newcommand{\KnowsZ}{\mathbf{Knows_{0}}}
\newcommand{\PKnowsZ}{\mathbf{PKnows_{0}}}
\newcommand{\PKnows}{\mathbf{PKnows}}
\newcommand{\KTrans}{\mathbf{KDo}}
\newcommand{\KDo}{\mathbf{KDo}}
\newcommand{\EDo}{\mathbf{EDo}}
\newcommand{\vars}[1]{\bar{#1}}
\newcommand{\PbU}{PbU}


\title
{Complex Epistemic Modalities \\in the Situation Calculus}

\author
{Ryan F. Kelly\\
and Adrian R. Pearce}

\institute[The University of Melbourne]
{
  Department of Computer Science and Software Engineering\\
  The University of Melbourne\\
  Victoria, 3010, Australia\\
  \{rfk,adrian\}@csse.unimelb.edu.au
}

\date[The University of Melbourne]
{19th September 2007}

\pgfdeclareimage[height=1.2cm]{university-logo}{MINSilvr}
\logo{\pgfuseimage{university-logo}}

\begin{document}

\begin{frame}
  \titlepage
\end{frame}

\section{Introduction}

\begin{frame}
\centering "Complex Epistemic Modalities in the Situation Calculus"
\ \\
\ \\
\begin{itemize}
\item TODO: outline
\end{itemize}
\end{frame}

\begin{frame}
\frametitle{Knowledge in the Situation Calculus}
Reify possible worlds using "possible situations":
\begin{equation*}
 K(agt,s',s)
\end{equation*}
And define knowledge using a simple macro:
\begin{equation*}
\Knows(agt,\phi,s) \isdef \forall s': \left( K(agt,s',s) \rightarrow \phi[s']\right)
\end{equation*}
\end{frame}

\begin{frame}
\frametitle{Knowledge and Action}
Successor state axiom for $K$:
\begin{multline*}
K(agt,s'',do(c,s)) \equiv \exists c',s': s'' = do(c',s') \wedge K(agt,s',s)\\
  Poss(c',s') \wedge Obs(agt,c',s') = Obs(agt,c,s)
\end{multline*}
\end{frame}

\begin{frame}
\frametitle{Reasoning about Knowledge}
Regression rule for knowledge
\begin{multline*}
\Reg(\Knows(agt,\phi,do(c,s))) \isdef \exists y: y = Obs(agt,c,s) \wedge \Knows(agt,\\
  \forall c': \Reg(Poss(c')\wedgeObs(agt,c')=y) \rightarrow \Reg(\phi,c'),s)
\end{multline*}

Repeated applications can reduce it to a query about knowledge in the initial situation.
\end{frame}

\begin{frame}
\frametitle{Group-Level Knowledge}
Simple group-level modalities can be expanded down to individual level knowledge.
$\EKnows(G,\phi,s) \isdef \bigwedge_{agt \in G} \Knows(agt,\phi,s)$
But common knowledge cannot be handled in this way.
\end{frame}

\begin{frame}
\frametitle{Common Knowledge}
Instead, it is usually introduced using an explicit second-order axioms.

This precludes the use of regression for reasoning about common knowledge.
\end{frame}

\begin{frame}
\frametitle{Common Knowledge}
Perhaps we can formulate a separate regression rule?
 
Impossible!
\end{frame}

\begin{frame}
\frametitle{Common Knowledge}
Diagram explaining why this is not possible.
Need a richer epistemic language.
\end{frame}

\begin{frame}
\frametitle{Dynamic Logic}
Dynamic logic typically thought of as a logic of action.

More generally, it is a language for exploring Kripke structures.
\end{frame}

\begin{frame}
\frametitle{Epistemic Path Language}
vanBentham et al, LCC: let's interpret dynamic logic epistemically.
\end{frame}


\begin{frame}
\centering \large Thank You\\
\end{frame}

\end{document}
