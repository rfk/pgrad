%\documentclass[handout,a4paper]{beamer}
\documentclass{beamer}

\usepackage{amsmath}
\usepackage{graphicx}

\newcommand{\isdef}{\hbox{$\stackrel{\mbox{\tiny def}}{=}$}}

\mode<presentation>{\usetheme{Dresden}}

\title{Common Knowledge, Hidden Actions,\\ and the Frame Problem}
\author[Ryan Kelly (rfk@csse.unimelb.edu.au)]{Ryan Kelly \and Adrian Pearce}
\begin{document}

\begin{frame}
  \titlepage
\end{frame}

\section{Motivation}

\begin{frame}
\frametitle{Motivation}
What do we want?\ \\
\ \\
\begin{itemize}
\item A powerful account of \emph{multi-agent knowledge and action}:
  \begin{itemize}
    \item with \emph{Common Knowledge} and other group-level operators
    \item with partial observability and potential for \emph{Hidden Actions}
    \item using regression to deal with the \emph{Frame Problem}
  \end{itemize}
\item that supports complex symbolic reasoning
\item that allows agents to reason about their own world
\item but is still computationally feasible
\end{itemize}
\end{frame}

\begin{frame}
  \frametitle{Outline}
  \tableofcontents
\end{frame}

\section{Knowledge \& Action}

\begin{frame}
\frametitle{Knowledge}

Knowledge is a modal operator
\begin{itemize}
\item Written like this:\ \ \  $\mathbf{Knows}(agt,\phi)$
\item $\neg\mathbf{Knows}(agt,\phi)\ \ \not\rightarrow\ \ \mathbf{Knows}(agt,\neg\phi)$
\end{itemize}
\ \\
\ \\
Knowledge is "true belief":
\[ \mathbf{Knows}(agt,\phi)\ \rightarrow\ \phi \]
\ \\
\ \\
Typically approached using modal logic.

Knowledge is given semantics in terms of "possible worlds".

\end{frame}

\begin{frame}
\frametitle{Possible Worlds Semantics}

If an agent is unsure about the state of the world, there must be several
different states of the world that it considers possible.

\begin{center}
  \includegraphics[scale=0.4]{poss_worlds.png}
\end{center}

The agent \emph{knows} $\phi$ iff $\phi$ is true in all possible worlds.

\end{frame}

\begin{frame}
\frametitle{Knowledge and Action}
"Possible worlds" works well for reasoning about a static knowledge base,
but what if the world itself is changing?
\begin{itemize}
\item  Does the agent \emph{know} that the world has changed?
\item  How should the worlds considered possible change in response?
\item  What of all the associated reasoning-about-actions problems? (frame, quantification, ramification, ...)
\end{itemize}
\ \\
\ \\
Need to integrate with a powerful \emph{theory of action}.
\pause
\ \\
\ \\
I happen to know just the thing...
\end{frame}

\begin{frame}
\frametitle{The Situation Calculus}
\emph{Actions} are instantaneous events causing the world to change
\begin{itemize}
  \item $pickup(Thomas,Bowl)$, $beginTask(Richard,mix(Bowl,1))$
\end{itemize}
\emph{Situations} are histories of actions that have been performed
\begin{itemize}
  \item $S_0$, $do(pickup(Harriet,Knife),S_0)$
\end{itemize}
\emph{Fluents} are situation-dependent properties of the world
\begin{itemize}
  \item $Poss(a,s)$, $Holding(Harriet,Knife,s)$
\end{itemize}
\emph{Successor State Axioms} define what holds after an action in terms of what held before the action
\begin{itemize}
  \item $Holding(agt,obj,do(a,s)) \equiv a = pickup(agt,obj)$ \\
        $\,\,\,\,\,\,\,\,\,\,\,\,\,\vee Holding(agt,obj,s) \wedge \neg\left(a = drop(agt,obj)\right)$
\end{itemize}
\end{frame}

\begin{frame}
\frametitle{Knowledge in the Situation Calculus}

Why the situation calculus?
\begin{itemize}
\item Elegant monotonic solution to the frame problem
\item Effective reasoning procedure
\item situations provide a direct encoding of "possible worlds"
\end{itemize}
\ \\
\ \\
Standard approach due (mainly) to Scherl and Levesque:\\
"Knowledge, Action and the Frame Problem", AI, 2003

\end{frame}

\begin{frame}
\frametitle{Possible Situations}
We want to say "$agt$ knows $\phi$ in situation $s$":
\begin{equation*}
\mathbf{Knows}(agt,\phi,s)
\end{equation*}
To do this, we introduce a fluent meaning that "in situation $s$, $agt$ considers it possible that the world may be in situation $s'$":
\begin{equation*}
K(agt,s',s)
\end{equation*}
\end{frame}

\begin{frame}
\frametitle{Possible Situations}
\begin{center}
  \includegraphics[scale=0.5]{k_relation.png}
\end{center}

We can then define knowledge as a simple macro:
\[ \mathbf{Knows}(agt,\phi,s)\ \isdef\ \forall s':\left[K(agt,s',s)\rightarrow \phi(s')\right] \]
\end{frame}

\begin{frame}
\frametitle{The Frame Problem}
\begin{center}
  \includegraphics[scale=0.5]{frame_soln.png}
\end{center}
\end{frame}

\begin{frame}
\frametitle{The Frame Solution}
Since $K$ is just a fluent, we specify how it changes  between situations
with a successor state axiom:
\begin{equation*}
 K(agt,s'',do(a,s)) \equiv \exists s': \ s''=do(a,s')
 \wedge K(agt,s',s) \wedge Poss(a,s')
\end{equation*}

To specify what the agent knows initially, restrict the properties of situations
K-related to $S_0$:
\begin{gather*}
  \mathbf{Knows}(Richard,\neg Holding(Thomas,Knife),S_0) \\
  \Rightarrow \ \ \forall s: \left[K(Richard,s,S_0) \rightarrow \neg Holding(Thomas,Knife,s) \right]
\end{gather*}

Automatically inherits the solution to the frame problem!
\end{frame}

\begin{frame}
\frametitle{The Frame Solution}
We can then reason about knowledge using \emph{regression}.

This is a syntactic transformation denoted by $\mathcal{R}$:
\begin{equation*}
\mathcal{D} \models\ \ \phi[do(a,s)]\ \equiv\ \mathcal{R}\{\phi\}[s]
\end{equation*}

For knowledge, the rule is:
\begin{multline*}
  \mathcal{R}\{\mathbf{Knows}(agt,\phi,do(a,s))\}\ \Rightarrow\ \\
    \mathbf{Knows}(agt,Poss(a) \rightarrow\ \mathcal{R}\{\phi[do(a,s)]\},s)
\end{multline*}
\end{frame}

\begin{frame}
\frametitle{Why Regression?}
By repeatedly applying $\mathcal{R}$, we reach an expression at $S_0$.

This can be decided without many of the axioms in $\mathcal{D}$:
\begin{equation*}
  \Sigma \cup \mathcal{D}_{S_0} \cup \mathcal{D}_{Poss} \cup \mathcal{D}_{ssa} \cup \mathcal{D}_{una} \models \phi[s]
\end{equation*}
\begin{center}
iff
\end{center}
\begin{equation*}
  \mathcal{D}_{S_0} \cup \mathcal{D}_{una} \models \mathcal{R}^{*}\{\phi\}[S_0]
\end{equation*}
This is usually easier to answer, since regression has already done some
of the reasoning for us.
\end{frame}

\begin{frame}
\frametitle{Extensions}
Can easily add several powerful extensions:

\begin{itemize}
\item Sensing Actions
\end{itemize}
\begin{equation*}
SR(a,s)=r\ \equiv\ \Phi(a,r,s)
\end{equation*}

\begin{itemize}
\item Concurrent Actions
\end{itemize}
\begin{equation*}
  do(\left\{pickup(Richard,Knife),drop(Thomas,Bowl)\right\},s)
\end{equation*}
\end{frame}


\begin{frame}
\frametitle{Shortcomings}
This is a very powerful framework, but it was designed for single-agent systems:
\begin{itemize}
 \item Agents must be aware of \emph{all} actions that occur
 \item No formal treatment of group-level knowledge operators
 \item Requires omniscient viewpoint for reasoning
\end{itemize}
\ \\
\ \\
In short, it's not suitable for complex multi-agent domains.
\ \\
\ \\
Can we do better, while maintaining the theoretical and practical elegance
of this approach?
\end{frame}

\begin{frame}
\frametitle{Observations}

With multiple agents:
\begin{itemize}
\item might not know all actions that have occurred
\item might not know all sensing results
\item but might have \emph{partial} knowledge of these things
\end{itemize}
\ \\
\ \\
\pause
Capture all this with the notion of \emph{Observations}:
\begin{gather*}
  Obs(agt,c,s) = \{o1,o2,...\} \\
  View(agt,do(c,s)) = Obs(agt,c,s) \cdot View(agt,s)
\end{gather*}
\end{frame}

\begin{frame}
\frametitle{The Local Perspective}
\begin{itemize}
\item \emph{Action}: global event, changes state of world
\item \emph{Observation}: local event, changes state of agent's knowledge\\
\ \\
\item \emph{Situation}: history of actions, omniscient viewpoint
\item \emph{View}: history of observations, local viewpoint
\end{itemize}
\ \\
\ \\
Clearly, we require:
\begin{equation*}
K(agt,s',s)\ \equiv\ View(agt,s') = View(agt,s)
\end{equation*}
\end{frame}

\begin{frame}
\frametitle{Knowledge and Observation}

For the moment, assume a \emph{synchronous} domain:
\begin{equation*}
Obs(agt,c,s)\ \neq\ \{\}
\end{equation*}
Then individual knowledge is straightforward:
\begin{multline*}
\mathcal{R}\{\mathbf{Knows_0}(agt,\phi,do(c,s))\}\ \Rightarrow \\
\exists o: Obs(agt,c,s)=o\ \wedge\\
\forall c': \mathbf{Knows_0}(agt,Poss(c') \wedge Obs(agt,c')=o \rightarrow \mathcal{R}\{\phi[do(c',s)]\},s)
\end{multline*}
\end{frame}

\section{Common Knowledge}
\begin{frame}
\frametitle{Group-Level Knowledge}
The basic group-level operator is "Everyone Knows":
\begin{gather*}
\mathbf{EKnows}(G,\phi,s)\ \isdef \bigwedge_{agt \in G} \mathbf{Knows}(agt,\phi,s)\\
\mathbf{EKnows}^{2}(G,\phi,s)\ \isdef\ \mathbf{EKnows}(G,\mathbf{EKnows}(G,\phi),s)\\
\dots \\
\mathbf{EKnows}^{n}(G,\phi,s)\ \isdef\ \mathbf{EKnows}(G,\mathbf{EKnows}^{n-1}(G,\phi),s)
\end{gather*}
Eventually, we get "Common Knowledge":
\begin{gather*}
\mathbf{CKnows}(G,\phi,s)\ \isdef\ \bigwedge_{n\in\mathbb{N}} \mathbf{EKnows}^{n}(agt,\phi,s)
\end{gather*}
\end{frame}

\begin{frame}
\frametitle{Graphically}
\begin{center}
  \includegraphics[scale=0.55]{pk_relation.png}
\end{center}
\end{frame}

\begin{frame}
\frametitle{Regressing Group Knowledge}
Since $\mathbf{EKnows}$ is finite, it can be expanded to perform regression.
\ \\
\ \\
$\mathbf{CKnows}$ is infinitary, so this won't work for common knowledge.

We need to regress it directly.  Maybe like this?
\begin{multline*}
\mathcal{R}\{\mathbf{CKnows}(G,\phi,do(c,s))\}\ \Rightarrow \\
\exists o: \mathbf{CObs}(G,c,s)=o\ \wedge\\
\forall c': \mathbf{CKnows}(G,Poss(c') \wedge \mathbf{CObs}(agt,c')=o \rightarrow \mathcal{R}\{\phi[do(c',s)]\},s)
\end{multline*}
\ \\
\ \\
\pause
It is \alert{impossible} to express $\mathcal{R}\{\mathbf{CKnows}\}$ in terms of $\mathbf{CKnows}$
\end{frame}

\begin{frame}
\frametitle{Graphically}
\begin{center}
  \includegraphics[scale=0.55]{pk_relation.png}
\end{center}
\end{frame}

\begin{frame}
\frametitle{Epistemic Paths}
We can interpret Dynamic Logic epistemically:
\begin{gather*}
\mathbf{KDo}(agt,s,s')\ \isdef\ K(agt,s',s)\\
\mathbf{KDo}(?\phi,s,s')\ \isdef\ s'=s \wedge \phi[s]\\
\mathbf{KDo}(\pi_1;\pi_2,s,s')\ \isdef\ \exists s'': \mathbf{KDo}(\pi_1,s,s'') \wedge \mathbf{KDo}(\pi_2,s'',s')\\
\mathbf{KDo}(\pi_1\cup\pi_2,s,s')\ \isdef\ \mathbf{KDo}(\pi_1,s,s'') \vee \mathbf{KDo}(\pi_2,s,s')\\
\mathbf{KDo}(\pi^*,s,s')\ \isdef\ \mathrm{refl. tran. clos.}\left[\mathbf{KDo}(\pi,s,s'')\right]\\
\end{gather*}

Idea from van Bentham, van Eijck and Kooi\\
"Logics of Communication and Change", Info. \& Comp., 2006
\end{frame}

\begin{frame}
\frametitle{Epistemic Paths}
New macro for path-based knowledge:
\begin{equation*}
\mathbf{PKnows}(\pi,\phi,s)\ \isdef\ \forall s': \mathbf{KDo}(\pi,s,s') \rightarrow \phi[s']
\end{equation*}
Used like so:
\begin{gather*}
\mathbf{Knows}(agt,\phi,s)\ \equiv\ \mathbf{PKnows}(agt,\phi,s)\\
\mathbf{Knows}(agt_1,\mathbf{Knows}(agt_2,\phi),s)\ \equiv\ \mathbf{PKnows}(agt_1 ; agt_2,\phi,s)\\
\mathbf{EKnows}(G,\phi,s)\ \equiv\ \mathbf{PKnows}(\bigcup_{agt \in G}agt,\phi,s)\\
\mathbf{CKnows}(G,\phi,s)\ \equiv\ \mathbf{PKnows}((\bigcup_{agt \in G}agt)^*,\phi,s)
\end{gather*}
\end{frame}

\begin{frame}
\frametitle{Regressing Epistemic Paths}
It's now possible to formulate a regression rule for $\mathbf{PKnows}$ in synchronous domains:
\begin{multline*}
\mathcal{R}\{\mathbf{PKnows_0}(\pi,\phi,do(c,s))\} \Rightarrow \\
 \forall c': \mathbf{PKnows_0}(\mathcal{T}\{\pi,c,c'\},\mathcal{R}\{\phi[c']\},s)
\end{multline*}
$\mathcal{T}$ basically encodes the semantics of $K$ and $\mathbf{KDo}$
\begin{gather*}
\mathcal{T}\{agt\}\ \isdef\ \mathrm{s.s.a.\ for}\ K\ \mathrm{fluent} \\
\mathcal{T}\{?\phi\}\ \isdef\ ?\mathcal{R}\{\phi\} \\
\mathcal{T}\{\pi_1\cup\pi_2\}\ \isdef\ \mathcal{T}\{\pi_1\}\cup\mathcal{T}\{\pi_2\} \\
\mathcal{T}\{\pi^*\} \isdef\ \mathcal{T}\{\pi\}^*
\end{gather*}
\end{frame}

\section{Hidden Actions}

\begin{frame}
\frametitle{Hidden Actions}
What if the domain is asynchronous?\\
How do we handle actions that are possible but unobservable?
\begin{equation*}
\mathbf{PbU}(agt,c,s)\ \isdef\ Poss(c,s) \wedge Obs(agt,c,s)=\{\}
\end{equation*}
\ \\
\ \\
There may be arbitrarily-long sequences of such hidden actions.\\
We cannot use standard regression for this.
\end{frame}

\begin{frame}
\frametitle{Hidden Actions}
For individual knowledge, we can use \emph{property persistence}:
\begin{equation*}
\mathcal{P}\{\phi,\alpha\}\ \equiv\ \forall s': s \leq_{\alpha} s' \rightarrow \phi[s']
\end{equation*}
Combined with regression, we get:
\begin{multline*}
\mathcal{R}\{\mathbf{Knows}(agt,\phi,do(c,s))\}\ \Rightarrow \\
\exists o: Obs(agt,c,s)=o\ \wedge\\
\forall c': \mathbf{Knows}(agt,Poss(c') \wedge Obs(agt,c')=o \rightarrow \\
 \mathcal{R}\{\mathcal{P}\{\phi,\mathbf{PbU}(agt)\}[do(c',s)]\},s)
\end{multline*}
\end{frame}

\begin{frame}
\frametitle{Hidden Actions}
For $\mathbf{PKnows}$, we fake it using $\mathbf{PKnows_0}$ and a stack of empty actions:
\begin{gather*}
\mathcal{E}\{do(c,s)\}\ \isdef\ do(\{\},do(c,\mathcal{E}\{s\})) \\
\mathcal{E}^n\{s\}\ \isdef\ \mathcal{E}\{\mathcal{E}^{n-1}\{s\}\}
\end{gather*}
Using a fixpoint of this construction:
\begin{equation*}
\mathbf{PKnows}(\pi,\phi,s)\ \isdef\ \mathbf{PKnows_0}(\pi,\phi,\mathcal{E}^{\infty}\{s\})
\end{equation*}
We can prove $\mathbf{PKnows}(agt,\phi,s) \equiv \mathbf{Knows}(agt,\phi,s)$ using these definitions.
\end{frame}

\section{The Frame Problem}

\begin{frame}
\frametitle{The Frame Problem}
Representational Frame Problem:
\begin{itemize}
\item Knowledge evolves based on the effects of actions, without requiring additional axioms
\end{itemize}
\ \\
\ \\
Inferential Frame Problem:
\begin{itemize}
\item Regression can be used for more effective reasoning
\end{itemize}
\ \\
\ \\
Omniscient Viewpoint Problem:
\begin{itemize}
\item Regression can be applied using a view rather than a full situation term
\end{itemize}
\end{frame}

\section{Conclusions}

\begin{frame}
\frametitle{Conclusions}
What did we want?
\begin{itemize}
\item A powerful account of \emph{multi-agent knowledge and action}:
  \begin{itemize}
    \item with \emph{Common Knowledge} and other group-level operators
    \item with partial observability and potential for \emph{Hidden Actions}
    \item using regression to deal with the \emph{Frame Problem}
  \end{itemize}
\item that supports complex symbolic reasoning
\item that allows agents to reason about their own world
\item but is still computationally feasible
\end{itemize}
\end{frame}

\begin{frame}
\centering \large Thank You\\
\end{frame}
\end{document}

