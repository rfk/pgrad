\documentclass{beamer}

\newcommand{\isdef}{\hbox{$\stackrel{\mbox{\tiny def}}{=}$}}

\mode<presentation>{\usetheme{Dresden}}

\usepackage{fancyvrb}

\title{High-Level Program Execution\\ for Multi-Agent Teams}
\author{Ryan Kelly\\ (Supervisor: Dr. Adrian Pearce)}

\begin{document}

\begin{frame}
  \titlepage
\end{frame}

\begin{frame}
  \frametitle{Outline}
  \tableofcontents
\end{frame}

\section{Motivation}

\begin{frame}
\frametitle{Multi-Agent Teams}
What is an agent?
\ \\
\ \\
Two categories of multi-agent system:
\begin{itemize}
  \item Open multi-agent systems
  \item Multi-agent teams
\end{itemize}
\ \\
\ \\
\pause
Teams may be conceptualised as a single agent, with distributed sensing,
reasoning and acting capabilities.\\
\ \\
\pause
What ideas can we borrow from single-agent programming?
\end{frame}

\begin{frame}
\frametitle{High-Level Program Execution}
A practical alternative to planning for single-agent systems. Specify
a program made up of:
\begin{itemize}
  \item Actions from the world
  \item Connected by high-level programming constructs
  \item Including nondeterminism where reasoning is required
\end{itemize}
\ \\
\ \\
\pause
Why?
\begin{itemize}
  \item Controlled Nondeterminism
  \item Clear, simple programs
  \item Symbolic reasoning about the world
  \item Can prove program properties (safety/liveness/etc...)
\end{itemize}
\end{frame}

\begin{frame}
\frametitle{Motivating Example: The Cooking Agents}
Several robotic chefs inhabit a kitchen, along with various ingredients,
appliances and utensils.  They must cooperate to produce a meal consisting
of several dishes.\\
\ \\
\pause
Challenges:
\begin{itemize}
  \item Encoding procedural knowledge
  \item Concurrent execution of tasks
  \item Conflict over shared resources
  \item Coordination of shared actions
\end{itemize}

\pause
Assumptions:
\begin{itemize}
  \item Deterministic, fully observable world ?
  \item Complete and reliable communication ?
\end{itemize}
\end{frame}


\section{The Situation Calculus \& Golog}

\begin{frame}
\frametitle{The Situation Calculus}

A F.O.L. formalism for reasoning about dynamic worlds:

\begin{itemize}
\pause
\item Actions: $wait$, $pickup(obj)$, $goto(x,y)$
\pause
\item Situations: $S_{0}$, $do(a_{1},S_{0})$, $do(a_{2},do(a_{1},S_{0}))$
\pause
\item Fluents: $holding(obj,s)$, $(x,y)=curPosition(s)$
\pause
\item Precondition Axioms: $Poss(drop(obj),s) \equiv holding(obj,s)$
\pause
\item Successor State Axioms:\[
\begin{array}{cc}
holding(obj,do(a,s))\equiv & a=pickup(obj)\,\vee\\
 & \left[holding(obj,s)\wedge a\neq drop(obj)\right]\end{array}\]
\end{itemize}
\pause
Successor State Axioms can provide an elegant solution to the infamous
frame problem.
\end{frame}


\begin{frame}
\frametitle{The Situation Calculus}
These axioms combine to give a first-order \emph{Theory of Action} $\mathbf{D}$,
 with which agents can:
\begin{itemize}
  \item Reason about when actions are possible
  \item Reason about the effects of actions
  \item Represent sequences of actions as situations:\[
\left[a_1, a_2, a_3, a_4\right] \rightarrow do(a_4,do(a_3,do(a_2,do(a_1,S_0))))
\]
\end{itemize}
\pause
But, what about more complex actions and procedures?
\begin{itemize}
  \item $\mathbf{if}\ available(Item)\ \mathbf{then}\ pickup(Item)\ \mathbf{else}\ beSad$
  \item $\mathbf{while}\ (\exists block)\ ontable(block)\ \mathbf{do}\ remove(block)\ \mathbf{endWhile}$
  \item $\mathbf{proc}\ remove(block)\ [pickup(block);putaway(block)]\ \mathbf{endProc}$
\end{itemize}
\end{frame}

\begin{frame}
\frametitle{GOLOG: "alGOL in LOGic"}
GOLOG \cite{levesque97golog} introduces programming to the SitCalc via "complex actions".
Primitive actions are composed using operators such as:

\begin{itemize}
  \pause
  \item $\delta_1;\delta_2$ - Perform two programs in sequence
  \pause
  \item $\phi?$ - Assert that a condition holds
  \pause
  \item $\delta_1|\delta_2$ - Choose between programs to execute
  \pause
  \item $(\pi x)\delta(x)$ - Choose suitable bindings for variables
  \pause
  \item $\delta^*$ - Execute a program zero or more times
\end{itemize}
\ \\
\ \\
\pause
Key Point:  programs can include \alert{nondeterminism}
\end{frame}

\begin{frame}
\frametitle{A Simple Example}
$[(\pi\ block)(goNextTo(block);pickup(block))];holding(Block1)?$\\
\ \\
\pause
Assuming that $\neg holding(Block1,S_0)$, this program has a unique
\emph{legal execution}:\[
s = do(pickup(Block1),do(goNextTo(Block1),S_0)\]

Corresponding to the action sequence:\[
[goNextTo(Block1), pickup(Block1)]\]

\pause
The agent reasons about the world to inform its nondeterministic choices.
\end{frame}

\begin{frame}
\frametitle{A More Complicated Example}
As a slightly more complex example, consider a Golog program for
getting to uni of a morning:\[
\begin{array}{c}
ringAlarm;(hitSnooze; ringAlarm)^*;turnOffAlarm;\\
(\pi\ food)(edible(food)?;eat(food)); haveShower; brushTeeth\\
(driveToUni\ |\ trainToUni); (time<9:00)?
\end{array}\]

There are potentially many legal executions, depending on which actions
are possible in the world.
\end{frame}

\begin{frame}
\frametitle{The Golog Family}
ConGolog provides concurrency via interleaving (think: threads)
\begin{itemize}
  \item $\delta_1\ ||\ \delta_2$: Execute two programs concurrently
  \item $\delta_1\ \gg\ \delta_2$: Prioritised concurrency
  \item $<\phi \rightarrow \delta>$: Execute $\delta$ when $\phi$ becomes true
\end{itemize}
\ \\
\ \\
\pause
IndiGolog interleaves planning with action
\begin{itemize}
  \item $\Sigma[\delta]$: Plan a full execution of $\delta$, then perform it
\end{itemize}

\end{frame}

\begin{frame}
\frametitle{Getting to Uni Reliably}
Getting to Uni could fail at the final test because I snoozed too long:\[
\begin{array}{c}
ringAlarm;(hitSnooze; ringAlarm)^*;turnOffAlarm;\\
(\pi\ food)(edible(food)?;eat(food)); haveShower; brushTeeth\\
(driveToUni\ |\ trainToUni); (time<9:00)?
\end{array}\]\\
\ \\
\pause
Better to plan over just the nondeterministic parts.\\
I could also brush my teeth while in the shower:
\[
\begin{array}{c}
ringAlarm;\Sigma[(hitSnooze; ringAlarm)^*;turnOffAlarm;(time<7:30)?]\\
(\pi\ food)(edible(food)?;eat(food)); (haveShower || brushTeeth)\\
\Sigma[(driveToUni\ |\ trainToUni); (time<9:00)?]
\end{array}\]\\
\end{frame}

\begin{frame}
\frametitle{Golog Semantics}
Single-steps of computation are specified by  $Trans(\delta,s,\delta',s')$.
\begin{itemize}
  \item $Trans(a,s,nil,s')\ \equiv\ Poss(a,s)\wedge s'=do(a,s)$
  \item $Trans(\delta_1|\delta_2,s,\delta',s')\ \equiv\ Trans(\delta_1,s,\delta',s')\vee Trans(\delta_2,s,\delta',s')$
  \item $\dots$
  \pause
  \item $\mathbf{if}\ \phi\ \mathbf{then}\ \delta_1\ \mathbf{else}\ \delta_2\ \mathbf{endIf}\ \isdef\ [\phi?;\delta_1]\ |\ [\neg\phi?;\delta_2]$
  \pause
  \item $\mathbf{while}\ \phi\ \mathbf{do}\ \delta\ \mathbf{endWhile}\ \isdef\ [[\phi?;\delta]*;\neg\phi?]$
\end{itemize}
\ \\
\pause
Termination is indicated by $Final(\delta,s)$
\begin{itemize}
  \item $Final(nil,s) \equiv true$
  \item $Final(\delta_1|\delta_2,s) \equiv Final(\delta_1,s)\vee Final(\delta_2,s)$
  \item $\dots$
\end{itemize}
\end{frame}

\begin{frame}
\frametitle{Executing a Golog Program}
Intuitively, an agent must find a \emph{legal execution} of its program - some set of choices for the nondeterministic components so that all
actions performed are possible, all test conditions are satisfied, and the
program runs to completion.\\
\ \\
Formally, the agent must find a situation $s$ such that:\[
\mathbf{D} \models \exists s,\delta' . Trans*(\delta,S_0,\delta',s) \wedge Final(\delta',s)\]

Remember, a situation is a sequence of actions - in this case, the actions which must be carried out to execute the program.
\end{frame}

\begin{frame}
\frametitle{Programming or Planning?}
\begin{itemize}
  \item May require reasoning about effects of actions to resolve nondeterminism
$\rightarrow$ Planning
  \item Or, procedure may be completely specified $\rightarrow$ Programming
  \item Golog is the middle ground $\rightarrow$ "High-Level Programming"
\end{itemize}
\pause
Full planning is possible in Golog:\[
\mathbf{while}\ \neg Goal\ \mathbf{do}\ (\pi\ a)(a)\ \mathbf{endWhile}\]
\pause
Or you can specify more of the procedure yourself, to make the agent's job
easier.\\
\ \\
Key Point: \alert{Controlled} Nondeterminism
\end{frame}


\begin{frame}
\frametitle{Why is Golog popular?}
\begin{itemize}
  \item Good level of abstraction:
  \begin{itemize}
    \item Programs based directly on actions from the domain
    \item Nondeterminism makes programs simpler and more powerful
    \item Symbolic reasoning effortlessly available
  \end{itemize}
  \item Tradeoff between programming and planning:
  \begin{itemize}
    \item Amount of nondeterminism controlled by the programmer
    \item Procedural knowledge easy to encode
    \item Full planning still available
  \end{itemize}
  \item Logic-based:
  \begin{itemize}
    \item Compact implementation in Prolog
    \item Possible to prove safety/liveness properties
  \end{itemize}
\end{itemize}
\end{frame}


\section{MIndiGolog}

\begin{frame}
\frametitle{Golog for Multi-Agent Teams?}
Is Golog a useful formalism for multi-agent teams?
Consider programming our cooking agents to make a simple salad:
\ \\
\begin{columns}
  \begin{column}{0.5\textwidth}
\[
\begin{array}{c}
\mathbf{proc}\ ChopTypeInto(type,dest)\\
\pi(obj)[IsType(obj,type)?\ ;\\
chop(obj)\ ;\ place\_in(obj,dest)]\\
\mathbf{end}\end{array}\]
  \end{column}
  \begin{column}{0.5\textwidth}
\[
\begin{array}{c}
\mathbf{proc}\ MakeSalad(dest)\\
ChopTypeInto(Lettuce,dest)\ ;\\
ChopTypeInto(Carrot,dest)\ ;\\
ChopTypeInto(Tomato,dest)\ ;\\
mix(dest,1)\\
\mathbf{end}\end{array}\]
  \end{column}
\end{columns}
\ \\
\ \\
\pause
\alert{Idea}:  have the team cooperate to execute a shared Golog program
\end{frame}

\begin{frame}
\frametitle{Multi-Agent Situation Calculus}
What do we need for a multi-agent situation calculus?
\begin{itemize}
  \item Identify which agent performs which action
  \item True concurrency of actions
  \item Explicit account of time, for coordination
  \item Natural actions, the predict the behavior of the world
  \item Knowledge about state of the world
  \item Knowledge about what actions have been performed
\end{itemize}
\ \\
\ \\
\pause
Many of these have been addressed independently by other authors.

We extend and combine them to form a compelling formalism for multi-agent
worlds.
\end{frame}

\begin{frame}
\frametitle{Multiple Agents, Explicit Time}
Multiple agents can be handled by having the first argument of each action
identify the agent performing it.
\begin{itemize}
  \item $pickup(Richard,Block1)$
  \item $goto(Thomas,x,y)$
  \item $chop(Harriet,Tomato1)$
\end{itemize}
\pause
Adding time to the SitCalc has a long history \cite{pinto94temporal},\cite{reiter96sc_nat_conc}.
We opt for a simple solution: attach an explicit time to each situation,
indicating when it was brought about:\[
\begin{array}{c}
do(a_1,t_1,S_0),\ do(a_2,t_2,do(a_1,t_1,S_0))\\
Poss(a,t,s) \equiv\ \dots\\
start(do(a,t,s)) = t
\end{array}\]
\end{frame}

\begin{frame}
\frametitle{True vs Interleaved Concurrency}
ConGolog provides \emph{interleaved concurrency} allowing multiple
high-level programs to be executed together.\\
\ \\
With mulitple agents, the world exhibits \emph{true concurrency}. Several
actions can occur at the same instant.\\
\ \\
\pause
Need to:
\begin{itemize}
  \item allow true concurrency to be represented
  \item modify the $||$ operator to take advantage of true concurrency
  \item ensure that this modification is well-behaved
\end{itemize}
\end{frame}

\begin{frame}
\frametitle{The Concurrent Situation Calculus}
Replace action terms with sets of actions, all of which are performed
at the same time \cite{reiter96sc_nat_conc}:\[
do(\{pickup(Thomas,Lettuce1),pickup(Richard,Tomato1)\},t_1,S_0)\]
\pause
\ \\
\ \\
Precondition interaction:\[
Poss(\{pickup(Richard,Obj),pickup(Harriet,Obj)\},t,s)\ ?\]
\pause
Basic solution:\[
Poss(c,s) \equiv \forall a \in c.Poss(a,s)\ \wedge\ \neg Conflicts(c,s)\]
\end{frame}

\begin{frame}
\frametitle{Continuous Actions}
Some actions have a finite duration: $mix(Thomas,Bowl1,5)$
Others are instantaneous: $pickup(Robert,Lettuce1)$.
So this concurrent action is problematic:\[
\{mix(Thomas,Bowl1,5),pickup(Robert,Lettuce1)\}\]
Standard solution: decompose continuous actions into instantaneous $begin$
and $end$ actions:\[
begin\_mix(Thomas,Bowl1,5) \dots end\_mix(Thomas,Bowl1,5)\]
But $end\_mix()$ is not a standard action.  It cannot be performed at
any time.  Rather, it \emph{must} occur whenever it is possible.
\end{frame}

\begin{frame}
\frametitle{Natural Actions}
Natural Action:  an action that must occur whenever it is possible,
unless some earlier action prevents it from occuring.
\begin{itemize}
  \item $Poss(na,t,s)$ predicts the times at which a natural action will occur.
  \item Legal Situation: one in which any natural actions that could have occured, did occur:\[
\begin{array}{c}
legal(do(c,t,s)) \equiv legal(s) \wedge\\
Poss(c,t,s) \wedge start(s) \leq t \wedge\\
\forall a,t_a [natural(a) \wedge Poss(a,t_a,s) \rightarrow (a \in c \vee t < t_a)]
\end{array}\]
\end{itemize}
\end{frame}

\begin{frame}
\frametitle{MIndiGolog}
We have created a new Golog variant that:
\begin{itemize}
  \item Integrates both true and interleaved concurrency
  \item Has an explicit notion of time
  \item Automatically handles natural actions
\end{itemize}
\ \\
\ \\
Implemented in Prolog, it functions as a \emph{centralised} mult-agent planner.
\ \\
\ \\
Using the distributed logic programming capabilities of Oz, the planning
workload can be \emph{distributed} amongst the agents.
\end{frame}

\begin{frame}
\frametitle{Making a Salad}
When making a salad, the order of ingredients doesnt matter.  This can
be expressed using concurrency:\[
\begin{array}{c}
\mathbf{proc}\ MakeSalad(dest)\\
\left[\ ChopTypeInto(Lettuce,dest)\ ||\right.\\
ChopTypeInto(Carrot,dest)\ ||\\
\left.ChopTypeInto(Tomato,dest)\ \right]\ ;\\
mix(dest,1)\\
\mathbf{end}\end{array}\]
\end{frame}

\begin{frame}
\frametitle{Making a Salad}
We must expand the continuous $mix$ action, and specify the agent performing
each action.
\ \\
\pause
But, it doesnt matter who does what: more nondeterminism
\[
\begin{array}{c}
\mathbf{proc}\, MakeSalad(dest)\\
\left[\pi(agt)(ChopTypeInto(agt,Lettuce,dest))\,||\right.\\
\pi(agt)(ChopTypeInto(agt,Carrot,dest))\,||\\
\left.\pi(agt)(ChopTypeInto(agt,Tomato,dest))\right]\,;\\
\pi(agt)\left[begin\_task(agt,mix(dest,1))\,;\right.\\
\left.\,\,\,\, end\_task(agt,mix(dest,1))\right]\,\,\mathbf{end}\end{array}\]
Of course, we could hide this with syntactic sugar...
\end{frame}

\begin{frame}
\frametitle{Actually Making Salad}
\begin{columns}
  \begin{column}{0.5\textwidth}
    \VerbatimInput[fontsize=\tiny]{output_makeSalad_1.txt}
  \end{column}
  \begin{column}{0.5\textwidth}
    \VerbatimInput[fontsize=\tiny]{output_makeSalad_2.txt}
  \end{column}
\end{columns}
\ \\
\ \\
\pause
Can give idle agents other work using:\[
MakeSalad(Bowl1)\ ||\ MakeCake(Bowl2)\ ||\ \dots\]
\end{frame}

\begin{frame}
\frametitle{Golog for Multi-Agent Teams!}
We get all the benefits of Golog for single-agent systems:
\begin{itemize}
  \item Controlled Nondeterminism
  \item Clear, simple programs
  \item Symbolic reasoning about the world
  \item Can prove safety/liveness/etc properties
\end{itemize}
\ \\
Plus:
\begin{itemize}
  \item The same program can be given to \emph{any} team of agents
  \item Conflicts and concurrency handled declaratively
  \item Planning workload easily shared between agents
\end{itemize}
\ \\
Golog has been sucessfully used in the RoboCup simulation leage \cite{Ferrein2005readylog}.
\end{frame}


\section{Ongoing Work}

\begin{frame}
\frametitle{Ongoing Work}

The current implementation assumes the world in fully observable.
\begin{itemize}
  \item What if agents have different ideas about the state of the world?
  \item What is they cant see what the other agents are doing?
\end{itemize}
\ \\
\ \\
Although the workload can be distributed, \emph{control} over planning 
an execution still centralised.

\begin{itemize}
  \item How can the execution be truly distributed?
\end{itemize}
\end{frame}

\begin{frame}
\frametitle{Knowledge}
Knowledge in the S.C. is based on the standard "possible worlds" model:
\[
\begin{array}{c}
\mathbf{Knows}(agt,\phi,s)\ \isdef\ \forall s'\ K(s',s) \rightarrow \phi(s')\\
\\
K(agt,s'',do(a,s)) \equiv \exists s'\ s''=do(a,s') \wedge K(agt,s',s)
\end{array}
\]
\pause
This assumes that all actions are public.  This is unreasonable
in a multi-agent setting, so we explicitly axiomatise action observability:
\[
\begin{array}{c}
Unobs(agt,s'',s)\ \isdef\ \neg\exists s',a\ s<do(a,s')\leq s'' \wedge CanObs(agt,a,s')\\
\\
K(agt,s'',do(a,s)) \equiv \exists s'\ Unobs(agt,do(a,s'),s'') \wedge K(agt,s',s)
\end{array}
\]

\end{frame}

\begin{frame}
\frametitle{Reasoning about Knowledge}
Our account of knowledge requires reasoning about all possible future
sitations.  This typically requires second-order reasoning (Uh-Oh!)
\pause
\ \\
\ \\
We have developed an algorithm to calculate the "persistence conditions"
of a formula - the conditions under which a formula is be guaranteed to
remain true as long as a certain class of actions do not occur:
\[
\mathcal{P}[\phi,\alpha] \equiv \forall s'' . s \leq s'' \rightarrow \phi(s'') \vee \exists a,s'\ s<do(a,s')\leq s'' \wedge \alpha(a)
\]
\pause
We can thus handle multi-agent knowledge using purely first-order reasoning
(Phew!)
\end{frame}

\begin{frame}
\frametitle{When are actions safe to perform?}
In the absence of centralised control, it is "safe" for me to perform
an action $A$ when:
\begin{itemize}
  \item I \emph{know} that $A$ is a legal next step in the program
  \item For any actions $C$ that I believe other agents \emph{might} consider
safe, I \emph{know} that $A$ can legally be performed concurrently with $C$
\end{itemize}
\ \\
As long as everyone only performs actions that they know to be safe, all
actions performed will be legal steps of the shared program.
\end{frame}

\begin{frame}
\frametitle{The Goal}
\begin{itemize}
  \item Have agents cooperate to execute a shared MIndiGolog program
  \item When the world is fully observable, they need not communicate
  \item In partially observable worlds, they automatically communicate
 to enable cooperation
\end{itemize}
\end{frame}

\section{Conclusion}

\begin{frame}
\frametitle{Conclusion}
\begin{itemize}
  \item High-Level Program Execution is a promising paradigm for multi-agent teams
  \item Programmer specifies as much or as little of the procedure as
desired
  \item Agents cooperate to resolve the nondeterminism according to a
Theory of Action
\end{itemize}
\end{frame}

\bibliography{/storage/uni/pgrad/library/references}
\bibliographystyle{aaai}

\end{document}
